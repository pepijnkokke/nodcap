\documentclass[12pt,a4paper,UKenglish,mscres,logo,twoside,notimes,parskip,lfcs]{infthesis}
\shieldtype{3}
\usepackage{cmap}
\usepackage{alphabeta}
\usepackage[greek,english]{babel}
\languageattribute{greek}{polutoniko}
\usepackage{hyperref}
\usepackage{comment}
\usepackage{lmodern}

%%% TikZ and QTree

\usepackage{tikz}
\usepackage{tikz-qtree}

%%% Term Language
\newcommand{\subst}[3]{\ensuremath{#1 \{ #2 / #3 \}}}
\newcommand{\case}[3]{%
\toks0={#2#3}%
\edef\param{\the\toks0}%
\ifx\param\empty
  \ensuremath{\text{case} \; #1 \; \{\}}
\else
  \ensuremath{\text{case} \; #1 \; \{ #2 ; #3 \}}
\fi}
\newcommand{\inl}[2]{%
  \ensuremath{#1[\text{inl}].#2}}
\newcommand{\inr}[2]{%
  \ensuremath{#1[\text{inr}].#2}}

%%% Local Variables:
%%% TeX-master: "main"
%%% End:
\title{Races in Classical Linear Logic}
\author{Wen Kokke}
\addbibresource{main.bib}
\begin{document}
\begin{preliminary}
  \abstract{%
    Process calculi based in logic, such as CP, provide a foundation for
    deadlock-free concurrent programming, but at the cost of excluding
    non-determinism and races.  We introduce \nodcap (nodcap), which extends CP
    with a novel account of non-determinism.  Our approach draws on bounded linear
    logic to provide a strongly-typed account of standard process calculus
    expressions of non-determinism.  We show that our extension is expressive
    enough to capture many uses of non-determinism in untyped calculi, such as
    non-deterministic choice, while preserving CP's meta-theoretic properties,
    including deadlock freedom.  We have formalized our calculus and its
    properties using Agda.
  }
  \maketitle
  \begin{acknowledgements}
  \end{acknowledgements}
  \standarddeclaration
  \tableofcontents
\end{preliminary}

\wen{Replace references to \cpRedAxCut1 and \cpRedAxCut2 with \emph{just} a
  reference to the first, and insert a note explaining how we deviate from \cp
  in this particular instance.}
%% Introduction
\chapter{Introduction}\label{sec:introduction}
%% - Motivating examples
Consider the following scenario:
\begin{quote}
  John and Mary are working from home one morning when they get a craving for a
  slice of cake. Being denizens of the web, they quickly find the nearest store
  which does home deliveries.
  Unfortunately for them, they both order their cake at the \emph{same} store,
  which has only one slice left. After that, all it can deliver is
  disappointment.
\end{quote}
This is an example of a race condition. We can model this scenario in the
\textpi-calculus, assuming \john, \mary and \store are three processes
modeling John, Mary and the store, and \sliceofcake and \nope are two channels
giving access to a slice of cake and disappointment, respectively.
As expected, this process has two possible outcomes: either John gets the cake,
and Mary gets disappointment, or vice versa.
\[
  \begin{array}{c}
    \tm{(\piPar{%
    \piSend{x}{\sliceofcake}{\piSend{x}{\nope}{\store}}
    }{%
    \piPar{\piRecv{x}{y}{\john}}{\piRecv{x}{z}{\mary}}
    })}
    \\[1ex]
    \rotatebox[origin=c]{270}{$\Longrightarrow^{\star}$}
    \\[1ex]
    \tm{(\piPar{\store}{\piPar{\piSub{\sliceofcake}{y}{\john}}{\piSub{\nope}{z}{\mary}}})}
    \quad
    \text{or}
    \quad
    \tm{(\piPar{\store}{\piPar{\piSub{\nope}{y}{\john}}{\piSub{\sliceofcake}{z}{\mary}}})}
  \end{array}
\]
While John or Mary may not like all of the outcomes, it is the store which is
responsible for implementing the online delivery service, and the store is happy
with either outcome. Thus, the above is program we would like to be able to
write.
\\[\baselineskip]\noindent
Now consider another scenario, which takes place \emph{after} John has already
bought the cake:
\begin{quote}
  Mary is \emph{really} disappointed when she finds out the cake has sold out.
  John, always looking to make some money, offers to sell the slice to her for a
  profit. Mary agrees to engage in a little bit of back-alley cake resale, but
  sadly there is no trust between the two.
  John demands payment first.
  Mary would rather get her slice of cake before she gives John the money.
\end{quote}
This is an example of a deadlock. We can also model this scenario in the
\textpi-calculus, assuming that \bill\ is a channel giving access to some
adequate amount of money.
\[
  \begin{array}{c}
    \tm{(\piPar{%
    \piRecv{x}{z}{\piSend{y}{\sliceofcake}{\john}}
    }{%
    \piRecv{y}{w}{\piSend{x}{\bill}{\mary}}
    })}
    \quad
    \centernot\Longrightarrow^{\star}
  \end{array}  
\]
The above process does not reduce. As both John and Mary would prefer the
exchange to be made, this program is desired by \emph{neither}. Thus, the above
is a program we would \emph{somehow} like to exclude.

%% - Overview:
Session types~\cite{honda1993} can provide a static guarantee that concurrent
programs, such as those above, respect communication protocols.
Session-typed calculi with logical foundations, such as
\piDILL~\cite{caires2010} and CP~\cite{wadler2012}, obtain deadlock freedom as a
result of a close correspondence with logic.
The same correspondence, however, also rules out non-determinism and race
conditions.

We present \nodcap (nodcap), an extension of CP~\cite{wadler2012} with
a novel account of non-determinism and races.  Inspired by bounded linear
logic~\cite{girard1992}, we introduce a form of shared channels, in which the
type of a shared channel tracks how many times it is reused.  As in the untyped
$\pi$-calculus, sharing introduces the potential of non-determinism.  We show
that our approach is sufficient to capture practical examples of races, such as
the web store, as well as other formal characterizations of non-determinism,
such as non-deterministic choice.  However, \nodcap does not lose the
metatheoretical benefits of CP: we show that it enjoys termination and
deadlock-freedom.
%%% Local Variables:
%%% TeX-master: "main"
%%% End:

%% Background
\chapter{Background}
\label{sec:background}

\section{Classical Processes}
\label{sec:cp}
In this section, we will discuss a rudimentary subset of the typed process
calculus \cp~\parencite{wadler2012,lindley2015semantics}, which we will refer to as
\rcp. 
We have chosen to discuss only a subset in order to keep the discussion of
\nodcap in \cref{sec:main} as simple as possible.
\rcp is the subset of \cp which corresponds to rudimentary linear
logic~\parencite[RLL]{girard1992}, also known as multiplicative-applicative linear
logic. 
We foresee no problems in extending the proofs from \cref{sec:main} to
cover the remaining features of \cp, polymophism and the exponentials $\ty{!A}$
and $\ty{?A}$. 

This chapter proceeds as follows.
In \cref{sec:cp-terms-and-types}, we introduce the terms, the structural
congruence, and the types of \rcp. 
In \cref{sec:cp-multiplicatives,sec:cp-additives,sec:cp-duality,sec:cp-commuting-conversions}, 
we discuss the terms and their corresponding types, in small groups,
together with their typing and reduction rules.
In \cref{sec:cp-properties}, we prove preservation, progress and
termination for \rcp.

\subsection{Terms and types}
\label{sec:cp-terms-and-types}
The term language for \rcp is a variant of the
\textpi-calculus~\parencite{milner1992b}.
Its terms are defined by the following grammar:
\begin{definition}[Terms]\label{def:cp-terms}
  \[\!
    \begin{aligned}
      \tm{P}, \tm{Q}, \tm{R}
           :=& \; \tm{\cpLink{x}{y}}       &&\text{link}
      \\ \mid& \; \tm{\cpCut{x}{P}{Q}}     &&\text{parallel composition, or ``cut'''}
      \\ \mid& \; \tm{\cpSend{x}{y}{P}{Q}} &&\text{``output''}
      \\ \mid& \; \tm{\cpRecv{x}{y}{P}}    &&\text{``input''}
      \\ \mid& \; \tm{\cpHalt{x}}          &&\text{halt}
      \\ \mid& \; \tm{\cpWait{x}{P}}       &&\text{wait}
      \\ \mid& \; \tm{\cpInl{x}{P}}        &&\text{select left choice}
      \\ \mid& \; \tm{\cpInr{x}{P}}        &&\text{select right choice}
      \\ \mid& \; \tm{\cpCase{x}{P}{Q}}    &&\text{offer binary choice}
      \\ \mid& \; \tm{\cpAbsurd{x}}        &&\text{offer nullary choice}
    \end{aligned}
  \]  
\end{definition}
%%% Local Variables:
%%% TeX-master: "main"
%%% End:

The variables \tm{x}, \tm{y}, \tm{z} and \tm{w} range over channel names.
The construct \tm{\cpLink{x}{y}} links two
channels~\parencite{sangiorgi1996,boreale1998}, forwarding messages received on
\tm{x} to \tm{y} and vice versa.
The construct \tm{\cpCut{x}{P}{Q}} creates a new channel \tm{x}, and composes
two processes, \tm{P} and \tm{Q}, which communicate on \tm{x}, in parallel.
Therefore, in \tm{\cpCut{x}{P}{Q}} the name \tm{x} is bound in both \tm{P} and
\tm{Q}.
In \tm{\cpRecv{x}{y}{P}} and \tm{\cpSend{x}{y}{P}{Q}}, round brackets denote
input, square brackets denote output. 
We use bound output~\parencite{sangiorgi1996}, meaning that both input and output
bind a new name. 
In \tm{\cpRecv{x}{y}{P}} the new name \tm{y} is bound in \tm{P}.
In \tm{\cpSend{x}{y}{P}{Q}}, the new name \tm{y} is only bound in \tm{P}, while
\tm{x} is only bound in \tm{Q}.

Terms in \rcp are identified up to structural congruence, which states that
parallel compositions \tm{\cpCut{x}{P}{Q}} are associative and commutative.
It is defined as follows:
\begin{definition}[Structural congruence]\label{def:cp-equiv}
  We define the structural congruence $\equiv$ as a reflexive, transitive
  congruence over terms which satisfies the following additional axioms:
  \[
    \begin{array}{llll}
      \cpEquivLinkComm
      & \tm{\cpLink{x}{y}}
      & \equiv \;
      & \tm{\cpLink{y}{x}}
      \\
      \cpEquivCutComm
      & \tm{\cpCut{x}{P}{Q}}
      & \equiv \;
      & \tm{\cpCut{x}{Q}{P}}
      \\
      \cpEquivCutAss1
      & \tm{\cpCut{x}{P}{\cpCut{y}{Q}{R}}}
      & \equiv \;
      & \tm{\cpCut{y}{\cpCut{x}{P}{Q}}{R}}
        \quad \text{if} \; \notFreeIn{x}{R} \; \text{and} \; \notFreeIn{y}{P}
    \end{array}
  \]
\end{definition}
%%% Local Variables:
%%% TeX-master: "main"
%%% End:

We deviate from the original presentation of \cp here, as the structural
congruence defined by \textcite{wadler2012} does not include \cpEquivLinkComm.
We include it here because it fits well with our notion of links, and it
simplifies our reduction system and the proofs of meta-theoretical properties of
\cp. 
We do not add an axiom for \cpEquivCutAss2, as it follows from
\cref{def:cp-equiv}.
Throughout this dissertation, we will leave uses of the transitivity and
congruence rules implicit.
\begin{lemma}[\cpEquivCutAssNoParen2]\label{thm:cp-cut-assoc2}
  If $\tm{x}\not\in\tm{R}$ and $\tm{y}\not\in\tm{P}$, then 
  \(
    \tm{\cpCut{y}{\cpCut{x}{P}{Q}}{R}} \equiv
    \tm{\cpCut{x}{P}{\cpCut{y}{Q}{R}}}
  \).
\end{lemma}
  \begin{proof}
    \begin{align*}
      \tm{\cpCut{y}{\cpCut{x}{P}{Q}}{R}} &\equiv \qquad \text{by \cpEquivCutComm} \\
      \tm{\cpCut{y}{\cpCut{x}{Q}{P}}{R}} &\equiv \qquad \text{by \cpEquivCutComm} \\
      \tm{\cpCut{y}{R}{\cpCut{x}{Q}{P}}} &\equiv \qquad \text{by \cpEquivCutAss1} \\
      \tm{\cpCut{x}{\cpCut{y}{R}{Q}}{P}} &\equiv \qquad \text{by \cpEquivCutComm} \\
      \tm{\cpCut{x}{P}{\cpCut{y}{R}{Q}}} &\equiv \qquad \text{by \cpEquivCutComm} \\
      \tm{\cpCut{x}{P}{\cpCut{y}{Q}{R}}}
    \end{align*}
    The side conditions for \cpEquivCutAss1 are given.
  \end{proof}
%%% Local Variables:
%%% TeX-master: "main"
%%% End:

Furthermore, structural congruence is a symmetric relation.
\begin{theorem}[Symmetry]\label{thm:cp-symmetry}
  If $\tm{P} \equiv \tm{Q}$, then $\tm{Q} \equiv \tm{P}$.
\end{theorem}
\begin{proof}
  By induction on the structure of the equivalence proof.
\end{proof}
%%% Local Variables:
%%% TeX-master: "main"
%%% End:

%
Channels in \rcp are typed using a session type system which corresponds to RLL,
the multiplicative, additive fragment of linear logic.
The types are defined by the following grammar:
\begin{definition}[Types]\label{def:cp-types}
  \[\!
    \begin{aligned}
      \ty{A}, \ty{B}, \ty{C}
           :=& \; \ty{A \tens B} &&\text{pair of independent processes}
      \\ \mid& \; \ty{A \parr B} &&\text{pair of interdependent processes}
      \\ \mid& \; \ty{\one}      &&\text{unit for} \; {\tens}
      \\ \mid& \; \ty{\bot}      &&\text{unit for} \; {\parr}
      \\ \mid& \; \ty{A \plus B} &&\text{internal choice}
      \\ \mid& \; \ty{A \with B} &&\text{external choice}
      \\ \mid& \; \ty{\nil}      &&\text{unit for} \; {\plus}
      \\ \mid& \; \ty{\top}      &&\text{unit for} \; {\with}
    \end{aligned}
  \]  
\end{definition}
%%% Local Variables:
%%% TeX-master: "main"
%%% End:

Duality plays a crucial role in both linear logic and session types.
In \rcp, the two endpoints of a channel are assigned dual types.
This ensures that, for instance, whenever a process \emph{sends} across a
channel, the process on the other end of that channel is waiting to
\emph{receive}.
Each type \ty{A} has a dual, written \ty{A^\bot}.
\begin{definition}[Negation]\label{def:cp-negation}
  \[\!
    \begin{array}{lclclcl}
              \ty{(A \tens B)^\bot} &=& \ty{A^\bot \parr B^\bot}
      &\quad& \ty{\one^\bot}        &=& \ty{\bot}
      \\      \ty{(A \parr B)^\bot} &=& \ty{A^\bot \tens B^\bot}
      &\quad& \ty{\bot^\bot}        &=& \ty{\one}
      \\      \ty{(A \plus B)^\bot} &=& \ty{A^\bot \with B^\bot}
      &\quad& \ty{\nil^\bot}        &=& \ty{\top}
      \\      \ty{(A \with B)^\bot} &=& \ty{A^\bot \plus B^\bot}
      &\quad& \ty{\top^\bot}        &=& \ty{\nil}
    \end{array}
  \]
\end{definition}
%%% Local Variables:
%%% TeX-master: "main"
%%% End:

Duality is an involutive function.
\begin{lemma}[Involutive]\label{thm:negation-involutive}
  We have $\ty{A^{\bot\bot}} = \ty{A}$.
\end{lemma}
\begin{proof}
  By induction on the structure of the type $\ty{A}$.
\end{proof}
%%% Local Variables:
%%% TeX-master: "main"
%%% End:

%
Environments associate channels with types.
\begin{definition}[Environments]\label{def:cp-environments}
  We define environments as follows:
  \[
    \ty{\Gamma}, \ty{\Delta}, \ty{\Theta}
    ::= \tmty{x_1}{A_1}\dots\tmty{x_n}{A_n}
  \] 
  Names in environments must be unique, and environments \ty{\Gamma} and
  \ty{\Delta} can only be combined as $\ty{\Gamma}, \ty{\Delta}$ if
  $\text{fv}(\ty{\Gamma}) \cap \text{fv}(\ty{\Delta}) = \varnothing$. 
\end{definition}
%%% Local Variables:
%%% TeX-master: "main"
%%% End:

Typing judgements associate processes with collections of typed channels.
\begin{definition}[Typing judgements]\label{def:cp-typing-judgement}
  A typing judgement $\seq[{ P }]{\tmty{x_1}{A_1}\dots\tmty{x_n}{A_n}}$ denotes
  that the process \tm{P} communicates along channels $\tm{x_1}\dots\tm{x_n}$
  following protocols $\ty{A_1}\dots\ty{A_n}$.
  Typing judgements can be constructed using the inference rules in
  \cref{fig:cp-typing-judgement}.
\end{definition}
%%% Local Variables:
%%% TeX-master: "main"
%%% End:

\newcommand{\cpInfAx}{%
  \begin{prooftree*}
    \AXC{$\vphantom{\seq[ Q ]{ \Delta, \tmty{y}{A^\bot} }}$}
    \NOM{Ax}
    \UIC{$\seq[ \cpLink{x}{y} ]{ \tmty{x}{A}, \tmty{y}{A^\bot} }$}
  \end{prooftree*}
}
\newcommand{\cpInfCut}{%
  \begin{prooftree*}
    \AXC{$\seq[ P ]{ \Gamma, \tmty{x}{A} }$}
    \AXC{$\seq[ Q ]{ \Delta, \tmty{y}{A^\bot} }$}
    \NOM{Cut}
    \BIC{$\seq[ \cpCut{x}{P}{Q} ]{ \Gamma, \Delta }$}
  \end{prooftree*}
}
\newcommand{\cpInfTens}{%
  \begin{prooftree*}
    \AXC{$\seq[ P ]{ \Gamma , \tmty{y}{A} }$}
    \AXC{$\seq[ Q ]{ \Delta , \tmty{x}{B} }$}
    \SYM{(\tens)}
    \BIC{$\seq[ \cpSend{x}{y}{P}{Q} ]{ \Gamma , \Delta , \tmty{x}{A \tens B} }$}
  \end{prooftree*}
}
\newcommand{\cpInfParr}{%
  \begin{prooftree*}
    \AXC{$\seq[ P ]{ \Gamma , \tmty{y}{A} , \tmty{x}{B} }$}
    \SYM{(\parr)}
    \UIC{$\seq[ \cpRecv{x}{y}{P} ]{ \Gamma , \tmty{x}{A \parr B} }$}
  \end{prooftree*}
}
\newcommand{\cpInfOne}{%
  \begin{prooftree*}
    \AXC{$\vphantom{\seq[ P ]{ \Gamma }}$}
    \SYM{(\one)}
    \UIC{$\seq[ \cpHalt{x} ]{ \tmty{x}{\one} }$}
  \end{prooftree*}
}
\newcommand{\cpInfBot}{%
  \begin{prooftree*}
    \AXC{$\seq[ P ]{ \Gamma }$}
    \SYM{(\bot)}
    \UIC{$\seq[ \cpWait{x}{P} ]{ \Gamma , \tmty{x}{\bot} }$}
  \end{prooftree*}
}
\newcommand{\cpInfPlus}[1]{%
  \ifdim#1pt=1pt
  \begin{prooftree*}
    \AXC{$\seq[ P ]{ \Gamma , \tmty{x}{A} }$}
    \SYM{(\plus_1)}
    \UIC{$\seq[{ \cpInl{x}{P} }]{ \Gamma , \tmty{x}{A \plus B} }$}
  \end{prooftree*}
  \else%
  \ifdim#1pt=2pt
  \begin{prooftree*}
    \AXC{$\seq[ P ]{ \Gamma , \tmty{x}{B} }$}
    \SYM{(\plus_2)}
    \UIC{$\seq[ \cpInr{x}{P} ]{ \Gamma , \tmty{x}{A \plus B} }$}
  \end{prooftree*}
  \else%
  \fi%
  \fi%
}
\newcommand{\cpInfWith}{%
  \begin{prooftree*}
    \AXC{$\seq[ P ]{ \Gamma , \tmty{x}{A} }$}
    \AXC{$\seq[ Q ]{ \Delta , \tmty{x}{B} }$}
    \SYM{(\with)}
    \BIC{$\seq[ \cpCase{x}{P}{Q} ]{ \Gamma , \Delta , \tmty{x}{A \with B} }$}
  \end{prooftree*}
}
\newcommand{\cpInfNil}{%
  (no rule for \ty{\nil})
}
\newcommand{\cpInfTop}{%
  \begin{prooftree*}
    \AXC{}
    \SYM{(\top)}
    \UIC{$\seq[ \cpAbsurd{x} ]{ \Gamma, \tmty{x}{\top} }$}
  \end{prooftree*}
}
\begin{figure*}[b]
  \begin{center}
    \cpInfAx
    \cpInfCut
  \end{center}
  \begin{center}
    \cpInfTens
    \cpInfParr
  \end{center}
  \begin{center}
    \cpInfOne
    \cpInfBot
  \end{center}
  \begin{center}
    \cpInfPlus1
    \cpInfPlus2
  \end{center}
  \begin{center}
    \cpInfWith
  \end{center}
  \begin{center}
    \cpInfNil
    \cpInfTop
  \end{center}
  \caption{Typing judgement for the multiplicative applicative subset of \rcp.}
  \label{fig:cp-typing-judgement}
\end{figure*}
%%% Local Variables:
%%% TeX-master: "main"
%%% End:

Reductions relate processes with their reduced forms.
They are defined as follows:
\begin{definition}[Term reduction]\label{def:cp-term-reduction-1}
  A reduction $\reducesto{P}{P'}$ denotes that the process \tm{P} can reduce to
  the process \tm{P'} in a single step. Reductions can be constructed using the
  rules in~\cref{fig:cp-term-reduction-1,fig:cp-term-reduction-2}. 
\end{definition}
%%% Local Variables:
%%% TeX-master: "main"
%%% End:

\begin{figure*}[b]
  \[
    \begin{array}{llll}
      \cpRedAxCut1
      & \tm{\cpCut{x}{\cpLink{w}{x}}{P}}
      & \Longrightarrow \;
      & \tm{\cpSub{w}{x}{P}} 
      \\
      \cpRedAxCut2
      & \tm{\cpCut{x}{\cpLink{x}{w}}{P}}
      & \Longrightarrow \;
      & \tm{\cpSub{w}{x}{P}} 
      \\
      \cpRedBetaTensParr
      & \tm{\cpCut{x}{\cpSend{x}{y}{P}{Q}}{\cpRecv{x}{z}{R}}}
      & \Longrightarrow \;
      & \tm{\cpCut{y}{P}{\cpCut{x}{Q}{\cpSub{y}{z}{R}}}}
      \\
      \cpRedBetaOneBot
      & \tm{\cpCut{x}{\cpHalt{x}}{\cpWait{x}{P}}}
      & \Longrightarrow \;
      & \tm{P}
      \\
      \cpRedBetaPlusWith1
      & \tm{\cpCut{x}{\cpInl{x}{P}}{\cpCase{x}{Q}{R}}}
      & \Longrightarrow \;
      & \tm{\cpCut{x}{P}{Q}}
      \\
      \cpRedBetaPlusWith2
      & \tm{\cpCut{x}{\cpInr{x}{P}}{\cpCase{x}{Q}{R}}}
      & \Longrightarrow \;
      & \tm{\cpCut{x}{P}{R}}
    \end{array}
  \]
  \begin{prooftree}
    \AXC{\reducesto{P}{P^\prime}}
    \SYM{\cpRedGammaCut}
    \UIC{\reducesto{\cpCut{x}{P}{Q}}{\cpCut{x}{P^\prime}{Q}}}
  \end{prooftree}
  \begin{prooftree}
    \AXC{$\tm{P}\equiv\tm{Q}$}
    \AXC{\reducesto{Q}{Q^\prime}}
    \AXC{$\tm{Q^\prime}\equiv\tm{P^\prime}$}
    \SYM{\cpRedGammaEquiv}
    \TIC{\reducesto{P}{P^\prime}}
  \end{prooftree}
  \caption{Term reduction rules for \rcp.}
  \label{fig:cp-term-reduction-1}
\end{figure*}
%%% Local Variables:
%%% TeX-master: "main"
%%% End:

\begin{figure*}[b]
  \[
    \begin{array}{llll}
      \cpRedKappaTens1
      & \tm{\cpCut{x}{\cpSend{y}{z}{P}{Q}}{R}}
      & \Longrightarrow \;
      & \tm{\cpSend{y}{z}{\cpCut{x}{P}{R}}{Q}}
        \quad \text{if} \; \notFreeIn{x}{Q}
      \\
      \cpRedKappaTens2
      & \tm{\cpCut{x}{\cpSend{y}{z}{P}{Q}}{R}}
      & \Longrightarrow \;
      & \tm{\cpSend{y}{z}{P}{\cpCut{x}{Q}{R}}}
        \quad \text{if} \; \notFreeIn{x}{P}
      \\
      \cpRedKappaParr
      & \tm{\cpCut{x}{\cpRecv{y}{z}{P}}{R}}
      & \Longrightarrow \;
      & \tm{\cpRecv{y}{z}{\cpCut{x}{P}{R}}}
      \\
      \cpRedKappaBot
      & \tm{\cpCut{x}{\cpWait{y}{P}}{R}}
      & \Longrightarrow \;
      & \tm{\cpWait{y}{\cpCut{x}{P}{R}}}
      \\
      \cpRedKappaPlus1
      & \tm{\cpCut{x}{\cpInl{y}{P}}{R}}
      & \Longrightarrow \;
      & \tm{\cpInl{y}{\cpCut{x}{P}{R}}}
      \\
      \cpRedKappaPlus2
      &\tm{\cpCut{x}{\cpInr{y}{P}}{R}}
      & \Longrightarrow \;
      & \tm{\cpInr{y}{\cpCut{x}{P}{R}}}
      \\
      \cpRedKappaWith
      & \tm{\cpCut{x}{\cpCase{y}{P}{Q}}{R}}
      & \Longrightarrow \;
      & \tm{\cpCase{y}{\cpCut{x}{P}{R}}{\cpCut{x}{Q}{R}}}
      \\
      \cpRedKappaTop
      & \tm{\cpCut{x}{\cpAbsurd{y}}{R}}
      & \Longrightarrow \;
      & \tm{\cpAbsurd{y}}
    \end{array}
  \]
  \caption{Commutative conversions for \cp.}
  \label{fig:cp-term-reduction-2}
\end{figure*}
%%% Local Variables:
%%% TeX-master: "main"
%%% End:

We will discuss the interpretations of each connective, together with their
typing and reduction rules, in \cref{sec:cp-multiplicatives,sec:cp-additives,sec:cp-duality}.

\subsection{Multiplicatives and in- and interdependence}
\label{sec:cp-multiplicatives}
The multiplicatives ($\ty{\tens}, \ty{\parr}$) deal with independence and
interdependence:
\begin{itemize}
\item
  A channel of type \ty{A \tens B} represents a pair of channels, which
  communicate with two independent processes---that is to say, two
  processes who share no channels.
  A process acting on a channel of type \ty{A \tens B} will send one endpoint of
  a fresh channel, and then split into a pair of independent processes.
  One of these processes will be responsible for an interaction of type \ty{A}
  over the fresh channel, while the other process continues to interact as
  \ty{B}.
\item
  A channel of type \ty{A \parr B} represents a pair of interdependent channels,
  which are used within a single process.
  A process acting on a channel of type \ty{A \parr B} will receive a channel to
  act on, and communicate on its channels in whatever order it pleases.
  This means that the usage of one channel can depend on that of
  another---e.g.\ the interaction of type \ty{B} could depend on the result of
  the interaction of type \ty{A}, or vise versa, and if \ty{A} and \ty{B} are
  complex types, their interactions could likewise interweave in complex ways.
\end{itemize}
While the rules for \ty{\tens} and \ty{\parr} introduce input and output
operations, these are inessential---the essential distinction lies two in the
fact that (\tens) composes two independent processes, and therefore \emph{must}
split the environment between them, whereas ($\parr$) uses a single process, which
then can---and must---use all the channels in the environment.
\begin{center}
  \cpInfTens
  \cpInfParr
\end{center}
The \textbeta-reduction rule for terms introduced by $(\tens)$ and $(\parr)$
implements the behaviour outlined above:
\[
  \tm{\cpCut{x}{\cpSend{x}{y}{P}{Q}}{\cpRecv{x}{z}{R}}}
  \Longrightarrow
  \tm{\cpCut{y}{P}{\cpCut{x}{Q}{\cpSub{y}{z}{R}}}}
\]
The function \tm{\cpSub{x}{y}{P}} denotes the substitution of the name \tm{x}
for the name \tm{y} in the term \tm{P}.
%
The rules for the multiplicative units ($\ty{\one}, \ty{\bot}$) follow the same
pattern, except for the nullary instead of the binary case:
\begin{itemize}
\item
  A term constructed by $(\one)$ must composes \emph{zero} independent
  processes, and thus must halt. Furthermore, it must be able to split its
  environment between zero processes, and thus its environment must be empty.
\item
  A term constructed by $(\bot)$, on the other hand, consists of a single
  process, which is not further restricted.
\end{itemize}
The rules for $\ty{\one}$ and $\ty{\bot}$ introduce a nullary send and receive
operation, such as those found in the polyadic \textpi-calculus~\parencite{milner1993}. 
\begin{center}
  \cpInfOne
  \cpInfBot
\end{center}
The \textbeta-reduction rule for terms introduced by $(\one)$ and $(\bot)$
implements the behaviour outlined above:
\[
  \tm{\cpCut{x}{\cpHalt{x}}{\cpWait{x}{P}}}
  \Longrightarrow
  \tm{P}
\]

\subsection{Additives and choice}
\label{sec:cp-additives}
The additives ($\ty{\plus}, \ty{\with}$) deal with choice:
\begin{itemize}
\item
  A process acting on a channel of type \ty{A \plus B} either sends the value
  \tm{\text{inl}} to select an interaction of type \ty{A} or the value
  \tm{\text{inr}} to select one of type \ty{B}. 
\item
  A process acting on a channel of type \ty{A \with B} receives such a value,
  and then offers an interaction of either type \ty{A} or \ty{B},
  correspondingly.
\end{itemize}
In essence, the additive operations implement sending and receiving of a single
bit of information (\tm{\text{inl}} or \tm{\text{inr}}) and branching based on
the value of that bit. 
\begin{center}
  \cpInfPlus1
  \cpInfWith
\end{center}
The rule for constructing a process which sends \tm{\text{inr}}, $(\plus_2)$,
has been omitted, but can be found in~\cref{fig:cp-typing-judgement}.
The \textbeta-reduction rules for terms introduced by $(\plus_1)$, $(\plus_2)$
and $(\with)$ implement the behaviour outlined above.
\[
  \begin{array}{c}
    \tm{\cpCut{x}{\cpInl{x}{P}}{\cpCase{x}{Q}{R}}} \Longrightarrow \tm{\cpCut{x}{P}{Q}}
    \\
    \tm{\cpCut{x}{\cpInr{x}{P}}{\cpCase{x}{Q}{R}}} \Longrightarrow \tm{\cpCut{x}{P}{R}}
  \end{array}
\]
%
The rules for the additive units ($\ty{\nil}, \ty{\top}$) follow the same
pattern, except for a nullary choice:
\begin{itemize}
\item
  There is \emph{no} rule for \ty{\nil}, as a process acting on a channel of
  that type would have to select one of \emph{zero} options, which is clearly
  impossible.
\item
  A process acting on a channel of type \ty{\top} will wait to receive a choice
  of out \emph{zero} options. Since this will clearly never arrive, we have two
  options: either we block, waiting forever, or we simply crash.
\end{itemize}
It may seem odd at first to include a type for the process which cannot possibly
exist, and for the process which waits forever, but these make sensible units
for choice.
When offered a choice of type \ty{A \plus \nil}, one can either choose to
interact as \ty{A}, or choose to commit to doing the impossible.
Similarly, when offering a choice of type \ty{A \with \top}, one can safely
implement the right branch with a process which waits forever, as no sound
process will ever be able to select that branch anyway.
\begin{center}
  \cpInfNil
  \cpInfTop
\end{center}
As there is no way to construct a process of type \ty{\nil}, there is no
reduction rule for the additive units.


\subsection{Structural rules and duality}
\label{sec:cp-duality}
Duality plays a crucial role in session type systems.
In~\cref{sec:cp-additives}, we saw that duality ensures a process offering a choice
is always matched with a process making a choice.
In~\cref{sec:cp-multiplicatives}, we saw that it also ensures that, for instance
a process which uses communication on \tm{x} to decide what to send on \tm{y} is
matched with a pair of independent processes on \tm{x} and \tm{y}, a property
which is crucial to deadlock freedom, as it prevents circular dependencies. 

Duality appears in the typing rules for two \rcp term constructs.
Forwarding, \tm{\cpLink{x}{y}}, connects two dual channels with dual endpoints,
while composition, \tm{\cpCut{x}{P}{Q}}, composes two processes \tm{P} and
\tm{Q} with a shared channel \tm{x}, requiring that they follow dual protocols
on \tm{x}.
\begin{center}
  \cpInfAx
  \cpInfCut
\end{center}
There are two reduction rules which deal with the interactions between
forwarding and compositions. These implement the intuition that if a process is
meant to communicate on \tm{x}, \tm{x} is forwarding to \tm{y}, and nobody else
is listening on \tm{x}, then the process might as well start communicating on
\tm{y}.
\[
  \tm{\cpCut{x}{\cpLink{w}{x}}{P}} \Longrightarrow \tm{\cpSub{w}{x}{P}} 
\]
We can do this because \rcp implements a binary session type system, meaning
that each communication has only two participants, and therefore we know that no
other process is communicating on \tm{x}.

\textcite{wadler2012} has two reduction rules which deal with links. The first of
these can be seen above, and the second of these deals with the case in which
the link is flipped.
As we consider links commutative, this rule is derivable.

\subsection{Commuting conversions}
\label{sec:cp-commuting-conversions}
The commuting conversions are not reductions typically found in the
\textpi-calculus.
Instead, they are based on permutation cuts, which are commonly used in logic to
define a procedure for cut elimination.
These cut elimination steps push a cut deeper into a term, past unrelated
inference rules.
When viewed from the perspective of process calculus, they allow us to pull
actions upwards, past unrelated cuts.
The commuting conversion are defined in \cref{fig:cp-term-reduction-2}.

\subsection{Example}
\label{sec:cp-example}
The multiplicatives are responsible for structuring communication, and
it is this structure which rules out deadlocked interactions.
Let us go back to our example of a deadlocked interaction from
\cref{sec:introduction}:
\[
  \tm{(\piPar{%
      \piRecv{x}{z}{\piSend{y}{\sliceofcake}{\john}}
    }{%
      \piRecv{y}{w}{\piSend{x}{\bill}{\mary}}
    })}
\]

If we want to translate this interaction to \cp, we run into a problem: there is
no \emph{plain} sending construct in \cp---we only have \tm{\cpSend{x}{y}{P}{Q}},
which requires that the remainder of the interaction is split in two independent
processes.
This enforces a certain structure on the program. Either John will already have
to have the cake in his hands, or Mary will already have to have the money in
the bank.
We model the second scenario below, assuming \john, \mary and \bank are
processes modeling John, Mary, and Mary's bank, and \cake and \money are the
types of two channels which give access to a slice of cake and appropriate
payment.
\begin{prooftree}
  \AXC{$\seq[{ \john }]{ \Gamma, \tmty{y}{\money^\bot}, \tmty{x}{\cake}}$}
  \SYM{\parr}
  \UIC{$\seq[{ \cpRecv{x}{y}{\john} }]{ \Gamma, \tmty{x}{\money^\bot \parr \cake} }$}
  \AXC{$\seq[{ \bank }]{ \Delta, \tmty{z}{\money} }$}
  \AXC{$\seq[{ \mary }]{ \Theta, \tmty{x}{\cake^\bot} }$}
  \SYM{\tens}
  \BIC{$\seq[{\cpSend{x}{z}{\bank}{\mary} }]{ \Delta, \tmty{x}{\money \tens \cake^\bot} }$}
  \NOM{Cut}
  \BIC{$\seq[{ \cpCut{x}{\cpRecv{x}{y}{\john}}{\cpSend{x}{z}{\bank}{\mary}} }]{\Gamma, \Delta, \Theta }$}
\end{prooftree}
The resulting process reduces, as expected:
\[
  \reducesto
  {\cpCut{x}{\cpRecv{x}{y}{\john}}{\cpSend{x}{z}{\bank}{\mary}}}
  {\cpCut{y}{\bank}{\cpCut{x}{\john}{\mary}}}
\]


\subsection{Properties of \rcp}
\label{sec:cp-properties}
In this section, we will prove three important properties of \rcp, namely
preservation, progress and termination.

\subsubsection{Preservation}
Preservation is the fact that term reduction preserves typing. In order to prove
this, we will first need to prove that structural congruence preserves typing.
\begin{theorem}[Preservation for $\equiv$]\label{thm:cp-preservation-equiv}
  If $\seq[{ P }]{ \Gamma }$ and $\tm{P} \equiv \tm{Q}$,
  then $\seq[{ Q }]{\Gamma }$.
\end{theorem}
\begin{proof}
  By induction on the structure of the equivalence. The cases for reflexivity,
  transitivity and congruence are trivial. The two interesting cases, for
  \cpEquivCutComm and \cpEquivCutAss1 are given in \cref{fig:cp-preservation-equiv}
\end{proof}
%%% Local Variables:
%%% TeX-master: "main"
%%% End:

%
\begin{figure*}[ht]
  \centering
  \begin{tabular}{ll}
    \cpEquivCutComm
    &
      \begin{prooftree*}
        \AXC{$\seq[{ P }]{ \Gamma, \tmty{x}{A} }$}
        \AXC{$\seq[{ Q }]{ \Delta, \tmty{x}{A^\bot} }$}
        \NOM{Cut}
        \BIC{$\seq[{ \cpCut{x}{P}{Q} }]{ \Gamma, \Delta }$}
      \end{prooftree*}
    \\[30pt]
    $\equiv$
    &
      \begin{prooftree*}
        \AXC{$\seq[{ Q }]{ \Delta, \tmty{x}{A^\bot} }$}
        \AXC{$\seq[{ P }]{ \Gamma, \tmty{x}{A} }$}
        \NOM{\cref{thm:cp-negation-involutive}}
        \UIC{$\seq[{ P }]{ \Gamma, \tmty{x}{A^{\bot\bot}} }$}
        \NOM{Cut}
        \BIC{$\seq[{ \cpCut{x}{Q}{P} }]{ \Gamma, \Delta }$}
      \end{prooftree*}
    \\[40pt]
    \cpEquivCutAss1
    &
      \begin{prooftree*}
        \AXC{$\seq[{ P }]{ \Gamma, \tmty{x}{A} }$}
        \AXC{$\seq[{ Q }]{ \Delta, \tmty{x}{A^\bot}, \tmty{y}{B} }$}
        \AXC{$\seq[{ R }]{ \Theta, \tmty{y}{B^\bot} }$}
        \NOM{Cut}
        \BIC{$\seq[{ \cpCut{y}{Q}{R} }]{ \Delta, \Theta, \tmty{x}{A^\bot} }$}
        \NOM{Cut}
        \BIC{$\seq[{ \cpCut{x}{P}{\cpCut{y}{Q}{R}} }]{ \Gamma, \Delta, \Theta }$}
      \end{prooftree*}
    \\[30pt]
    $\equiv$
    &
      \begin{prooftree*}
        \AXC{$\seq[{ P }]{ \Gamma, \tmty{x}{A} }$}
        \AXC{$\seq[{ Q }]{ \Delta, \tmty{x}{A^\bot}, \tmty{y}{B} }$}
        \NOM{Cut}
        \BIC{$\seq[{ \cpCut{x}{P}{Q} }]{ \Gamma, \Delta, \tmty{y}{B} }$}
        \AXC{$\seq[{ R }]{ \Theta, \tmty{y}{B^\bot} }$}
        \NOM{Cut}
        \BIC{$\seq[{ \cpCut{y}{\cpCut{x}{P}{Q}}{R} }]{ \Gamma, \Delta, \Theta }$}
      \end{prooftree*}
  \end{tabular}
  \caption{Type preservation for the structural congruence of \rcp}
  \label{fig:cp-preservation-equiv}
\end{figure*}
%%% Local Variables:
%%% TeX-master: "main"
%%% End:

%
Then, we can prove preservation.
\begin{theorem}[Preservation]\label{thm:cp-preservation}
  If \reducesto{P}{Q} and $\seq[{ P }]{ \Gamma }$, then $\seq[{ Q }]{ \Gamma }$.
\end{theorem}
\begin{proof}
  By induction on the structure of the reduction. See
  \cref{fig:cp-preservation-1} for the \cpRedAxCut{i} and \textbeta-reduction
  rules, and \cref{fig:cp-preservation-2a,fig:cp-preservation-2b} for the
  commutative conversions.
  %
  The case for \cpRedGammaCut is trivial by the induction hypothesis, and the
  case for \cpRedGammaEquiv is trivial by the induction hypothesis and
  \cref{thm:cp-preservation-equiv}. 
\end{proof}
%%% Local Variables:
%%% TeX-master: "main"
%%% End:

%
\begin{figure*}[ht]
  \begin{tabular}{ll}
    \cpRedAxCut1
    &
      \begin{prooftree*}
        \AXC{}
        \NOM{Ax}
        \UIC{$\seq[{ \cpLink{w}{x} }]{ \tmty{w}{A}, \tmty{x}{A^\bot} }$}
        \AXC{$\seq[{ R }]{ \tmty{x}{A}, \Gamma }$}
        \NOM{Cut}
        \BIC{$\seq[{ \cpCut{x}{\cpLink{w}{x}}{R} }]{ \tmty{w}{A}, \Gamma }$}
      \end{prooftree*}
    \\[30pt]
    $\Longrightarrow$
    &
      \begin{prooftree*}
        \AXC{$\seq[{ \cpSub{w}{x}{R} }]{ \tmty{w}{A}, \Gamma }$}
      \end{prooftree*}
    \\[30pt]
    
    \cpRedAxCut2
    &
      (as above)
    \\[30pt]
    
    \cpRedBetaTensParr
    &
      \begin{prooftree*}
        \AXC{$\seq[{ P }]{ \Gamma, \tmty{x}{A^\bot}, \tmty{y}{B^\bot} }$}
        \SYM{\parr}
        \UIC{$\seq[{ \cpRecv{y}{x}P }]{ \Gamma, \tmty{y}{A^\bot \parr B^\bot} }$}
        \AXC{$\seq[{ Q }]{ \Delta, \tmty{x}{A} }$}
        \AXC{$\seq[{ R }]{ \Theta, \tmty{y}{B} }$}
        \SYM{\tens}
        \BIC{$\seq[{ \cpSend{y}{x}(Q \mid R) }]{ \Delta, \Theta, \tmty{y}{A \tens B} }$}
        \NOM{Cut}
        \BIC{$\seq[{ \cpCut{y}{\cpRecv{y}{x}{P}}{\cpSend{y}{x}{Q}{R}} }]{ \Gamma, \Delta, \Theta }$}
      \end{prooftree*}
    \\[30pt]
    $\Longrightarrow$
    &
      \begin{prooftree*}
        \AXC{$\seq[{ P }]{ \Gamma, \tmty{x}{A^\bot}, \tmty{y}{B^\bot} }$}
        \AXC{$\seq[{ Q }]{ \Delta, \tmty{x}{A} }$}
        \NOM{Cut}
        \BIC{$\seq[{ \cpCut{x}{P}{Q} }]{ \Gamma, \Delta, \tmty{y}{B^\bot} }$}
        \AXC{$\seq[{ R }]{ \Theta, \tmty{y}{B} }$}      
        \NOM{Cut}
        \BIC{$\seq[{ \cpCut{y}{\cpCut{x}{P}{Q}}{R} }]{ \Gamma, \Delta, \Theta }$}
      \end{prooftree*}
    \\[40pt]
    
    \cpRedBetaOneBot
    &
      \begin{prooftree*}
        \AXC{$\seq[{ P }]{ \Gamma }$}
        \SYM{\bot}
        \UIC{$\seq[{ \cpWait{x}{P} }]{ \Gamma, \tmty{x}{\bot} }$}
        \AXC{}
        \SYM{\one}
        \UIC{$\seq[{ \cpHalt{x} }]{ \tmty{x}{\one} }$}
        \NOM{Cut}
        \BIC{$\seq[{ \cpCut{x}{\cpHalt{x}}{\cpWait{x}{P}} }]{ \Gamma }$}
      \end{prooftree*}
    \\[30pt]
    $\Longrightarrow$
    &
      \begin{prooftree*}
        \AXC{$\seq[{ P }]{ \Gamma }$}
      \end{prooftree*}
    \\[40pt]
    
    \cpRedBetaPlusWith1
    &
      \begin{prooftree*}
        \AXC{$\seq[{ P }]{ \Gamma, \tmty{x}{A^\bot} }$}
        \AXC{$\seq[{ Q }]{ \Gamma, \tmty{x}{B^\bot} }$}
        \SYM{\with}
        \BIC{$\seq[{ \cpCase{x}{P}{Q} }]{ \Gamma, \tmty{x}{A^\bot \with B^\bot} }$}
        \AXC{$\seq[{ R }]{ \Delta, \tmty{x}{A} }$}
        \SYM{\plus_1}
        \UIC{$\seq[{ \cpInl{x}{R}}]{ \Delta, \tmty{x}{A \plus B} }$}
        \NOM{Cut}
        \BIC{$\seq[{ \cpCut{x}{\cpCase{x}{P}{Q}}{\cpInl{x}{R}} }]{ \Gamma, \Delta }$}
      \end{prooftree*}
    \\[30pt]
    $\Longrightarrow$
    &
      \begin{prooftree*}
        \AXC{$\seq[{ P }]{ \Gamma, \tmty{x}{A^\bot} }$}
        \AXC{$\seq[{ R }]{ \Delta, \tmty{x}{A} }$}
        \NOM{Cut}
        \BIC{$\seq[{ \cpCut{x}{P}{R} }]{ \Gamma, \Delta }$} 
      \end{prooftree*}
    \\[40pt]
    
    \cpRedBetaPlusWith2
    &
      (as above)
  \end{tabular}
  
  \caption{Type preservation for the \cpRedAxCut{i} and \textbeta-reduction rules of \cp}
  \label{fig:cp-preservation-1}
\end{figure*}
%%% Local Variables:
%%% TeX-master: "main"
%%% End:
\begin{figure*}[ht]
  \centering
  \begin{tabular}{ll}
    \cpRedKappaTens1
    &
      \begin{prooftree*}
        \AXC{$\seq[{ P }]{ \Gamma, \tmty{x}{A}, \tmty{z}{B} }$}
        \AXC{$\seq[{ Q }]{ \Delta, \tmty{y}{C} }$}
        \SYM{\tens}
        \BIC{$\seq[{ \cpSend{y}{z}{P}{Q} }]{ \Gamma, \Delta, \tmty{y}{B \tens C} }$}
        \AXC{$\seq[{ R }]{ \Theta, \tmty{x}{A^\bot} }$}
        \NOM{Cut}
        \BIC{$\seq[{ \cpCut{x}{\cpSend{y}{z}{P}{Q}}{R} }]{ \Gamma, \Delta, \Theta, \tmty{y}{B \tens C} }$}
      \end{prooftree*}
    \\[30pt]
    $\Longrightarrow$
    &
       \begin{prooftree*}
         \AXC{$\seq[{ P }]{ \Gamma, \tmty{x}{A}, \tmty{z}{B} }$}
         \AXC{$\seq[{ R }]{ \Theta, \tmty{x}{A^\bot} }$}
         \NOM{Cut}
         \BIC{$\seq[{ \cpCut{x}{P}{Q} }]{ \Gamma, \Theta, \tmty{z}{B} }$}
         \AXC{$\seq[{ Q }]{ \Delta, \tmty{y}{C} }$}
         \SYM{\tens}
         \BIC{$\seq[{ \cpSend{y}{z}{\cpCut{x}{P}{Q}}{R} }]{ \Gamma, \Delta, \Theta, \tmty{y}{B \tens C} }$}
       \end{prooftree*}
    \\[30pt]
    \cpRedKappaTens2
    &
      (as above)
    \\[20pt]
    \cpRedKappaParr
    &
      \begin{prooftree*}
        \AXC{$\seq[{ P }]{ \Gamma, \tmty{x}{A}, \tmty{z}{B}, \tmty{y}{C} }$}
        \SYM{\parr}
        \UIC{$\seq[{ \cpRecv{y}{z}{P} }]{ \Gamma, \tmty{x}{A}, \tmty{y}{B \parr C} }$}
        \AXC{$\seq[{ R }]{ \Theta, \tmty{x}{A^\bot} }$}
        \NOM{Cut} 
        \BIC{$\seq[{ \cpCut{x}{\cpRecv{y}{z}{P}}{R} }]{ \Gamma, \Theta, \tmty{y}{B \parr C} }$}
      \end{prooftree*}
    \\[30pt]
    $\Longrightarrow$
    &
      \begin{prooftree*}
        \AXC{$\seq[{ P }]{ \Gamma, \tmty{x}{A}, \tmty{z}{B}, \tmty{y}{C} }$}
        \AXC{$\seq[{ R }]{ \Theta, \tmty{x}{A^\bot} }$}
        \NOM{Cut}
        \BIC{$\seq[{ \cpCut{x}{P}{R} }]{ \Gamma, \Theta, \tmty{z}{B}, \tmty{y}{C} }$}
        \SYM{\parr}
        \UIC{$\seq[{ \cpRecv{y}{z}{\cpCut{x}{P}{R}} }]{ \Gamma, \Theta, \tmty{y}{B \parr C} }$}
      \end{prooftree*}
    \\[40pt]
    \cpRedKappaBot
    &
      \begin{prooftree*}
        \AXC{$\seq[{ P }]{ \Gamma, \tmty{x}{A} }$}
        \SYM{\bot}
        \UIC{$\seq[{ \cpWait{y}{P} }]{ \Gamma, \tmty{x}{A}, \tmty{y}{\bot} }$}
        \AXC{$\seq[{ R }]{ \Theta, \tmty{x}{A^\bot} }$}
        \NOM{Cut} 
        \BIC{$\seq[{ \cpCut{x}{\cpWait{y}{P}}{R} }]{ \Gamma, \Theta, \tmty{y}{\bot} }$}
      \end{prooftree*}
    \\[30pt]
    $\Longrightarrow$
    &
      \begin{prooftree*}
        \AXC{$\seq[{ P }]{ \Gamma, \tmty{x}{A} }$}
        \AXC{$\seq[{ R }]{ \Theta, \tmty{x}{A^\bot} }$}
        \NOM{Cut} 
        \BIC{$\seq[{ \cpCut{x}{P}{R} }]{ \Gamma, \Theta }$}
        \SYM{\bot}
        \UIC{$\seq[{ \cpWait{y}{\cpCut{x}{P}{R}} }]{ \Gamma, \Theta, \tmty{y}{\bot} }$}
      \end{prooftree*}
  \end{tabular}

  \caption{Type preservation for the commutative conversions of \cp}
  \label{fig:cp-preservation-2a}
\end{figure*}

\begin{figure*}[ht]
  \makebox[\textwidth][c]{
    \begin{tabular}{ll}
      \cpRedKappaPlus1
      &
        \begin{prooftree*}
          \AXC{$\seq[{ P }]{ \Gamma, \tmty{x}{A}, \tmty{y}{B} }$}
          \SYM{\plus_1}
          \UIC{$\seq[{ \cpInl{y}{P} }]{ \Gamma, \tmty{x}{A}, \tmty{y}{B \plus C} }$}
          \AXC{$\seq[{ R }]{ \Theta, \tmty{x}{A^\bot} }$}
          \NOM{Cut}
          \BIC{$\seq[{ \cpCut{x}{\cpInl{y}{P}}{R} }]{ \Gamma, \Theta, \tmty{y}{B \plus C} }$}
        \end{prooftree*}
      \\[30pt]
      $\Longrightarrow$
      &
        \begin{prooftree*}
          \AXC{$\seq[{ P }]{ \Gamma, \tmty{x}{A}, \tmty{y}{B} }$}
          \AXC{$\seq[{ R }]{ \Theta, \tmty{x}{A^\bot} }$}
          \NOM{Cut}
          \BIC{$\seq[{ \cpCut{x}{P}{R} }]{ \Gamma, \Theta, \tmty{y}{B} }$}
          \SYM{\plus_1}
          \UIC{$\seq[{ \cpInl{y}{\cpCut{x}{P}{R}} }]{ \Gamma, \Theta, \tmty{y}{B \plus C} }$}
        \end{prooftree*}
      \\[30pt] 
      \cpRedKappaPlus2
      &
        (as above)
      \\[20pt]
      \cpRedKappaWith
      &
        \begin{prooftree*}
          \AXC{$\seq[{ P }]{ \Gamma, \tmty{x}{A}, \tmty{y}{B} }$}
          \AXC{$\seq[{ Q }]{ \Gamma, \tmty{x}{A}, \tmty{y}{C} }$}
          \SYM{\with}
          \BIC{$\seq[{ \cpCase{y}{P}{Q} }]{ \Gamma, \tmty{x}{A}, \tmty{y}{B \with C} }$}
          \AXC{$\seq[{ R }]{ \Theta, \tmty{x}{A^\bot} }$}
          \NOM{Cut}
          \BIC{$\seq[{ \cpCut{x}{\cpCase{y}{P}{Q}}{R} }]{ \Gamma, \Theta, \tmty{y}{B \with C} }$}
        \end{prooftree*}
      \\[30pt]
      $\Longrightarrow$
      &
        \begin{prooftree*}
          \AXC{$\seq[{ P }]{ \Gamma, \tmty{x}{A}, \tmty{y}{B} }$}
          \AXC{$\seq[{ R }]{ \Theta, \tmty{x}{A^\bot} }$}
          \NOM{Cut}
          \BIC{$\seq[{ \cpCut{x}{P}{R} }]{ \Gamma, \Theta, \tmty{y}{B} }$}
          \AXC{$\seq[{ Q }]{ \Gamma, \tmty{x}{A}, \tmty{y}{C} }$}
          \AXC{$\seq[{ R }]{ \Theta, \tmty{x}{A^\bot} }$}
          \NOM{Cut}
          \BIC{$\seq[{ \cpCut{x}{Q}{R} }]{ \Gamma, \Theta, \tmty{y}{C} }$}
          \SYM{\with}
          \BIC{$\seq[{ \cpCase{y}{\cpCut{x}{P}{R}}{\cpCut{x}{Q}{R}} }]{ \Gamma, \Theta, \tmty{y}{B \with C} }$}
        \end{prooftree*}
      \\[40pt] 
      \cpRedKappaTop
      &
        \begin{prooftree*}
          \AXC{}
          \SYM{\top}
          \UIC{$\seq[{ \cpAbsurd{y} }]{ \Gamma, \tmty{x}{A}, \tmty{y}{\top} }$}
          \AXC{$\seq[{ R }]{ \Theta, \tmty{x}{A^\bot} }$}
          \NOM{Cut}
          \BIC{$\seq[{ \cpCut{x}{\cpAbsurd{y}}{R} }]{ \Gamma, \Theta, \tmty{y}{\top} }$}
        \end{prooftree*}
      \\[30pt]
      $\Longrightarrow$
      &
        \begin{prooftree*}
          \AXC{}
          \SYM{\top}
          \UIC{$\seq[{ \cpAbsurd{y} }]{ \Gamma, \Theta, \tmty{y}{\top} }$}
        \end{prooftree*}
    \end{tabular}
  }
  \caption{Type preservation for the commutative conversions of \cp (cont'd)}
  \label{fig:cp-preservation-2b}
\end{figure*}
%%% Local Variables:
%%% TeX-master: "main"
%%% End:

\subsubsection{Progress}
Progress is the fact that every term is either in some canonical form, or can be
reduced further. In order for a statement of progress to make sense, we need a
definition of canonical form. The canonical form used by \cp is ``any term which
is not a cut.'' We will refer to this canonical form as \emph{weak head normal
form}, for its relation to the eponymous \textlambda-calculus normal form. 
\begin{definition}[Weak head normal form]\label{def:cp-weak-head-normal-form}
  A process \tm{P} is in weak head normal form if it is not a cut.
\end{definition}
%%% Local Variables:
%%% TeX-master: "main"
%%% End:

As \cp has a tight correspondence with classical linear logic, so does its proof
of progress have a tight correspondence with (part of) the procedure proof
normalisation for classical linear logic for classical linear
logic~\parencite{girard1987}.
\begin{theorem}[Progress, WHNF]\label{thm:cp-progress-1}
  If $\seq[{ P }]{ \Gamma }$, then either $\tm{P}$ is in weak head normal form,
  or there exists a $\tm{P'}$ such that $\reducesto{P}{P'}$.
\end{theorem}
\begin{proof}
  By induction on the structure of \tm{P}. The only interesting case is the
  case where \tm{P} is a cut \tm{\cpCut{x}{P'}{Q'}}. In every other case,
  \tm{P} is in weak head normal form. There are three possibilities:
  \begin{itemize}
  \item
    Both \tm{P'} and \tm{Q'} act on \tm{x}.
    \\
    We apply one of the \textbeta-reduction rules.
  \item
    Either \tm{P'} or \tm{Q'} acts on an external channel.
    \\
    We apply one of the commutative conversions.
  \item
    Otherwise one or both of \tm{P'} and \tm{Q'} are themselves cuts.
    \\
    We apply the induction hypothesis.
  \end{itemize}
\end{proof}
%%% Local Variables:
%%% TeX-master: "main"
%%% End:
  
If we extend the reduction system with all congruence rules---not just
\cpRedGammaCut for reduction under cuts, but for reduction under any term
context---then we can strengthen our canonical form, and extend our proof for
progress to the \emph{full} proof normalisation procedure.
\begin{definition}[Normal form]\label{def:cp-normal-form}
  A process \tm{P} is in normal form if it does not contain any cuts.
\end{definition}
%%% Local Variables:
%%% TeX-master: "main"
%%% End:

\begin{theorem}[Progress, NF]\label{thm:cp-progress-2}
  If $\seq[{ P }]{ \Gamma }$, then either \tm{P} is in normal form, or there
  exists a \tm{P'} such that \reducesto{P}{P'}.
\end{theorem}
\begin{proof}
  Either \tm{P} is in normal form, or there is a cut \emph{somewhere} in \tm{P}.
  If there is a cut, then we obtain a reduction by \cref{thm:cp-progress-1} and
  congruence.
\end{proof}
%%% Local Variables:
%%% TeX-master: "main"
%%% End:
 
\textcite{wadler2012} opts to leave these additional congruence rules out,
because ``such rules do not correspond well to our notion of computation on
processes'', and his choice is analogous to a common practice in the
\textlambda-calculus to not allow reduction under lambdas.

It might not be so odd to allow these reductions in the context of \cp.
If one conceives of the two sides of a parallel composition in
\tm{\cpRecv{y}{z}{\cpCut{x}{P}{Q}}} as separate processes, both waiting on an
external communication on \tm{y}, it does not seem odd to allow the
communication on \tm{x} to happen. It is simply eager a form of evaluation.

\subsubsection{Termination}
\label{sec:cp-termination}
Termination is the fact that if we iteratively apply progress to obtain a
reduction, and apply that reduction, we will eventually end up with a term in
canonical form.
Its proof is quite simple, owing to the fact that our reduction rules were all
inspired by cut reductions from classical linear logic.
\begin{theorem}[Termination]\label{thm:cp-termination}
  If $\seq[{ P }]{ \Gamma }$, then there are no infinite $\Longrightarrow$
  reduction sequences.
\end{theorem}
  \begin{proof}
    Every reduction reduces a single cut to zero, one or two cuts.
    However, each of these cuts is \emph{smaller}, in the sense that the type of
    the channel on which the communication takes place is smaller, as each
    reduction eliminates a connective---see
    \cref{fig:cp-preservation-1,fig:cp-preservation-2a,fig:cp-preservation-2b}.
    Therefore, there cannot be an infinite reduction sequence.
  \end{proof}
%%% Local Variables:
%%% TeX-master: "main"
%%% End:

\section{Non-determinism, logic, and session types}
\label{sec:local-choice}
In recent work, we have seen the extension of \piDILL and \cp with operators for
non-deterministic behaviour~\parencite{atkey2016,caires2014,caires2017}.
These extensions all implement an operator known as non-deterministic local
choice.
While this operator is written as \tm{P+Q}, it should not be confused with
input-guarded choice~\parencite{milner1992b} from the \textpi-calculus.
Essentially, non-deterministic local choice can be summarised by the following
typing and reduction rules: 
\begin{center}
  \begin{prooftree*}
    \AXC{$\seq[{ P }]{ \Gamma }$}
    \AXC{$\seq[{ Q }]{ \Gamma }$}
    \BIC{$\seq[{ P + Q }]{ \Gamma }$}
  \end{prooftree*}
  \hspace*{2cm}
  \(
  \begin{array}{c}
    \reducesto{P + Q}{P}\\
    \reducesto{P + Q}{Q}
  \end{array}
  \)
\end{center}
There are some problems with non-deterministic local choice. First of all, the
non-determinism arises from the fact that for any term \tm{P+Q}, two different
reduction rules apply simultaneously. These reduction rules are written
specifically to introduce non-determinism. This is unlike the \textpi-calculus,
where non-determinism arises due to multiple processes communicating on a
single, shared channel.
We can easily implement this operator in the \textpi-calculus, using a nullary
communication:
\[
  \begin{array}{c}
    \tm{( \piPar{\piPar{\piSend{x}{}{\piHalt}}{\piRecv{x}{}{P}}}{\piRecv{x}{}{Q}} )}
    \\[1ex]
    \rotatebox[origin=c]{270}{$\Longrightarrow^{\star}$}
    \\[1ex]
    \tm{( \piPar{P}{\piRecv{x}{}{Q}} )}
    \quad
    \text{or}
    \quad
    \tm{( \piPar{\piRecv{x}{}{P}}{Q} )}
  \end{array}
\]
In this implementation, the process \tm{\piSend{x}{}{0}} will ``unlock'' either
\tm{P} or \tm{Q}, leaving the other process deadlocked. Or we could use
input-guarded choice:
\[
  \tm{( \piPar{\piSend{x}{}{\piHalt}}{( \piRecv{x}{}{P} + \piRecv{x}{}{Q} )} )}
\]
However, there are many non-deterministic processes in the \textpi-calculus
which are awkward to encode using non-deterministic local choice.
Let us look at our example:
\[
  \begin{array}{c}
    \tm{(\piPar{%
    \piSend{x}{\sliceofcake}{\piSend{x}{\nope}{\store}}
    }{%
    \piPar{\piRecv{x}{y}{\john}}{\piRecv{x}{z}{\mary}}
    })}
    \\[1ex]
    \rotatebox[origin=c]{270}{$\Longrightarrow^{\star}$}
    \\[1ex]
    \tm{(\piPar{\store}{\piPar{\piSub{\sliceofcake}{y}{\john}}{\piSub{\nope}{z}{\mary}}})}
    \quad
    \text{or}
    \quad
    \tm{(\piPar{\store}{\piPar{\piSub{\nope}{y}{\john}}{\piSub{\sliceofcake}{z}{\mary}}})}
  \end{array}
\]
This non-deterministic interaction involves communication. If we wanted to write
down a process which reduced to the same result using non-deterministic local
choice, we would have to write the following process: 
\[
  \begin{array}{c}
    \tm{
    (\piPar{\store}{\piPar{\piSub{\sliceofcake}{y}{\john}}{\piSub{\nope}{z}{\mary}}})
    +
    (\piPar{\store}{\piPar{\piSub{\nope}{y}{\john}}{\piSub{\sliceofcake}{z}{\mary}}})
    }
    \\[1ex]
    \rotatebox[origin=c]{270}{$\Longrightarrow^{\star}$}
    \\[1ex]
    \tm{(\piPar{\store}{\piPar{\piSub{\sliceofcake}{y}{\john}}{\piSub{\nope}{z}{\mary}}})}
    \quad
    \text{or}
    \quad
    \tm{(\piPar{\store}{\piPar{\piSub{\nope}{y}{\john}}{\piSub{\sliceofcake}{z}{\mary}}})}
  \end{array}
\]
In essence, instead of modelling a non-deterministic interaction, we are
enumerating the outcomes of such an interaction.
This means non-deterministic local choice does not adequately model
non-determinism in the way the \textpi-calculus does.
%
Enumerating all possible outcomes becomes worse the more processes are involved
in an interaction. Imagine a scenario the following scenario:
\begin{quote}
  Three customers, Alice, John and Mary, have a craving for cake. Should cake be
  sold out, however, well... a doughnut will do. They prepare to order their
  goods via an online store. Unfortunately, they all decide to use the
  same \emph{shockingly} under-stocked store, which has only one slice of cake,
  and a single doughnut. After that, all it can deliver is disappointment.
\end{quote}
We can model this scenario in the \textpi-calculus, assuming \alice, \john,
\mary, and \store are four processes modelling Alice, John, Mary and the store,
and \sliceofcake, \doughnut, and \nope are three channels giving access to a
slice of cake, a so-so doughnut, and disappointment, respectively.
\begin{center}
  \makebox[\textwidth][c]{\ensuremath{
      \begin{array}{c}
        \tm{(\piPar{%
        \piSend{x}{\sliceofcake}{\piSend{x}{\doughnut}{\piSend{x}{\nope}{\store}}}
        }{%
        \piPar{\piRecv{x}{y}{\alice}}{\piPar{\piRecv{x}{z}{\john}}{\piRecv{x}{w}{\mary}}
        })}}
        \\[1ex]
        \rotatebox[origin=c]{270}{$\Longrightarrow^{\star}$}
        \\[1ex]
        \tm{(\piPar{\store}{\piPar{\piSub{\sliceofcake}{y}{\alice}}{\piPar{\piSub{\doughnut}{z}{\john}}{\piSub{\nope}{w}{\mary}}}})}
        \;
        \text{or}
        \;
        \tm{(\piPar{\store}{\piPar{\piSub{\sliceofcake}{y}{\alice}}{\piPar{\piSub{\nope}{z}{\john}}{\piSub{\doughnut}{w}{\mary}}}})}
        \\[1ex]
        \tm{(\piPar{\store}{\piPar{\piSub{\doughnut}{y}{\alice}}{\piPar{\piSub{\nope}{z}{\john}}{\piSub{\sliceofcake}{w}{\mary}}}})}
        \;
        \text{or}
        \;
        \tm{(\piPar{\store}{\piPar{\piSub{\doughnut}{y}{\alice}}{\piPar{\piSub{\sliceofcake}{z}{\john}}{\piSub{\nope}{w}{\mary}}}})}
        \\[1ex]
        \tm{(\piPar{\store}{\piPar{\piSub{\nope}{y}{\alice}}{\piPar{\piSub{\sliceofcake}{z}{\john}}{\piSub{\doughnut}{w}{\mary}}}})}
        \;
        \text{or}
        \;
        \tm{(\piPar{\store}{\piPar{\piSub{\nope}{y}{\alice}}{\piPar{\piSub{\doughnut}{z}{\john}}{\piSub{\sliceofcake}{w}{\mary}}}})}
        \\[1ex]
      \end{array}
    }}
\end{center}
With the addition of one process, Alice, we have increased the number of possible
outcomes enormously! In general, the number of outcomes for these types of
scenarios is $n!$, where $n$ is the number of processes. This means that if we
wish to translate any non-deterministic process to one using non-deterministic
local choice, we can expect a factorial growth in the size of the term!
%%% Local Variables:
%%% TeX-master: "main"
%%% End:

\chapter{\cp as a type system for the \textpi-calculus}
\cp has a tight correspondence with classical linear logic.
This has many advantages.
It is deadlock free and terminating, and as seen in \cref{sec:cp-properties},
the proofs of its meta-theoretical properties are brief and aesthetically
pleasing.

The price paid for this is a somewhat weaker correspondence with the
\textpi-calculus.
It would be useful to be able to think of \cp as a type system for the
\textpi-calculus.
However, as it stands there are many differences between these systems.
Most prominent of these are the commutative conversions. These reduction rules
are taken directly from the proof normalisation procedure of classical linear
logic, and do not correspond to any reductions in the \textpi-calculus.

\citenat{lindley2015semantics} observed that, using a different reduction
strategy, which more closely resembles that of the \textpi-calculus, we can
ensure that the commutative conversions are always applied \emph{last}.
That is to say, they define two separate reduction relations:
$\longrightarrow_{C}$ for \cpRedAxCut1, \cpRedAxCut2 and \textbeta-reductions,
and $\longrightarrow_{CC}$ for commutative conversions, and show that any
sequence of reductions has the following form:
\[
  P \longrightarrow_{C} \dots \longrightarrow_{C} Q \longrightarrow_{CC} \dots \longrightarrow_{CC} R
\]
In this dissertation, we use a reduction strategy which follows
\citenat{lindley2015semantics}, but opts to drop the suffix of commutative
conversions.
As a consequence of this, we can drop the commutative conversions from our
reduction system, which therefore more tightly corresponds to that of the
\textpi-calculus. 
The price we pay for this is a weaker correspondence to classical linear logic.
This shows in our notion of canonical form, which is weaker, and in our proof of
progress, which is much more involved.

For clarity's sake: whenever we refer to \cp or \rcp, for the remainder of this
dissertation, we refer to the variant \emph{without} commutative conversions,
i.e.\ which uses the following definition of term reduction.
\begin{definition}[Term reduction]\label{def:cp-term-reduction-2}
  A reduction $\reducesto{P}{P'}$ denotes that the process \tm{P} can reduce to
  the process \tm{P'} in a single step. Reductions can be constructed using the
  rules in~\cref{fig:cp-term-reduction-2}.
\end{definition}
%%% Local Variables:
%%% TeX-master: "main"
%%% End:


This chapter will proceed as follows.
In \cref{sec:cp-canonical-forms}, we will describe what it means for a term to
be in canonical form.
In \cref{sec:cp-evaluation-contexts}, we will define evaluation contexts.
In \cref{sec:cp-progress}, we will give a new proof of progress, which follows
\citenat{lindley2015semantics}.

\section{Canonical forms}\label{sec:cp-canonical-forms}
The reduction strategy described by \citeauthor{lindley2015semantics} applies
\cpRedAxCut1, \cpRedAxCut2, and \textbeta-reductions until the process blocks on
one or more external communications, and then applies the commuting conversions
to bubble one of those external communications to the front of the term.
This allows them to define canonical forms as any term which is not a cut.
For us, terms in canonical form will be those terms which are blocked on an
external communication, before any commutative conversions are applied.
In this section, we will describe the form of such terms.

We have informally used the phrase ``act on'' in previous sections. It is time
to formally define what it means when we say a process \emph{acts on} some
channel.
\begin{definition}[Action]\label{def:cp-action}
  A process \tm{P} \emph{acts on} a channel \tm{x} if it is in one of the
  following forms:
  \begin{multicols}{3}
    \begin{itemize}[noitemsep,topsep=0pt,parsep=0pt,partopsep=0pt]
    \item \tm{\cpLink{x}{y}}
    \item \tm{\cpLink{y}{x}}
    \item \tm{\cpSend{x}{y}{P'}{Q'}}
    \item \tm{\cpRecv{x}{y}{P'}}
    \item \tm{\cpHalt{x}}
    \item \tm{\cpWait{x}{P'}}
    \item \tm{\cpInl{x}{P'}}
    \item \tm{\cpInr{x}{P'}}
    \item \tm{\cpCase{x}{P'}{Q'}}
    \item \tm{\cpAbsurd{x}}
    \end{itemize}
  \end{multicols}
\end{definition}
%%% Local Variables:
%%% TeX-master: "main"
%%% End:

Furthermore, we will need the notion of an \emph{evaluation prefix}.
Intuitively, evaluation prefixes are multi-holed contexts consisting solely of
cuts. We will use evaluation prefixes in order to have a view of every
\emph{action} in a process at once.
\begin{definition}[Evaluation prefixes]\label{def:cp-evaluation-prefixes}
  We define evaluation prefixes as:
  \begin{align*}
    \tm{G}, \tm{H} := \tm{\Box} \mid \tm{\cpCut{x}{G}{H}}
  \end{align*}
\end{definition}
\begin{definition}[Plugging]\label{def:cp-evaluation-prefix-plugging}
  We define plugging for an evaluation prefix with $n$ holes as:
  \[
    \begin{array}{ll}
      \tm{\cpPlug{\Box}{R}} & := \; \tm{R} \\
      \tm{\cpPlug{\cpCut{x}{G}{H}}{R_1 \dots R_m, R_{m+1} \dots R_{n}}}
                            & := \; \tm{\cpCut{x}{\cpPlug{G}{R_1 \dots R_m}}{\cpPlug{H}{R_{m+1} \dots R_n}}}
    \end{array}
  \]
  Note that in the second case, \tm{G} is an evaluation prefix with $m$ holes,
  and \tm{H} is an evaluation prefix with $(n-m)$ holes.
\end{definition}
%%% Local Variables:
%%% TeX-master: "main"
%%% End:

Intuitively, we can say that every term of the form
\tm{\cpPlug{G}{P_1 \dots P_n}} is equivalent to some term of the form
\tm{\cpCut{x_1}{P_1}{\cpCut{x_2}{P_2}{\dots \cpCut{x_n}{P_{n-1}}{P_n} \dots}}} 
where $\tm{x_1} \dots \tm{x_{n-1}}$ are the channels bound in \tm{G}.
In fact, a similar equivalence was used by \citeauthor{lindley2015semantics}
\cite{lindley2015semantics} in their semantics for \cp. 
\begin{definition}[Maximum evaluation prefix]\label{def:cp-maximum-evaluation-prefix}
  We say that \tm{G} is the evaluation prefix of \tm{P} when there exist terms
  $\tm{P_1} \dots \tm{P_n}$ such that $\tm{P} = \tm{\cpPlug{G}{P_1 \dots P_n}}$.
  We say that \tm{G} is the maximum evaluation prefix if each \tm{P_i} is an
  action. 
\end{definition}
\begin{lemma}\label{thm:cp-maximum-evaluation-prefix}
  Every term \tm{P} has a maximum evaluation prefix.
\end{lemma}
\begin{proof}
  By induction on the structure of \tm{P}.
\end{proof}
%%% Local Variables:
%%% TeX-master: "main"
%%% End:

We can now define what it means for a term to be in canonical form. Intuitively,
a process is in canonical form either when there is no top-level cut, or when it
is blocked on an external communication. We state this formally as follows:
\begin{definition}[Canonical forms]\label{def:cp-canonical-forms}
  We define the following forms to be canonical:
  \begin{center}
    \begin{prooftree*}
      \AXC{}
      \UIC{\canonical{\cpLink{x}{y}}}
    \end{prooftree*}
    \begin{prooftree*}
      \AXC{}
      \UIC{\canonical{\cpSend{x}{y}{P}{Q}}}
    \end{prooftree*}
    \begin{prooftree*}
      \AXC{}
      \UIC{\canonical{\cpRecv{x}{y}{P}}}
    \end{prooftree*}
  \end{center}
  \begin{center}
    \begin{prooftree*}
      \AXC{}
      \UIC{\canonical{\cpHalt{x}}}
    \end{prooftree*}
    \begin{prooftree*}
      \AXC{}
      \UIC{\canonical{\cpWait{x}{P}}}
    \end{prooftree*}
    \begin{prooftree*}
      \AXC{}
      \UIC{\canonical{\cpAbsurd{x}}}
    \end{prooftree*}
  \end{center}
  \begin{center}
    \begin{prooftree*}
      \AXC{}
      \UIC{\canonical{\cpInl{x}{P}}}
    \end{prooftree*}
    \begin{prooftree*}
      \AXC{}
      \UIC{\canonical{\cpInr{x}{P}}}
    \end{prooftree*}
    \begin{prooftree*}
      \AXC{}
      \UIC{\canonical{\cpCase{x}{P}{Q}}}
    \end{prooftree*}
  \end{center}
\end{definition}
%%% Local Variables:
%%% TeX-master: "main"
%%% End:

This definition is adequate, as it matches our intuition. We will see this in
our proof for \cref{thm:cp-progress-3}.

\section{Evaluation contexts}\label{sec:cp-evaluation-contexts}
Intuitively, evaluation contexts are one-holed term contexts under which
reduction can take place. For \rcp, these consist solely of cuts.
\begin{definition}[Evaluation contexts]\label{def:cp-evaluation-contexts}
  We define evaluation contexts as:
  \begin{align*}
    \tm{G}, \tm{H} := \tm{\Box}
    \mid \tm{\cpCut{x}{G}{P}}
    \mid \tm{\cpCut{x}{P}{G}}
  \end{align*}
\end{definition}
\begin{definition}[Plugging]\label{def:cp-evaluation-context-plugging}
  We define plugging for evaluation contexts as:
  \begin{gather*}
    \begin{array}{ll}
      \tm{\cpPlug{\Box}{R}}            
      & := \; \tm{R}
      \\
      \tm{\cpPlug{\cpCut{x}{G}{P}}{R}}
      & := \; \tm{\cpCut{x}{\cpPlug{G}{R}}{P}}
      \\
      \tm{\cpPlug{\cpCut{x}{P}{G}}{R}}
      & := \; \tm{\cpCut{x}{P}{\cpPlug{G}{R}}}
    \end{array}
  \end{gather*}
\end{definition}
%%% Local Variables:
%%% TeX-master: "main"
%%% End:

We can prove that we can push any cut downwards under an evaluation contexts, as
long as the channel it binds does not occur in the context itself.
\begin{lemmaB}\label{thm:cp-display-cut-1}
  If $\seq[{ \tm{\cpCut{x}{\cpPlug{E}{P}}{Q}} }]{ \Gamma }$ and
  $\notFreeIn{x}{E}$, then $\tm{\cpCut{x}{\cpPlug{E}{P}}{Q}} \equiv
  \tm{\cpPlug{E}{\cpCut{x}{P}{Q}}}$. 
\end{lemmaB}
  \begin{proof}
    By induction on the structure of the evaluation context \tm{E}.
    \begin{itemize}
    \item
      Case $\tm{\Box}$. By reflexivity.
    \item
      Case $\tm{\cpCut{y}{E}{R}}$.
      \[\!
        \begin{array}{ll}
          \tm{\cpCut{x}{\cpCut{y}{\cpPlug{E}{P}}{R}}{Q}} & \equiv \quad \text{by \cpEquivCutComm}\\
          \tm{\cpCut{x}{\cpCut{y}{R}{\cpPlug{E}{P}}}{Q}} & \equiv \quad \text{by \cpEquivCutAss2}\\
          \tm{\cpCut{y}{R}{\cpCut{x}{\cpPlug{E}{P}}{Q}}} & \equiv \quad \text{by \cpEquivCutComm}\\
          \tm{\cpCut{y}{\cpCut{x}{\cpPlug{E}{P}}{Q}}{R}} & \equiv \quad \text{by the induction hypothesis}\\ 
          \tm{\cpCut{y}{\cpPlug{E}{\cpCut{x}{P}{Q}}}{R}} &
        \end{array}
      \]
    \item
      Case $\tm{\cpCut{y}{R}{E}}$.
      \[\!
        \begin{array}{ll}
          \tm{\cpCut{x}{\cpCut{y}{R}{\cpPlug{E}{P}}}{Q}} & \equiv \quad \text{by \cpEquivCutAss2}\\
          \tm{\cpCut{y}{R}{\cpCut{x}{\cpPlug{E}{P}}{Q}}} & \equiv \quad \text{by the induction hypothesis}\\
          \tm{\cpCut{y}{R}{\cpPlug{E}{\cpCut{x}{P}{Q}}}}
        \end{array}
      \]
    \end{itemize}
    In each case, the side conditions for \cpEquivCutAss2, $\notFreeIn{x}{R}$ and
    $\notFreeIn{y}{Q}$, can be inferred from $\notFreeIn{x}{E}$ and the fact that
    $\tm{\cpCut{x}{\cpPlug{E}{P}}{Q}}$ is well-typed.
  \end{proof}
%%% Local Variables:
%%% TeX-master: "main"
%%% End:
And vice versa. However, we will not use the following lemma in this
dissertation, and leave its proof as an exercise to the reader.
\begin{lemmaB}\label{thm:cp-display-cut-2}
  If $\seq[{ \cpPlug{E}{\cpCut{x}{P}{Q}} }]{ \Gamma }$ and $\notFreeIn{x}{E}$,
  then 
  $\tm{\cpPlug{E}{\cpCut{x}{P}{Q}}} \equiv \tm{\cpCut{x}{\cpPlug{E}{P}}{Q}}$. 
\end{lemmaB}
%%% Local Variables:
%%% TeX-master: "main"
%%% End:

\section{Progress}\label{sec:cp-progress}
Progress is the fact that every term is either in some canonical form, or can be
reduced further.
%
There are two important lemmas which we will need in order to prove progress.
These relate evaluation prefixes to evaluation contexts.
%
Specifically, if a process under an evaluation prefix is a link, we can rewrite
the entire process in such a way as to reveal the cut which introduced one of
the channels acted upon by that link.
\begin{lemmaB}\label{thm:cp-progress-link}
  If $\seq[{ \cpPlug{G}{P_1 \dots P_n} }]{ \Gamma }$, and some \tm{P_i} is a
  link \tm{\cpLink{x}{y}}, then either \tm{x} and \tm{y} are not bound by
  \tm{G}, or there exist \tm{H}, \tm{H'} and \tm{Q} such that either
  $\tm{\cpPlug{G}{P_1 \dots P_n}} \equiv
  \tm{\cpPlug{H}{\cpCut{x}{\cpPlug{H'}{\cpLink{x}{y}}}{Q}}}$ or 
  $\tm{\cpPlug{G}{P_1 \dots P_n}} \equiv
  \tm{\cpPlug{H}{\cpCut{y}{\cpPlug{H'}{\cpLink{x}{y}}}{Q}}}$. 
\end{lemmaB}
\begin{proof}
  By induction on the structure of \tm{G}.
  \begin{itemize}
  \item
    Case \tm{\Box}. Clearly \tm{x} and \tm{y} are not bound.
  \item
    Case \tm{\cpCut{z}{\cpPlug{G'}{P_1 \dots P_i \dots P_m}}{\cpPlug{G''}{P_{m+1} \dots P_n}}}.\\
    Case \tm{\cpCut{z}{\cpPlug{G'}{P_1 \dots P_m}}{\cpPlug{G''}{P_{m+1} \dots P_i \dots P_n}}}.\\
    We apply the induction hypothesis. There are two cases:
    \begin{itemize}
    \item
      If \tm{x} and \tm{y} were not bound, they remain unbound.
    \item
      If \tm{x} or \tm{y} is bound deeper in \tm{G}, we prepend one of
      \tm{\cpCut{z}{\Box}{\cpPlug{G''}{P_{m+1} \dots P_n}}},
      \tm{\cpCut{z}{\cpPlug{G'}{P_1 \dots P_m}}{\Box}},
      \tm{\ncPool{\Box}{\cpPlug{G''}{P_{m+1} \dots P_n}}}, or
      \tm{\ncPool{\cpPlug{G'}{P_1 \dots P_m}}{\Box}} to \tm{H}.
      \\
      The desired equality follows by congruence.
    \end{itemize}
  \item
    Case \tm{\cpCut{x}{\cpPlug{G'}{P_1 \dots P_i \dots P_m}}{\cpPlug{G''}{P_{m+1} \dots P_n}}}.\\
    Case \tm{\cpCut{y}{\cpPlug{G'}{P_1 \dots P_i \dots P_m}}{\cpPlug{G''}{P_{m+1} \dots P_n}}}.\\
    Case \tm{\cpCut{x}{\cpPlug{G'}{P_1 \dots P_m}}{\cpPlug{G''}{P_{m+1} \dots P_i \dots P_n}}}.\\
    Case \tm{\cpCut{y}{\cpPlug{G'}{P_1 \dots P_m}}{\cpPlug{G''}{P_{m+1} \dots P_i \dots P_n}}}.
    \\
    Let $\tm{H} := \tm{\Box}$ and 
    \(\arraycolsep=0pt\begin{array}[t]{ll}
      \text{let}
      & \; \tm{H_i} := \tm{\cpPlug{G'}{P_1 \dots P_{i-1}, \Box, P_{i+1}, \dots P_m}}
      \\
      \text{or}
      & \; \tm{H_i} := \tm{\cpPlug{G''}{P_{m+1} \dots P_{i-1}, \Box, P_{i+1}, \dots P_n}}
    \end{array}\)
    \\[1ex]
    By reflexivity and \cpEquivCutComm.
  \end{itemize}
\end{proof}
%%% Local Variables:
%%% TeX-master: "main"
%%% End:
And if two processes under an evaluation prefix act on the same channel, then we
can rewrite the entire process in such a way as to reveal the cut which
introduced that channel. 
\begin{lemmaB}\label{thm:cp-progress-beta}
  If $\seq[{ \cpPlug{G}{P_1 \dots P_n} }]{ \Gamma }$, and some \tm{P_i} and
  \tm{P_j}, on different sides of at least one cut, act on the same bound
  channel \tm{x}, then there exist \tm{H}, \tm{H_i} and \tm{H_j} such that 
  \(
  \tm{\cpPlug{G}{P_1 \dots P_n}} =
  \tm{\cpPlug{H}{\cpCut{x}{\cpPlug{H_i}{P_i}}{\cpPlug{H_j}{P_j}}}}
  \).
\end{lemmaB}
\begin{proof}
  By induction on the structure of \tm{G}.
  \begin{itemize}
  \item
    Case \tm{\cpCut{x}{\cpPlug{G'}{P_1\dots P_i\dots P_m}}{\cpPlug{G''}{P_{m+1}\dots P_j\dots P_n}}}. 
    \\
    \(\arraycolsep=0pt\begin{array}[t]{lll}
      \text{Let}&\ \tm{H}  &\ :=\ \tm{\Box}, \\
                &\ \tm{H_i}&\ :=\ \tm{\cpPlug{G'}{P_1\dots P_{i-1},\Box,P_{i+1}\dots P_m}}, \\
                &\ \tm{H_j}&\ :=\ \tm{\cpPlug{G''}{P_{m+1}\dots P_{j-1},\Box,P_{j+1}\dots P_n}},
    \end{array}\)
    \\[1ex]
    By reflexivity.
  \item
    Case \tm{\cpCut{x}{\cpPlug{G'}{P_1\dots P_j\dots P_m}}{\cpPlug{G''}{P_{m+1}\dots P_i\dots P_n}}}.
    \\
    As above.
  \item
    Case \tm{\cpCut{y}{\cpPlug{G'}{P_1\dots P_i\dots P_j\dots P_m}}{\cpPlug{G''}{P_{m+1}\dots P_n}}}. \\
    Case \tm{\cpCut{y}{\cpPlug{G'}{P_1\dots P_m}}{\cpPlug{G''}{P_{m+1}\dots P_i\dots P_j\dots P_n}}}.
    \\
    We obtain \tm{H}, \tm{H_1} and \tm{H_2} from the induction hypothesis and
    \cref{thm:cp-preservation-equiv}, and then prepend either
    \tm{\cpCut{y}{\Box}{\cpPlug{G''}{P_{m+1}\dots P_n}}} or
    \tm{\cpCut{y}{\cpPlug{G'}{P_1\dots P_m}}{\Box}} to \tm{H}.
    The desired equality follows by congruence.
  \end{itemize}
  The case for \tm{\Box} is excluded because $n > 1$.
  The cases in which \tm{P_i} and \tm{P_j} are on the \emph{same} side of the
  cut, but the cut binds \tm{x}, and the cases in which \tm{P_i} and \tm{P_j}
  are on different sides of the cut, but the cut binds some other channel
  \tm{y}, are excluded by the type system.
\end{proof}
%%% Local Variables:
%%% TeX-master: "main"
%%% End:


Finally, we are ready to prove progress.
Note that in proofs throughout this dissertation, we will leave uses of
\cpRedGammaEquiv and the congruence rules for reductions implicit. 
\begin{theorem}[Progress]\label{thm:cp-progress-3}
  If $\seq[{ P }]{ \Gamma }$, then either $\tm{P}$ is in canonical form, or
  there exists a $\tm{P'}$ such that $\reducesto{P}{P'}$. 
\end{theorem}
\begin{proof}
  By induction on the structure of derivation for $\seq[{ P }]{ \Gamma }$.
  The only interesting case is when the last rule of the derivation is
  \textsc{Cut}---in every other case, the typing rule constructs a term in which 
  is in canonical form. 
  \\
  We consider the maximum evaluation prefix \tm{G} of \tm{P}, such that $\tm{P}
  = \tm{\cpPlug{G}{P_1 \dots P_{n+1}}}$ and each \tm{P_i} is an action.
  The prefix \tm{G} consists of $n$ cuts, and introduces $n$ channels, but
  composes $n+1$ actions. Therefore, one of the following must be true:
  \begin{itemize}
  \item
    One of the processes is a link \tm{\cpLink{x}{y}} acting on a bound channel.
    We have:
    \begin{gather*}
      \begin{array}{ll}
        \tm{\cpPlug{G}{P_1 \dots \cpLink{x}{y} \dots P_{n+1}}}
        & \equiv \quad \text{by \cref{thm:cp-progress-link}}
        \\
        \tm{\cpPlug{E}{\cpCut{z}{\cpPlug{E'}{\cpLink{x}{y}}}{Q}}}
        & \equiv \quad \text{by \cref{thm:cp-display-cut-1}}
        \\
        \tm{\cpPlug{E}{\cpPlug{E'}{\cpCut{z}{\cpLink{x}{y}}{Q}}}}
      \end{array}
    \end{gather*}
    Where $\tm{z} = \tm{x}$ or $\tm{z} = \tm{y}$.
    We then apply \cpRedAxCut1.
  \item
    Two of the processes, \tm{P_i} and \tm{P_j}, act on the same bound channel
    \tm{x}. We have:
    \begin{gather*}
      \begin{array}{ll}
        \tm{\cpPlug{G}{P_1 \dots P_i \dots P_j \dots P_{n+1}}}
        & = \quad \text{by \cref{thm:cp-progress-beta}}
        \\
        \tm{\cpPlug{G}{\cpCut{x}{\cpPlug{E_i}{P_i}}{\cpPlug{E_j}{P_j}}}}
        & \equiv \quad \text{by \cref{thm:cp-display-cut-1}} 
        \\
        \tm{\cpPlug{G}{\cpPlug{E_i}{\cpCut{x}{P_i}{\cpPlug{E_j}{P_j}}}}}
        & \equiv \quad \text{by \cref{thm:cp-display-cut-1}} 
        \\
        \tm{\cpPlug{G}{\cpPlug{E_i}{\cpPlug{E_j}{\cpCut{x}{P_i}{P_j}}}}} 
      \end{array}
    \end{gather*}
    We then apply one of the \textbeta-reduction rules.
  \item
    Otherwise (at least) one of the processes acts on an external channel.
    \\
    No process \tm{P_i} is a link.
    No two processes \tm{P_i} and \tm{P_j} act on the same channel \tm{x}.
    Therefore, \tm{P} is canonical.
  \end{itemize}
\end{proof}
%%% Local Variables:
%%% TeX-master: "main"
%%% End:

The proof of progress described in this section is novel, though it takes
inspiration from the reduction system for \cp described by \citenat{lindley2015semantics}. 
The proof of itself is somewhat involved. The reason for
this is that we want our reduction strategy to match that of the
\textpi-calculus as closely as possible.
It is also for this reason that our proof of progress involves some
non-determinism. For instance, in the second case of the proof, we do not
specify how to select the two processes \tm{P_i} and \tm{P_j} if there are
multiple options available.

\section{Example}
Let's have a look at the differences between the reduction strategy we have
defined in this chapter, and the reduction strategy which \citenat{wadler2012}
defines. Let's imagine the following scenario:
\begin{quote}
  Alice, John, and Mary went on a lovely trip together.
  However, John and Mary can be a bit scatterbrained sometimes, and it just so
  happens that they both forgot to bring their wallets.
  They both owe Alice some money, which they're now trying to pay back at the
  same time.
\end{quote}
We can model this interaction in \cp as the following term, assuming \alice,
\john, and \mary are three processes representing Alice, John and Mary, and
\bankjohn and \bankmary are two processes representing John and Mary's
respective banks.
\begin{align*}
  \tm{\cpCut{x}{\cpCut{y}{
  \cpRecv{x}{z}{\cpRecv{y}{w}{\alice}}}{
  \cpSend{y}{w}{\bankjohn}{\john}}}{
  \cpSend{x}{z}{\bankmary}{\mary}}}
\end{align*}
Note that in the above interaction, no \textbeta-reduction rule applies
\emph{immediately}. This means that in the reduction system described by
\citenat{wadler2012}, we will have to apply a commutative conversion.
\begin{align*}
  \begin{array}{lll}
    \tm{\cpCut{x}{\cpCut{y}{
    \cpRecv{x}{z}{\cpRecv{y}{w}{\alice}}}{
    \cpSend{y}{w}{\bankjohn}{\john}}}{
    \cpSend{x}{z}{\bankmary}{\mary}}}
    & \Longrightarrow & \text{by \cpRedKappaParr}
    \\
    \tm{\cpCut{x}{\cpRecv{x}{z}{\cpCut{y}{
    \cpRecv{y}{w}{\alice}}{
    \cpSend{y}{w}{\bankjohn}{\john}}}}{
    \cpSend{x}{z}{\bankmary}{\mary}}}
    & \Longrightarrow & \text{by \cpRedBetaTensParr}
    \\
    \tm{\cpCut{z}{\cpCut{x}{\cpCut{y}{
    \cpRecv{y}{w}{\alice}}{
    \cpSend{y}{w}{\bankjohn}{\john}}}{
    \mary}}{
    \bankmary}}
    & \Longrightarrow & \text{by \cpRedBetaTensParr}
    \\
    \tm{\cpCut{z}{\cpCut{x}{\cpCut{w}{\cpCut{y}{
    \alice}{
    \john}}{
    \bankjohn}}{
    \mary}}{
    \bankmary}}
  \end{array}
\end{align*}
Whereas in the reduction system described in this chapter, we will have to
rewrite using the equivalence.
\begin{align*}
  \begin{array}{lll}
    \tm{\cpCut{x}{\cpCut{y}{
    \cpRecv{x}{z}{\cpRecv{y}{w}{\alice}}}{
    \cpSend{y}{w}{\bankjohn}{\john}}}{
    \cpSend{x}{z}{\bankmary}{\mary}}}
    & \equiv          & \text{by \cref{thm:cp-display-cut-1}}
    \\
    \tm{\cpCut{y}{\cpCut{x}{
    \cpRecv{x}{z}{\cpRecv{y}{w}{\alice}}}{
    \cpSend{x}{z}{\bankmary}{\mary}}}{
    \cpSend{y}{w}{\bankjohn}{\john}}}
    & \Longrightarrow & \text{by \cpRedBetaTensParr}
    \\
    \tm{\cpCut{y}{\cpCut{z}{\cpCut{x}{
    \cpRecv{y}{w}{\alice}}{
    \mary}}{
    \bankmary}}{
    \cpSend{y}{w}{\bankjohn}{\john}}} 
    & \equiv          & \text{by \cref{thm:cp-display-cut-1}}
    \\
    \tm{\cpCut{z}{\cpCut{x}{\cpCut{y}{
    \cpRecv{y}{w}{\alice}}{
    \cpSend{y}{w}{\bankjohn}{\john}}}{
    \mary}}{
    \bankmary}}
    & \Longrightarrow & \text{by \cpRedBetaTensParr}
    \\
    \tm{\cpCut{z}{\cpCut{x}{\cpCut{w}{\cpCut{y}{
    \alice}{
    \john}}{
    \bankjohn}}{
    \mary}}{
    \bankmary}}
  \end{array}
\end{align*}
Note that the second reduction sequence is a \emph{valid} reduction sequence in
the either system---it is simply not the sequence \emph{chosen} by the reduction
strategy described by \citenat{wadler2012}.

%%% Local Variables:
%%% TeX-master: "main"
%%% End:

\chapter{Non-deterministic Classical Processes}\label{sec:main}
In this section, we will discuss our main contribution: an extension of \cp
which allows for races while still excluding deadlocks. 
We have seen in \cref{sec:cp-example} how \cp excludes deadlocks, but how exactly
does \cp exclude races?
Let us return to our first example from \cref{sec:introduction}, to the
interaction between John, Mary and the store.
\[
  \begin{array}{c}
    \tm{(\piPar{%
    \piSend{x}{\sliceofcake}{\piSend{x}{\nope}{\store}}
    }{%
    \piPar{\piRecv{x}{y}{\john}}{\piRecv{x}{z}{\mary}}
    })}
    \\[1ex]
    \rotatebox[origin=c]{270}{$\Longrightarrow^{\star}$}
    \\[1ex]
    \tm{(\piPar{\store}{\piPar{\piSub{\sliceofcake}{y}{\john}}{\piSub{\nope}{z}{\mary}}})}
    \quad
    \text{or}
    \quad
    \tm{(\piPar{\store}{\piPar{\piSub{\nope}{y}{\john}}{\piSub{\sliceofcake}{z}{\mary}}})}
  \end{array}
\]
Races occur when more than two processes attempt to communicate simultaneously
over the \emph{same} channel. However, the \text{Cut} rule of \cp requires that
\emph{exactly two} processes communicate over each channel:
\begin{center}
  \cpInfCut
\end{center}
We could attempt write down a protocol for our example, stating that the store
has a pair of channels $\tm{x}, \tm{y} : \ty{\cake}$ with which it communicates
with John and Mary, taking \cake to be the type of interactions in which cake
\emph{may} be obtained, i.e.\ of both \sliceofcake and \nope, and state that the
store communicates with John \emph{and} Mary over a channel of type \ty{\cake
  \parr \cake}.
However, this \emph{only} models interactions such as the following:
\begin{prooftree}
  \AXC{$\seq[{ \john }]{ \Gamma, \tmty{x}{\cake^\bot} }$}
  \AXC{$\seq[{ \mary }]{ \Delta, \tmty{y}{\cake^\bot} }$}
  \SYM{\tens}
  \BIC{$\seq[{ \cpSend{y}{x}{\john}{\mary} }]{
      \Gamma, \Delta, \tmty{y}{\cake^\bot \tens \cake^\bot} }$}
  \AXC{$\seq[{ \store }]{ \Theta, \tmty{x}{\cake}, \tmty{y}{\cake} }$}
  \SYM{\parr}
  \UIC{$\seq[{ \cpRecv{y}{x}{\store} }]{
      \Theta, \tmty{y}{\cake \parr \cake} }$}
  \NOM{Cut}
  \BIC{$\seq[{ \cpCut{y}{\cpSend{y}{x}{\john}{\mary}}{\cpRecv{y}{x}{\store}} }]{
      \Gamma, \Delta, \Theta }$}
\end{prooftree}
Note that in this interaction, John will get whatever the store decides to send
on \tm{x}, and Mary will get whatever the store decides to send on \tm{y}.
This means that this interactions gives the choice of who receives what \emph{to
the store}. This is not an accurate model of our original example, where
the choice of who receives the cake is non-deterministic and depends on factors
outside of any of the participants' control!
And to make matters worse, the term which models our example is entirely
different from the one we initially wrote down in the \textpi-calculus!

The ability to model racy behaviour, such as that in our example, is essential
to describing the interactions that take place in realistic concurrent systems.
Therefore, we would like to extend \cp to allow such races.
Specifically, we would like to do it in a way which mirrors the way in which the
\textpi-calculus handles non-determinism.
We will base our extension on \rcp, a subset of \cp which we introduced in
\cref{sec:background}.
We have chosen to do this to keep our discussion as simple as possible.
Furthermore, as compatibility with the \textpi-calculus is of interest, we will
use the reduction system without commutative conversions, which we introduced in
\cref{sec:cppi}.

This chapter proceeds as follows.
\wen{I should probably write this bit last.}

%% * Terms and types
\section{Terms and types}\label{sec:nc-terms-and-types}
Let us return, briefly, to our example.
\begin{gather*}
  \tm{(\piPar{%
      \piSend{x}{\sliceofcake}{\piSend{x}{\nope}{\store}}
    }{%
      \piPar{\piRecv{x}{y}{\john}}{\piRecv{x}{z}{\mary}}
    })}
\end{gather*}
In this interaction, we see that the channel \tm{x} is used only as a way to
connect the various clients, John and Mary, to the store.
The \emph{real} communication, sending the slice of cake and disappointment,
takes places on the channels \tm{\sliceofcake}, \tm{\nope}, \tm{y} and \tm{z}.

Inspired by this, we add two new constructs to the term language of \cp: sending
and receiving on a \emph{shared} channel.
These actions are marked with a \tm{\star} in order to distinguish them
syntactically from ordinary sending and receiving.
To group clients, we add another form of parallel composition, which we refer to
as \emph{pooling}. 
\begin{definition}[Terms]\label{def:nc-terms}
  We extend \cref{def:cp-terms} with the following constructs:
  \[\!
    \begin{aligned}
      \tm{P}, \tm{Q}, \tm{R}
           :=& \; \dots
      \\ \mid& \; \tm{\ncCnt{x}{y}{P}} &&\text{create client}
      \\ \mid& \; \tm{\ncSrv{x}{y}{P}} &&\text{create server interaction}
      \\ \mid& \; \tm{\ncPool{P}{Q}}   &&\text{parallel composition of clients}
    \end{aligned}
  \]
\end{definition}
%%% Local Variables:
%%% TeX-master: "main"
%%% End:

As before, round brackets denote input, square brackets denote output.
Note that \tm{\ncCnt{x}{y}{P}}, much like \tm{\cpSend{x}{y}{P}{Q}}, is a bound
output---this means that both client creation and server interaction bind a new
name.

In \rcp, we terms are identified up to the commutativity and associativity of
parallel composition. In \nodcap, we add another form of parallel composition,
and therefore must extend our structural congruence:
\begin{definition}[Structural congruence]\label{def:nc-equiv}
  We extend \cref{def:cp-equiv} with the following equivalences:
  \[
    \begin{array}{llll}
      \ncEquivPoolComm
      & \tm{\ncPool{P}{Q}}
      & \equiv \;
      & \tm{\ncPool{Q}{P}}
      \\
      \ncEquivPoolAss1
      & \tm{\ncPool{P}{\ncPool{Q}{R}}}
      & \equiv \;
      & \tm{\ncPool{\ncPool{P}{Q}}{R}}
      \\
      \ncRedKappaPool
      & \tm{\cpCut{x}{\ncPool{P}{Q}}{R}}
      & \equiv \;
      & \tm{\ncPool{P}{\cpCut{x}{Q}{R}}} \quad \text{if} \; \notFreeIn{x}{P} 
    \end{array}
  \]
\end{definition} 
%%% Local Variables:
%%% TeX-master: "main"
%%% End:

We add axioms for the commutativity and associativity of pooling.
We do not add an axiom for \ncEquivPoolAss2, as it follows from
\cref{def:nc-equiv}, see \cref{thm:nc-pool-assoc2}.
It should be noted that \tm{\cpCut{x}{P}{Q}} is considered a single,
\emph{atomic} construct.
Therefore you \emph{cannot} use \ncEquivPoolAss1 to rewrite
\tm{\cpCut{x}{P}{\ncPool{Q}{R}}} to \tm{\cpCut{x}{\ncPool{P}{Q}}{R}}.
We do, however, add two axioms which relate cuts and pool.
We call these \emph{extrusion}, because they closely resemble the
\textpi-calculus axiom for scope extrusion.
We add both \ncRedKappaPool1 and \ncRedKappaPool2, as these relate two different
constructs, and therefore we cannot use the one to derive the other. 
\begin{lemma}[\ncEquivPoolAssNoParen2]\label{thm:nc-pool-assoc2}
  We have
  \[
    \tm{\ncPool{\ncPool{P}{Q}}{R}} \equiv
    \tm{\ncPool{P}{\ncPool{Q}{R}}}
  \]
\end{lemma}
\begin{proof}
  \begin{align*}
    \tm{\ncPool{\ncPool{P}{Q}}{R}} &\equiv \qquad \text{by \ncEquivPoolComm and congruence} \\
    \tm{\ncPool{\ncPool{Q}{P}}{R}} &\equiv \qquad \text{by \ncEquivPoolComm} \\
    \tm{\ncPool{R}{\ncPool{Q}{P}}} &\equiv \qquad \text{by \ncEquivPoolAss1} \\
    \tm{\ncPool{\ncPool{R}{Q}}{P}} &\equiv \qquad \text{by \ncEquivPoolComm} \\
    \tm{\ncPool{P}{\ncPool{R}{Q}}} &\equiv \qquad \text{by \ncEquivPoolComm and congruence} \\
    \tm{\ncPool{P}{\ncPool{Q}{R}}}
  \end{align*}
\end{proof}
%%% Local Variables:
%%% TeX-master: "main"
%%% End:

Furthermore, the extensions to structural congruence preserve symmetry.
\begin{theorem}[Symmetry]\label{thm:nc-symmetry}
  If $\tm{P} \equiv \tm{Q}$, then $\tm{Q} \equiv \tm{P}$.
\end{theorem}
\begin{proof}
  By induction on the structure of the equivalence proof.
\end{proof}
%%% Local Variables:
%%% TeX-master: "main"
%%% End:

We can make another observation from our examples.
In every example in which a server interacts with a pool of clients, and which
does not deadlock, there are \emph{exactly} as many clients as there are
server interactions.
Therefore, we add two new \emph{dual} types for client pools and servers, which
track how many clients or server interactions they represent.
\begin{definition}[Types]\label{def:nc-types}
  \[\!
    \begin{aligned}
      \ty{A}, \ty{B}, \ty{C}
           :=& \; \dots
      \\ \mid& \; \ty{\take[n]{A}} &&\text{pool of} \; n \; \text{clients}
      \\ \mid& \; \ty{\give[n]{A}} &&n \; \text{server interactions}
    \end{aligned}
  \]  
\end{definition}
%%% Local Variables:
%%% TeX-master: "main"
%%% End:

\begin{definition}[Negation]\label{def:nc-negation}
  We extend \cref{def:cp-negation} with the following cases:
  \[\!
    \begin{array}{lclclcl}
              \ty{(\take[n]{A})^\bot} &=& \ty{\give[n]{A^\bot}}
      &\quad& \ty{(\give[n]{A})^\bot} &=& \ty{\take[n]{A^\bot}}
    \end{array}
  \]
\end{definition}
%%% Local Variables:
%%% TeX-master: "main"
%%% End:

With these new types, duality remains an involutive function.
\begin{lemma}[Involutive]\label{thm:nc-negation-involutive}
  We have $\ty{A^{\bot\bot}} = \ty{A}$.
\end{lemma}
\begin{proof}
  By induction on the structure of the type $\ty{A}$.
\end{proof}
%%% Local Variables:
%%% TeX-master: "main"
%%% End:


\section{Typing clients and servers}\label{sec:nc-clients-and-servers}
We have to add typing rules to associate our new client and server
interactions with their types.
The definition for environments will remain unchanged, but we will extend the
definition for the typing judgement.
To determine the new typing rules, we essentially have to answer the question
``What typing constructs do we need to complete the following proof?''
\begin{prooftree}
  \AXC{$\seq[{ \john }]{ \Gamma, \tmty{y}{\cake^\bot} }$}
  \noLine\UIC{$\smash{\vdots}\vphantom{\vdash}$}
  \AXC{$\seq[{ \mary }]{ \Delta, \tmty{y'}{\cake^\bot} }$}
  \noLine\UIC{$\smash{\vdots}\vphantom{\vdash}$}
  \AXC{$\seq[{ \store }]{ \Theta, \tmty{z}{\cake}, \tmty{z'}{\cake} }$}
  \noLine\UIC{$\smash{\vdots}\vphantom{\vdash}$}
  \noLine\TIC{$\seq[{
      \cpCut{x}{\ncPool{\ncCnt{x}{y}{\john}}{\ncCnt{x}{y'}{\mary}}}{
        \ncSrv{x}{z}{\ncSrv{x}{z'}{\store}}} }]{
      \Gamma, \Delta, \Theta }$}
\end{prooftree}
Ideally, we would still like the composition of the client pool and the server
to be a cut. This seems reasonable, as the left-hand side of the term above has
2 clients, and the right-hand side has two server interactions, so \tm{x} is
used at type \ty{\take[2]{\cake^\bot}} on the left, and as \ty{\give[2]{\cake}}
on the right.
\begin{prooftree}
  \AXC{$\seq[{ \john }]{ \Gamma, \tmty{y}{\cake^\bot} }$}
  \noLine\UIC{$\smash{\vdots}\vphantom{\vdash}$}
  \AXC{$\seq[{ \mary }]{ \Delta, \tmty{y'}{\cake^\bot} }$}
  \noLine\UIC{$\smash{\vdots}\vphantom{\vdash}$}
  \noLine\BIC{$\seq[{ \ncPool{\ncCnt{x}{y}{\john}}{\ncCnt{x}{y'}{\mary}} }]{
      \Gamma, \Delta, \tmty{x}{\take[2]{\cake^\bot}} }$}

  \AXC{$\seq[{ \store }]{ \Theta, \tmty{z}{\cake}, \tmty{z'}{\cake} }$}
  \noLine\UIC{$\smash{\vdots}\vphantom{\vdash}$}
  \noLine\UIC{$\seq[{ \ncSrv{x}{z}{\ncSrv{x}{z'}{\store}} }]{
      \Theta, \tmty{x}{\give[2]{\cake}} }$}

  \NOM{Cut}
  \BIC{$\seq[{
      \cpCut{x}{\ncPool{\ncCnt{x}{y}{\john}}{\ncCnt{x}{y'}{\mary}}}{
        \ncSrv{x}{z}{\ncSrv{x}{z'}{\store}}} }]{
      \Gamma, \Delta, \Theta }$}
\end{prooftree}
We will define the typing judgement, and then discuss servers and clients, the
two sides of the above cut, describe the rules we add, and show how they allow
us to complete our proof.
\begin{definition}[Typing judgements]\label{def:cp-typing-judgement}
  A typing judgement $\seq[{ P }]{\tmty{x_1}{A_1}\dots\tmty{x_n}{A_n}}$ denotes
  that the process \tm{P} communicates along channels $\tm{x_1}\dots\tm{x_n}$
  following protocols $\ty{A_1}\dots\ty{A_n}$.
  Typing judgements can be constructed using the inference rules in
  \cref{fig:cp-typing-judgement,fig:nc-typing-judgement}.
\end{definition}
%%% Local Variables:
%%% TeX-master: "main"
%%% End:

\begin{figure*}[bh]
  \begin{center}
    \ncInfTake1
    \ncInfGive1
  \end{center}
  \vspace*{1\baselineskip}
  \begin{center}
    \ncInfPool
  \end{center}
  \vspace*{1\baselineskip}
  \begin{center}
    \ncInfCont
  \end{center}
  \caption{Typing judgement for \nodcap extending that of Figure~\ref{fig:cp-typing-judgement}}
  \label{fig:nc-typing-judgement}
\end{figure*}
%%% Local Variables:
%%% TeX-master: "main"
%%% End:


\subsection{Clients and pooling}\label{sec:clients-and-pooling}
A client pool represents a number of independent processes, each wanting to
interact with the server. Examples of such a pool include John and Mary from our
example, customers for online stores in general, and any number of processes
which interact with a single, centralised server.

We introduce two new rules: one to construct clients, and one to pool them
together. The first rule, $(\take[1]{})$, marks interaction over some channel as
a client interaction. It does this by receiving a channel \tm{y} over a
\emph{shared} channel \tm{x}. The channel \tm{y} is the channel across which the
actual interaction will eventually take place.
The second rule, \textsc{Pool}, allows us to pool together clients. This is
implemented, as in the \textpi-calculus, using parallel composition.
\begin{center}
  \ncInfTake1
  \ncInfPool
\end{center}
Using these rules, we can derive the left-hand side of our proof by marking John
and Mary as clients, and pooling them together.
\begin{prooftree}
  \AXC{$\seq[{ \john }]{ \Gamma, \tmty{y}{\cake^\bot} }$}
  \SYM{(\take[1]{})}
  \UIC{$\seq[{ \ncCnt{x}{y}{\john} }]{ \Gamma, \tmty{z}{\take[1]{\cake^\bot}} }$}

  \AXC{$\seq[{ \mary }]{ \Delta, \tmty{y'}{\cake^\bot} }$}
  \SYM{(\take[1]{})}
  \UIC{$\seq[{ \ncCnt{x}{y'}{\mary} }]{ \Delta, \tmty{y'}{\take[1]{\cake^\bot}} }$}

  \NOM{Pool}
  \BIC{$\seq[{ \ncPool{\ncCnt{x}{y}{\john}}{\ncCnt{x}{y'}{\mary}} }]{
      \Gamma, \Delta, \tmty{x}{\take[2]{\cake^\bot}} }$}
\end{prooftree}

\subsection{Servers and contraction}\label{sec:servers-and-contraction}
Dual to a pool of clients is a server. Our interpretation of a server is a
process which offers up some number of interdependent interactions of the same
type. Examples include the store from our example, which gives out slices of
cake and disappointment, online stores in general, and any central server which
interacts with some number of client processes.

We introduce two new rules to construct servers. The first rule, $(\give[1]{})$,
marks a interaction over some channel as a server interaction. It does this by
sending a channel \tm{y} over a \emph{shared} channel \tm{x}. The channel \tm{y}
is the channel across which the actual interaction will eventually take place.
The second rule, \textsc{Cont}, short for contraction, allows us to contract
several server interactions into a single server. This allows us to construct a
server which has multiple interactions of the same type, across the same shared
channel.\footnote{%
  While it ultimately does not matter whether $(\give[1]{})$ and $(\take[1]{})$
  are implemented with a send or a receive action, it feels more natural to have
  the server do the sending.
  Clients indicate their interest in interacting with the server by connecting
  to the shared channel, but it is up to the server to decide \emph{when} to
  interact with each channel.}
\begin{center}
  \ncInfGive1
  \ncInfCont
\end{center}
Using these rules, we can derive the right-hand side of our proof, by marking
each of the store's interactions as server interactions, and then contracting
them.
\begin{prooftree}
  \AXC{$\seq[{ \store }]{ \Theta, \tmty{z}{\cake}, \tmty{z'}{\cake} }$}
  \SYM{(\give[1]{})}
  \UIC{$\seq[{ \ncSrv{x'}{z'}{\store} }]{
      \Theta, \tmty{z}{\cake}, \tmty{x'}{\give[1]{\cake}} }$}
  \SYM{(\give[1]{})}
  \UIC{$\seq[{ \ncSrv{x}{z}{\ncSrv{x'}{z'}{\store}} }]{
      \Theta, \tmty{x}{\give[1]{\cake}}, \tmty{x'}{\give[1]{\cake}} }$}
  \NOM{Cont}
  \UIC{$\seq[{ \ncSrv{x}{z}{\ncSrv{x}{z'}{\store}} }]{
      \Theta, \tmty{x}{\give[2]{\cake}} }$}
\end{prooftree}
Thus, we complete the typing derivation of our example.

\section{Running clients and servers}
Once we have a client/server interaction, how do we run it? Ideally, we would
simply use the reduction rule closest to the one used in the \textpi-calculus. 
\begin{gather*}
  \reducesto
  {\tm{\cpCut{x}{\ncCnt{x}{y}{P}}{\ncSrv{x}{z}{R}}}}
  {\tm{\cpCut{y}{P}{\cpSub{y}{z}{R}}}}
\end{gather*}
However, our case is complicated by the fact that in \tm{\cpCut{x}{P}{Q}} the
name restriction is an inseparable part of the composition, and therefore has to
be part of our reduction rule. 
Because of this, the above reduction can only apply in the singleton case.
If the client pool contains more than one client, such as in the term below,
then there is no way to isolate a single client together with the server,
because \tm{x} occurs in both \tm{\ncCnt{x}{y}{P}} and \tm{\ncCnt{x}{z}{Q}}.
\begin{gather*}
  \tm{\cpCut{x}{\ncPool{\ncCnt{x}{y}{P}}{\ncCnt{x}{z}{Q}}}{\ncSrv{x}{w}{R}}}
  \centernot\Longrightarrow
\end{gather*}
Therefore, we add a second reduction rule, which handles communication between a
one client in a pool of multiple clients and a server.
\begin{gather*}
  \reducesto
  {\tm{\cpCut{x}{\ncPool{\ncCnt{x}{y}{P}}{Q}}{\ncSrv{x}{z}{R}}}}
  {\tm{\cpCut{x}{Q}{\cpCut{y}{P}{\cpSub{y}{z}{R}}}}}
\end{gather*}
Lastly, because we have added another form of parallel composition, we add
another congruence rule, to allow for reduction inside client pools.
\begin{definition}[Term reduction]\label{def:nc-reduction}
  We extend \cref{def:cp-reduction} with the following reductions:
  \[
    \begin{array}{llll}
      \ncRedBetaStar{1}
      & \tm{\cpCut{x}{\ncCnt{x}{y}{P}}{\ncSrv{x}{z}{R}}}
      & \Longrightarrow \;
      & \tm{\cpCut{y}{P}{\cpSub{y}{z}{R}}}
      \\
      \ncRedBetaStar{n+1}
      & \tm{\cpCut{x}{\ncPool{\ncCnt{x}{y}{P}}{Q}}{\ncSrv{x}{z}{R}}}
      & \Longrightarrow \;
      & \tm{\cpCut{x}{Q}{\cpCut{y}{P}{\cpSub{y}{z}{R}}}}
      \\
      \\
      \ncRedKappaTake
      & \tm{\cpCut{x}{\ncCnt{y}{z}{P}}{R}}
      & \Longrightarrow \;
      & \tm{\ncCnt{y}{z}{\cpCut{x}{P}{R}}}
      \\
      \ncRedKappaGive
      & \tm{\cpCut{x}{\ncSrv{y}{z}{P}}{R}}
      & \Longrightarrow \;
      & \tm{\ncSrv{y}{z}{\cpCut{x}{P}{R}}}
    \end{array}
  \]
  \begin{prooftree}
    \AXC{\reducesto{P}{P'}}
    \SYM{\ncRedGammaPool}
    \UIC{\reducesto{\ncPool{P}{Q}}{\ncPool{P'}{Q}}}
  \end{prooftree}
\end{definition}
%%% Local Variables:
%%% TeX-master: "main"
%%% End:


The rules \ncRedBetaStar1 and \ncRedBetaStar{n+1} seem like the elimination
rules for a list-like construct. This may come as a surprise, as our client
pools are built up like binary trees, and the typing rules for both sides are
tree-like, with $(\take[1]{})$ and $(\give[1]{})$ playing the role of leaves,
and \textsc{Pool} and \textsc{Cont} merging two trees with $m$ and $n$ leaves
into one with $m+n$ leaves.
However, the server process imposes a sequential ordering on its interactions,
and it is because of this that we have to use list-like elimination rules.

\section{Properties of \nodcap}
In this section, we will revisit the proofs for three important properties of
\rcp, namely preservation, progress, and termination, and show that our
extensions preserve these properties.

\subsection{Preservation}
Preservation is the fact that term reduction preserves typing. There are two
proofs involved in this. First, we show that structural congruence preserves
typing.
\begin{theorem}[Preservation for $\equiv$]\label{thm:nc-preservation-equiv}
  If $\seq[{ P }]{ \Gamma }$ and $\tm{P} \equiv \tm{Q}$,
  then $\seq[{ Q }]{ \Gamma }$.
\end{theorem}
\begin{proof}
  By induction on the structure of the equivalence. The cases for reflexivity,
  transitivity and congruence are trivial. The cases for \cpEquivCutComm and
  \cpEquivCutAss1 are given in \cref{fig:cp-preservation-equiv}.
  The cases for \ncEquivPoolComm and \ncEquivPoolAss1 are given in
  \cref{fig:nc-preservation-equiv}.
\end{proof}
%%% Local Variables:
%%% TeX-master: "main"
%%% End:

\begin{figure*}[b]
  \centering
  \begin{tabular}{ll}
    \ncEquivPoolComm
    &
      \begin{prooftree*}
        \AXC{$\seq[{ P }]{ \Gamma, \tmty{x}{\take[m]{A}} }$}
        \AXC{$\seq[{ Q }]{ \Delta, \tmty{x}{\take[n]{A}} }$}
        \NOM{Pool}
        \BIC{$\seq[{ \ncPool{P}{Q} }]{ \Gamma, \Delta, \take[m+n]{A} }$}
      \end{prooftree*}
    \\[30pt]
    $\equiv$
    &
      \begin{prooftree*}
        \AXC{$\seq[{ Q }]{ \Delta, \tmty{x}{\take[n]{A}} }$}
        \AXC{$\seq[{ P }]{ \Gamma, \tmty{x}{\take[m]{A}} }$}
        \NOM{Pool}
        \BIC{$\seq[{ \ncPool{Q}{P} }]{ \Gamma, \Delta, \tmty{x}{\take[m+n]{A}} }$}
      \end{prooftree*}
    \\[40pt]
    \ncEquivPoolAss1
    &
      \begin{prooftree*}
        \AXC{$\seq[{ P }]{ \Gamma, \tmty{x}{\take[l]{A}} }$}
        \AXC{$\seq[{ Q }]{ \Delta, \tmty{x}{\take[m]{A}} }$}
        \AXC{$\seq[{ R }]{ \Theta, \tmty{x}{\take[n]{A}} }$}
        \NOM{Pool}
        \BIC{$\seq[{ \ncPool{Q}{R} }]{ \Delta, \Theta, \tmty{x}{\take[m+n]{A}} }$}
        \NOM{Pool}
        \BIC{$\seq[{ \ncPool{P}{\ncPool{Q}{R}} }]{ \Gamma, \Delta, \Theta, \tmty{x}{\take[l+m+n]{A}} }$}
      \end{prooftree*}
    \\[30pt]
    $\equiv$
    &
      \begin{prooftree*}
        \AXC{$\seq[{ P }]{ \Gamma, \tmty{x}{\take[l]{A}} }$}
        \AXC{$\seq[{ Q }]{ \Delta, \tmty{x}{\take[m]{A}} }$}
        \NOM{Pool}
        \BIC{$\seq[{ \ncPool{P}{Q} }]{ \Gamma, \Delta, \tmty{x}{\take[l+m]{A}} }$}
        \AXC{$\seq[{ R }]{ \Theta, \tmty{x}{\take[n]{A}} }$}
        \NOM{Pool}
        \BIC{$\seq[{ \ncPool{\ncPool{P}{Q}}{R} }]{ \Gamma, \Delta, \Theta, \tmty{x}{\take[l+m+n]{A}} }$}
      \end{prooftree*}
    \\[20pt]
    or
    \\[20pt]
    \ncEquivPoolAss1
    &
      \begin{prooftree*}
        \AXC{$\seq[{ P }]{ \Gamma, \tmty{x}{\take[k]{A}} }$}
        \AXC{$\seq[{ Q }]{ \Delta, \tmty{x}{\take[l]{A}}, \tmty{y}{\take[m]{B}} }$}
        \AXC{$\seq[{ R }]{ \Theta, \tmty{x}{\take[n]{B}} }$}
        \NOM{Pool}
        \BIC{$\seq[{ \ncPool{Q}{R} }]{ \Delta, \Theta, \tmty{x}{\take[l]{A}}, \tmty{y}{\take[m+n]{B}} }$}
        \NOM{Pool}
        \BIC{$\seq[{ \ncPool{P}{\ncPool{Q}{R}} }]{ \Gamma, \Delta, \Theta, \tmty{x}{\take[k+l]{A}}, \tmty{y}{\take[m+n]{B}} }$}
      \end{prooftree*}
    \\[30pt]
    $\equiv$
    &
      \begin{prooftree*}
        \AXC{$\seq[{ P }]{ \Gamma, \tmty{x}{\take[k]{A}} }$}
        \AXC{$\seq[{ Q }]{ \Delta, \tmty{x}{\take[l]{A}}, \tmty{y}{\take[m]{B}} }$}
        \NOM{Pool}
        \BIC{$\seq[{ \ncPool{Q}{R} }]{ \Gamma, \Delta, \tmty{x}{\take[k+l]{A}}, \tmty{y}{\take[m]{B}} }$}
        \AXC{$\seq[{ R }]{ \Theta, \tmty{x}{\take[o]{B}} }$}
        \NOM{Pool}
        \BIC{$\seq[{ \ncPool{P}{\ncPool{Q}{R}} }]{ \Gamma, \Delta, \Theta, \tmty{x}{\take[k+l]{A}}, \tmty{y}{\take[m+n]{B}} }$}
      \end{prooftree*}
    \\[30pt]
    \ncRedKappaPool1
    &
      \begin{prooftree*}
        \AXC{$\seq[{ P }]{ \Gamma, \tmty{y}{\take[m]{B}} }$}
        \AXC{$\seq[{ Q }]{ \Delta, \tmty{x}{A}, \tmty{y}{\take[n]{B}} }$}
        \AXC{$\seq[{ R }]{ \Theta, \tmty{x}{A^\bot} }$}
        \NOM{Cut}
        \BIC{$\seq[{ \cpCut{x}{Q}{R} }]{ \Gamma, \Delta, \Theta, \tmty{y}{\take[n]{B}} }$}
        \NOM{Pool}
        \BIC{$\seq[{ \ncPool{P}{\cpCut{x}{Q}{R}} }]{ \Gamma, \Delta, \Theta, \tmty{y}{\take[m+n]{B}} }$}
      \end{prooftree*}
    \\[30pt]
    $\equiv$
    &
      \begin{prooftree*}
        \AXC{$\seq[{ P }]{ \Gamma, \tmty{y}{\take[m]{B}} }$}
        \AXC{$\seq[{ Q }]{ \Delta, \tmty{x}{A}, \tmty{y}{\take[n]{B}} }$}
        \NOM{Pool}
        \BIC{$\seq[{ \ncPool{P}{Q} }]{ \Gamma, \Delta, \tmty{y}{\take[m+n]{B}} }$}
        \AXC{$\seq[{ R }]{ \Theta, \tmty{x}{A^\bot} }$}
        \NOM{Cut}
        \BIC{$\seq[{ \cpCut{x}{\ncPool{P}{Q}}{R} }]{ \Gamma, \Delta, \Theta, \tmty{y}{\take[m+n]{B}} }$}
      \end{prooftree*} 
    \\[30pt]
    \ncRedKappaPool2
    &
      (as above)
  \end{tabular}
  \caption{Type preservation for the structural congruence of \nodcap}
  \label{fig:nc-preservation-equiv}
\end{figure*}
%%% Local Variables:
%%% TeX-master: "main"
%%% End:

Secondly, we prove that term reduction preserves typing.
\begin{theorem}[Preservation]\label{thm:nc-preservation}
  If \reducesto{P}{Q} and $\seq[{ P }]{ \Gamma }$, then $\seq[{ Q }]{ \Gamma }$.
\end{theorem}
\begin{proof}
  By induction on the structure of the reduction. See
  \cref{fig:cp-preservation-1} for the AxCut and \textbeta-reduction rules from
  \cp, and \cref{fig:cp-preservation-2a,fig:cp-preservation-2b} for the
  commutative conversions from \cp.
  See \cref{fig:nc-preservation} for the \textbeta-reduction rules and
  commutative conversions from \nc.
  The cases for \cpRedGammaCut and \ncRedGammaPool are trivial by call to the
  induction hypothesis, and the case for \cpRedGammaEquiv is trivial by call to
  the induction hypothesis and \cref{thm:cp-preservation-equiv}.
\end{proof}
%%% Local Variables:
%%% TeX-master: "main"
%%% End:

\begin{figure*}[ht]
  \makebox[\textwidth][c]{
    \begin{tabular}{ll}
      \ncRedBetaStar1
      &
        \begin{prooftree*}
          \AXC{$\seq[{ P }]{ \Gamma, \tmty{y}{A} }$}
          \SYM{\take[1]{}}
          \UIC{$\seq[{ \ncCnt{x}{y}{P} }]{ \Gamma, \tmty{x}{\take[1]{A}} }$}
          \AXC{$\seq[{ Q }]{ \Delta, \tmty{z}{A^\bot} }$}
          \SYM{\give[1]{}}
          \UIC{$\seq[{ \ncSrv{x}{z}{Q} }]{ \Delta, \tmty{x}{\give[1]{A}} }$}
          \NOM{Cut}
          \BIC{$\seq[{ \cpCut{x}{\ncCnt{x}{y}{P}}{\ncSrv{x}{z}{Q}} }]{ \Gamma, \Delta }$}
        \end{prooftree*}
      \\[30pt]
      $\Longrightarrow$
      &
        \begin{prooftree*}
          \AXC{$\seq[{ P }]{ \Gamma, \tmty{y}{A} }$}
          \AXC{$\seq[{ Q }]{ \Delta, \tmty{z}{A^\bot} }$}
          \NOM{Cut}
          \BIC{$\seq[{ \cpCut{y}{P}{\cpSub{y}{z}{Q}} }]{ \Gamma, \Delta }$}
        \end{prooftree*}
      \\[40pt]
      \ncRedBetaStar{n+1}
      &
        \begin{prooftree*}
          \AXC{$\seq[{ P }]{ \Gamma, \tmty{y}{A} }$}
          \SYM{\take[1]{}}
          \UIC{$\seq[{ \ncCnt{x}{y}{P} }]{ \Gamma, \tmty{x}{\take[1]{A}} }$}
          \AXC{$\seq[{ Q }]{ \Delta, \tmty{x}{\take[n]{A}} }$}
          \NOM{Pool}
          \BIC{$\seq[{ \ncPool{\ncCnt{x}{y}{P}}{Q} }]{ \Gamma, \Delta, \tmty{x}{\take[n+1]{A}} }$}
          \AXC{$\seq[{ R }]{ \Theta, \tmty{z}{A^\bot}, \tmty{x}{\give[n]{A}} }$}
          \SYM{\give[1]{}}
          \UIC{$\seq[{ \ncSrv{x'}{z}{Q} }]{ \Theta, \tmty{x'}{\give[1]{A}}, \tmty{x}{\give[n]{A}} }$}
          \NOM{Cont}
          \UIC{$\seq[{ \ncSrv{x}{z}{Q} }]{ \Theta, \tmty{x}{\give[n+1]{A}} }$}
          \NOM{Cut}
          \BIC{$\seq[{ \cpCut{x}{\ncPool{\ncCnt{x}{y}{P}}{Q}}{\ncSrv{x}{z}{Q}} }]{ \Gamma, \Delta, \Theta }$}
        \end{prooftree*}
      \\[40pt]
      $\Longrightarrow$
      &
        \begin{prooftree*}
          \AXC{$\seq[{ Q }]{ \Delta, \tmty{x}{\take[n]{A}} }$}
          \AXC{$\seq[{ P }]{ \Gamma, \tmty{y}{A} }$}
          \AXC{$\seq[{ R }]{ \Theta, \tmty{z}{A^\bot}, \tmty{x}{\give[n]{A}} }$}
          \NOM{Cut}
          \BIC{$\seq[{ \cpCut{y}{P}{\cpSub{y}{z}{R}} }]{ \Gamma, \Theta, \tmty{x}{\give[n]{A}} }$}
          \NOM{Cut}
          \BIC{$\seq[{ \cpCut{x}{Q}{\cpCut{y}{P}{\cpSub{y}{z}{R}}} }]{ \Gamma, \Delta, \Theta }$}
        \end{prooftree*}
      \\[40pt]
      \ncRedKappaTake
      &
        \begin{prooftree*}
          \AXC{$\seq[{ P }]{ \Gamma, \tmty{x}{A}, \tmty{z}{B} }$}
          \SYM{\take[1]{}}
          \UIC{$\seq[{ \ncCnt{y}{z}{P} }]{ \Gamma, \tmty{x}{A}, \tmty{y}{\take[1]{B}} }$}
          \AXC{$\seq[{ R }]{ \Delta, \tmty{x}{A^\bot} }$}
          \NOM{Cut}
          \BIC{$\seq[{ \cpCut{x}{\ncCnt{y}{z}{P}}{R} }]{ \Gamma, \Delta, \tmty{y}{\take[1]{B}} }$}
        \end{prooftree*}
      \\[30pt]
      $\Longrightarrow$
      &
        \begin{prooftree*}
          \AXC{$\seq[{ P }]{ \Gamma, \tmty{x}{A}, \tmty{z}{B} }$}
          \AXC{$\seq[{ R }]{ \Delta, \tmty{x}{A^\bot} }$}
          \NOM{Cut}
          \BIC{$\seq[{ \cpCut{x}{P}{R} }]{ \Gamma, \Delta, \tmty{z}{B} }$}
          \SYM{\take[1]{}}
          \UIC{$\seq[{ \ncCnt{y}{z}{\cpCut{x}{P}{R}} }]{ \Gamma, \Delta, \tmty{z}{\take[1]{B}} }$}
        \end{prooftree*}
      \\[30pt]
      \ncRedKappaGive
      &
        (as above)
    \end{tabular}
  }   

  \caption{Type preservation for the \textbeta-reduction rules and commutative conversions of \nodcap}
  \label{fig:nc-preservation-1}
\end{figure*}
%%% Local Variables:
%%% TeX-master: "main"
%%% End: 

\subsection{Canonical forms and progress}
In this section, we will extend the definition of canonical forms and the proof
of progress progress given in \cref{sec:cppi}.

\subsubsection{Canonical forms}
First, we extend the definitions of actions with our actions for client and
server creation. 
\begin{definition}[Action]\label{def:nc-action}
  We extend \cref{def:cp-action} with the following cases:
  \begin{itemize}[noitemsep,topsep=0pt,parsep=0pt,partopsep=0pt]
  \item \tm{\ncCnt{x}{y}{P'}}
  \item \tm{\ncSrv{x}{y}{P'}}
  \end{itemize}
\end{definition}
%%% Local Variables:
%%% TeX-master: "main"
%%% End:

Secondly, as we can reduce inside client pools, we will add pooling to our
definition of evaluation prefixes.
\begin{definition}[Evaluation prefixes]\label{def:nc-evaluation-prefixes}
  We extend \cref{def:cp-evaluation-prefixes} with the following constructs:
  \begin{align*}
    \tm{G}, \tm{H} := \dots \mid \tm{\ncPool{G}{H}}
  \end{align*}
  We also define a special case of evaluation prefixes, which we will refer to
  as \emph{pooling prefixes}. These are evaluation prefixes which consist solely
  of pooling operators and holes. 
\end{definition}
\begin{definition}[Plugging]\label{def:nc-evaluation-prefix-plugging}
  We extend \cref{def:cp-evaluation-prefix-plugging} with the following case:
  \[
    \begin{array}{ll}
      \tm{\cpPlug{\ncPool{G}{H}}{R_1 \dots R_m, R_{m+1} \dots R_{n}}}
                            & := \; \tm{\ncPool{\cpPlug{G}{R_1 \dots R_m}}{\cpPlug{H}{R_{m+1} \dots R_n}}}
    \end{array}
  \]
  Note that in the this case, \tm{G} is an evaluation prefix with $m$ holes,
  and \tm{H} is an evaluation prefix with $(n-m)$ holes.
\end{definition}
%%% Local Variables:
%%% TeX-master: "main"
%%% End:

The definition for the maximum evaluation prefix is unchanged.

There are some subtleties to our definition of canonical forms. The type system
for \cp guarantees that all links directly under an evaluation context act on a
bound channel. Not so for \nodcap.
\begin{scprooftree}
  \AXC{}
  \NOM{Ax}
  \UIC{$\seq[{ \cpLink{x}{y} }]{ \tmty{x}{\take[m]{A}}, \tmty{y}{\give[m]{A^\bot}} }$}
  \AXC{$\seq[{ P }]{ \Gamma, \tmty{x}{\take[n]{A}} }$}
  \NOM{Pool}
  \BIC{$\seq[{ \ncPool{\cpLink{x}{y}}{P} }]{ \Gamma, \tmty{x}{\take[m+n]{A}}, \tmty{y}{\give[m]{A^\bot}} }$}
\end{scprooftree}
There is no way to sensibly reduce this link.
Furthermore, in \cp, if two processes act on the same channel, then they must be
on different sides of the cut introducing that channel. The addition of shared
channels and client pools invalidates this property.
Therefore, we will have to be more careful about the way we define canonical
forms.
We restate the definition of canonical forms below. The additions have been
italicised.
\begin{definition}[Canonical forms]\label{def:nc-canonical-forms}
  A process \tm{P} is in canonical form if it is an action, or if it is of the
  form \tm{\cpPlug{G}{P_1 \dots P_n}}, where \tm{G} is the maximum evaluation
  prefix of \tm{P}, no \tm{P_i} is a link \emph{which acts on a bound channel},
  and no \tm{P_i} and \tm{P_j}, \emph{on different sides of at least one cut}, act on
  the same channel.
\end{definition}
%%% Local Variables:
%%% TeX-master: "main"
%%% End:


\subsection{Evaluation contexts}
Evaluation contexts are one-holed term contexts under which reduction can take
place. Since we have added another congruence rule, stating that reduction can
take place inside client pools, we extend our definition of evaluation contexts
to match this.
\begin{definition}[Evaluation contexts]\label{def:nc-evaluation-contexts}
  We extend \cref{def:cp-evaluation-contexts} with the following constructs:
  \begin{align*}
    \tm{E} := \dots \mid \tm{\ncPool{E}{P}} \mid \tm{\ncPool{P}{E}}
  \end{align*}
  We also define a special case of evaluation contexts, which we will refer to
  as \emph{pooling contexts}. These are evaluation contexts which consist solely
  of pooling operators and holes.
\end{definition}
\begin{definition}[Plugging]\label{def:nc-evaluation-context-plugging}
  We extend \cref{def:cp-evaluation-context-plugging} with the following cases:
  \begin{gather*}
    \begin{array}{ll}
      \tm{\cpPlug{\ncPool{E}{P}}{R}}
      & := \; \tm{\ncPool{\cpPlug{E}{R}}{P}}
      \\
      \tm{\cpPlug{\ncPool{P}{E}}{R}}
      & := \; \tm{\ncPool{P}{\cpPlug{E}{R}}}
    \end{array}
  \end{gather*}
\end{definition}
%%% Local Variables:
%%% TeX-master: "main"
%%% End:

We also restate \cref{thm:cp-display-cut-1}, and prove that our extension
preserves the property.
\begin{lemmaB}\label{thm:nc-display-cut-1}
  If $\seq[{ \tm{\cpCut{x}{\cpPlug{G}{P}}{Q}} }]{ \Gamma }$ and
  $\notFreeIn{x}{G}$, then $\tm{\cpCut{x}{\cpPlug{G}{P}}{Q}} \equiv
  \tm{\cpPlug{G}{\cpCut{x}{P}{Q}}}$.
\end{lemmaB}
\begin{proof}
  By induction on the structure of the evaluation context \tm{G}.
  \begin{itemize}
  \item
    Case $\tm{\Box}$, $\tm{\cpCut{y}{H}{R}}$, and $\tm{\cpCut{y}{R}{H}}$. See \cref{thm:cp-display-cut-1}.
  \item
    Case $\tm{\ncPool{G}{R}}$.
    \[\!
      \begin{array}{ll}
        \tm{\cpCut{x}{\ncPool{\cpPlug{G}{P}}{R}}{Q}} & \equiv \quad \text{by} \; \ncEquivPoolComm \\
        \tm{\cpCut{x}{\ncPool{R}{\cpPlug{G}{P}}}{Q}} & \equiv \quad \text{by} \; \ncRedKappaPool1 \\
        \tm{\ncPool{R}{\cpCut{x}{\cpPlug{G}{P}}{Q}}} & \equiv \quad \text{by} \; \ncEquivPoolComm \\
        \tm{\ncPool{\cpCut{x}{\cpPlug{G}{P}}{Q}}{R}} & \equiv \quad \text{by the induction hypothesis}\\
        \tm{\ncPool{\cpPlug{G}{\cpCut{x}{P}{Q}}}{R}} &
      \end{array}
    \]
  \item
    Case $\tm{\ncPool{R}{G}}$.
    \[\!
      \begin{array}{ll}
        \tm{\cpCut{x}{\ncPool{R}{\cpPlug{G}{P}}}{Q}} & \equiv \quad \text{by} \; \ncRedKappaPool1 \\
        \tm{\ncPool{R}{\cpCut{x}{\cpPlug{G}{P}}{Q}}} & \equiv \quad \text{by the induction hypothesis}\\
        \tm{\ncPool{R}{\cpPlug{G}{\cpCut{x}{P}{Q}}}} &
      \end{array}
    \]
  \end{itemize}
  In each case, the side condition for \ncRedKappaPool1, $\notFreeIn{x}{R}$, can
  be inferred from $\notFreeIn{x}{G}$, and the side conditions for the induction
  hypothesis can be inferred from \cref{thm:nc-preservation-equiv} and
  $\notFreeIn{x}{G}$.
\end{proof}
%%% Local Variables:
%%% TeX-master: "main"
%%% End:
Furthermore, it will be useful to prove a similar lemma, which shows that we can
push any pooling downwards under an evaluation context.
\begin{lemmaB}\label{thm:nc-display-pool-1}
  If $\seq[{ \ncPool{\cpPlug{G}{P}}{Q} }]{ \Gamma }$, then
  $\tm{\ncPool{\cpPlug{G}{P}}{Q}} \equiv \tm{\cpPlug{G}{\ncPool{P}{Q}}}$. 
\end{lemmaB}
\begin{proof}
  By induction on the structure of the evaluation context \tm{G}.
  \begin{itemize}
  \item
    Case $\tm{\Box}$. By reflexivity.
  \item
    Case $\tm{\cpCut{y}{G}{R}}$.
    \[\!
      \begin{array}{ll}
        \tm{\ncPool{\cpCut{y}{\cpPlug{G}{P}}{R}}{Q}} & \equiv \quad \text{by} \; \ncEquivPoolComm \\
        \tm{\ncPool{Q}{\cpCut{y}{\cpPlug{G}{P}}{R}}} & \equiv \quad \text{by} \; \ncRedKappaPool2 \\
        \tm{\cpCut{y}{\ncPool{Q}{\cpPlug{G}{P}}}{R}} & \equiv \quad \text{by} \; \cpEquivCutComm \\
        \tm{\cpCut{y}{\ncPool{\cpPlug{G}{P}}{Q}}{R}} & \equiv \quad \text{by the induction hypothesis}\\
        \tm{\cpCut{y}{\cpPlug{G}{\ncPool{P}{Q}}}{R}} &
      \end{array}
    \]
  \item
    Case $\tm{\cpCut{y}{R}{G}}$.
    \[\!
      \begin{array}{ll}
        \tm{\ncPool{\cpCut{y}{R}{\cpPlug{G}{P}}}{Q}} & \equiv \quad \text{by} \; \ncEquivPoolComm \\
        \tm{\ncPool{Q}{\cpCut{y}{\cpPlug{G}{P}}{R}}} & \equiv \quad \text{by} \; \ncRedKappaPool2 \\
        \tm{\cpCut{y}{\ncPool{Q}{\cpPlug{G}{P}}}{R}} & \equiv \quad \text{by} \; \cpEquivCutComm \\
        \tm{\cpCut{y}{R}{\ncPool{Q}{\cpPlug{G}{P}}}} & \equiv \quad \text{by the induction hypothesis} \\
        \tm{\cpCut{y}{R}{\cpPlug{G}{\ncPool{P}{Q}}}}
      \end{array}
    \]
  \item
    Case $\tm{\ncPool{G}{R}}$.
    \[\!
      \begin{array}{ll}
        \tm{\ncPool{\ncPool{\cpPlug{G}{P}}{R}}{Q}} & \equiv \quad \text{by} \; \ncEquivPoolComm \\
        \tm{\ncPool{\ncPool{R}{\cpPlug{G}{P}}}{Q}} & \equiv \quad \text{by} \; \ncEquivPoolAss2 \\
        \tm{\ncPool{R}{\ncPool{\cpPlug{G}{P}}{Q}}} & \equiv \quad \text{by} \; \ncEquivPoolComm \\
        \tm{\ncPool{\ncPool{\cpPlug{G}{P}}{Q}}{R}} & \equiv \quad \text{by the induction hypothesis} \\
        \tm{\ncPool{\cpPlug{G}{\ncPool{P}{Q}}}{R}} &
      \end{array}
    \]
  \item
    Case $\tm{\ncPool{R}{G}}$.
    \[\!
      \begin{array}{ll}
        \tm{\ncPool{\ncPool{R}{\cpPlug{G}{P}}}{Q}} & \equiv \quad \text{by} \; \ncEquivPoolAss2 \\
        \tm{\ncPool{R}{\ncPool{\cpPlug{G}{P}}{Q}}} & \equiv \quad \text{by the induction hypothesis} \\
        \tm{\ncPool{R}{\cpPlug{G}{\ncPool{P}{Q}}}} &
      \end{array}
    \]
  \end{itemize}
  In each case, the side conditions for \ncRedKappaPool2, $\notFreeIn{y}{Q}$,
  can be inferred from the fact that
  $\tm{\ncPool{Q}{\cpCut{y}{\cpPlug{G}{P}}{R}}}$ is well-typed, the side
  conditions for \cpEquivCutAss2, $\notFreeIn{x}{R}$ and $\notFreeIn{y}{Q}$, can
  be inferred from $\notFreeIn{x}{G}$ and the fact that
  $\tm{\ncPool{\cpPlug{G}{P}}{Q}}$ is well-typed, and the side conditions for
  the induction hypothesis can be inferred from the fact that
  $\tm{\ncPool{\cpPlug{G}{P}}{Q}}$ is well-typed,
  \cref{thm:nc-preservation-equiv} and $\notFreeIn{x}{G}$. 
\end{proof}
%%% Local Variables:
%%% TeX-master: "main"
%%% End:

\subsection{Progress}
Progress is the fact that every term is either in some canonical form, or can be
reduced further.
First, we will restate \cref{thm:cp-progress-link} and \cref{thm:cp-progress-beta},
which relate evaluation prefixes and evaluation contexts, and show that our
extension preserves these properties. 
\begin{lemmaB}\label{thm:nc-progress-link}
  If $\seq[{ \cpPlug{G}{P_1 \dots P_n} }]{ \Gamma }$, and some \tm{P_i} is a
  link \tm{\cpLink{x}{y}}, then either $\tm{\cpPlug{G}{P_1 \dots P_n}} =
  \tm{\cpLink{x}{y}}$, or there exist \tm{H}, \tm{H'} and \tm{Q} such that
  \(
  \tm{\cpPlug{G}{P_1 \dots P_n}} \equiv
  \tm{\cpPlug{H}{\cpCut{z}{\cpPlug{H'}{\cpLink{x}{y}}}{Q}}}
  \)
  with either $\tm{z} = \tm{x}$ or $\tm{z} = \tm{y}$.
\end{lemmaB}
\begin{proof}
  \textcolor{red}{\bfseries TODO}
\end{proof}
%%% Local Variables:
%%% TeX-master: "main"
%%% End:
\begin{lemmaB}\label{thm:nc-progress-beta}
  If $\seq[{ \cpPlug{G}{P_1 \dots P_n} }]{ \Gamma }$, and some \tm{P_i} and
  \tm{P_j} (with $i \neq j$) act on the same channel \tm{x}, then there exist \tm{H},
  \tm{H_i} and \tm{H_j} such that 
  \(
  \tm{\cpPlug{G}{P_1 \dots P_n}} =
  \tm{\cpPlug{H}{\cpCut{x}{\cpPlug{H_i}{P_i}}{\cpPlug{H_j}{P_j}}}}
  \).
\end{lemmaB}
\begin{proof}
  \textcolor{red}{\bfseries TODO}
\end{proof}
%%% Local Variables:
%%% TeX-master: "main"
%%% End:

In essence, \cref{thm:nc-progress-link} and \cref{thm:nc-progress-beta} cover
the cases in which either \cpRedAxCut1 or a \textbeta-reduction rule will be
applied.
However, after applying \cref{thm:nc-progress-beta}, we cannot immediately apply
\ncRedBetaStar{n+1}. For that, we must uncover at least one layer of pooling.
We prove a lemma which states that if we have an interaction on a shared channel
\tm{x}, we can push all pooling rules which pool clients communicating on \tm{x}
inwards. 
\begin{lemmaB}\label{thm:nc-progress-shared}
  If $\seq[{ \cpPlug{G}{P} }]{ \Gamma, \tmty{x}{\take[n]{A}} }$ and
  $\freeIn{x}{P}$, then there exists an $\tm{H}$ and $\tm{R_1}\dots\tm{R_{n-1}}$
  such that $\tm{\cpPlug{G}{P}} \equiv
  \tm{\cpPlug{H}{\ncPool{P}{\ncPool{R_1}{\ncPool{\dots}{R_{n-1}}\dots}}}}$,
  where $\notFreeIn{x}{H}$ and $\freeIn{x}{R_1},\dots,\freeIn{x}{R_{n-1}}$.
\end{lemmaB}
\begin{proof}
  By induction on the structure of the evaluation context \tm{G}.
  \begin{itemize}
  \item
    Case \tm{\Box}. By reflexivity.
  \item
    Case \tm{\cpCut{y}{G}{R}}. \\
    Case \tm{\cpCut{y}{R}{G}}.
    \\
    By the induction hypothesis.
  \item
    Case \tm{\ncPool{G}{R}}. There are two cases:
    \begin{itemize}
    \item Case $\freeIn{x}{R}$.
      \begin{flalign*}
        \begin{array}{l}
          \tm{\ncPool{\cpPlug{G}{P}}{R_{n-1}}} \\
          \qquad \equiv \quad \text{by the induction hypothesis} \\ 
          \tm{\ncPool{\cpPlug{H}{
          \ncPool{P}{\ncPool{R_1}{
          \ncPool{\dots}{R_{n-2}}\dots}}}}{R_{n-1}}} \\
          \qquad \equiv \quad \text{by \cref{thm:nc-display-pool-1}} \\
          \tm{\cpPlug{H}{\ncPool{P}{
          \ncPool{R_1}{\ncPool{
          \dots}{\ncPool{R_{n-2}}{R_{n-1}}}\dots}}}}
        \end{array}
      \end{flalign*}
    \item Case $\notFreeIn{x}{R}$. By the induction hypothesis.
    \end{itemize}
  \item Case \tm{\ncPool{R}{G}}. There are two cases:
    \begin{itemize}
    \item Case $\freeIn{x}{R}$.
      \begin{flalign*}
        \begin{array}{ll}
          \tm{\ncPool{R_{n-1}}{\cpPlug{G}{P}}} \\
          \qquad \equiv \quad \text{by} \; \ncEquivPoolComm \\ 
          \tm{\ncPool{\cpPlug{G}{P}}{R_{n-1}}} \\
          \qquad \equiv \quad \text{by the induction hypothesis} \\ 
          \tm{\ncPool{\cpPlug{H}{
          \ncPool{P}{\ncPool{R_1}{
          \ncPool{\dots}{R_{n-2}}\dots}}}}{R_{n-1}}} \\
          \qquad \equiv \quad \text{by \cref{thm:nc-display-pool-1}} \\
          \tm{\cpPlug{H}{\ncPool{P}{
          \ncPool{R_1}{\ncPool{
          \dots}{\ncPool{R_{n-2}}{R_{n-1}}}\dots}}}}
        \end{array}
      \end{flalign*}
    \item Case $\notFreeIn{x}{R}$. By the induction hypothesis.
    \end{itemize}
  \end{itemize}
\end{proof}
%%% Local Variables:
%%% TeX-master: "main"
%%% End:

Finally, we are ready to extend our proof of progress. The overall structure of
the proof remains the same, though the addition of pooling makes the wording
slightly more subtle.
\begin{theorem}[Progress]\label{thm:nc-progress}
  If $\seq[{ P }]{ \Gamma }$, then $\tm{P}$ is in canonical form, or there
  exists a $\tm{P'}$ s.t.\ $\reducesto{P}{P'}$.
\end{theorem}
\begin{proof}
  By induction on the structure of derivation for $\seq[{ P }]{ \Gamma }$.
  There only interesting cases are when the last rule of the derivation is
  \textsc{Cut} or \textsc{Pool}. In every other case, the typing rule constructs
  a term in which is in canonical form. 
  \\
  If the last rule in the derivation is \textsc{Cut} or \textsc{Pool}, we
  consider the prefix of the derivation for $\seq[{ P }]{ \Gamma}$ which
  consists of all top-level cuts and pooling rules. A prefix of $n$ cuts and $m$
  pooling rules introduces $n$ variables, but composes $n+m+1$ actions, at most
  $m+1$ of which are on the same side of all cut rules.
  Therefore, one of the following must be true:
  \begin{itemize}
  \item
    One of these actions was introduced by an application of \textsc{Ax}.
    \\
    We proceed as in \cref{thm:cp-progress}. 
  \item
    Two of these actions, on different sides of a \textsc{Cut}, act on the same
    channel. Let us name these processes \tm{P_i} and \tm{P_j}, and their shared
    channel \tm{y}. We have
    $\tm{P} = \tm{\cpPlug{G}{\cpCut{y}{\cpPlug{H_i}{P_i}}{\cpPlug{H_j}{P_j}}}}$.
    We distinguish the following cases:
    \begin{itemize}
    \item
      We have either
      $\seq[{ \cpPlug{H_i}{P_i}}]{ \Delta, \tmty{y}{\take[n]{A}} }$ or
      $\seq[{ \cpPlug{H_j}{P_j} }]{\Delta, \tmty{y}{\take[n]{A}} }$.
      \\
      We rewrite by \cref{thm:nc-display-3}, then apply one of \ncRedBetaStar{1}
      and \ncRedBetaStar{n+1}. 
    \item
      Otherwise, we can infer $\notFreeIn{y}{H_i}$ and $\notFreeIn{y}{H_j}$.
      \\
      We proceed as in \cref{thm:cp-progress}. 
    \end{itemize}
  \item
    The process is in canonical form \tm{
      \ncPool{\ncCnt{x_1}{y_1}{P_1}}{
        \ncPool{\dots}{\ncCnt{x_n}{y_n}{P_n}}\dots}}. 
  \item 
    Otherwise (at least) one of the actions acts on a free variable.
    \\
    We apply one of the commutative conversions.
  \end{itemize}
\end{proof}
%%% Local Variables:
%%% TeX-master: "main"
%%% End:


\subsection{Termination}
Termination is the fact that if we iteratively apply progress to obtain a
reduction, and apply that reduction, we will eventually end up with a term in
canonical form.
We restate its proof here for the sake of completeness, but its wording is
unchanged, module references to figures.
\begin{theorem}[Termination]\label{thm:nc-termination}
  If $\seq[{ P }]{ \Gamma }$, then there are no infinite $\Longrightarrow$
  reduction sequences.
\end{theorem}
\begin{proof}
  Every reduction reduces a single cut to zero, one or two cuts.
  However, each of these cuts is \emph{smaller}, in the sense that the type of
  the channel on which the communication takes place is smaller.
  Each reduction either eliminates a connective, or decreases a resource index
  on the type of a shared channel.
  See \cref{fig:cp-preservation-1,fig:nc-preservation-1}.
  Furthermore, each instance of the structural congruence preserves the size
  of the cut---see \cref{fig:cp-preservation-equiv,fig:nc-preservation-equiv}.
  Therefore, there cannot be an infinite $\Longrightarrow$ reduction sequence.
\end{proof}
%%% Local Variables:
%%% TeX-master: "main"
%%% End:

\section{\nodcap and non-deterministic local choice}
In \cref{sec:local-choice}, we discussed the non-deterministic local choice
operator, which is used in several extensions of \piDILL and
\cp~\cite{atkey2016,caires2014,caires2017}.
This operator is admissible in \nodcap.
We can derive the non-deterministic choice \tm{P+Q} by constructing the
following term, provided that \tm{P} and \tm{Q} only communicate over a single
channel: 
\[
  \arraycolsep=0pt
  \tm{
  \begin{array}{lrl}
    \nu x.&((  & \; \ncCnt{x}{y}{\cpCase{y}{P}{\cpHalt{y}}} \\
          &\mid& \; \ncCnt{x}{z}{\cpCase{z}{Q}{\cpHalt{z}}} \; )\\
          &\mid& \; \ncSrv{x}{y}{\ncSrv{x}{z}{\cpInl{y}{\cpWait{y}{\cpInr{z}{\cpLink{z}{w}}}}}} \; )
  \end{array}
  }
\]
The term is a cut between two processes. Let us unpack what each side is doing. 
On the left-hand side, we have a pool with both \tm{P} and \tm{Q} as clients,
wrapped in a case statement which allows the process they are interacting with
to choose whether or not they want them to run. Each of these processes is
constructed as follows: 
\begin{prooftree}
  \AXC{$\seq[{ P }]{ \tmty{y}{A} }$}
  \AXC{}
  \SYM{(\one)}
  \UIC{$\seq[{ \cpHalt{y} }]{ \tmty{y}{\one} }$}
  \SYM{(\with)}
  \BIC{$\seq[{ \cpCase{y}{P}{\cpHalt{y}} }]{ \tmty{y}{A \with \one} }$}
  \SYM{(\take[1]{})}
  \UIC{$\seq[{ \ncCnt{x}{y}{\cpCase{y}{P}{\cpHalt{y}}} }]{
      \tmty{y}{\take[1]{A \with \one}} }$}
\end{prooftree}
On the right-hand side, we have a server which connects to two clients of the
above form, and chooses to terminate one, while allowing the other to run.
This process is constructed as follows:
\begin{prooftree}
  \AXC{}
  \NOM{Ax}
  \UIC{$\seq[{ \cpLink{z}{w} }]{
      \tmty{z}{A^\bot}, \tmty{w}{A} }$}
  \SYM{(\plus_2)}
  \UIC{$\seq[{ \cpInr{z}{\cpLink{z}{w}} }]{
      \tmty{z}{A^\bot \plus \bot}, \tmty{w}{A} }$}
  \SYM{(\bot)}
  \UIC{$\seq[{ \cpWait{y}{\cpInr{z}{\cpLink{z}{w}}} }]{
      \tmty{y}{\bot}, \tmty{z}{A^\bot \plus \bot}, \tmty{w}{A}}$}
  \SYM{(\plus_1)}
  \UIC{$\seq[{ \cpInl{y}{\cpWait{y}{\cpInr{z}{\cpLink{z}{w}}}} }]{
      \tmty{y}{A^\bot \plus \bot}, \tmty{z}{A^\bot \plus \bot}, \tmty{w}{A}}$}
  \SYM{(\give[1]{})}
  \UIC{$\seq[{ \ncSrv{x'}{z}{\cpInl{y}{\cpWait{y}{\cpInr{z}{\cpLink{z}{w}}}}} }]{
      \tmty{y}{(A^\bot \plus \bot)}, \tmty{x'}{\give[1]{(A^\bot \plus \bot)}}, \tmty{w}{A} }$}
  \SYM{(\give[1]{})}
  \UIC{$\seq[{ \ncSrv{x}{y}{\ncSrv{x'}{z}{\cpInl{y}{\cpWait{y}{\cpInr{z}{\cpLink{z}{w}}}}}} }]{
      \tmty{x}{\give[1]{(A^\bot \plus \bot)}}, \tmty{x'}{\give[1]{(A^\bot \plus \bot)}}, \tmty{w}{A} }$}
  \NOM{Cont}
  \UIC{$\seq[{ \ncSrv{x}{y}{\ncSrv{x}{z}{\cpInl{y}{\cpWait{y}{\cpInr{z}{\cpLink{z}{w}}}}}} }]{
      \tmty{x}{\give[2]{(A^\bot \plus \bot)}, \tmty{w}{A}}}$}
\end{prooftree}
When we compose these two processes, we get a process of the type
$\seq[{ P + Q}]{ \tmty{w}{A} }$, using \tm{P+Q} as an abbreviation for the
encoding of non-deterministic local choice.

The full version of local choice, which allows you to choose between two
processes with an arbitrary number of channels, is admissible.
We can construct it using the invertibility of $(\parr)$, the logical property
that we can turn any context $\tmty{x_1}{A_1}\dots\tmty{x_n}{A_n}$ into a single
formula $\tmty{x}{A_1 \parr \dots \parr A_n}$, and vice versa.
In terms of the process calculus, this means that for any number of channels
there exist processes which can systematically pack or unpack those channels.
\begin{lemmaB}
  If $\seq[{ P }]{ \Gamma, \tmty{x_1}{A_1}\dots\tmty{x_n}{A_n} }$, then
  $\seq[{ \cpRecv{x_1}{x_n}{\dots\cpRecv{x_1}{x_2}{P}} }]{
    \Gamma, \tmty{x_1}{A_1 \parr \dots \parr A_n} }$.
\end{lemmaB}
\begin{proof}
  By repeated application of $(\parr)$.
\end{proof}

\begin{lemmaB}
  If $\seq[{ \cpRecv{x_1}{x_n}{\dots\cpRecv{x_1}{x_2}{P}} }]{
    \Gamma, \tmty{x_1}{A_1 \parr \dots \parr A_n} }$, then there exists a term
  \tm{P'} such that
  $\seq[{ P' }]{ \Gamma, \tmty{x_1}{A_1}\dots\tmty{x_n}{A_n}}$
  and $\tm{P'} \Longrightarrow^{3(n-1)} \tm{P}$.
\end{lemmaB}
\begin{proof}
  By induction on $n$.
  \begin{itemize}
  \item
    Case $n = 1$.
    We have $\seq[{ P }]{ \Gamma, \tmty{x_1}{A_1} }$.
    This is our desired result.
  \item
    Case $n = k + 1$. We have:
    \[
      \seq[{
      \cpRecv{x_1}{x_n}{\cpRecv{x_1}{x_k}{\dots\cpRecv{x_1}{x_2}{P}}} }]{
      \Gamma, \tmty{x_1}{A_1 \parr \dots \parr A_k \parr A_n} }.
    \]
    We construct the process:

    \[
      \tm{\cpCut{x_1}{
          \cpRecv{x_1}{x_n}{\cpRecv{x_1}{x_k}{\dots\cpRecv{x_1}{x_2}{P}}}}{
          \cpSend{x_1}{x_n}{\cpLink{x_n}{x_n'}}{\cpLink{x_1}{x_1'}}
        }}.
    \]
    In three steps, this reduces to
    \(
      \tm{\cpRecv{x_1'}{x_k}{\dots\cpRecv{x_1'}{x_2}{\cpSub{x_1'}{x_1}{\cpSub{x_n'}{x_n}{P}}}}}.
    \)
    We then substitute \tm{x_1} for \tm{x_1'} and \tm{x_n} for \tm{x_n'}, and
    apply the induction hypothesis.
  \end{itemize}
\end{proof}
While there is a certain amount of overhead involved in this encoding, it scales
linearly in terms of the number of processes.
The reverse---encoding the non-determinism present in \nodcap using
non-deterministic local choice---scales exponentially, as with the \textpi-calculus.


\wen{Add a discussion of non-determinism in \nodcap versus non-determinism in
  the \textpi-calculus. Mention at least that terms which have clients who are
  and who aren't blocked on an external communication will prefer those clients
  who aren't blocked when non-deterministically reducing.}
%%% Local Variables:
%%% TeX-master: "main"
%%% End:

\chapter{Discussion}\label{sec:discussion}

\wen{
  What to discuss? I've done the things I've promised. I can see the following
  valuable points:
  \begin{itemize}
  \item
    It would be worthwhile to decouple the name restriction from the parallel
    composition in \cp, as this would greatly simplify our reduction system.
  \item
    It would be worthwhile to investigate if, for instance, we could define a
    cut on a client/server interaction such that it would naturally leave
    leftovers. This would mash well with client and server pools of arbitrary
    size.
  \end{itemize}
}

\section{Relation to bounded linear logic}
We mentioned in \cref{sec:introduction} that \nodcap was inspired by
bounded linear logic~(BLL)~\cite{girard1992}. BLL is a typed lambda calculus
based on intuitionistic linear logic which guarantees that its programs are
polynomial-time functions.
It too uses resource-indexed exponentials. However, instead of interpreting
these as client and server interactions, BLL interprets them as accesses to a
memory cell, as is a common interpretation in linear logic~\cite{girard1987}.
There are some superficial differences between BLL and \nodcap, e.g.\ the former
is intuitionistic while the latter is classical, but the main difference between
the two lies in storage versus pooling. In BLL, \ty{\take[n]{A}} denotes a memory
cell which can be accessed $n$ times, whereas in \nodcap, \ty{\take[n]{A}}
represents a pool of $n$ different values, computed independently by $n$
different processes.

\section{Recursion and resource variables}
Our formalism so far has only captured servers that provide for a fixed number
of clients.  More realistically, we would want to define servers that provide
for arbitrary numbers of clients.  This poses two problems: how would we define
arbitrarily-interacting stateful processes, and how would we extend the
typing discipline of \nodcap to account for them without losing its static
guarantees.

One approach to defining server processes would be to combine \nodcap with
structural recursion and corecursion, following the $\mu\text{CP}$ extension of Lindley
and Morris~\cite{lindley2016}.  Their approach can express processes which
produce streams of \ty{A} channels; such a process would expose a channel with the
corecursive type \ty{\nu X. A \parr (1 \plus X)}.  Given such a process, it is
possible to produce a channel of type \ty{A \parr A \parr \cdots \parr A} for any
number of $A$s, allowing us to satisfy the type \ty{\give[n]{A}} for an arbitrary
$n$.

We would also need to extend the typing discipline to capture arbitrary use of
shared channels.  One approach would be to introduce resource variables and
quantification.  Following this approach, in addition to having types $\give[n]
A$ and $\take[n] A$ for concrete $n$, we would also have types $\give[x] A$ and
$\take[x] A$ for resource variables $x$.  These variables would be introduced by
quantifiers $\forall x A$ and $\exists x A$.  Defining terms
corresponding to $\forall x A$, and its relationship with structured recursion,
presents an interesting area of further work.

\wen{Update to reflect how I currently think about this.}

\section{Relation to exponentials in CP}
Our account of CP has not included the exponentials \ty{\give A} and \ty{\take A}.
The type \ty{\take A} denotes arbitrarily many independent instances of \ty{A}, while
the type \ty{\give A} denotes a concrete (if unspecified) number of
potentially-dependent instances of \ty{A}.  Existing interpretations of linear
logic as session types have taken \ty{\take A} to denote \ty{A}-servers, while
\ty{\give A} denotes \ty{A}-clients.  However, the analogy is imperfect: while we
expect servers to provide arbitrarily many instances of their behavior, we also
expect those instances to be interdependent.

With quantification over resource variables, we can give precise accounts
of both \cp's exponentials and idealised servers and clients.
\cp exponentials could be embedded into this framework using the definitions
$\ty{\take{A}} := \ty{\forall{n}\take[n]{A}}$ and $\ty{\give{A}} :=
\ty{\exists{n}{\give[n]{A}}}$.
We would also have types that precisely matched our intuitions for server and
client behavior: an \ty{A} server is of type \ty{\forall{n}{\give[n] A}}, being
unbounded but dependent, while a collection of \ty{A} clients is of type
\ty{\exists{n}{\take[n] A}}, being definitely sized by independent.

%%% Local Variables:
%%% TeX-master: "main"
%%% End:


%% Bibliography
\printbibliography
\end{document}

%%% Local Variables:
%%% TeX-master: "main"
%%% End:
