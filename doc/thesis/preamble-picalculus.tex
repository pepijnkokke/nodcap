% \usepackage[fleqn]{amsmath}
% \usepackage{textgreek}
% \usepackage{xspace}
% \xspaceaddexceptions{]\}}

%%% The π-calculus term syntax:
\newcommand{\piCalc}[0]{\textpi-calculus\xspace}
\newcommand{\piSend}[3]{\ensuremath{\overline{#1}\langle #2 \rangle.#3}}
\newcommand{\piRecv}[3]{\ensuremath{#1(#2).#3}}
\newcommand{\piPar}[2]{\ensuremath{#1 \mid #2}}
\newcommand{\piNew}[2]{\ensuremath{(\nu #1)#2}}
\newcommand{\piRepl}[1]{\ensuremath{!#1}}
\newcommand{\piHalt}[0]{\ensuremath{0}}
\newcommand{\piSub}[3]{\ensuremath{#3\{#1/#2\}}}

%%% The CP term syntax:
\newcommand{\cp}{CP\xspace}
\newcommand{\cpLink}[2]{\ensuremath{#1{\leftrightarrow}#2}}
\newcommand{\cpCut}[3]{\ensuremath{\nu #1.(\piPar{#2}{#3})}}
\newcommand{\cpSend}[4]{\ensuremath{#1[#2].(\piPar{#3}{#4})}}
\newcommand{\cpRecv}[3]{\ensuremath{\piRecv{#1}{#2}{#3}}}
\newcommand{\cpWait}[2]{\ensuremath{#1().#2}}
\newcommand{\cpHalt}[1]{\ensuremath{#1[].0}}
\newcommand{\cpInl}[2]{\ensuremath{#1[\text{inl}].#2}}
\newcommand{\cpInr}[2]{\ensuremath{#1[\text{inr}].#2}}
\newcommand{\cpCase}[3]{\ensuremath{\text{case}\;#1\;\{#2;#3\}}}
\newcommand{\cpAbsurd}[1]{\ensuremath{\text{case}\;#1\;\{\}}}
\newcommand{\cpSub}[3]{\ensuremath{\piSub{#1}{#2}{#3}}}

%%% The nodcap term syntax:
\newcommand{\nc}{\ensuremath{\text{CP}_{\text{ND}}}\xspace}
\newcommand{\ncExpn}[3]{\ensuremath{#1\uparrow #2.#3}}
\newcommand{\ncIntl}[3]{\ensuremath{#1\downarrow #2.#3}}
\newcommand{\ncSrv}[3]{\ensuremath{{\star}{#1}(#2).#3}}
\newcommand{\ncCnt}[3]{\ensuremath{{\star}{#1}[#2].#3}}
\newcommand{\ncPool}[2]{\ensuremath{(\piPar{#1}{#2})}}
\newcommand{\ncCont}[3]{\ensuremath{\cpSub{#1}{#2}{#3}}}
%%% Local Variables:
%%% TeX-master: "main"
%%% End:
