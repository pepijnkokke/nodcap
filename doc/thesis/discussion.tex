\chapter{Discussion}\label{sec:discussion}

\wen{
  What to discuss? I've done the things I've promised. I can see the following
  valuable points:
  \begin{itemize}
  \item
    It would be worthwhile to decouple the name restriction from the parallel
    composition in \cp, as this would greatly simplify our reduction system.
  \item
    It would be worthwhile to investigate if, for instance, we could define a
    cut on a client/server interaction such that it would naturally leave
    leftovers. This would mash well with client and server pools of arbitrary
    size.
  \end{itemize}
}

\section{Relation to bounded linear logic}
We mentioned in \cref{sec:introduction} that \nodcap was inspired by
bounded linear logic~(BLL)~\cite{girard1992}. BLL is a typed lambda calculus
based on intuitionistic linear logic which guarantees that its programs are
polynomial-time functions.
It too uses resource-indexed exponentials. However, instead of interpreting
these as client and server interactions, BLL interprets them as accesses to a
memory cell, as is a common interpretation in linear logic~\cite{girard1987}.
There are some superficial differences between BLL and \nodcap, e.g.\ the former
is intuitionistic while the latter is classical, but the main difference between
the two lies in storage versus pooling. In BLL, \ty{\take[n]{A}} denotes a memory
cell which can be accessed $n$ times, whereas in \nodcap, \ty{\take[n]{A}}
represents a pool of $n$ different values, computed independently by $n$
different processes.

\section{Recursion and resource variables}
Our formalism so far has only captured servers that provide for a fixed number
of clients.  More realistically, we would want to define servers that provide
for arbitrary numbers of clients.  This poses two problems: how would we define
arbitrarily-interacting stateful processes, and how would we extend the
typing discipline of \nodcap to account for them without losing its static
guarantees.

One approach to defining server processes would be to combine \nodcap with
structural recursion and corecursion, following the $\mu\text{CP}$ extension of Lindley
and Morris~\cite{lindley2016}.  Their approach can express processes which
produce streams of \ty{A} channels; such a process would expose a channel with the
corecursive type \ty{\nu X. A \parr (1 \plus X)}.  Given such a process, it is
possible to produce a channel of type \ty{A \parr A \parr \cdots \parr A} for any
number of $A$s, allowing us to satisfy the type \ty{\give[n]{A}} for an arbitrary
$n$.

We would also need to extend the typing discipline to capture arbitrary use of
shared channels.  One approach would be to introduce resource variables and
quantification.  Following this approach, in addition to having types $\give[n]
A$ and $\take[n] A$ for concrete $n$, we would also have types $\give[x] A$ and
$\take[x] A$ for resource variables $x$.  These variables would be introduced by
quantifiers $\forall x A$ and $\exists x A$.  Defining terms
corresponding to $\forall x A$, and its relationship with structured recursion,
presents an interesting area of further work.

\wen{Update to reflect how I currently think about this.}

\section{Relation to exponentials in CP}
Our account of CP has not included the exponentials \ty{\give A} and \ty{\take A}.
The type \ty{\take A} denotes arbitrarily many independent instances of \ty{A}, while
the type \ty{\give A} denotes a concrete (if unspecified) number of
potentially-dependent instances of \ty{A}.  Existing interpretations of linear
logic as session types have taken \ty{\take A} to denote \ty{A}-servers, while
\ty{\give A} denotes \ty{A}-clients.  However, the analogy is imperfect: while we
expect servers to provide arbitrarily many instances of their behavior, we also
expect those instances to be interdependent.

With quantification over resource variables, we can give precise accounts
of both \cp's exponentials and idealised servers and clients.
\cp exponentials could be embedded into this framework using the definitions
$\ty{\take{A}} := \ty{\forall{n}\take[n]{A}}$ and $\ty{\give{A}} :=
\ty{\exists{n}{\give[n]{A}}}$.
We would also have types that precisely matched our intuitions for server and
client behavior: an \ty{A} server is of type \ty{\forall{n}{\give[n] A}}, being
unbounded but dependent, while a collection of \ty{A} clients is of type
\ty{\exists{n}{\take[n] A}}, being definitely sized by independent.

%%% Local Variables:
%%% TeX-master: "main"
%%% End:
