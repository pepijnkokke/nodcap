\chapter{Conclusions and future work}\label{sec:discussion}
We have presented \nodcap, an extension of \cp~\parencite{wadler2012} which permits
non-deterministic communication. 
We have given proofs for preservation, progress, and termination for the term
reduction system of \nodcap.
We have shown that we can define non-deterministic local choice in \nodcap.

We have also presented an alternative reduction system for \cp, based on the
work by \textcite{lindley2015semantics}, which more closely resembles reduction
in the \textpi-calculus.

\section{Mechanisation of \nodcap}
We have mechanised a variant of \nodcap, which has the same terms and types, but
has a different reduction system for the non-deterministic terms.
While the existence of this mechanisation gives us some confidence in the
correctness of the proofs given in this dissertation, and shows that the type
system of \nodcap corresponds to a sound logical system, it would be worthwhile
to extend this mechanisation to cover the work described in this dissertation.

\section{Relation to bounded linear logic}
We mentioned in \cref{sec:introduction} that \nodcap was inspired by
bounded linear logic~(BLL)~\parencite{girard1992}. BLL is a typed lambda calculus
based on intuitionistic linear logic which guarantees that its programs are
polynomial-time functions.
It too uses resource-indexed exponentials. However, instead of interpreting
these as client and server interactions, BLL interprets them as accesses to a
memory cell, as is a common interpretation in linear logic~\parencite{girard1987}.
There are some superficial differences between BLL and \nodcap, e.g.\ the former
is intuitionistic while the latter is classical, but the main difference between
the two lies in storage versus pooling. In BLL, \ty{\take[n]{A}} denotes a memory
cell which can be accessed $n$ times, whereas in \nodcap, \ty{\take[n]{A}}
represents a pool of $n$ different values, computed independently by $n$
different processes.

\section{Name restriction and parallel composition}
It would be worthwhile to decouple the name restriction from the parallel
composition in \cp, as this would greatly simplify our reduction system.
We could do this, for instance, using a two-layered environment, which tracks
which sets of channels are interdependent.
\begin{center}
  \[
    \ty{X}, \ty{Y}, \ty{Z} := \ty{\Gamma_1} ; \dots ; \ty{\Gamma_n}
  \]
  \vspace*{1\baselineskip}
  \begin{prooftree*}
    \AXC{$\seq[{ P }]{ X }$}
    \AXC{$\seq[{ Q }]{ Y }$}
    \BIC{$\seq[{ \piPar{P}{Q} }]{ X ; Y }$}
  \end{prooftree*}
  \begin{prooftree*}
    \AXC{$\seq[{ P }]{ \tmty{x}{A} , \Gamma ; \tmty{x}{A^\bot}, \Delta ; X }$}
    \UIC{$\seq[{ \piNew{x}{P} }]{ \Gamma, \Delta ; X }$}
  \end{prooftree*}
\end{center}
This decoupling would allow us to reduce the number of reduction rules for
servers and clients, and strengthen our correspondence to the \textpi-calculus.

\section{Recursion and resource variables}
Our formalism so far has only captured servers that provide for a fixed number
of clients.  More realistically, we would want to define servers that provide
for arbitrary numbers of clients.  This poses two problems: how would we define
arbitrarily-interacting stateful processes, and how would we extend the
typing discipline of \nodcap to account for them without losing its static
guarantees.

One approach to defining server processes would be to combine \nodcap with
structural recursion and corecursion, following the $\mu\text{CP}$ extension of Lindley
and Morris~\parencite{lindley2016}.  Their approach can express processes which
produce streams of \ty{A} channels. Such a process would expose a channel with the
corecursive type \ty{\nu X. A \parr (1 \plus X)}.  Given such a process, it is
possible to produce a channel of type \ty{A \parr A \parr \cdots \parr A} for any
number of \ty{A}s, allowing us to satisfy the type \ty{\give[n]{A}} for an arbitrary
$n$.

We would also need to extend the typing discipline to capture arbitrary use of
shared channels.  One approach would be to introduce resource variables and
quantification.  Following this approach, in addition to having types \ty{\give[n]
A} and \ty{\take[n] A} for concrete $n$, we would also have types \ty{\give[x] A} and
\ty{\take[x] A} for resource variables $x$.  These variables would be introduced by
quantifiers \ty{\forall x A} and \ty{\exists x A}.  Defining terms
corresponding to \ty{\forall x A}, and its relationship with structured recursion,
presents an interesting area of further work.

\section{Cuts with leftovers}
So far, our account of non-determinism in client/server interactions only allows
interactions between equal numbers of clients and server interactions.
A natural extension of this would be to investigate if we could define a special
case of cut on a client/server interaction, such that e.g.\ the clients only
consume part of the server resources.
\begin{scprooftree}
  \AXC{$\seq[{ P }]{ \Gamma, \tmty{x}{\take[n]{A}} }$}
  \AXC{$\seq[{ Q }]{ \Delta, \tmty{x}{\give[m]{A^\bot}} }$}
  \AXC{$n < m$}
  \TIC{$\seq[{ \cpCut{x}{P}{Q} }]{ \Gamma, \Delta, \tmty{x}{\give[m-n]{A^\bot}} }$}
\end{scprooftree}
Such an extension would work well together with an extension allowing clients
and servers to provide an arbitrary number of interactions.

\section{Relation to exponentials in CP}
Our account of CP has not included the exponentials \ty{\give A} and \ty{\take A}.
The type \ty{\take A} denotes arbitrarily many independent instances of \ty{A}, while
the type \ty{\give A} denotes a concrete (if unspecified) number of
potentially-dependent instances of \ty{A}.  Existing interpretations of linear
logic as session types have taken \ty{\take A} to denote \ty{A}-servers, while
\ty{\give A} denotes \ty{A}-clients.  However, the analogy is imperfect: while we
expect servers to provide arbitrarily many instances of their behavior, we also
expect those instances to be interdependent.

With quantification over resource variables, we can give precise accounts
of both \cp's exponentials and idealised servers and clients.
\cp exponentials could be embedded into this framework using the definitions
$\ty{\take{A}} := \ty{\forall{n}\take[n]{A}}$ and $\ty{\give{A}} :=
\ty{\exists{n}{\give[n]{A}}}$.
We would also have types that precisely matched our intuitions for server and
client behavior: an \ty{A} server is of type \ty{\forall{n}{\give[n] A}}, being
unbounded but dependent, while a collection of \ty{A} clients is of type
\ty{\exists{n}{\take[n] A}}, being definitely sized by independent.

%%% Local Variables:
%%% TeX-master: "main"
%%% End:
