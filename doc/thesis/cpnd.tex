\chapter{Non-deterministic Classical Processes}\label{sec:main}
In this section, we will discuss our main contribution: an extension of \cp
which allows for races while still excluding deadlocks. 
We have seen in \cref{sec:cp-example} how \cp excludes deadlocks, but how exactly
does \cp exclude races?
Let us return to our first example from \cref{sec:introduction}, to the
interaction between John, Mary and the store.
\[
  \begin{array}{c}
    \tm{(\piPar{%
    \piSend{x}{\sliceofcake}{\piSend{x}{\nope}{\store}}
    }{%
    \piPar{\piRecv{x}{y}{\john}}{\piRecv{x}{z}{\mary}}
    })}
    \\[1ex]
    \rotatebox[origin=c]{270}{$\Longrightarrow^{\star}$}
    \\[1ex]
    \tm{(\piPar{\store}{\piPar{\piSub{\sliceofcake}{y}{\john}}{\piSub{\nope}{z}{\mary}}})}
    \quad
    \text{or}
    \quad
    \tm{(\piPar{\store}{\piPar{\piSub{\nope}{y}{\john}}{\piSub{\sliceofcake}{z}{\mary}}})}
  \end{array}
\]
Races occur when more than two processes attempt to communicate simultaneously
over the \emph{same} channel. However, the \text{Cut} rule of \cp requires that
\emph{exactly two} processes communicate over each channel:
\begin{center}
  \cpInfCut
\end{center}
We could attempt write down a protocol for our example, stating that the store
has a pair of channels $\tm{x}, \tm{y} : \ty{\cake}$ with which it communicates
with John and Mary, taking \cake to be the type of interactions in which cake
\emph{may} be obtained, i.e.\ of both \sliceofcake and \nope, and state that the
store communicates with John \emph{and} Mary over a channel of type \ty{\cake
  \parr \cake}.
However, this \emph{only} models interactions such as the following:
\begin{prooftree}
  \AXC{$\seq[{ \john }]{ \Gamma, \tmty{x}{\cake^\bot} }$}
  \AXC{$\seq[{ \mary }]{ \Delta, \tmty{y}{\cake^\bot} }$}
  \SYM{\tens}
  \BIC{$\seq[{ \cpSend{y}{x}{\john}{\mary} }]{
      \Gamma, \Delta, \tmty{y}{\cake^\bot \tens \cake^\bot} }$}
  \AXC{$\seq[{ \store }]{ \Theta, \tmty{x}{\cake}, \tmty{y}{\cake} }$}
  \SYM{\parr}
  \UIC{$\seq[{ \cpRecv{y}{x}{\store} }]{
      \Theta, \tmty{y}{\cake \parr \cake} }$}
  \NOM{Cut}
  \BIC{$\seq[{ \cpCut{y}{\cpSend{y}{x}{\john}{\mary}}{\cpRecv{y}{x}{\store}} }]{
      \Gamma, \Delta, \Theta }$}
\end{prooftree}
Note that in this interaction, John will get whatever the store decides to send
on \tm{x}, and Mary will get whatever the store decides to send on \tm{y}.
This means that this interactions gives the choice of who receives what \emph{to
the store}. This is not an accurate model of our original example, where
the choice of who receives the cake is non-deterministic and depends on factors
outside of any of the participants' control!
And to make matters worse, the term which models our example is entirely
different from the one we initially wrote down in the \textpi-calculus!

The ability to model racy behaviour, such as that in our example, is essential
to describing the interactions that take place in realistic concurrent systems.
Therefore, we would like to extend \cp to allow such races.
Specifically, we would like to do it in a way which mirrors the way in which the
\textpi-calculus handles non-determinism.
We will base our extension on \rcp, a subset of \cp which we introduced in
\cref{sec:background}.
We have chosen to do this to keep our discussion as simple as possible.
Furthermore, as compatibility with the \textpi-calculus is of interest, we will
use the reduction system without commuting conversions, which we introduced in
\cref{sec:cppi}.

This chapter proceeds as follows.
In \cref{sec:nc-terms-and-types}, we introduce the extensions to the terms,
structural congruence and types of \cp.
In \cref{sec:nc-typing-clients-and-servers}, we introduce the typing rules for
\nodcap. 
In \cref{sec:nc-running-clients-and-servers}, we introduce the reduction rules
for \nodcap.
In \cref{sec:nc-properties}, we prove that our extension preserves the
meta-theoretical of \cp.
Finally, in \cref{sec:nc-local-choice}, we discuss the relation between
non-determinism in \nodcap and the non-determinism introduced by the addition of
non-deterministic local choice.

\section{Terms and types}\label{sec:nc-terms-and-types}
Let us return, briefly, to our example.
\[
  \tm{(\piPar{%
      \piSend{x}{\sliceofcake}{\piSend{x}{\nope}{\store}}
    }{%
      \piPar{\piRecv{x}{y}{\john}}{\piRecv{x}{z}{\mary}}
    })}
\]
In this interaction, we see that the channel \tm{x} is used only as a way to
connect the various clients, John and Mary, to the store.
The \emph{real} communication, sending the slice of cake and disappointment,
takes places on the channels \tm{\sliceofcake}, \tm{\nope}, \tm{y} and \tm{z}.

Inspired by this, we add two new constructs to the term language of \cp: sending
and receiving on a \emph{shared} channel.
These actions are marked with a \tm{\star} in order to distinguish them
syntactically from ordinary sending and receiving.
To group clients, we add another form of parallel composition, which we refer to
as \emph{pooling}. 
\begin{definition}[Terms]\label{def:nc-terms}
  We extend \cref{def:cp-terms} with the following constructs:
  \[\!
    \begin{aligned}
      \tm{P}, \tm{Q}, \tm{R}
           :=& \; \dots
      \\ \mid& \; \tm{\ncCnt{x}{y}{P}} &&\text{create client}
      \\ \mid& \; \tm{\ncSrv{x}{y}{P}} &&\text{create server interaction}
      \\ \mid& \; \tm{\ncPool{P}{Q}}   &&\text{parallel composition of clients}
    \end{aligned}
  \]
\end{definition}
%%% Local Variables:
%%% TeX-master: "main"
%%% End:

As before, round brackets denote input, square brackets denote output.
Note that \tm{\ncCnt{x}{y}{P}}, much like \tm{\cpSend{x}{y}{P}{Q}}, is a bound
output---this means that both client creation and server interaction bind a new
name.

In \rcp, we terms are identified up to the commutativity and associativity of
parallel composition. In \nodcap, we add another form of parallel composition,
and therefore must extend our structural congruence:
\begin{definition}[Structural congruence]\label{def:nc-equiv}
  We extend \cref{def:cp-equiv} with the following equivalences:
  \[
    \begin{array}{llll}
      \ncEquivPoolComm
      & \tm{\ncPool{P}{Q}}
      & \equiv \;
      & \tm{\ncPool{Q}{P}}
      \\
      \ncEquivPoolAss1
      & \tm{\ncPool{P}{\ncPool{Q}{R}}}
      & \equiv \;
      & \tm{\ncPool{\ncPool{P}{Q}}{R}}
      \\
      \ncRedKappaPool
      & \tm{\cpCut{x}{\ncPool{P}{Q}}{R}}
      & \equiv \;
      & \tm{\ncPool{P}{\cpCut{x}{Q}{R}}} \quad \text{if} \; \notFreeIn{x}{P} 
    \end{array}
  \]
\end{definition} 
%%% Local Variables:
%%% TeX-master: "main"
%%% End:

We add axioms for the commutativity and associativity of pooling.
We do not add an axiom for \ncEquivPoolAss2, as it follows from
\cref{def:nc-equiv}, see \cref{thm:nc-pool-assoc2}.
It should be noted that \tm{\cpCut{x}{P}{Q}} is considered a single,
\emph{atomic} construct.
Therefore you \emph{cannot} use \ncEquivPoolAss1 to rewrite
\tm{\cpCut{x}{P}{\ncPool{Q}{R}}} to \tm{\cpCut{x}{\ncPool{P}{Q}}{R}}.
We do, however, add two axioms which relate cuts and pool.
We call these \emph{extrusion}, because they closely resemble the
\textpi-calculus axiom for scope extrusion.
We add both \ncRedKappaPool1 and \ncRedKappaPool2, as these relate two different
constructs, and therefore we cannot use the one to derive the other. 
\begin{lemma}[\ncEquivPoolAssNoParen2]\label{thm:nc-pool-assoc2}
  We have
  \[
    \tm{\ncPool{\ncPool{P}{Q}}{R}} \equiv
    \tm{\ncPool{P}{\ncPool{Q}{R}}}
  \]
\end{lemma}
\begin{proof}
  \begin{align*}
    \tm{\ncPool{\ncPool{P}{Q}}{R}} &\equiv \qquad \text{by \ncEquivPoolComm and congruence} \\
    \tm{\ncPool{\ncPool{Q}{P}}{R}} &\equiv \qquad \text{by \ncEquivPoolComm} \\
    \tm{\ncPool{R}{\ncPool{Q}{P}}} &\equiv \qquad \text{by \ncEquivPoolAss1} \\
    \tm{\ncPool{\ncPool{R}{Q}}{P}} &\equiv \qquad \text{by \ncEquivPoolComm} \\
    \tm{\ncPool{P}{\ncPool{R}{Q}}} &\equiv \qquad \text{by \ncEquivPoolComm and congruence} \\
    \tm{\ncPool{P}{\ncPool{Q}{R}}}
  \end{align*}
\end{proof}
%%% Local Variables:
%%% TeX-master: "main"
%%% End:

Furthermore, the extensions to structural congruence preserve symmetry.
\begin{theorem}[Symmetry]\label{thm:nc-symmetry}
  If $\tm{P} \equiv \tm{Q}$, then $\tm{Q} \equiv \tm{P}$.
\end{theorem}
\begin{proof}
  By induction on the structure of the equivalence proof.
\end{proof}
%%% Local Variables:
%%% TeX-master: "main"
%%% End:

We can make another observation from our examples.
In every example in which a server interacts with a pool of clients, and which
does not deadlock, there are \emph{exactly} as many clients as there are
server interactions.
Therefore, we add two new \emph{dual} types for client pools and servers, which
track how many clients or server interactions they represent.
\begin{definition}[Types]\label{def:nc-types}
  \[\!
    \begin{aligned}
      \ty{A}, \ty{B}, \ty{C}
           :=& \; \dots
      \\ \mid& \; \ty{\take[n]{A}} &&\text{pool of} \; n \; \text{clients}
      \\ \mid& \; \ty{\give[n]{A}} &&n \; \text{server interactions}
    \end{aligned}
  \]  
\end{definition}
%%% Local Variables:
%%% TeX-master: "main"
%%% End:

\begin{definition}[Negation]\label{def:nc-negation}
  We extend \cref{def:cp-negation} with the following cases:
  \[\!
    \begin{array}{lclclcl}
              \ty{(\take[n]{A})^\bot} &=& \ty{\give[n]{A^\bot}}
      &\quad& \ty{(\give[n]{A})^\bot} &=& \ty{\take[n]{A^\bot}}
    \end{array}
  \]
\end{definition}
%%% Local Variables:
%%% TeX-master: "main"
%%% End:

With these new types, duality remains an involutive function.
\begin{lemma}[Involutive]\label{thm:nc-negation-involutive}
  We have $\ty{A^{\bot\bot}} = \ty{A}$.
\end{lemma}
\begin{proof}
  By induction on the structure of the type $\ty{A}$.
\end{proof}
%%% Local Variables:
%%% TeX-master: "main"
%%% End:


\section{Typing clients and servers}\label{sec:nc-typing-clients-and-servers}
We have to add typing rules to associate our new client and server
interactions with their types.
The definition for environments will remain unchanged, but we will extend the
definition for the typing judgement.
To determine the new typing rules, we essentially have to answer the question
``What typing constructs do we need to complete the following proof?''
\begin{prooftree}
  \AXC{$\seq[{ \john }]{ \Gamma, \tmty{y}{\cake^\bot} }$}
  \noLine\UIC{$\smash{\vdots}\vphantom{\vdash}$}
  \AXC{$\seq[{ \mary }]{ \Delta, \tmty{y'}{\cake^\bot} }$}
  \noLine\UIC{$\smash{\vdots}\vphantom{\vdash}$}
  \AXC{$\seq[{ \store }]{ \Theta, \tmty{z}{\cake}, \tmty{z'}{\cake} }$}
  \noLine\UIC{$\smash{\vdots}\vphantom{\vdash}$}
  \noLine\TIC{$\seq[{
      \cpCut{x}{\ncPool{\ncCnt{x}{y}{\john}}{\ncCnt{x}{y'}{\mary}}}{
        \ncSrv{x}{z}{\ncSrv{x}{z'}{\store}}} }]{
      \Gamma, \Delta, \Theta }$}
\end{prooftree}
Ideally, we would still like the composition of the client pool and the server
to be a cut. This seems reasonable, as the left-hand side of the term above has
2 clients, and the right-hand side has two server interactions, so \tm{x} is
used at type \ty{\take[2]{\cake^\bot}} on the left, and as \ty{\give[2]{\cake}}
on the right.
\begin{prooftree}
  \AXC{$\seq[{ \john }]{ \Gamma, \tmty{y}{\cake^\bot} }$}
  \noLine\UIC{$\smash{\vdots}\vphantom{\vdash}$}
  \AXC{$\seq[{ \mary }]{ \Delta, \tmty{y'}{\cake^\bot} }$}
  \noLine\UIC{$\smash{\vdots}\vphantom{\vdash}$}
  \noLine\BIC{$\seq[{ \ncPool{\ncCnt{x}{y}{\john}}{\ncCnt{x}{y'}{\mary}} }]{
      \Gamma, \Delta, \tmty{x}{\take[2]{\cake^\bot}} }$}

  \AXC{$\seq[{ \store }]{ \Theta, \tmty{z}{\cake}, \tmty{z'}{\cake} }$}
  \noLine\UIC{$\smash{\vdots}\vphantom{\vdash}$}
  \noLine\UIC{$\seq[{ \ncSrv{x}{z}{\ncSrv{x}{z'}{\store}} }]{
      \Theta, \tmty{x}{\give[2]{\cake}} }$}

  \NOM{Cut}
  \BIC{$\seq[{
      \cpCut{x}{\ncPool{\ncCnt{x}{y}{\john}}{\ncCnt{x}{y'}{\mary}}}{
        \ncSrv{x}{z}{\ncSrv{x}{z'}{\store}}} }]{
      \Gamma, \Delta, \Theta }$}
\end{prooftree}
We will define the typing judgement, and then discuss servers and clients, the
two sides of the above cut, describe the rules we add, and show how they allow
us to complete our proof.
\begin{definition}[Typing judgements]\label{def:cp-typing-judgement}
  A typing judgement $\seq[{ P }]{\tmty{x_1}{A_1}\dots\tmty{x_n}{A_n}}$ denotes
  that the process \tm{P} communicates along channels $\tm{x_1}\dots\tm{x_n}$
  following protocols $\ty{A_1}\dots\ty{A_n}$.
  Typing judgements can be constructed using the inference rules in
  \cref{fig:cp-typing-judgement,fig:nc-typing-judgement}.
\end{definition}
%%% Local Variables:
%%% TeX-master: "main"
%%% End:

\begin{figure*}[bh]
  \begin{center}
    \ncInfTake1
    \ncInfGive1
  \end{center}
  \vspace*{1\baselineskip}
  \begin{center}
    \ncInfPool
  \end{center}
  \vspace*{1\baselineskip}
  \begin{center}
    \ncInfCont
  \end{center}
  \caption{Typing judgement for \nodcap extending that of Figure~\ref{fig:cp-typing-judgement}}
  \label{fig:nc-typing-judgement}
\end{figure*}
%%% Local Variables:
%%% TeX-master: "main"
%%% End:


\subsection{Clients and pooling}\label{sec:clients-and-pooling}
A client pool represents a number of independent processes, each wanting to
interact with the server. Examples of such a pool include John and Mary from our
example, customers for online stores in general, and any number of processes
which interact with a single, centralised server.

We introduce two new rules: one to construct clients, and one to pool them
together. The first rule, $(\take[1]{})$, marks interaction over some channel as
a client interaction. It does this by receiving a channel \tm{y} over a
\emph{shared} channel \tm{x}. The channel \tm{y} is the channel across which the
actual interaction will eventually take place.
The second rule, \textsc{Pool}, allows us to pool together clients. This is
implemented, as in the \textpi-calculus, using parallel composition.
\begin{center}
  \ncInfTake1
  \ncInfPool
\end{center}
Using these rules, we can derive the left-hand side of our proof by marking John
and Mary as clients, and pooling them together.
\begin{prooftree}
  \AXC{$\seq[{ \john }]{ \Gamma, \tmty{y}{\cake^\bot} }$}
  \SYM{(\take[1]{})}
  \UIC{$\seq[{ \ncCnt{x}{y}{\john} }]{ \Gamma, \tmty{z}{\take[1]{\cake^\bot}} }$}

  \AXC{$\seq[{ \mary }]{ \Delta, \tmty{y'}{\cake^\bot} }$}
  \SYM{(\take[1]{})}
  \UIC{$\seq[{ \ncCnt{x}{y'}{\mary} }]{ \Delta, \tmty{y'}{\take[1]{\cake^\bot}} }$}

  \NOM{Pool}
  \BIC{$\seq[{ \ncPool{\ncCnt{x}{y}{\john}}{\ncCnt{x}{y'}{\mary}} }]{
      \Gamma, \Delta, \tmty{x}{\take[2]{\cake^\bot}} }$}
\end{prooftree}

\subsection{Servers and contraction}\label{sec:servers-and-contraction}
Dual to a pool of clients is a server. Our interpretation of a server is a
process which offers up some number of interdependent interactions of the same
type. Examples include the store from our example, which gives out slices of
cake and disappointment, online stores in general, and any central server which
interacts with some number of client processes.

We introduce two new rules to construct servers. The first rule, $(\give[1]{})$,
marks a interaction over some channel as a server interaction. It does this by
sending a channel \tm{y} over a \emph{shared} channel \tm{x}. The channel \tm{y}
is the channel across which the actual interaction will eventually take place.
The second rule, \textsc{Cont}, short for contraction, allows us to contract
several server interactions into a single server. This allows us to construct a
server which has multiple interactions of the same type, across the same shared
channel.\footnote{%
  While it ultimately does not matter whether $(\give[1]{})$ and $(\take[1]{})$
  are implemented with a send or a receive action, it feels more natural to have
  the server do the sending.
  Clients indicate their interest in interacting with the server by connecting
  to the shared channel, but it is up to the server to decide \emph{when} to
  interact with each channel.}
\begin{center}
  \ncInfGive1
  \ncInfCont
\end{center}
Using these rules, we can derive the right-hand side of our proof, by marking
each of the store's interactions as server interactions, and then contracting
them.
\begin{prooftree}
  \AXC{$\seq[{ \store }]{ \Theta, \tmty{z}{\cake}, \tmty{z'}{\cake} }$}
  \SYM{(\give[1]{})}
  \UIC{$\seq[{ \ncSrv{x'}{z'}{\store} }]{
      \Theta, \tmty{z}{\cake}, \tmty{x'}{\give[1]{\cake}} }$}
  \SYM{(\give[1]{})}
  \UIC{$\seq[{ \ncSrv{x}{z}{\ncSrv{x'}{z'}{\store}} }]{
      \Theta, \tmty{x}{\give[1]{\cake}}, \tmty{x'}{\give[1]{\cake}} }$}
  \NOM{Cont}
  \UIC{$\seq[{ \ncSrv{x}{z}{\ncSrv{x}{z'}{\store}} }]{
      \Theta, \tmty{x}{\give[2]{\cake}} }$}
\end{prooftree}
Thus, we complete the typing derivation of our example.

\section{Running clients and servers}\label{sec:nc-running-clients-and-servers}
Once we have a client/server interaction, how do we run it? Ideally, we would
simply use the reduction rule closest to the one used in the \textpi-calculus. 
\[
  \reducesto
  {\tm{\cpCut{x}{\ncCnt{x}{y}{P}}{\ncSrv{x}{z}{R}}}}
  {\tm{\cpCut{y}{P}{\cpSub{y}{z}{R}}}}
\]
However, our case is complicated by the fact that in \tm{\cpCut{x}{P}{Q}} the
name restriction is an inseparable part of the composition, and therefore has to
be part of our reduction rule. 
Because of this, the above reduction can only apply in the singleton case.
If the client pool contains more than one client, such as in the term below,
then there is no way to isolate a single client together with the server,
because \tm{x} occurs in both \tm{\ncCnt{x}{y}{P}} and \tm{\ncCnt{x}{z}{Q}}.
\[
  \tm{\cpCut{x}{\ncPool{\ncCnt{x}{y}{P}}{\ncCnt{x}{z}{Q}}}{\ncSrv{x}{w}{R}}}
  \centernot\Longrightarrow
\]
Therefore, we add a second reduction rule, which handles communication between a
one client in a pool of multiple clients and a server.
\[
  \reducesto
  {\tm{\cpCut{x}{\ncPool{\ncCnt{x}{y}{P}}{Q}}{\ncSrv{x}{z}{R}}}}
  {\tm{\cpCut{x}{Q}{\cpCut{y}{P}{\cpSub{y}{z}{R}}}}}
\]
Lastly, because we have added another form of parallel composition, we add
another congruence rule, to allow for reduction inside client pools.
\begin{definition}[Term reduction]\label{def:nc-reduction}
  We extend \cref{def:cp-reduction} with the following reductions:
  \[
    \begin{array}{llll}
      \ncRedBetaStar{1}
      & \tm{\cpCut{x}{\ncCnt{x}{y}{P}}{\ncSrv{x}{z}{R}}}
      & \Longrightarrow \;
      & \tm{\cpCut{y}{P}{\cpSub{y}{z}{R}}}
      \\
      \ncRedBetaStar{n+1}
      & \tm{\cpCut{x}{\ncPool{\ncCnt{x}{y}{P}}{Q}}{\ncSrv{x}{z}{R}}}
      & \Longrightarrow \;
      & \tm{\cpCut{x}{Q}{\cpCut{y}{P}{\cpSub{y}{z}{R}}}}
      \\
      \\
      \ncRedKappaTake
      & \tm{\cpCut{x}{\ncCnt{y}{z}{P}}{R}}
      & \Longrightarrow \;
      & \tm{\ncCnt{y}{z}{\cpCut{x}{P}{R}}}
      \\
      \ncRedKappaGive
      & \tm{\cpCut{x}{\ncSrv{y}{z}{P}}{R}}
      & \Longrightarrow \;
      & \tm{\ncSrv{y}{z}{\cpCut{x}{P}{R}}}
    \end{array}
  \]
  \begin{prooftree}
    \AXC{\reducesto{P}{P'}}
    \SYM{\ncRedGammaPool}
    \UIC{\reducesto{\ncPool{P}{Q}}{\ncPool{P'}{Q}}}
  \end{prooftree}
\end{definition}
%%% Local Variables:
%%% TeX-master: "main"
%%% End:


The rules \ncRedBetaStar1 and \ncRedBetaStar{n+1} seem like the elimination
rules for a list-like construct. This may come as a surprise, as our client
pools are built up like binary trees, and the typing rules for both sides are
tree-like, with $(\take[1]{})$ and $(\give[1]{})$ playing the role of leaves,
and \textsc{Pool} and \textsc{Cont} merging two trees with $m$ and $n$ leaves
into one with $m+n$ leaves.
However, the server process imposes a sequential ordering on its interactions,
and it is because of this that we have to use list-like elimination rules.

So where does the non-determinism in \nodcap come from? Let us say we have a
term of the following form:
\[
  \tm{
    \cpCut{x}
    {\ncPool{\ncCnt{x}{y_1}{P_1}}{\dots \mid \ncCnt{x}{y_n}{P_n}}}
    {\ncSrv{x}{y}{Q}}
  }
\]
Because pooling is commutative and associative, we can rewrite this term to
bring any client in the pool to the front, before applying \ncRedBetaStar{n+1}.
Thus, like in the \textpi-calculus, the non-determinism is introduced by the
structural congruence.

Does this mean that, for an arbitrary client pool \tm{P} in
\tm{\cpCut{x}{P}{\ncSrv{x}{z}{Q}}}, every client in that pool is competing for
the server interaction on \tm{x}?
Not necessarily, as some portion of the clients can be blocked on an external
communication. For instance, in the term below, clients
$\tm{\ncCnt{x}{y_{n+1}}{P_{n+1}}} \dots \tm{\ncCnt{x}{y_m}{P_m}}$ are blocked
on a communication on the external channel \tm{a}.
\[
  \arraycolsep=0pt
  \tm{
  \begin{array}{lrl}
    \nu x.&  ((&\; \ncPool{\ncCnt{x}{y_1}{P_1}}{\dots\mid\ncCnt{x}{y_n}{P_n}}\\
          &\mid&\; \cpWait{a}{\ncPool{\ncCnt{x}{y_{n+1}}{P_{n+1}}}{\dots \mid \ncCnt{x}{y_m}{P_m}}}\;)\\
          &\mid&\; \ncSrv{x}{y_1}{\dots\ncSrv{x}{y_m}{Q}}\;)
  \end{array}}
\]
If we reduce this term, then only the clients
$\tm{\ncCnt{x}{y_1}{P_1}} \dots \tm{\ncCnt{x}{y_n}{P_n}}$
will be assigned server interactions, and we end up with the following canonical
form term. 
\[
  \arraycolsep=0pt
  \tm{
  \begin{array}{lrl}
    \nu x.&   (&\; \cpWait{a}{\ncPool{\ncCnt{x}{y_{n+1}}{P_{n+1}}}{\dots\mid\ncCnt{x}{y_m}{P_m}}}\\
          &\mid&\; \ncSrv{x}{y_{n+1}}{\dots\ncSrv{x}{y_m}{Q}}\;)
  \end{array}}
\]
This matches the reduction behaviour of the \textpi-calculus, and it fits with
out notion of computation with processes.

\section{Properties of \nodcap}\label{sec:nc-properties}
In this section, we will revisit the proofs for three important properties of
\rcp, namely preservation, progress, and termination, and show that our
extensions preserve these properties.

\subsection{Preservation}
Preservation is the fact that term reduction preserves typing. There are two
proofs involved in this. First, we show that structural congruence preserves
typing.
\begin{theorem}[Preservation for $\equiv$]\label{thm:nc-preservation-equiv}
  If $\seq[{ P }]{ \Gamma }$ and $\tm{P} \equiv \tm{Q}$,
  then $\seq[{ Q }]{ \Gamma }$.
\end{theorem}
\begin{proof}
  By induction on the structure of the equivalence. The cases for reflexivity,
  transitivity and congruence are trivial. The cases for \cpEquivCutComm and
  \cpEquivCutAss1 are given in \cref{fig:cp-preservation-equiv}.
  The cases for \ncEquivPoolComm and \ncEquivPoolAss1 are given in
  \cref{fig:nc-preservation-equiv}.
\end{proof}
%%% Local Variables:
%%% TeX-master: "main"
%%% End:

\begin{figure*}[b]
  \centering
  \begin{tabular}{ll}
    \ncEquivPoolComm
    &
      \begin{prooftree*}
        \AXC{$\seq[{ P }]{ \Gamma, \tmty{x}{\take[m]{A}} }$}
        \AXC{$\seq[{ Q }]{ \Delta, \tmty{x}{\take[n]{A}} }$}
        \NOM{Pool}
        \BIC{$\seq[{ \ncPool{P}{Q} }]{ \Gamma, \Delta, \take[m+n]{A} }$}
      \end{prooftree*}
    \\[30pt]
    $\equiv$
    &
      \begin{prooftree*}
        \AXC{$\seq[{ Q }]{ \Delta, \tmty{x}{\take[n]{A}} }$}
        \AXC{$\seq[{ P }]{ \Gamma, \tmty{x}{\take[m]{A}} }$}
        \NOM{Pool}
        \BIC{$\seq[{ \ncPool{Q}{P} }]{ \Gamma, \Delta, \tmty{x}{\take[m+n]{A}} }$}
      \end{prooftree*}
    \\[40pt]
    \ncEquivPoolAss1
    &
      \begin{prooftree*}
        \AXC{$\seq[{ P }]{ \Gamma, \tmty{x}{\take[l]{A}} }$}
        \AXC{$\seq[{ Q }]{ \Delta, \tmty{x}{\take[m]{A}} }$}
        \AXC{$\seq[{ R }]{ \Theta, \tmty{x}{\take[n]{A}} }$}
        \NOM{Pool}
        \BIC{$\seq[{ \ncPool{Q}{R} }]{ \Delta, \Theta, \tmty{x}{\take[m+n]{A}} }$}
        \NOM{Pool}
        \BIC{$\seq[{ \ncPool{P}{\ncPool{Q}{R}} }]{ \Gamma, \Delta, \Theta, \tmty{x}{\take[l+m+n]{A}} }$}
      \end{prooftree*}
    \\[30pt]
    $\equiv$
    &
      \begin{prooftree*}
        \AXC{$\seq[{ P }]{ \Gamma, \tmty{x}{\take[l]{A}} }$}
        \AXC{$\seq[{ Q }]{ \Delta, \tmty{x}{\take[m]{A}} }$}
        \NOM{Pool}
        \BIC{$\seq[{ \ncPool{P}{Q} }]{ \Gamma, \Delta, \tmty{x}{\take[l+m]{A}} }$}
        \AXC{$\seq[{ R }]{ \Theta, \tmty{x}{\take[n]{A}} }$}
        \NOM{Pool}
        \BIC{$\seq[{ \ncPool{\ncPool{P}{Q}}{R} }]{ \Gamma, \Delta, \Theta, \tmty{x}{\take[l+m+n]{A}} }$}
      \end{prooftree*}
    \\[20pt]
    or
    \\[20pt]
    \ncEquivPoolAss1
    &
      \begin{prooftree*}
        \AXC{$\seq[{ P }]{ \Gamma, \tmty{x}{\take[k]{A}} }$}
        \AXC{$\seq[{ Q }]{ \Delta, \tmty{x}{\take[l]{A}}, \tmty{y}{\take[m]{B}} }$}
        \AXC{$\seq[{ R }]{ \Theta, \tmty{x}{\take[n]{B}} }$}
        \NOM{Pool}
        \BIC{$\seq[{ \ncPool{Q}{R} }]{ \Delta, \Theta, \tmty{x}{\take[l]{A}}, \tmty{y}{\take[m+n]{B}} }$}
        \NOM{Pool}
        \BIC{$\seq[{ \ncPool{P}{\ncPool{Q}{R}} }]{ \Gamma, \Delta, \Theta, \tmty{x}{\take[k+l]{A}}, \tmty{y}{\take[m+n]{B}} }$}
      \end{prooftree*}
    \\[30pt]
    $\equiv$
    &
      \begin{prooftree*}
        \AXC{$\seq[{ P }]{ \Gamma, \tmty{x}{\take[k]{A}} }$}
        \AXC{$\seq[{ Q }]{ \Delta, \tmty{x}{\take[l]{A}}, \tmty{y}{\take[m]{B}} }$}
        \NOM{Pool}
        \BIC{$\seq[{ \ncPool{Q}{R} }]{ \Gamma, \Delta, \tmty{x}{\take[k+l]{A}}, \tmty{y}{\take[m]{B}} }$}
        \AXC{$\seq[{ R }]{ \Theta, \tmty{x}{\take[o]{B}} }$}
        \NOM{Pool}
        \BIC{$\seq[{ \ncPool{P}{\ncPool{Q}{R}} }]{ \Gamma, \Delta, \Theta, \tmty{x}{\take[k+l]{A}}, \tmty{y}{\take[m+n]{B}} }$}
      \end{prooftree*}
    \\[30pt]
    \ncRedKappaPool1
    &
      \begin{prooftree*}
        \AXC{$\seq[{ P }]{ \Gamma, \tmty{y}{\take[m]{B}} }$}
        \AXC{$\seq[{ Q }]{ \Delta, \tmty{x}{A}, \tmty{y}{\take[n]{B}} }$}
        \AXC{$\seq[{ R }]{ \Theta, \tmty{x}{A^\bot} }$}
        \NOM{Cut}
        \BIC{$\seq[{ \cpCut{x}{Q}{R} }]{ \Gamma, \Delta, \Theta, \tmty{y}{\take[n]{B}} }$}
        \NOM{Pool}
        \BIC{$\seq[{ \ncPool{P}{\cpCut{x}{Q}{R}} }]{ \Gamma, \Delta, \Theta, \tmty{y}{\take[m+n]{B}} }$}
      \end{prooftree*}
    \\[30pt]
    $\equiv$
    &
      \begin{prooftree*}
        \AXC{$\seq[{ P }]{ \Gamma, \tmty{y}{\take[m]{B}} }$}
        \AXC{$\seq[{ Q }]{ \Delta, \tmty{x}{A}, \tmty{y}{\take[n]{B}} }$}
        \NOM{Pool}
        \BIC{$\seq[{ \ncPool{P}{Q} }]{ \Gamma, \Delta, \tmty{y}{\take[m+n]{B}} }$}
        \AXC{$\seq[{ R }]{ \Theta, \tmty{x}{A^\bot} }$}
        \NOM{Cut}
        \BIC{$\seq[{ \cpCut{x}{\ncPool{P}{Q}}{R} }]{ \Gamma, \Delta, \Theta, \tmty{y}{\take[m+n]{B}} }$}
      \end{prooftree*} 
    \\[30pt]
    \ncRedKappaPool2
    &
      (as above)
  \end{tabular}
  \caption{Type preservation for the structural congruence of \nodcap}
  \label{fig:nc-preservation-equiv}
\end{figure*}
%%% Local Variables:
%%% TeX-master: "main"
%%% End:

Secondly, we prove that term reduction preserves typing.
\begin{theorem}[Preservation]\label{thm:nc-preservation}
  If \reducesto{P}{Q} and $\seq[{ P }]{ \Gamma }$, then $\seq[{ Q }]{ \Gamma }$.
\end{theorem}
\begin{proof}
  By induction on the structure of the reduction. See
  \cref{fig:cp-preservation-1} for the AxCut and \textbeta-reduction rules from
  \cp, and \cref{fig:cp-preservation-2a,fig:cp-preservation-2b} for the
  commutative conversions from \cp.
  See \cref{fig:nc-preservation} for the \textbeta-reduction rules and
  commutative conversions from \nc.
  The cases for \cpRedGammaCut and \ncRedGammaPool are trivial by call to the
  induction hypothesis, and the case for \cpRedGammaEquiv is trivial by call to
  the induction hypothesis and \cref{thm:cp-preservation-equiv}.
\end{proof}
%%% Local Variables:
%%% TeX-master: "main"
%%% End:

\begin{figure*}[ht]
  \makebox[\textwidth][c]{
    \begin{tabular}{ll}
      \ncRedBetaStar1
      &
        \begin{prooftree*}
          \AXC{$\seq[{ P }]{ \Gamma, \tmty{y}{A} }$}
          \SYM{\take[1]{}}
          \UIC{$\seq[{ \ncCnt{x}{y}{P} }]{ \Gamma, \tmty{x}{\take[1]{A}} }$}
          \AXC{$\seq[{ Q }]{ \Delta, \tmty{z}{A^\bot} }$}
          \SYM{\give[1]{}}
          \UIC{$\seq[{ \ncSrv{x}{z}{Q} }]{ \Delta, \tmty{x}{\give[1]{A}} }$}
          \NOM{Cut}
          \BIC{$\seq[{ \cpCut{x}{\ncCnt{x}{y}{P}}{\ncSrv{x}{z}{Q}} }]{ \Gamma, \Delta }$}
        \end{prooftree*}
      \\[30pt]
      $\Longrightarrow$
      &
        \begin{prooftree*}
          \AXC{$\seq[{ P }]{ \Gamma, \tmty{y}{A} }$}
          \AXC{$\seq[{ Q }]{ \Delta, \tmty{z}{A^\bot} }$}
          \NOM{Cut}
          \BIC{$\seq[{ \cpCut{y}{P}{\cpSub{y}{z}{Q}} }]{ \Gamma, \Delta }$}
        \end{prooftree*}
      \\[40pt]
      \ncRedBetaStar{n+1}
      &
        \begin{prooftree*}
          \AXC{$\seq[{ P }]{ \Gamma, \tmty{y}{A} }$}
          \SYM{\take[1]{}}
          \UIC{$\seq[{ \ncCnt{x}{y}{P} }]{ \Gamma, \tmty{x}{\take[1]{A}} }$}
          \AXC{$\seq[{ Q }]{ \Delta, \tmty{x}{\take[n]{A}} }$}
          \NOM{Pool}
          \BIC{$\seq[{ \ncPool{\ncCnt{x}{y}{P}}{Q} }]{ \Gamma, \Delta, \tmty{x}{\take[n+1]{A}} }$}
          \AXC{$\seq[{ R }]{ \Theta, \tmty{z}{A^\bot}, \tmty{x}{\give[n]{A}} }$}
          \SYM{\give[1]{}}
          \UIC{$\seq[{ \ncSrv{x'}{z}{Q} }]{ \Theta, \tmty{x'}{\give[1]{A}}, \tmty{x}{\give[n]{A}} }$}
          \NOM{Cont}
          \UIC{$\seq[{ \ncSrv{x}{z}{Q} }]{ \Theta, \tmty{x}{\give[n+1]{A}} }$}
          \NOM{Cut}
          \BIC{$\seq[{ \cpCut{x}{\ncPool{\ncCnt{x}{y}{P}}{Q}}{\ncSrv{x}{z}{Q}} }]{ \Gamma, \Delta, \Theta }$}
        \end{prooftree*}
      \\[40pt]
      $\Longrightarrow$
      &
        \begin{prooftree*}
          \AXC{$\seq[{ Q }]{ \Delta, \tmty{x}{\take[n]{A}} }$}
          \AXC{$\seq[{ P }]{ \Gamma, \tmty{y}{A} }$}
          \AXC{$\seq[{ R }]{ \Theta, \tmty{z}{A^\bot}, \tmty{x}{\give[n]{A}} }$}
          \NOM{Cut}
          \BIC{$\seq[{ \cpCut{y}{P}{\cpSub{y}{z}{R}} }]{ \Gamma, \Theta, \tmty{x}{\give[n]{A}} }$}
          \NOM{Cut}
          \BIC{$\seq[{ \cpCut{x}{Q}{\cpCut{y}{P}{\cpSub{y}{z}{R}}} }]{ \Gamma, \Delta, \Theta }$}
        \end{prooftree*}
      \\[40pt]
      \ncRedKappaTake
      &
        \begin{prooftree*}
          \AXC{$\seq[{ P }]{ \Gamma, \tmty{x}{A}, \tmty{z}{B} }$}
          \SYM{\take[1]{}}
          \UIC{$\seq[{ \ncCnt{y}{z}{P} }]{ \Gamma, \tmty{x}{A}, \tmty{y}{\take[1]{B}} }$}
          \AXC{$\seq[{ R }]{ \Delta, \tmty{x}{A^\bot} }$}
          \NOM{Cut}
          \BIC{$\seq[{ \cpCut{x}{\ncCnt{y}{z}{P}}{R} }]{ \Gamma, \Delta, \tmty{y}{\take[1]{B}} }$}
        \end{prooftree*}
      \\[30pt]
      $\Longrightarrow$
      &
        \begin{prooftree*}
          \AXC{$\seq[{ P }]{ \Gamma, \tmty{x}{A}, \tmty{z}{B} }$}
          \AXC{$\seq[{ R }]{ \Delta, \tmty{x}{A^\bot} }$}
          \NOM{Cut}
          \BIC{$\seq[{ \cpCut{x}{P}{R} }]{ \Gamma, \Delta, \tmty{z}{B} }$}
          \SYM{\take[1]{}}
          \UIC{$\seq[{ \ncCnt{y}{z}{\cpCut{x}{P}{R}} }]{ \Gamma, \Delta, \tmty{z}{\take[1]{B}} }$}
        \end{prooftree*}
      \\[30pt]
      \ncRedKappaGive
      &
        (as above)
    \end{tabular}
  }   

  \caption{Type preservation for the \textbeta-reduction rules and commutative conversions of \nodcap}
  \label{fig:nc-preservation-1}
\end{figure*}
%%% Local Variables:
%%% TeX-master: "main"
%%% End: 

\subsection{Canonical forms and progress}
In this section, we will extend the definition of canonical forms and the proof
of progress progress given in \cref{sec:cppi}.

\subsubsection{Canonical forms}
First, we extend the definitions of actions with our actions for client and
server creation. 
\begin{definition}[Action]\label{def:nc-action}
  We extend \cref{def:cp-action} with the following cases:
  \begin{itemize}[noitemsep,topsep=0pt,parsep=0pt,partopsep=0pt]
  \item \tm{\ncCnt{x}{y}{P'}}
  \item \tm{\ncSrv{x}{y}{P'}}
  \end{itemize}
\end{definition}
%%% Local Variables:
%%% TeX-master: "main"
%%% End:

Secondly, as we can reduce inside client pools, we will add pooling to our
definition of evaluation prefixes.
\begin{definition}[Evaluation prefixes]\label{def:nc-evaluation-prefixes}
  We extend \cref{def:cp-evaluation-prefixes} with the following constructs:
  \begin{align*}
    \tm{G}, \tm{H} := \dots \mid \tm{\ncPool{G}{H}}
  \end{align*}
  We also define a special case of evaluation prefixes, which we will refer to
  as \emph{pooling prefixes}. These are evaluation prefixes which consist solely
  of pooling operators and holes. 
\end{definition}
\begin{definition}[Plugging]\label{def:nc-evaluation-prefix-plugging}
  We extend \cref{def:cp-evaluation-prefix-plugging} with the following case:
  \[
    \begin{array}{ll}
      \tm{\cpPlug{\ncPool{G}{H}}{R_1 \dots R_m, R_{m+1} \dots R_{n}}}
                            & := \; \tm{\ncPool{\cpPlug{G}{R_1 \dots R_m}}{\cpPlug{H}{R_{m+1} \dots R_n}}}
    \end{array}
  \]
  Note that in the this case, \tm{G} is an evaluation prefix with $m$ holes,
  and \tm{H} is an evaluation prefix with $(n-m)$ holes.
\end{definition}
%%% Local Variables:
%%% TeX-master: "main"
%%% End:

The definition for the maximum evaluation prefix is unchanged.

There are some subtleties to our definition of canonical forms. The type system
for \cp guarantees that all links directly under an evaluation context act on a
bound channel. Not so for \nodcap.
\begin{scprooftree}
  \AXC{}
  \NOM{Ax}
  \UIC{$\seq[{ \cpLink{x}{y} }]{ \tmty{x}{\take[m]{A}}, \tmty{y}{\give[m]{A^\bot}} }$}
  \AXC{$\seq[{ P }]{ \Gamma, \tmty{x}{\take[n]{A}} }$}
  \NOM{Pool}
  \BIC{$\seq[{ \ncPool{\cpLink{x}{y}}{P} }]{ \Gamma, \tmty{x}{\take[m+n]{A}}, \tmty{y}{\give[m]{A^\bot}} }$}
\end{scprooftree}
There is no way to sensibly reduce this link.
Furthermore, in \cp, if two processes act on the same channel, then they must be
on different sides of the cut introducing that channel. The addition of shared
channels and client pools invalidates this property.
Therefore, we will have to be more careful about the way we define canonical
forms.
We restate the definition of canonical forms below. The additions have been
italicised.
\begin{definition}[Canonical forms]\label{def:nc-canonical-forms}
  A process \tm{P} is in canonical form if it is an action, or if it is of the
  form \tm{\cpPlug{G}{P_1 \dots P_n}}, where \tm{G} is the maximum evaluation
  prefix of \tm{P}, no \tm{P_i} is a link \emph{which acts on a bound channel},
  and no \tm{P_i} and \tm{P_j}, \emph{on different sides of at least one cut}, act on
  the same channel.
\end{definition}
%%% Local Variables:
%%% TeX-master: "main"
%%% End:


\subsection{Evaluation contexts}
Evaluation contexts are one-holed term contexts under which reduction can take
place. Since we have added another congruence rule, stating that reduction can
take place inside client pools, we extend our definition of evaluation contexts
to match this.
\begin{definition}[Evaluation contexts]\label{def:nc-evaluation-contexts}
  We extend \cref{def:cp-evaluation-contexts} with the following constructs:
  \begin{align*}
    \tm{E} := \dots \mid \tm{\ncPool{E}{P}} \mid \tm{\ncPool{P}{E}}
  \end{align*}
  We also define a special case of evaluation contexts, which we will refer to
  as \emph{pooling contexts}. These are evaluation contexts which consist solely
  of pooling operators and holes.
\end{definition}
\begin{definition}[Plugging]\label{def:nc-evaluation-context-plugging}
  We extend \cref{def:cp-evaluation-context-plugging} with the following cases:
  \begin{gather*}
    \begin{array}{ll}
      \tm{\cpPlug{\ncPool{E}{P}}{R}}
      & := \; \tm{\ncPool{\cpPlug{E}{R}}{P}}
      \\
      \tm{\cpPlug{\ncPool{P}{E}}{R}}
      & := \; \tm{\ncPool{P}{\cpPlug{E}{R}}}
    \end{array}
  \end{gather*}
\end{definition}
%%% Local Variables:
%%% TeX-master: "main"
%%% End:

We also restate \cref{thm:cp-display-cut-1}, and prove that our extension
preserves the property.
\begin{lemmaB}\label{thm:nc-display-cut-1}
  If $\seq[{ \tm{\cpCut{x}{\cpPlug{G}{P}}{Q}} }]{ \Gamma }$ and
  $\notFreeIn{x}{G}$, then $\tm{\cpCut{x}{\cpPlug{G}{P}}{Q}} \equiv
  \tm{\cpPlug{G}{\cpCut{x}{P}{Q}}}$.
\end{lemmaB}
\begin{proof}
  By induction on the structure of the evaluation context \tm{G}.
  \begin{itemize}
  \item
    Case $\tm{\Box}$, $\tm{\cpCut{y}{H}{R}}$, and $\tm{\cpCut{y}{R}{H}}$. See \cref{thm:cp-display-cut-1}.
  \item
    Case $\tm{\ncPool{G}{R}}$.
    \[\!
      \begin{array}{ll}
        \tm{\cpCut{x}{\ncPool{\cpPlug{G}{P}}{R}}{Q}} & \equiv \quad \text{by} \; \ncEquivPoolComm \\
        \tm{\cpCut{x}{\ncPool{R}{\cpPlug{G}{P}}}{Q}} & \equiv \quad \text{by} \; \ncRedKappaPool1 \\
        \tm{\ncPool{R}{\cpCut{x}{\cpPlug{G}{P}}{Q}}} & \equiv \quad \text{by} \; \ncEquivPoolComm \\
        \tm{\ncPool{\cpCut{x}{\cpPlug{G}{P}}{Q}}{R}} & \equiv \quad \text{by the induction hypothesis}\\
        \tm{\ncPool{\cpPlug{G}{\cpCut{x}{P}{Q}}}{R}} &
      \end{array}
    \]
  \item
    Case $\tm{\ncPool{R}{G}}$.
    \[\!
      \begin{array}{ll}
        \tm{\cpCut{x}{\ncPool{R}{\cpPlug{G}{P}}}{Q}} & \equiv \quad \text{by} \; \ncRedKappaPool1 \\
        \tm{\ncPool{R}{\cpCut{x}{\cpPlug{G}{P}}{Q}}} & \equiv \quad \text{by the induction hypothesis}\\
        \tm{\ncPool{R}{\cpPlug{G}{\cpCut{x}{P}{Q}}}} &
      \end{array}
    \]
  \end{itemize}
  In each case, the side condition for \ncRedKappaPool1, $\notFreeIn{x}{R}$, can
  be inferred from $\notFreeIn{x}{G}$, and the side conditions for the induction
  hypothesis can be inferred from \cref{thm:nc-preservation-equiv} and
  $\notFreeIn{x}{G}$.
\end{proof}
%%% Local Variables:
%%% TeX-master: "main"
%%% End:
Furthermore, it will be useful to prove a similar lemma, which shows that we can
push any pooling downwards under an evaluation context.
\begin{lemmaB}\label{thm:nc-display-pool-1}
  If $\seq[{ \ncPool{\cpPlug{G}{P}}{Q} }]{ \Gamma }$, then
  $\tm{\ncPool{\cpPlug{G}{P}}{Q}} \equiv \tm{\cpPlug{G}{\ncPool{P}{Q}}}$. 
\end{lemmaB}
\begin{proof}
  By induction on the structure of the evaluation context \tm{G}.
  \begin{itemize}
  \item
    Case $\tm{\Box}$. By reflexivity.
  \item
    Case $\tm{\cpCut{y}{G}{R}}$.
    \[\!
      \begin{array}{ll}
        \tm{\ncPool{\cpCut{y}{\cpPlug{G}{P}}{R}}{Q}} & \equiv \quad \text{by} \; \ncEquivPoolComm \\
        \tm{\ncPool{Q}{\cpCut{y}{\cpPlug{G}{P}}{R}}} & \equiv \quad \text{by} \; \ncRedKappaPool2 \\
        \tm{\cpCut{y}{\ncPool{Q}{\cpPlug{G}{P}}}{R}} & \equiv \quad \text{by} \; \cpEquivCutComm \\
        \tm{\cpCut{y}{\ncPool{\cpPlug{G}{P}}{Q}}{R}} & \equiv \quad \text{by the induction hypothesis}\\
        \tm{\cpCut{y}{\cpPlug{G}{\ncPool{P}{Q}}}{R}} &
      \end{array}
    \]
  \item
    Case $\tm{\cpCut{y}{R}{G}}$.
    \[\!
      \begin{array}{ll}
        \tm{\ncPool{\cpCut{y}{R}{\cpPlug{G}{P}}}{Q}} & \equiv \quad \text{by} \; \ncEquivPoolComm \\
        \tm{\ncPool{Q}{\cpCut{y}{\cpPlug{G}{P}}{R}}} & \equiv \quad \text{by} \; \ncRedKappaPool2 \\
        \tm{\cpCut{y}{\ncPool{Q}{\cpPlug{G}{P}}}{R}} & \equiv \quad \text{by} \; \cpEquivCutComm \\
        \tm{\cpCut{y}{R}{\ncPool{Q}{\cpPlug{G}{P}}}} & \equiv \quad \text{by the induction hypothesis} \\
        \tm{\cpCut{y}{R}{\cpPlug{G}{\ncPool{P}{Q}}}}
      \end{array}
    \]
  \item
    Case $\tm{\ncPool{G}{R}}$.
    \[\!
      \begin{array}{ll}
        \tm{\ncPool{\ncPool{\cpPlug{G}{P}}{R}}{Q}} & \equiv \quad \text{by} \; \ncEquivPoolComm \\
        \tm{\ncPool{\ncPool{R}{\cpPlug{G}{P}}}{Q}} & \equiv \quad \text{by} \; \ncEquivPoolAss2 \\
        \tm{\ncPool{R}{\ncPool{\cpPlug{G}{P}}{Q}}} & \equiv \quad \text{by} \; \ncEquivPoolComm \\
        \tm{\ncPool{\ncPool{\cpPlug{G}{P}}{Q}}{R}} & \equiv \quad \text{by the induction hypothesis} \\
        \tm{\ncPool{\cpPlug{G}{\ncPool{P}{Q}}}{R}} &
      \end{array}
    \]
  \item
    Case $\tm{\ncPool{R}{G}}$.
    \[\!
      \begin{array}{ll}
        \tm{\ncPool{\ncPool{R}{\cpPlug{G}{P}}}{Q}} & \equiv \quad \text{by} \; \ncEquivPoolAss2 \\
        \tm{\ncPool{R}{\ncPool{\cpPlug{G}{P}}{Q}}} & \equiv \quad \text{by the induction hypothesis} \\
        \tm{\ncPool{R}{\cpPlug{G}{\ncPool{P}{Q}}}} &
      \end{array}
    \]
  \end{itemize}
  In each case, the side conditions for \ncRedKappaPool2, $\notFreeIn{y}{Q}$,
  can be inferred from the fact that
  $\tm{\ncPool{Q}{\cpCut{y}{\cpPlug{G}{P}}{R}}}$ is well-typed, the side
  conditions for \cpEquivCutAss2, $\notFreeIn{x}{R}$ and $\notFreeIn{y}{Q}$, can
  be inferred from $\notFreeIn{x}{G}$ and the fact that
  $\tm{\ncPool{\cpPlug{G}{P}}{Q}}$ is well-typed, and the side conditions for
  the induction hypothesis can be inferred from the fact that
  $\tm{\ncPool{\cpPlug{G}{P}}{Q}}$ is well-typed,
  \cref{thm:nc-preservation-equiv} and $\notFreeIn{x}{G}$. 
\end{proof}
%%% Local Variables:
%%% TeX-master: "main"
%%% End:

\subsection{Progress}
Progress is the fact that every term is either in some canonical form, or can be
reduced further.
First, we will restate \cref{thm:cp-progress-link} and \cref{thm:cp-progress-beta},
which relate evaluation prefixes and evaluation contexts, and show that our
extension preserves these properties. 
\begin{lemmaB}\label{thm:nc-progress-link}
  If $\seq[{ \cpPlug{G}{P_1 \dots P_n} }]{ \Gamma }$, and some \tm{P_i} is a
  link \tm{\cpLink{x}{y}}, then either $\tm{\cpPlug{G}{P_1 \dots P_n}} =
  \tm{\cpLink{x}{y}}$, or there exist \tm{H}, \tm{H'} and \tm{Q} such that
  \(
  \tm{\cpPlug{G}{P_1 \dots P_n}} \equiv
  \tm{\cpPlug{H}{\cpCut{z}{\cpPlug{H'}{\cpLink{x}{y}}}{Q}}}
  \)
  with either $\tm{z} = \tm{x}$ or $\tm{z} = \tm{y}$.
\end{lemmaB}
\begin{proof}
  \textcolor{red}{\bfseries TODO}
\end{proof}
%%% Local Variables:
%%% TeX-master: "main"
%%% End:
\begin{lemmaB}\label{thm:nc-progress-beta}
  If $\seq[{ \cpPlug{G}{P_1 \dots P_n} }]{ \Gamma }$, and some \tm{P_i} and
  \tm{P_j} (with $i \neq j$) act on the same channel \tm{x}, then there exist \tm{H},
  \tm{H_i} and \tm{H_j} such that 
  \(
  \tm{\cpPlug{G}{P_1 \dots P_n}} =
  \tm{\cpPlug{H}{\cpCut{x}{\cpPlug{H_i}{P_i}}{\cpPlug{H_j}{P_j}}}}
  \).
\end{lemmaB}
\begin{proof}
  \textcolor{red}{\bfseries TODO}
\end{proof}
%%% Local Variables:
%%% TeX-master: "main"
%%% End:

In essence, \cref{thm:nc-progress-link} and \cref{thm:nc-progress-beta} cover
the cases in which either \cpRedAxCut1 or a \textbeta-reduction rule will be
applied.
However, after applying \cref{thm:nc-progress-beta}, we cannot immediately apply
\ncRedBetaStar{n+1}. For that, we must uncover at least one layer of pooling.
We prove a lemma which states that if we have an interaction on a shared channel
\tm{x}, we can push all pooling rules which pool clients communicating on \tm{x}
inwards. 
\begin{lemmaB}\label{thm:nc-progress-shared}
  If $\seq[{ \cpPlug{G}{P} }]{ \Gamma, \tmty{x}{\take[n]{A}} }$ and
  $\freeIn{x}{P}$, then there exists an $\tm{H}$ and $\tm{R_1}\dots\tm{R_{n-1}}$
  such that $\tm{\cpPlug{G}{P}} \equiv
  \tm{\cpPlug{H}{\ncPool{P}{\ncPool{R_1}{\ncPool{\dots}{R_{n-1}}\dots}}}}$,
  where $\notFreeIn{x}{H}$ and $\freeIn{x}{R_1},\dots,\freeIn{x}{R_{n-1}}$.
\end{lemmaB}
\begin{proof}
  By induction on the structure of the evaluation context \tm{G}.
  \begin{itemize}
  \item
    Case \tm{\Box}. By reflexivity.
  \item
    Case \tm{\cpCut{y}{G}{R}}. \\
    Case \tm{\cpCut{y}{R}{G}}.
    \\
    By the induction hypothesis.
  \item
    Case \tm{\ncPool{G}{R}}. There are two cases:
    \begin{itemize}
    \item Case $\freeIn{x}{R}$.
      \begin{flalign*}
        \begin{array}{l}
          \tm{\ncPool{\cpPlug{G}{P}}{R_{n-1}}} \\
          \qquad \equiv \quad \text{by the induction hypothesis} \\ 
          \tm{\ncPool{\cpPlug{H}{
          \ncPool{P}{\ncPool{R_1}{
          \ncPool{\dots}{R_{n-2}}\dots}}}}{R_{n-1}}} \\
          \qquad \equiv \quad \text{by \cref{thm:nc-display-pool-1}} \\
          \tm{\cpPlug{H}{\ncPool{P}{
          \ncPool{R_1}{\ncPool{
          \dots}{\ncPool{R_{n-2}}{R_{n-1}}}\dots}}}}
        \end{array}
      \end{flalign*}
    \item Case $\notFreeIn{x}{R}$. By the induction hypothesis.
    \end{itemize}
  \item Case \tm{\ncPool{R}{G}}. There are two cases:
    \begin{itemize}
    \item Case $\freeIn{x}{R}$.
      \begin{flalign*}
        \begin{array}{ll}
          \tm{\ncPool{R_{n-1}}{\cpPlug{G}{P}}} \\
          \qquad \equiv \quad \text{by} \; \ncEquivPoolComm \\ 
          \tm{\ncPool{\cpPlug{G}{P}}{R_{n-1}}} \\
          \qquad \equiv \quad \text{by the induction hypothesis} \\ 
          \tm{\ncPool{\cpPlug{H}{
          \ncPool{P}{\ncPool{R_1}{
          \ncPool{\dots}{R_{n-2}}\dots}}}}{R_{n-1}}} \\
          \qquad \equiv \quad \text{by \cref{thm:nc-display-pool-1}} \\
          \tm{\cpPlug{H}{\ncPool{P}{
          \ncPool{R_1}{\ncPool{
          \dots}{\ncPool{R_{n-2}}{R_{n-1}}}\dots}}}}
        \end{array}
      \end{flalign*}
    \item Case $\notFreeIn{x}{R}$. By the induction hypothesis.
    \end{itemize}
  \end{itemize}
\end{proof}
%%% Local Variables:
%%% TeX-master: "main"
%%% End:

Finally, we are ready to extend our proof of progress. The overall structure of
the proof remains the same, though the addition of pooling makes the wording
slightly more subtle.
\begin{theorem}[Progress]\label{thm:nc-progress}
  If $\seq[{ P }]{ \Gamma }$, then $\tm{P}$ is in canonical form, or there
  exists a $\tm{P'}$ s.t.\ $\reducesto{P}{P'}$.
\end{theorem}
\begin{proof}
  By induction on the structure of derivation for $\seq[{ P }]{ \Gamma }$.
  There only interesting cases are when the last rule of the derivation is
  \textsc{Cut} or \textsc{Pool}. In every other case, the typing rule constructs
  a term in which is in canonical form. 
  \\
  If the last rule in the derivation is \textsc{Cut} or \textsc{Pool}, we
  consider the prefix of the derivation for $\seq[{ P }]{ \Gamma}$ which
  consists of all top-level cuts and pooling rules. A prefix of $n$ cuts and $m$
  pooling rules introduces $n$ variables, but composes $n+m+1$ actions, at most
  $m+1$ of which are on the same side of all cut rules.
  Therefore, one of the following must be true:
  \begin{itemize}
  \item
    One of these actions was introduced by an application of \textsc{Ax}.
    \\
    We proceed as in \cref{thm:cp-progress}. 
  \item
    Two of these actions, on different sides of a \textsc{Cut}, act on the same
    channel. Let us name these processes \tm{P_i} and \tm{P_j}, and their shared
    channel \tm{y}. We have
    $\tm{P} = \tm{\cpPlug{G}{\cpCut{y}{\cpPlug{H_i}{P_i}}{\cpPlug{H_j}{P_j}}}}$.
    We distinguish the following cases:
    \begin{itemize}
    \item
      We have either
      $\seq[{ \cpPlug{H_i}{P_i}}]{ \Delta, \tmty{y}{\take[n]{A}} }$ or
      $\seq[{ \cpPlug{H_j}{P_j} }]{\Delta, \tmty{y}{\take[n]{A}} }$.
      \\
      We rewrite by \cref{thm:nc-display-3}, then apply one of \ncRedBetaStar{1}
      and \ncRedBetaStar{n+1}. 
    \item
      Otherwise, we can infer $\notFreeIn{y}{H_i}$ and $\notFreeIn{y}{H_j}$.
      \\
      We proceed as in \cref{thm:cp-progress}. 
    \end{itemize}
  \item
    The process is in canonical form \tm{
      \ncPool{\ncCnt{x_1}{y_1}{P_1}}{
        \ncPool{\dots}{\ncCnt{x_n}{y_n}{P_n}}\dots}}. 
  \item 
    Otherwise (at least) one of the actions acts on a free variable.
    \\
    We apply one of the commutative conversions.
  \end{itemize}
\end{proof}
%%% Local Variables:
%%% TeX-master: "main"
%%% End:


\subsection{Termination}
Termination is the fact that if we iteratively apply progress to obtain a
reduction, and apply that reduction, we will eventually end up with a term in
canonical form.
We restate its proof here for the sake of completeness, but its wording is
unchanged, modulo references to figures.
\begin{theorem}[Termination]\label{thm:nc-termination}
  If $\seq[{ P }]{ \Gamma }$, then there are no infinite $\Longrightarrow$
  reduction sequences.
\end{theorem}
\begin{proof}
  Every reduction reduces a single cut to zero, one or two cuts.
  However, each of these cuts is \emph{smaller}, in the sense that the type of
  the channel on which the communication takes place is smaller.
  Each reduction either eliminates a connective, or decreases a resource index
  on the type of a shared channel.
  See \cref{fig:cp-preservation-1,fig:nc-preservation-1}.
  Furthermore, each instance of the structural congruence preserves the size
  of the cut---see \cref{fig:cp-preservation-equiv,fig:nc-preservation-equiv}.
  Therefore, there cannot be an infinite $\Longrightarrow$ reduction sequence.
\end{proof}
%%% Local Variables:
%%% TeX-master: "main"
%%% End:

\section{\nodcap and non-deterministic local choice}\label{sec:nc-local-choice}
In \cref{sec:local-choice}, we discussed the non-deterministic local choice
operator, which is used in several extensions of \piDILL and
\cp~\cite{atkey2016,caires2014,caires2017}.
This operator is admissible in \nodcap.
We can derive the non-deterministic choice \tm{P+Q} by constructing the
following term:
\[
  \arraycolsep=0pt
  \tm{
  \begin{array}{lrlrl}
    \nu x.&((  & \; \ncCnt{x}{y}{\cpInr{y}{\cpHalt{y}}} \\
          &\mid& \; \ncCnt{x}{z}{\cpInr{z}{\cpHalt{z}}} \; )\\
          &\mid& \; \ncSrv{x}{y}{\ncSrv{x}{z}{}}\text{case}\;y\;
          &\;\{& \; \cpWait{y}{\cpCase{y}{\cpWait{y}{P}}{\cpWait{y}{P}}}\\
          &&& ;& \; \cpWait{y}{\cpCase{y}{\cpWait{y}{Q}}{\cpWait{y}{Q}}}\;\}\;)
  \end{array}
  }
\]
The term is a cut between two processes.
Both sides are well-typed, see~\cref{fig:nc-local-choice}.
\begin{sidewaysfigure}[hb]
  \begin{landscape}
    \begin{prooftree}
      \AXC{}
      \SYM{(\one)}
      \UIC{$\seq[{ \cpHalt{y} }]{ \tmty{y}{\one} }$}
      \SYM{(\plus_1)}
      \UIC{$\seq[{ \cpInl{y}{\cpHalt{y}} }]{ \tmty{y}{\one \plus \one} }$}
      \SYM{(\take[1]{})}
      \UIC{$\seq[{ \ncCnt{x}{y}{\cpInr{y}{\cpHalt{y}}} }]{ \tmty{x}{\take[1]{\one \plus \one}} }$}
      
      \AXC{}
      \SYM{(\one)}
      \UIC{$\seq[{ \cpHalt{z} }]{ \tmty{z}{\one} }$}
      \SYM{(\plus_2)}
      \UIC{$\seq[{ \cpInr{z}{\cpHalt{z}} }]{ \tmty{z}{\one \plus \one} }$}
      \SYM{(\take[1]{})}
      \UIC{$\seq[{ \ncCnt{x}{z}{\cpInr{z}{\cpHalt{z}}} }]{ \tmty{x}{\take[1]{\one \plus \one}} }$}
      
      \NOM{Pool}
      \BIC{$\seq[{
          \ncPool{\ncCnt{x}{y}{\cpInr{y}{\cpHalt{y}}}}{\ncCnt{x}{y}{\cpInr{y}{\cpHalt{y}}}}
        }]{
          \tmty{x}{\take[2]{\one \plus \one}}
        }$}
    \end{prooftree}
    \vspace*{1\baselineskip}
    \begin{prooftree}
      \AXC{}
      \SYM{(\one)}
      \UIC{$\seq[{ \cpHalt{w} }]{ \tmty{w}{\one} }$}
      \SYM{(\bot)}
      \AXC{$\seq[{ P }]{ \Gamma }$}
      \SYM{(\bot)} 
      \UIC{$\seq[{ \cpWait{w}{P} }]{ \Gamma, \tmty{w}{\bot} }$}
      
      \AXC{$\seq[{ P }]{ \Gamma }$}
      \SYM{(\bot)} 
      \UIC{$\seq[{ \cpWait{y}{P} }]{ \Gamma, \tmty{y}{\bot} }$}
      \SYM{(\with)}
      \BIC{$\seq[{ \cpCase{y}{\cpWait{y}{P}}{\cpWait{y}{P}} }]{
          \Gamma, \tmty{y}{\bot\with\bot} }$}
      \SYM{(\bot)} 
      \UIC{$\seq[{ \cpWait{y}{\cpCase{y}{\cpWait{y}{P}}{\cpWait{y}{P}}} }]{
          \Gamma, \tmty{y}{\bot\with\bot}, \tmty{y}{\bot} }$}
      
      \AXC{$\seq[{ Q }]{ \Gamma }$}
      \SYM{(\bot)} 
      \UIC{$\seq[{ \cpWait{z}{Q} }]{ \Gamma, \tmty{z}{\bot} }$}
      \AXC{$\seq[{ Q }]{ \Gamma }$}
      \SYM{(\bot)} 
      \UIC{$\seq[{ \cpWait{z}{Q} }]{ \Gamma, \tmty{z}{\bot} }$}
      \SYM{(\with)}
      \BIC{$\seq[{ \cpCase{z}{\cpWait{z}{Q}}{\cpWait{z}{Q}} }]{
          \Gamma, \tmty{z}{\bot\with\bot} }$}
      \SYM{(\bot)} 
      \UIC{$\seq[{ \cpWait{z}{\cpCase{z}{\cpWait{z}{Q}}{\cpWait{z}{Q}}} }]{
          \Gamma, \tmty{z}{\bot\with\bot}, \tmty{y}{\bot} }$}
      
      \SYM{(\with)}
      \BIC{$\seq[{
          \cpCase{y}{\cpWait{y}{\cpCase{z}{\cpWait{z}{P}}{\cpWait{z}{P}}}}{\cpWait{y}{\cpCase{z}{\cpWait{z}{Q}}{\cpWait{z}{Q}}}}
        }]{ \Gamma, \tmty{z}{\bot\with\bot}, \tmty{y}{\bot\with\bot} }$} 
      \SYM{(\give[1]{})}
      \UIC{$\seq[{
          \ncSrv{x'}{z}{\cpCase{y}{\cpWait{y}{\cpCase{z}{\cpWait{z}{P}}{\cpWait{z}{P}}}}{\cpWait{y}{\cpCase{z}{\cpWait{z}{Q}}{\cpWait{z}{Q}}}}}
        }]{ \Gamma, \tmty{x'}{\give[1]{\bot\with\bot}}, \tmty{y}{\bot\with\bot} }$}
      \SYM{(\give[1]{})}
      \UIC{$\seq[{
          \ncSrv{x}{y}{\ncSrv{x'}{z}{\cpCase{y}{\cpWait{y}{\cpCase{z}{\cpWait{z}{P}}{\cpWait{z}{P}}}}{\cpWait{y}{\cpCase{z}{\cpWait{z}{Q}}{\cpWait{z}{Q}}}}}}
        }]{ \Gamma, \tmty{x'}{\give[1]{\bot\with\bot}}, \tmty{x}{\give[1]{\bot\with\bot}} }$} 
      \NOM{Cont}
      \UIC{$\seq[{
          \ncSrv{x}{y}{\ncSrv{x}{z}{\cpCase{y}{\cpWait{y}{\cpCase{z}{\cpWait{z}{P}}{\cpWait{z}{P}}}}{\cpWait{y}{\cpCase{z}{\cpWait{z}{Q}}{\cpWait{z}{Q}}}}}}
        }]{ \Gamma, \tmty{x}{\give[1]{\bot\with\bot}}, \tmty{x}{\give[1]{\bot\with\bot}} }$} 
    \end{prooftree}
  \end{landscape}
  \caption{Derivations for the encoding of non-deterministic local choice in \nodcap.}
  \label{fig:nc-local-choice}
\end{sidewaysfigure}
%%% Local Variables:
%%% TeX-master: "main"
%%% End:

Let us unpack what each side is doing. 
On the left-hand side, we have a pool of two processes,
\tm{\ncCnt{x}{y}{\cpInr{y}{\cpHalt{y}}}} and \tm{\ncCnt{x}{z}{\cpInr{z}{\cpHalt{z}}}}.
Each makes a choice---the first sends \tm{\text{inl}}, the second sends \tm{\text{inr}}.
On the right-hand side, we have a server with both \tm{P} and \tm{Q}.
This server has two channels on which a choice is offered, \tm{y} and
\tm{z}. However, the choice on \tm{z} does not affect the outcome of the
process.
When these clients and the server are put together, the choices offered by the
server will be non-deterministically lined up with the clients which make
choices, and either \tm{P} or \tm{Q} will run.

While there is a certain amount of overhead involved in this encoding, it scales
linearly in terms of the number of processes.
The reverse---encoding the non-determinism present in \nodcap using
non-deterministic local choice---scales exponentially, as with the \textpi-calculus.

Nonetheless, it is worrying that we duplicate each program \tm{P} and \tm{Q} in
order to encode non-deterministic local choice.
However, we can replace the each term of the form \tm{\cpCase{y}{\cpWait{y}{P}}{\cpWait{y}{P}}}
with \tm{\cpCut{w}{\cpCase{z}{\cpWait{z}{\cpHalt{w}}}{\cpWait{z}{\cpHalt{w}}}}{\cpWait{w}{P}}}.
This process is also well-typed, see below.
\begin{scprooftree}
  \AXC{}
  \SYM{(\one)}
  \UIC{$\seq[{ \cpHalt{w} }]{ \tmty{w}{\one} }$}
  \SYM{(\bot)}
  \UIC{$\seq[{ \cpWait{z}{\cpHalt{w}} }]{ \tmty{w}{\one}, \tmty{z}{\bot} }$} 
  \AXC{}
  \SYM{(\one)}
  \UIC{$\seq[{ \cpHalt{w} }]{ \tmty{w}{\one} }$}
  \SYM{(\bot)}
  \UIC{$\seq[{ \cpWait{z}{\cpHalt{w}} }]{ \tmty{w}{\one}, \tmty{z}{\bot} }$} 
  \SYM{(\with)}
  \BIC{$\seq[{
      \cpCase{z}{\cpWait{z}{\cpHalt{w}}}{\cpWait{z}{\cpHalt{w}}}
    }]{ \tmty{w}{\one}, \tmty{z}{\bot\with\bot} }$} 
  \AXC{$\seq[{ P }]{ \Gamma }$}
  \SYM{(\bot)} 
  \UIC{$\seq[{ \cpWait{w}{P} }]{ \Gamma, \tmty{w}{\bot} }$}
  \NOM{Cut}
  \BIC{$\seq[{
      \cpCut{w}{\cpCase{z}{\cpWait{z}{\cpHalt{w}}}{\cpWait{z}{\cpHalt{w}}}}{\cpWait{w}{P}}
    }]{ \Gamma, \tmty{z}{\bot\with\bot} }$} 
\end{scprooftree}
%%% Local Variables:
%%% TeX-master: "main"
%%% End:
