%%% The π-calculus term syntax:
\newcommand{\piCalc}[0]{\textpi-calculus\xspace}
\newcommand{\piSend}[3]{\ensuremath{\overline{#1}\langle #2 \rangle.#3}}
\newcommand{\piRecv}[3]{\ensuremath{#1(#2).#3}}
\newcommand{\piPar}[2]{\ensuremath{#1 \mid #2}}
\newcommand{\piNew}[2]{\ensuremath{(\nu #1)#2}}
\newcommand{\piRepl}[1]{\ensuremath{!#1}}
\newcommand{\piHalt}[0]{\ensuremath{0}}
\newcommand{\piSub}[3]{\ensuremath{#3\{#1/#2\}}}

%%% The CP term syntax:
\newcommand{\cp}{CP\xspace}
\newcommand{\cpLink}[2]{\ensuremath{#1{\leftrightarrow}#2}}
\newcommand{\cpCut}[3]{\ensuremath{\nu #1.(\piPar{#2}{#3})}}
\newcommand{\cpSend}[4]{\ensuremath{#1[#2].(\piPar{#3}{#4})}}
\newcommand{\cpRecv}[3]{\ensuremath{\piRecv{#1}{#2}{#3}}}
\newcommand{\cpWait}[2]{\ensuremath{#1().#2}}
\newcommand{\cpHalt}[1]{\ensuremath{#1[].0}}
\newcommand{\cpInl}[2]{\ensuremath{#1[\text{inl}].#2}}
\newcommand{\cpInr}[2]{\ensuremath{#1[\text{inr}].#2}}
\newcommand{\cpCase}[3]{\ensuremath{\text{case}\;#1\;\{#2;#3\}}}
\newcommand{\cpAbsurd}[1]{\ensuremath{\text{case}\;#1\;\{\}}}
\newcommand{\cpSub}[3]{\ensuremath{\piSub{#1}{#2}{#3}}}

%%% The CP reduction rules:
\newcommand{\cpEquivCutComm}{\ensuremath{(\nu\text{-comm})}\xspace}
\newcommand{\cpEquivCutAssNoParen}[1]{\ensuremath{\nu\text{-assoc}_{#1}}\xspace}
\newcommand{\cpEquivCutAss}[1]{\ensuremath{(\cpEquivCutAssNoParen{#1})}\xspace}
\newcommand{\cpRedAxCut}[1]{\ensuremath{(\text{AxCut}_{#1})}\xspace}
\newcommand{\cpRedBetaTensParr}{\ensuremath{(\beta{\tens}{\parr})}\xspace}
\newcommand{\cpRedBetaOneBot}{\ensuremath{(\beta{\one}{\bot})}\xspace}
\newcommand{\cpRedBetaPlusWith}[1]{\ensuremath{(\beta{\plus}{\with}_{#1})}\xspace}
\newcommand{\cpRedKappaTens}[1]{\ensuremath{(\kappa{\tens}_{#1})}\xspace}
\newcommand{\cpRedKappaParr}{\ensuremath{(\kappa{\parr})}\xspace}
\newcommand{\cpRedKappaBot}{\ensuremath{(\kappa{\bot})}\xspace}
\newcommand{\cpRedKappaPlus}[1]{\ensuremath{(\kappa{\plus}_{#1)}}\xspace}
\newcommand{\cpRedKappaWith}{\ensuremath{(\kappa{\with})}\xspace}
\newcommand{\cpRedKappaTop}{\ensuremath{(\kappa{\top})}\xspace}
\newcommand{\cpRedGammaCut}{\ensuremath{(\gamma\nu)}\xspace}
\newcommand{\cpRedGammaEquiv}{\ensuremath{(\gamma{\equiv})}\xspace}

%%% The nodcap term syntax:
\newcommand{\nc}{\ensuremath{\text{CP}_{\text{ND}}}\xspace}
\newcommand{\ncExpn}[3]{\ensuremath{#1\uparrow #2.#3}}
\newcommand{\ncIntl}[3]{\ensuremath{#1\downarrow #2.#3}}
\newcommand{\ncSrv}[3]{\ensuremath{{\star}{#1}(#2).#3}}
\newcommand{\ncCnt}[3]{\ensuremath{{\star}{#1}[#2].#3}}
\newcommand{\ncPool}[2]{\ensuremath{(\piPar{#1}{#2})}}
\newcommand{\ncCont}[3]{\ensuremath{\cpSub{#1}{#2}{#3}}}

%%% The nodcap reduction rules:
\newcommand{\ncEquivPoolComm}{\ensuremath{(|\text{-comm})}\xspace}
\newcommand{\ncEquivPoolAssNoParen}[1]{\ensuremath{|\text{-assoc}_#1}\xspace}
\newcommand{\ncEquivPoolAss}[1]{\ensuremath{(\ncEquivPoolAssNoParen{#1})}\xspace}
\newcommand{\ncEquivCutCong}{\ensuremath{(\nu\text{-cong})}\xspace}
\newcommand{\ncRedBetaStar}[1]{\ensuremath{(\beta{\star}_{#1})}\xspace}
\newcommand{\ncRedKappaTake}{\ensuremath{(\kappa{\take[1]{}})}\xspace}
\newcommand{\ncRedKappaGive}{\ensuremath{(\kappa{\give[1]{}})}\xspace}
\newcommand{\ncRedKappaPool}{\ensuremath{(\kappa{|})}\xspace}
\newcommand{\ncRedGammaPool}{\ensuremath{(\gamma{|})}\xspace}

%%% Local Variables:
%%% TeX-master: "main"
%%% End:
