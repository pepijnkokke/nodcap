%% Background
\chapter{Background}\label{sec:background}
\section{Classical Processes}\label{sec:cp}
In this section, we will discuss a rudimentary subset of the typed process
calculus \cp~\cite{wadler2012}, which we will refer to as \rcp.
We have chosen to discuss only a subset in order to keep our later discussion of
our extension to \cp in \cref{sec:main} as simple as possible.
\rcp is the subset which corresponds to rudimentary linear
logic~\cite[RLL]{girard1992}, also known as multiplicative-applicative linear
logic. 
However, we foresee no problems in extending the proofs from \cref{sec:main} to
cover the remaining features of \cp, polymophism and the exponentials $\ty{!A}$
and $\ty{?A}$. 

We will deviate from the formulation of \cp in one more way. In our formulation
of \rcp, we will leave out the commutative conversions of \cp, in order to
obtain a tighter correspondence with the reduction semantics of the
\textpi-calculus.

This section will proceed as follows.
In \cref{sec:cp-terms-and-types}, we will discuss the terms, the structural
congruence, and the types of \rcp. 
In \cref{sec:cp-dependence,sec:cp-choice,sec:cp-duality}, we will discuss the
terms and their corresponding types, in small groups, together with their typing
and reduction rules. 
In \cref{sec:cp-properties}, we will prove preservation, progress and
termination for \rcp.
And finally, in \cref{sec:cp-correspondence}, we will define an extended version
of the reduction relation, which corresponds to cut elimination in linear logic.

\subsection{Terms and types}\label{sec:cp-terms-and-types}
The term language for \rcp is a variant of the
\textpi-calculus~\cite{milner1992b}.
Its terms are defined by the following grammar:
\begin{definition}[Terms]\label{def:cp-terms}
  \[\!
    \begin{aligned}
      \tm{P}, \tm{Q}, \tm{R}
           :=& \; \tm{\cpLink{x}{y}}       &&\text{link}
      \\ \mid& \; \tm{\cpCut{x}{P}{Q}}     &&\text{parallel composition, or ``cut'''}
      \\ \mid& \; \tm{\cpSend{x}{y}{P}{Q}} &&\text{``output''}
      \\ \mid& \; \tm{\cpRecv{x}{y}{P}}    &&\text{``input''}
      \\ \mid& \; \tm{\cpHalt{x}}          &&\text{halt}
      \\ \mid& \; \tm{\cpWait{x}{P}}       &&\text{wait}
      \\ \mid& \; \tm{\cpInl{x}{P}}        &&\text{select left choice}
      \\ \mid& \; \tm{\cpInr{x}{P}}        &&\text{select right choice}
      \\ \mid& \; \tm{\cpCase{x}{P}{Q}}    &&\text{offer binary choice}
      \\ \mid& \; \tm{\cpAbsurd{x}}        &&\text{offer nullary choice}
    \end{aligned}
  \]  
\end{definition}
%%% Local Variables:
%%% TeX-master: "main"
%%% End:

The construct \tm{\cpLink{x}{y}} links two
channels~\cite{sangiorgi1996,boreale1998}, forwarding messages received on
\tm{x} to \tm{y} and vice versa.
The construct \tm{\cpCut{x}{P}{Q}} creates a new channel \tm{x}, and composes
two processes, \tm{P} and \tm{Q}, which communicate on \tm{x}, in parallel.
Therefore, in \tm{\cpCut{x}{P}{Q}} the name \tm{x} is bound in both \tm{P} and
\tm{Q}.
In \tm{\cpRecv{x}{y}{P}} and \tm{\cpSend{x}{y}{P}{Q}}, round brackets are used
for input, square brackets for output.
We use bound output~\cite{sangiorgi1996}.
This means that unlike in the \textpi-calculus, both input and output bind a new
name.
In \tm{\cpRecv{x}{y}{P}} the new name \tm{y} is bound in \tm{P}.
In \tm{\cpSend{x}{y}{P}{Q}}, the new name \tm{y} is only bound in \tm{P}, while
\tm{x} is only bound in \tm{Q}.

Terms in \rcp are identified up to structural congruence, which states that
parallel compositions \tm{\cpCut{x}{P}{Q}} are associative and commutative.
It is defined as follows:
\begin{definition}[Structural congruence]\label{def:cp-equiv}
  We define the structural congruence $\equiv$ as a reflexive, transitive
  congruence over terms which satisfies the following additional axioms:
  \[
    \begin{array}{llll}
      \cpEquivLinkComm
      & \tm{\cpLink{x}{y}}
      & \equiv \;
      & \tm{\cpLink{y}{x}}
      \\
      \cpEquivCutComm
      & \tm{\cpCut{x}{P}{Q}}
      & \equiv \;
      & \tm{\cpCut{x}{Q}{P}}
      \\
      \cpEquivCutAss1
      & \tm{\cpCut{x}{P}{\cpCut{y}{Q}{R}}}
      & \equiv \;
      & \tm{\cpCut{y}{\cpCut{x}{P}{Q}}{R}}
        \quad \text{if} \; \notFreeIn{x}{R} \; \text{and} \; \notFreeIn{y}{P}
    \end{array}
  \]
\end{definition}
%%% Local Variables:
%%% TeX-master: "main"
%%% End:

We do not add an axiom for \cpEquivCutAss2, as it follows from
\cref{def:cp-equiv}, see~\cref{thm:cp-cut-assoc2}.
Note that throughout this thesis, we will leave uses of the transitivity and
congruence rules implicit.
\begin{lemma}[\cpEquivCutAssNoParen2]\label{thm:cp-cut-assoc2}
  If $\tm{x}\not\in\tm{R}$ and $\tm{y}\not\in\tm{P}$, then 
  \(
    \tm{\cpCut{y}{\cpCut{x}{P}{Q}}{R}} \equiv
    \tm{\cpCut{x}{P}{\cpCut{y}{Q}{R}}}
  \).
\end{lemma}
  \begin{proof}
    \begin{align*}
      \tm{\cpCut{y}{\cpCut{x}{P}{Q}}{R}} &\equiv \qquad \text{by \cpEquivCutComm} \\
      \tm{\cpCut{y}{\cpCut{x}{Q}{P}}{R}} &\equiv \qquad \text{by \cpEquivCutComm} \\
      \tm{\cpCut{y}{R}{\cpCut{x}{Q}{P}}} &\equiv \qquad \text{by \cpEquivCutAss1} \\
      \tm{\cpCut{x}{\cpCut{y}{R}{Q}}{P}} &\equiv \qquad \text{by \cpEquivCutComm} \\
      \tm{\cpCut{x}{P}{\cpCut{y}{R}{Q}}} &\equiv \qquad \text{by \cpEquivCutComm} \\
      \tm{\cpCut{x}{P}{\cpCut{y}{Q}{R}}}
    \end{align*}
    The side conditions for \cpEquivCutAss1 are given.
  \end{proof}
%%% Local Variables:
%%% TeX-master: "main"
%%% End:

Furthermore, structural congruence is a symmetric relation.
\begin{theorem}[Symmetry]\label{thm:cp-symmetry}
  If $\tm{P} \equiv \tm{Q}$, then $\tm{Q} \equiv \tm{P}$.
\end{theorem}
\begin{proof}
  By induction on the structure of the equivalence proof.
\end{proof}
%%% Local Variables:
%%% TeX-master: "main"
%%% End:

%
Channels in \rcp are typed using a session type system which corresponds to RLL,
the multiplicative, additive fragment of linear logic.
These are defined using the following grammar:
\begin{definition}[Types]\label{def:cp-types}
  \[\!
    \begin{aligned}
      \ty{A}, \ty{B}, \ty{C}
           :=& \; \ty{A \tens B} &&\text{pair of independent processes}
      \\ \mid& \; \ty{A \parr B} &&\text{pair of interdependent processes}
      \\ \mid& \; \ty{\one}      &&\text{unit for} \; {\tens}
      \\ \mid& \; \ty{\bot}      &&\text{unit for} \; {\parr}
      \\ \mid& \; \ty{A \plus B} &&\text{internal choice}
      \\ \mid& \; \ty{A \with B} &&\text{external choice}
      \\ \mid& \; \ty{\nil}      &&\text{unit for} \; {\plus}
      \\ \mid& \; \ty{\top}      &&\text{unit for} \; {\with}
    \end{aligned}
  \]  
\end{definition}
%%% Local Variables:
%%% TeX-master: "main"
%%% End:

Duality plays a crucial role in both linear logic and session types.
In \cp, the two endpoints of a channel are assigned dual types.
This ensures that, for instance, whenever a process \emph{sends} across a
channel, the process on the other end of that channel is waiting to
\emph{receive}.
Each type \ty{A} has a dual, written \ty{A^\bot}, which is defined as follows:
\begin{definition}[Negation]\label{def:cp-negation}
  \[\!
    \begin{array}{lclclcl}
              \ty{(A \tens B)^\bot} &=& \ty{A^\bot \parr B^\bot}
      &\quad& \ty{\one^\bot}        &=& \ty{\bot}
      \\      \ty{(A \parr B)^\bot} &=& \ty{A^\bot \tens B^\bot}
      &\quad& \ty{\bot^\bot}        &=& \ty{\one}
      \\      \ty{(A \plus B)^\bot} &=& \ty{A^\bot \with B^\bot}
      &\quad& \ty{\nil^\bot}        &=& \ty{\top}
      \\      \ty{(A \with B)^\bot} &=& \ty{A^\bot \plus B^\bot}
      &\quad& \ty{\top^\bot}        &=& \ty{\nil}
    \end{array}
  \]
\end{definition}
%%% Local Variables:
%%% TeX-master: "main"
%%% End:

Duality is an involutive function.
\begin{lemma}[Involutive]\label{thm:negation-involutive}
  We have $\ty{A^{\bot\bot}} = \ty{A}$.
\end{lemma}
\begin{proof}
  By induction on the structure of the type $\ty{A}$.
\end{proof}
%%% Local Variables:
%%% TeX-master: "main"
%%% End:

%
Environments associate channels with types. They are defined as follows:
\begin{definition}[Environments]\label{def:cp-environments}
  We define environments as follows:
  \[
    \ty{\Gamma}, \ty{\Delta}, \ty{\Theta}
    ::= \tmty{x_1}{A_1}\dots\tmty{x_n}{A_n}
  \] 
  Names in environments must be unique, and environments \ty{\Gamma} and
  \ty{\Delta} can only be combined as $\ty{\Gamma}, \ty{\Delta}$ if
  $\text{fv}(\ty{\Gamma}) \cap \text{fv}(\ty{\Delta}) = \varnothing$. 
\end{definition}
%%% Local Variables:
%%% TeX-master: "main"
%%% End:

Typing judgements associative processes with their collection of channels, and
enforce the communication protocols specified by the types of those channels.
They are defined as follows:
\begin{definition}[Typing judgements]\label{def:cp-typing-judgement}
  A typing judgement $\seq[{ P }]{\tmty{x_1}{A_1}\dots\tmty{x_n}{A_n}}$ denotes
  that the process \tm{P} communicates along channels $\tm{x_1}\dots\tm{x_n}$
  following protocols $\ty{A_1}\dots\ty{A_n}$.
  Typing judgements can be constructed using the inference rules in
  \cref{fig:cp-typing-judgement}.
\end{definition}
%%% Local Variables:
%%% TeX-master: "main"
%%% End:

\newcommand{\cpInfAx}{%
  \begin{prooftree*}
    \AXC{$\vphantom{\seq[ Q ]{ \Delta, \tmty{y}{A^\bot} }}$}
    \NOM{Ax}
    \UIC{$\seq[ \cpLink{x}{y} ]{ \tmty{x}{A}, \tmty{y}{A^\bot} }$}
  \end{prooftree*}
}
\newcommand{\cpInfCut}{%
  \begin{prooftree*}
    \AXC{$\seq[ P ]{ \Gamma, \tmty{x}{A} }$}
    \AXC{$\seq[ Q ]{ \Delta, \tmty{y}{A^\bot} }$}
    \NOM{Cut}
    \BIC{$\seq[ \cpCut{x}{P}{Q} ]{ \Gamma, \Delta }$}
  \end{prooftree*}
}
\newcommand{\cpInfTens}{%
  \begin{prooftree*}
    \AXC{$\seq[ P ]{ \Gamma , \tmty{y}{A} }$}
    \AXC{$\seq[ Q ]{ \Delta , \tmty{x}{B} }$}
    \SYM{(\tens)}
    \BIC{$\seq[ \cpSend{x}{y}{P}{Q} ]{ \Gamma , \Delta , \tmty{x}{A \tens B} }$}
  \end{prooftree*}
}
\newcommand{\cpInfParr}{%
  \begin{prooftree*}
    \AXC{$\seq[ P ]{ \Gamma , \tmty{y}{A} , \tmty{x}{B} }$}
    \SYM{(\parr)}
    \UIC{$\seq[ \cpRecv{x}{y}{P} ]{ \Gamma , \tmty{x}{A \parr B} }$}
  \end{prooftree*}
}
\newcommand{\cpInfOne}{%
  \begin{prooftree*}
    \AXC{$\vphantom{\seq[ P ]{ \Gamma }}$}
    \SYM{(\one)}
    \UIC{$\seq[ \cpHalt{x} ]{ \tmty{x}{\one} }$}
  \end{prooftree*}
}
\newcommand{\cpInfBot}{%
  \begin{prooftree*}
    \AXC{$\seq[ P ]{ \Gamma }$}
    \SYM{(\bot)}
    \UIC{$\seq[ \cpWait{x}{P} ]{ \Gamma , \tmty{x}{\bot} }$}
  \end{prooftree*}
}
\newcommand{\cpInfPlus}[1]{%
  \ifdim#1pt=1pt
  \begin{prooftree*}
    \AXC{$\seq[ P ]{ \Gamma , \tmty{x}{A} }$}
    \SYM{(\plus_1)}
    \UIC{$\seq[{ \cpInl{x}{P} }]{ \Gamma , \tmty{x}{A \plus B} }$}
  \end{prooftree*}
  \else%
  \ifdim#1pt=2pt
  \begin{prooftree*}
    \AXC{$\seq[ P ]{ \Gamma , \tmty{x}{B} }$}
    \SYM{(\plus_2)}
    \UIC{$\seq[ \cpInr{x}{P} ]{ \Gamma , \tmty{x}{A \plus B} }$}
  \end{prooftree*}
  \else%
  \fi%
  \fi%
}
\newcommand{\cpInfWith}{%
  \begin{prooftree*}
    \AXC{$\seq[ P ]{ \Gamma , \tmty{x}{A} }$}
    \AXC{$\seq[ Q ]{ \Delta , \tmty{x}{B} }$}
    \SYM{(\with)}
    \BIC{$\seq[ \cpCase{x}{P}{Q} ]{ \Gamma , \Delta , \tmty{x}{A \with B} }$}
  \end{prooftree*}
}
\newcommand{\cpInfNil}{%
  (no rule for \ty{\nil})
}
\newcommand{\cpInfTop}{%
  \begin{prooftree*}
    \AXC{}
    \SYM{(\top)}
    \UIC{$\seq[ \cpAbsurd{x} ]{ \Gamma, \tmty{x}{\top} }$}
  \end{prooftree*}
}
\begin{figure*}[b]
  \begin{center}
    \cpInfAx
    \cpInfCut
  \end{center}
  \begin{center}
    \cpInfTens
    \cpInfParr
  \end{center}
  \begin{center}
    \cpInfOne
    \cpInfBot
  \end{center}
  \begin{center}
    \cpInfPlus1
    \cpInfPlus2
  \end{center}
  \begin{center}
    \cpInfWith
  \end{center}
  \begin{center}
    \cpInfNil
    \cpInfTop
  \end{center}
  \caption{Typing judgement for the multiplicative applicative subset of \rcp.}
  \label{fig:cp-typing-judgement}
\end{figure*}
%%% Local Variables:
%%% TeX-master: "main"
%%% End:

Reductions relate processes with their reduced forms.
They are defined as follows:
\begin{definition}[Term reduction]\label{def:cp-term-reduction-1}
  A reduction $\reducesto{P}{P'}$ denotes that the process \tm{P} can reduce to
  the process \tm{P'} in a single step. Reductions can be constructed using the
  rules in~\cref{fig:cp-term-reduction-1,fig:cp-term-reduction-2}. 
\end{definition}
%%% Local Variables:
%%% TeX-master: "main"
%%% End:

\begin{figure*}[b]
  \[
    \begin{array}{llll}
      \cpRedAxCut1
      & \tm{\cpCut{x}{\cpLink{w}{x}}{P}}
      & \Longrightarrow \;
      & \tm{\cpSub{w}{x}{P}} 
      \\
      \cpRedAxCut2
      & \tm{\cpCut{x}{\cpLink{x}{w}}{P}}
      & \Longrightarrow \;
      & \tm{\cpSub{w}{x}{P}} 
      \\
      \cpRedBetaTensParr
      & \tm{\cpCut{x}{\cpSend{x}{y}{P}{Q}}{\cpRecv{x}{z}{R}}}
      & \Longrightarrow \;
      & \tm{\cpCut{y}{P}{\cpCut{x}{Q}{\cpSub{y}{z}{R}}}}
      \\
      \cpRedBetaOneBot
      & \tm{\cpCut{x}{\cpHalt{x}}{\cpWait{x}{P}}}
      & \Longrightarrow \;
      & \tm{P}
      \\
      \cpRedBetaPlusWith1
      & \tm{\cpCut{x}{\cpInl{x}{P}}{\cpCase{x}{Q}{R}}}
      & \Longrightarrow \;
      & \tm{\cpCut{x}{P}{Q}}
      \\
      \cpRedBetaPlusWith2
      & \tm{\cpCut{x}{\cpInr{x}{P}}{\cpCase{x}{Q}{R}}}
      & \Longrightarrow \;
      & \tm{\cpCut{x}{P}{R}}
    \end{array}
  \]
  \begin{prooftree}
    \AXC{\reducesto{P}{P^\prime}}
    \SYM{\cpRedGammaCut}
    \UIC{\reducesto{\cpCut{x}{P}{Q}}{\cpCut{x}{P^\prime}{Q}}}
  \end{prooftree}
  \begin{prooftree}
    \AXC{$\tm{P}\equiv\tm{Q}$}
    \AXC{\reducesto{Q}{Q^\prime}}
    \AXC{$\tm{Q^\prime}\equiv\tm{P^\prime}$}
    \SYM{\cpRedGammaEquiv}
    \TIC{\reducesto{P}{P^\prime}}
  \end{prooftree}
  \caption{Term reduction rules for \rcp.}
  \label{fig:cp-term-reduction-1}
\end{figure*}
%%% Local Variables:
%%% TeX-master: "main"
%%% End:

We will discuss the interpretations of each connective, together with their
typing and reduction rules, in
\cref{sec:cp-dependence,sec:cp-choice,sec:cp-duality}.

\subsection{Multiplicatives and in- and interdependence}
\label{sec:cp-dependence}
The multiplicatives ($\ty{\tens}, \ty{\parr}$) deal with independence and
interdependence:
\begin{itemize}
\item
  A channel of type \ty{A \tens B} represents a pair of channels, which
  communicate with two \emph{independent} processes---that is to say, two
  processes who share no channels.
  A process acting on a channel of type \ty{A \tens B} will send one endpoint of
  a fresh channel, and then split into a pair of independent processes.
  One of these processes will be responsible for an interaction of type \ty{A}
  over the fresh channel, while the other process continues to interact as
  \ty{B}.
\item
  A channel of type \ty{A \parr B} represents a pair of interdependent channels,
  which are used within a single process.
  A process acting on a channel of type \ty{A \parr B} will receive a channel to
  act on, and communicate on its channels in whatever order it pleases.
  This means that the usage of one channel can depend on that of
  another---e.g.\ the interaction of type \ty{B} could depend on the result of
  the interaction of type \ty{A}, or vise versa, and if \ty{A} and \ty{B} are
  complex types, their interactions could likewise interweave in complex ways.
\end{itemize}
While the rules for \ty{\tens} and \ty{\parr} introduce input and output
operations, these are inessential---the essential distinction lies two in the
fact that (\tens) composes two independent processes, and therefore \emph{must}
split the environment between them, whereas (\parr) uses a single process, which
then can---and must---use all the channels in the environment.
\begin{center}
  \cpInfTens
  \cpInfParr
\end{center}
The \textbeta-reduction rule for terms introduced by $(\tens)$ and $(\parr)$
implements the behaviour outlined above:
\[
  \tm{\cpCut{x}{\cpSend{x}{y}{P}{Q}}{\cpRecv{x}{z}{R}}}
  \Longrightarrow
  \tm{\cpCut{y}{P}{\cpCut{x}{Q}{\cpSub{y}{z}{R}}}}
\]
%
The rules for the multiplicative units ($\ty{\one}, \ty{\bot}$) follow the same
pattern, except for the nullary instead of the binary case:
\begin{itemize}
\item
  A term constructed by $(\one)$ must composes \emph{zero} independent
  processes, and thus must halt. Furthermore, it must be able to split its
  environment between zero processes, and thus its environment must be empty.
\item
  A term constructed by $(\bot)$, on the other hand, uses a single process,
  which is not further restricted.
\end{itemize}
Note that the rules for $\ty{\one}$ and $\ty{\bot}$ introduce a nullary send and
receive operation, such as those found in the polyadic \textpi-calculus.
\begin{center}
  \cpInfOne
  \cpInfBot
\end{center}
The \textbeta-reduction rule for terms introduced by $(\one)$ and $(\bot)$
implements the behaviour outlined above:
\[
  \tm{\cpCut{x}{\cpHalt{x}}{\cpWait{x}{P}}}
  \Longrightarrow
  \tm{P}
\]

\subsection{Additives and choice}\label{sec:cp-choice}
The additives ($\ty{\plus}, \ty{\with}$) deal with choice:
\begin{itemize}
\item
  A process acting on a channel of type \ty{A \plus B} either sends the value
  \tm{inl} to select an interaction of type \ty{A} or the value \tm{inr} to
  select one of type \ty{B}.
\item
  A process acting on a channel of type \ty{A \with B} receives such a value,
  and then offers an interaction of either type \ty{A} or \ty{B},
  correspondingly.
\end{itemize}
Note that, in essence, the additive operations implement the sending and
receiving of a single bit of information, \tm{inl} or \tm{inr}, and branching
based on the value of that bit.
The rule for constructing a process which sends \tm{inr}, $(\plus_2)$, has been
omitted, but can be found in~\cref{fig:cp-typing-judgement}.
\begin{center}
  \cpInfPlus1
  \cpInfWith
\end{center}
The \textbeta-reduction rules for terms introduced by $(\plus_1)$, $(\plus_2)$
and $(\with)$ implements the behaviour outlined above.
\[
  \begin{array}{c}
    \tm{\cpCut{x}{\cpInl{x}{P}}{\cpCase{x}{Q}{R}}} \Longrightarrow \tm{\cpCut{x}{P}{Q}}
    \\
    \tm{\cpCut{x}{\cpInr{x}{P}}{\cpCase{x}{Q}{R}}} \Longrightarrow \tm{\cpCut{x}{P}{R}}
  \end{array}
\]
%
The rules for the additive units ($\ty{\nil}, \ty{\top}$) follow the same
pattern, except for a nullary choice:
\begin{itemize}
\item
  There is \emph{no} rule for \ty{\nil}, as a process acting on a channel of
  that type would have to select one of \emph{zero} options, which is clearly
  impossible.
\item
  A process acting on a channel of type \ty{\top} will wait to receive a choice
  of out \emph{zero} options. Since this will clearly never arrive, we have two
  options: either we block, waiting forever, or we simply crash.
\end{itemize}
It may seem odd at first to include a type for the process which cannot possibly
exist, and for the process which waits forever, but these make sensible units
for choice.
When offered a choice of type \ty{A \plus \nil}, one can either choose to
interact as \ty{A}, or choose to commit to doing the impossible.
Similarly, when offering a choice of type \ty{A \with \top}, one can safely
implement the right branch with a process which waits forever, as no sound
process will ever be able to select that branch anyway.
\begin{center}
  \cpInfNil
  \cpInfTop
\end{center}
As there is no way to construct a process of type \ty{\nil}, there is no
reduction rule for the additive units.


\subsection{Structural rules and duality}\label{sec:cp-duality}
Duality plays a crucial role in session type systems.
In~\cref{sec:cp-choice}, we saw that duality ensures a process offering a choice
is always matched with a process making a choice.
In~\cref{sec:cp-dependence}, we saw that it is also crucial to deadlock freedom,
as it ensures that, for instance, a process which uses communication on \tm{x}
to decide what to send on \tm{y} is matched with a pair of independent processes
on \tm{x} and \tm{y}, preventing circular dependencies.

Duality appears in the typing rules for two \rcp term constructs.
Forwarding, \tm{\cpLink{x}{y}}, connects two dual channels with dual endpoints,
while composition, \tm{\cpCut{x}{P}{Q}}, composes two processes \tm{P} and
\tm{Q} with a shared channel \tm{x}, requiring that they follow dual protocols
on \tm{x}.
\begin{center}
  \cpInfAx
  \cpInfCut
\end{center}
There are two reduction rules which deal with the interactions between
forwarding and compositions. These implement the intuition that if a process is
meant to communicate on \tm{x}, \tm{x} is forwarding to \tm{y}, and nobody else
is listening on \tm{x}, then the process might as well start communicating on
\tm{y}.
\[
  \begin{array}{c}
    \tm{\cpCut{x}{\cpLink{w}{x}}{P}} \Longrightarrow \tm{\cpSub{w}{x}{P}} 
    \\
    \tm{\cpCut{x}{\cpLink{x}{w}}{P}} \Longrightarrow \tm{\cpSub{w}{x}{P}}  
  \end{array}
\]
Note that we can do this \emph{solely} because \rcp implements a binary session
type system, meaning that each communication has only two participants, and
therefore we know that no other process is communicating on \tm{x}.

\subsection{Properties of \rcp}\label{sec:cp-properties}
In this section, we will prove three important properties of \rcp, namely
preservation, progress and termination.

\subsubsection{Preservation}
Preservation is the fact that term reduction preserves typing. In order to prove
this, we will first need to prove that equivalence preserves typing.
\begin{theorem}[Preservation for $\equiv$]\label{thm:cp-preservation-equiv}
  If $\seq[{ P }]{ \Gamma }$ and $\tm{P} \equiv \tm{Q}$,
  then $\seq[{ Q }]{\Gamma }$.
\end{theorem}
\begin{proof}
  By induction on the structure of the equivalence. The cases for reflexivity,
  transitivity and congruence are trivial. The two interesting cases, for
  \cpEquivCutComm and \cpEquivCutAss1 are given in \cref{fig:cp-preservation-equiv}
\end{proof}
%%% Local Variables:
%%% TeX-master: "main"
%%% End:

\begin{figure*}[ht]
  \centering
  \begin{tabular}{ll}
    \cpEquivCutComm
    &
      \begin{prooftree*}
        \AXC{$\seq[{ P }]{ \Gamma, \tmty{x}{A} }$}
        \AXC{$\seq[{ Q }]{ \Delta, \tmty{x}{A^\bot} }$}
        \NOM{Cut}
        \BIC{$\seq[{ \cpCut{x}{P}{Q} }]{ \Gamma, \Delta }$}
      \end{prooftree*}
    \\[30pt]
    $\equiv$
    &
      \begin{prooftree*}
        \AXC{$\seq[{ Q }]{ \Delta, \tmty{x}{A^\bot} }$}
        \AXC{$\seq[{ P }]{ \Gamma, \tmty{x}{A} }$}
        \NOM{\cref{thm:cp-negation-involutive}}
        \UIC{$\seq[{ P }]{ \Gamma, \tmty{x}{A^{\bot\bot}} }$}
        \NOM{Cut}
        \BIC{$\seq[{ \cpCut{x}{Q}{P} }]{ \Gamma, \Delta }$}
      \end{prooftree*}
    \\[40pt]
    \cpEquivCutAss1
    &
      \begin{prooftree*}
        \AXC{$\seq[{ P }]{ \Gamma, \tmty{x}{A} }$}
        \AXC{$\seq[{ Q }]{ \Delta, \tmty{x}{A^\bot}, \tmty{y}{B} }$}
        \AXC{$\seq[{ R }]{ \Theta, \tmty{y}{B^\bot} }$}
        \NOM{Cut}
        \BIC{$\seq[{ \cpCut{y}{Q}{R} }]{ \Delta, \Theta, \tmty{x}{A^\bot} }$}
        \NOM{Cut}
        \BIC{$\seq[{ \cpCut{x}{P}{\cpCut{y}{Q}{R}} }]{ \Gamma, \Delta, \Theta }$}
      \end{prooftree*}
    \\[30pt]
    $\equiv$
    &
      \begin{prooftree*}
        \AXC{$\seq[{ P }]{ \Gamma, \tmty{x}{A} }$}
        \AXC{$\seq[{ Q }]{ \Delta, \tmty{x}{A^\bot}, \tmty{y}{B} }$}
        \NOM{Cut}
        \BIC{$\seq[{ \cpCut{x}{P}{Q} }]{ \Gamma, \Delta, \tmty{y}{B} }$}
        \AXC{$\seq[{ R }]{ \Theta, \tmty{y}{B^\bot} }$}
        \NOM{Cut}
        \BIC{$\seq[{ \cpCut{y}{\cpCut{x}{P}{Q}}{R} }]{ \Gamma, \Delta, \Theta }$}
      \end{prooftree*}
  \end{tabular}
  \caption{Type preservation for the structural congruence of \rcp}
  \label{fig:cp-preservation-equiv}
\end{figure*}
%%% Local Variables:
%%% TeX-master: "main"
%%% End:

Then, we can prove preservation.
\begin{theorem}[Preservation]\label{thm:cp-preservation}
  If \reducesto{P}{Q} and $\seq[{ P }]{ \Gamma }$, then $\seq[{ Q }]{ \Gamma }$.
\end{theorem}
\begin{proof}
  By induction on the structure of the reduction. See
  \cref{fig:cp-preservation-1} for the \cpRedAxCut{i} and \textbeta-reduction
  rules, and \cref{fig:cp-preservation-2a,fig:cp-preservation-2b} for the
  commutative conversions.
  %
  The case for \cpRedGammaCut is trivial by the induction hypothesis, and the
  case for \cpRedGammaEquiv is trivial by the induction hypothesis and
  \cref{thm:cp-preservation-equiv}. 
\end{proof}
%%% Local Variables:
%%% TeX-master: "main"
%%% End:

\begin{figure*}[ht]
  \begin{tabular}{ll}
    \cpRedAxCut1
    &
      \begin{prooftree*}
        \AXC{}
        \NOM{Ax}
        \UIC{$\seq[{ \cpLink{w}{x} }]{ \tmty{w}{A}, \tmty{x}{A^\bot} }$}
        \AXC{$\seq[{ R }]{ \tmty{x}{A}, \Gamma }$}
        \NOM{Cut}
        \BIC{$\seq[{ \cpCut{x}{\cpLink{w}{x}}{R} }]{ \tmty{w}{A}, \Gamma }$}
      \end{prooftree*}
    \\[30pt]
    $\Longrightarrow$
    &
      \begin{prooftree*}
        \AXC{$\seq[{ \cpSub{w}{x}{R} }]{ \tmty{w}{A}, \Gamma }$}
      \end{prooftree*}
    \\[30pt]
    
    \cpRedAxCut2
    &
      (as above)
    \\[30pt]
    
    \cpRedBetaTensParr
    &
      \begin{prooftree*}
        \AXC{$\seq[{ P }]{ \Gamma, \tmty{x}{A^\bot}, \tmty{y}{B^\bot} }$}
        \SYM{\parr}
        \UIC{$\seq[{ \cpRecv{y}{x}P }]{ \Gamma, \tmty{y}{A^\bot \parr B^\bot} }$}
        \AXC{$\seq[{ Q }]{ \Delta, \tmty{x}{A} }$}
        \AXC{$\seq[{ R }]{ \Theta, \tmty{y}{B} }$}
        \SYM{\tens}
        \BIC{$\seq[{ \cpSend{y}{x}(Q \mid R) }]{ \Delta, \Theta, \tmty{y}{A \tens B} }$}
        \NOM{Cut}
        \BIC{$\seq[{ \cpCut{y}{\cpRecv{y}{x}{P}}{\cpSend{y}{x}{Q}{R}} }]{ \Gamma, \Delta, \Theta }$}
      \end{prooftree*}
    \\[30pt]
    $\Longrightarrow$
    &
      \begin{prooftree*}
        \AXC{$\seq[{ P }]{ \Gamma, \tmty{x}{A^\bot}, \tmty{y}{B^\bot} }$}
        \AXC{$\seq[{ Q }]{ \Delta, \tmty{x}{A} }$}
        \NOM{Cut}
        \BIC{$\seq[{ \cpCut{x}{P}{Q} }]{ \Gamma, \Delta, \tmty{y}{B^\bot} }$}
        \AXC{$\seq[{ R }]{ \Theta, \tmty{y}{B} }$}      
        \NOM{Cut}
        \BIC{$\seq[{ \cpCut{y}{\cpCut{x}{P}{Q}}{R} }]{ \Gamma, \Delta, \Theta }$}
      \end{prooftree*}
    \\[40pt]
    
    \cpRedBetaOneBot
    &
      \begin{prooftree*}
        \AXC{$\seq[{ P }]{ \Gamma }$}
        \SYM{\bot}
        \UIC{$\seq[{ \cpWait{x}{P} }]{ \Gamma, \tmty{x}{\bot} }$}
        \AXC{}
        \SYM{\one}
        \UIC{$\seq[{ \cpHalt{x} }]{ \tmty{x}{\one} }$}
        \NOM{Cut}
        \BIC{$\seq[{ \cpCut{x}{\cpHalt{x}}{\cpWait{x}{P}} }]{ \Gamma }$}
      \end{prooftree*}
    \\[30pt]
    $\Longrightarrow$
    &
      \begin{prooftree*}
        \AXC{$\seq[{ P }]{ \Gamma }$}
      \end{prooftree*}
    \\[40pt]
    
    \cpRedBetaPlusWith1
    &
      \begin{prooftree*}
        \AXC{$\seq[{ P }]{ \Gamma, \tmty{x}{A^\bot} }$}
        \AXC{$\seq[{ Q }]{ \Gamma, \tmty{x}{B^\bot} }$}
        \SYM{\with}
        \BIC{$\seq[{ \cpCase{x}{P}{Q} }]{ \Gamma, \tmty{x}{A^\bot \with B^\bot} }$}
        \AXC{$\seq[{ R }]{ \Delta, \tmty{x}{A} }$}
        \SYM{\plus_1}
        \UIC{$\seq[{ \cpInl{x}{R}}]{ \Delta, \tmty{x}{A \plus B} }$}
        \NOM{Cut}
        \BIC{$\seq[{ \cpCut{x}{\cpCase{x}{P}{Q}}{\cpInl{x}{R}} }]{ \Gamma, \Delta }$}
      \end{prooftree*}
    \\[30pt]
    $\Longrightarrow$
    &
      \begin{prooftree*}
        \AXC{$\seq[{ P }]{ \Gamma, \tmty{x}{A^\bot} }$}
        \AXC{$\seq[{ R }]{ \Delta, \tmty{x}{A} }$}
        \NOM{Cut}
        \BIC{$\seq[{ \cpCut{x}{P}{R} }]{ \Gamma, \Delta }$} 
      \end{prooftree*}
    \\[40pt]
    
    \cpRedBetaPlusWith2
    &
      (as above)
  \end{tabular}
  
  \caption{Type preservation for the \cpRedAxCut{i} and \textbeta-reduction rules of \cp}
  \label{fig:cp-preservation-1}
\end{figure*}
%%% Local Variables:
%%% TeX-master: "main"
%%% End:

\subsubsection{Progress}
Progress is the fact that every term is either in some canonical form, or can be
reduced further. In order for us to even be able to state progress, we will need
to define when a term is in canonical form.

We have informally used the phrase ``act on'' in previous sections. It is time
to formally define what it means when we say a process \emph{acts on} some
channel.
\begin{definition}[Action]\label{def:cp-action}
  A process \tm{P} \emph{acts on} a channel \tm{x} if it is in one of the
  following forms:
  \begin{multicols}{3}
    \begin{itemize}[noitemsep,topsep=0pt,parsep=0pt,partopsep=0pt]
    \item \tm{\cpLink{x}{y}}
    \item \tm{\cpLink{y}{x}}
    \item \tm{\cpSend{x}{y}{P'}{Q'}}
    \item \tm{\cpRecv{x}{y}{P'}}
    \item \tm{\cpHalt{x}}
    \item \tm{\cpWait{x}{P'}}
    \item \tm{\cpInl{x}{P'}}
    \item \tm{\cpInr{x}{P'}}
    \item \tm{\cpCase{x}{P'}{Q'}}
    \item \tm{\cpAbsurd{x}}
    \end{itemize}
  \end{multicols}
\end{definition}
%%% Local Variables:
%%% TeX-master: "main"
%%% End:

Furthermore, we will need the notion of an \emph{evaluation prefix}.
Intuitively, evaluation prefixes are multi-holed contexts consisting solely of
cuts. We will use evaluation prefixes in order to have a view of every
\emph{action} in a process at once.
\begin{definition}[Evaluation prefixes]\label{def:cp-evaluation-prefixes}
  We define evaluation prefixes as:
  \begin{align*}
    \tm{G}, \tm{H} := \tm{\Box} \mid \tm{\cpCut{x}{G}{H}}
  \end{align*}
\end{definition}
\begin{definition}[Plugging]\label{def:cp-evaluation-prefix-plugging}
  We define plugging for an evaluation prefix with $n$ holes as:
  \[
    \begin{array}{ll}
      \tm{\cpPlug{\Box}{R}} & := \; \tm{R} \\
      \tm{\cpPlug{\cpCut{x}{G}{H}}{R_1 \dots R_m, R_{m+1} \dots R_{n}}}
                            & := \; \tm{\cpCut{x}{\cpPlug{G}{R_1 \dots R_m}}{\cpPlug{H}{R_{m+1} \dots R_n}}}
    \end{array}
  \]
  Note that in the second case, \tm{G} is an evaluation prefix with $m$ holes,
  and \tm{H} is an evaluation prefix with $(n-m)$ holes.
\end{definition}
%%% Local Variables:
%%% TeX-master: "main"
%%% End:

Intuitively, we can say that every term of the form
\tm{\cpPlug{G}{P_1 \dots P_n}} is equivalent to some term of the form
\tm{\cpCut{x_1}{P_1}{\cpCut{x_2}{P_2}{\dots \cpCut{x_n}{P_{n-1}}{P_n} \dots}}} 
where $\tm{x_1} \dots \tm{x_{n-1}}$ are the channels bound in \tm{G}.
In fact, a similar equivalence was used by \citeauthor{lindley2015semantics}
\cite{lindley2015semantics} in their semantics for \cp. 
\begin{definition}[Maximum evaluation prefix]\label{def:cp-maximum-evaluation-prefix}
  We say that \tm{G} is the evaluation prefix of \tm{P} when there exist terms
  $\tm{P_1} \dots \tm{P_n}$ such that $\tm{P} = \tm{\cpPlug{G}{P_1 \dots P_n}}$.
  We say that \tm{G} is the maximum evaluation prefix if each \tm{P_i} is an
  action. 
\end{definition}
\begin{lemma}\label{thm:cp-maximum-evaluation-prefix}
  Every term \tm{P} has a maximum evaluation prefix.
\end{lemma}
\begin{proof}
  By induction on the structure of \tm{P}.
\end{proof}
%%% Local Variables:
%%% TeX-master: "main"
%%% End:

We can now define what it means for a term to be in canonical form. Intuitively,
a process is in canonical form either when there is no top-level cut, or when it
is blocked on an external communication. We state this formally as follows:
\begin{definition}[Canonical forms]\label{def:cp-canonical-forms}
  We define the following forms to be canonical:
  \begin{center}
    \begin{prooftree*}
      \AXC{}
      \UIC{\canonical{\cpLink{x}{y}}}
    \end{prooftree*}
    \begin{prooftree*}
      \AXC{}
      \UIC{\canonical{\cpSend{x}{y}{P}{Q}}}
    \end{prooftree*}
    \begin{prooftree*}
      \AXC{}
      \UIC{\canonical{\cpRecv{x}{y}{P}}}
    \end{prooftree*}
  \end{center}
  \begin{center}
    \begin{prooftree*}
      \AXC{}
      \UIC{\canonical{\cpHalt{x}}}
    \end{prooftree*}
    \begin{prooftree*}
      \AXC{}
      \UIC{\canonical{\cpWait{x}{P}}}
    \end{prooftree*}
    \begin{prooftree*}
      \AXC{}
      \UIC{\canonical{\cpAbsurd{x}}}
    \end{prooftree*}
  \end{center}
  \begin{center}
    \begin{prooftree*}
      \AXC{}
      \UIC{\canonical{\cpInl{x}{P}}}
    \end{prooftree*}
    \begin{prooftree*}
      \AXC{}
      \UIC{\canonical{\cpInr{x}{P}}}
    \end{prooftree*}
    \begin{prooftree*}
      \AXC{}
      \UIC{\canonical{\cpCase{x}{P}{Q}}}
    \end{prooftree*}
  \end{center}
\end{definition}
%%% Local Variables:
%%% TeX-master: "main"
%%% End:

This definition is adequate, as it matches our intuition. We will see this in
our proof for \cref{thm:cp-progress}. 

Lastly, we will need the notion of an \emph{evaluation context}.
Intuitively, evaluation contexts are one-holed term contexts under which
reduction can take place. For \rcp, these consist solely of cuts.
\begin{definition}[Evaluation contexts]\label{def:cp-evaluation-contexts}
  We define evaluation contexts as:
  \begin{align*}
    \tm{G}, \tm{H} := \tm{\Box}
    \mid \tm{\cpCut{x}{G}{P}}
    \mid \tm{\cpCut{x}{P}{G}}
  \end{align*}
\end{definition}
\begin{definition}[Plugging]\label{def:cp-evaluation-context-plugging}
  We define plugging for evaluation contexts as:
  \begin{gather*}
    \begin{array}{ll}
      \tm{\cpPlug{\Box}{R}}            
      & := \; \tm{R}
      \\
      \tm{\cpPlug{\cpCut{x}{G}{P}}{R}}
      & := \; \tm{\cpCut{x}{\cpPlug{G}{R}}{P}}
      \\
      \tm{\cpPlug{\cpCut{x}{P}{G}}{R}}
      & := \; \tm{\cpCut{x}{P}{\cpPlug{G}{R}}}
    \end{array}
  \end{gather*}
\end{definition}
%%% Local Variables:
%%% TeX-master: "main"
%%% End:

We can prove that we can push any cut downwards under an evaluation contexts, as
long as the channel it binds does not occur in the context itself.
\begin{lemmaB}\label{thm:cp-display-1}
  If $\seq[{ \tm{\cpCut{x}{\cpPlug{G}{P}}{Q}} }]{ \Gamma }$ and
  $\notFreeIn{x}{G}$, then $\tm{\cpCut{x}{\cpPlug{G}{P}}{Q}} \equiv
  \tm{\cpPlug{G}{\cpCut{x}{P}{Q}}}$. 
\end{lemmaB}
\begin{proof}
  By induction on the structure of the evaluation context \tm{G}.
  \begin{itemize}
  \item
    Case $\tm{\Box}$. By reflexivity.
  \item
    Case $\tm{\cpCut{y}{G}{R}}$.
    \[\!
      \begin{array}{ll}
        \tm{\cpCut{x}{\cpCut{y}{\cpPlug{G}{P}}{R}}{Q}} & \equiv \quad \text{by} \; \cpEquivCutComm\\
        \tm{\cpCut{x}{\cpCut{y}{R}{\cpPlug{G}{P}}}{Q}} & \equiv \quad \text{by} \; \cpEquivCutAss2 \\
        \tm{\cpCut{y}{R}{\cpCut{x}{\cpPlug{G}{P}}{Q}}} & \equiv \quad \text{by} \; \cpEquivCutComm \\
        \tm{\cpCut{y}{\cpCut{x}{\cpPlug{G}{P}}{Q}}{R}} & \equiv \quad \text{by the induction hypothesis and} \; \cref{thm:cp-preservation-equiv}\\
        \tm{\cpCut{y}{\cpPlug{G}{\cpCut{x}{P}{Q}}}{R}} &
      \end{array}
    \]
  \item
    Case $\tm{\cpCut{y}{R}{G}}$.
    \[\!
      \begin{array}{ll}
        \tm{\cpCut{x}{\cpCut{y}{R}{\cpPlug{G}{P}}}{Q}} & \equiv \quad \text{by} \; \cpEquivCutAss2\\
        \tm{\cpCut{y}{R}{\cpCut{x}{\cpPlug{G}{P}}{Q}}} & \equiv \quad \text{by the induction hypothesis and} \; \cref{thm:cp-preservation-equiv}\\
        \tm{\cpCut{y}{R}{\cpPlug{G}{\cpCut{x}{P}{Q}}}}
      \end{array}
    \]
  \end{itemize}
  In each case, the side conditions for \cpEquivCutAss2, $\notFreeIn{x}{R}$ and
  $\notFreeIn{y}{Q}$, can be inferred from $\notFreeIn{x}{G}$ and the fact that
  $\tm{\cpCut{x}{\cpPlug{G}{P}}{Q}}$ is well-typed.
\end{proof}
%%% Local Variables:
%%% TeX-master: "main"
%%% End:
And vice versa. However, we will not use the following lemma in this
dissertation, and leave its proof as an exercise to the reader.
\begin{lemmaB}\label{thm:cp-display-2}
  If $\seq[{ \cpPlug{G}{\cpCut{x}{P}{Q}} }]{ \Gamma }$ and $\notFreeIn{x}{G}$,
  then 
  $\tm{\cpPlug{G}{\cpCut{x}{P}{Q}}} \equiv \tm{\cpCut{x}{\cpPlug{G}{P}}{Q}}$. 
\end{lemmaB}
%%% Local Variables:
%%% TeX-master: "main"
%%% End:
We also prove two lemmas which relate evaluation prefixes to evaluation
contexts.
Specifically, if a process under an evaluation prefix is a link, we can rewrite
the entire process in such a way as to reveal the cut which introduced one of
the channels acted upon by that link.
\begin{lemmaB}\label{thm:cp-display-3}
  If $\seq[{ \cpPlug{G}{P_1 \dots P_n} }]{ \Gamma }$, and some \tm{P_i} is a
  link \tm{\cpLink{x}{y}}, then either $\tm{\cpPlug{G}{P_1 \dots P_n}} =
  \tm{\cpLink{x}{y}}$, or there exist \tm{H}, \tm{H'} and \tm{Q} such that
  \(
  \tm{\cpPlug{G}{P_1 \dots P_n}} \equiv
  \tm{\cpPlug{H}{\cpCut{z}{\cpPlug{H'}{\cpLink{x}{y}}}{Q}}}
  \)
  with either $\tm{z} = \tm{x}$ or $\tm{z} = \tm{y}$.
\end{lemmaB}
  \begin{proof}
    Either $\tm{G} = \tm{\Box}$, and the statement is trivially true, or we
    proceed by induction on the structure of \tm{G}. We have:
    \begin{gather*}
      \begin{array}[t]{ll}
        \tm{\cpPlug{G}{P_1 \dots P_n}} = \tm{\cpCut{z}{\cpPlug{G'}{P_1\dots P_i\dots P_m}}{\cpPlug{G''}{P_{m+1}\dots P_n}}},&\text{or}
        \\
        \tm{\cpPlug{G}{P_1 \dots P_n}} = \tm{\cpCut{z}{\cpPlug{G'}{P_1\dots P_m}}{\cpPlug{G''}{P_{m+1}\dots P_i\dots P_n}}}. 
      \end{array}
    \end{gather*}
    There are three cases:
    \begin{itemize}
    \item
      Case $\tm{z} = \tm{x}$ or $\tm{z} = \tm{y}$
      and $\tm{\cpCut{z}{\cpPlug{G'}{P_1\dots P_i\dots P_m}}{\cpPlug{G''}{P_{m+1}\dots P_n}}}$.
      \\
      Let $\tm{H} := \tm{\Box}$ and $\tm{H'} := \tm{\cpPlug{G'}{P_1\dots P_{j-1},\Box,P_{j+1}\dots P_m}}$.
      \\[1ex]
      By reflexivity.
    \item
      Case $\tm{z} = \tm{x}$ or $\tm{z} = \tm{y}$
      and $\tm{\cpCut{z}{\cpPlug{G'}{P_1\dots P_m}}{\cpPlug{G''}{P_{m+1}\dots P_i\dots P_n}}}$.
      \\
      Let $\tm{H} := \tm{\Box}$ and $\tm{H'} := \tm{\cpPlug{G''}{P_{m+1}\dots P_{i-1},\Box,P_{i+1}\dots P_n}}$.
      \\
      By reflexivity and \cpEquivCutComm.
    \item
      Otherwise, we obtain \tm{H}, \tm{H'} and \tm{Q} from the induction
      hypothesis and \cref{thm:cp-preservation-equiv}, and prepend either
      \tm{\cpCut{z}{\Box}{\cpPlug{G''}{P_{m+1}\dots P_n}}} or
      \tm{\cpCut{z}{\cpPlug{G'}{P_1\dots P_m}}{\Box}} to \tm{H}.
      The desired equality follows by congruence.
    \end{itemize}
    The case where neither \tm{x} nor \tm{y} is bound in \tm{G}, but $\tm{G} \neq
    \tm{\Box}$, is excluded by the type system.
  \end{proof}
%%% Local Variables:
%%% TeX-master: "main"
%%% End:
And if two processes under an evaluation prefix act on the same channel, then we
can rewrite the entire process in such a way as to reveal the cut which
introduced that channel. 
\begin{lemmaB}\label{thm:cp-display-4}
  If $\seq[{ \cpPlug{G}{P_1 \dots P_n} }]{ \Gamma }$, and some \tm{P_i} and
  \tm{P_j} (with $i \neq j$) act on the same channel \tm{x}, then there exist \tm{H},
  \tm{H_i} and \tm{H_j} such that 
  \(
  \tm{\cpPlug{G}{P_1 \dots P_n}} =
  \tm{\cpPlug{H}{\cpCut{x}{\cpPlug{H_i}{P_i}}{\cpPlug{H_j}{P_j}}}}
  \).
\end{lemmaB}
\begin{proof}
  By induction on the structure of \tm{G}.
  \begin{itemize}
  \item
    Case \tm{\cpCut{x}{\cpPlug{G'}{P_1\dots P_i\dots P_m}}{\cpPlug{G''}{P_{m+1}\dots P_j\dots P_n}}}. 
    \\
    \(\arraycolsep=0pt\begin{array}[t]{lll}
      \text{Let}&\ \tm{H}  &\ :=\ \tm{\Box}, \\
                &\ \tm{H_i}&\ :=\ \tm{\cpPlug{G'}{P_1\dots P_{i-1},\Box,P_{i+1}\dots P_m}}, \\
                &\ \tm{H_j}&\ :=\ \tm{\cpPlug{G''}{P_{m+1}\dots P_{j-1},\Box,P_{j+1}\dots P_n}},
    \end{array}\)
    \\[1ex]
    By reflexivity.
  \item
    Case \tm{\cpCut{x}{\cpPlug{G'}{P_1\dots P_j\dots P_m}}{\cpPlug{G''}{P_{m+1}\dots P_i\dots P_n}}}.
    \\
    As above.
  \item
    \(\arraycolsep=0pt\begin{array}[t]{ll}
      \text{Case}
      &\ \tm{\cpCut{y}{\cpPlug{G'}{P_1\dots P_i\dots P_j\dots P_m}}{\cpPlug{G''}{P_{m+1}\dots P_n}}}
      \\
      \text{or}
      &\ \tm{\cpCut{y}{\cpPlug{G'}{P_1\dots P_m}}{\cpPlug{G''}{P_{m+1}\dots P_i\dots P_j\dots P_n}}}.
    \end{array}\)
    \\[1ex]
    We obtain \tm{H}, \tm{H_1} and \tm{H_2} from the induction hypothesis and
    \cref{thm:cp-preservation-equiv}, and then prepend either
    \tm{\cpCut{y}{\Box}{\cpPlug{G''}{P_{m+1}\dots P_n}}} or
    \tm{\cpCut{y}{\cpPlug{G'}{P_1\dots P_m}}{\Box}} to \tm{H}.
    The desired equality follows by congruence.
  \end{itemize}
  The case for \tm{\Box} is excluded because $n > 1$.
  The cases in which \tm{P_i} and \tm{P_j} are on the \emph{same} side of the
  cut, but the cut binds \tm{x}, and the cases in which \tm{P_i} and \tm{P_j}
  are on different sides of the cut, but the cut binds some other channel
  \tm{y}, are excluded by the type system.
\end{proof}
%%% Local Variables:
%%% TeX-master: "main"
%%% End:


\begin{theorem}[Progress]\label{thm:cp-progress}
  If $\seq[{ P }]{ \Gamma }$, then $\tm{P}$ is in canonical form, or there
  exists a $\tm{P'}$ s.t.\ $\reducesto{P}{P'}$. 
\end{theorem}
  \begin{proof}
    By induction on the structure of derivation for $\seq[{ P }]{ \Gamma }$.
    The only interesting case is when the last rule of the derivation is
    \textsc{Cut}. In every other case, the typing rule constructs a term in which
    is in canonical form. 
    \\
    If the last rule in the derivation is \textsc{Cut}, we consider the maximum
    evaluation prefix \tm{G} of \tm{P}, such that $\tm{P} = \tm{\cpPlug{G}{P_1
        \dots P_{n+1}}}$ and each $P_i$ is an action.
    The prefix \tm{G} consists of $n$ cuts, and introduces $n$ variables, but
    composes $n+1$ actions. Therefore, one of the following must be true:
    \begin{itemize}
    \item
      One of the processes is a link \tm{\cpLink{x}{y}}.
      We have:
      \begin{gather*}
        \begin{array}{ll}
          \tm{\cpPlug{G}{P_1 \dots \cpLink{x}{y} \dots P_{n+1}}}
          & \equiv \quad \text{by \cref{thm:cp-display-3}}
          \\
          \tm{\cpPlug{H}{\cpCut{z}{\cpPlug{H'}{\cpLink{x}{y}}}{Q}}}
          & \equiv \quad \text{by \cref{thm:cp-display-1}}
          \\
          \tm{\cpPlug{H}{\cpPlug{H'}{\cpCut{z}{\cpLink{x}{y}}{Q}}}}
        \end{array}
      \end{gather*}
      Where $\tm{z} = \tm{x}$ or $\tm{z} = \tm{y}$.
      We then apply one of \cpRedAxCut1 or \cpRedAxCut2.
    \item
      Two of the processes, \tm{P_i} and \tm{P_j}, act on the same channel \tm{x}.
      We have:
      \begin{gather*}
        \begin{array}{ll}
          \tm{\cpPlug{G}{P_1 \dots P_i \dots P_j \dots P_{n+1}}}
          & = \quad \text{by \cref{thm:cp-display-4}}
          \\
          \tm{\cpPlug{G}{\cpCut{x}{\cpPlug{H_i}{P_i}}{\cpPlug{H_j}{P_j}}}}
          & \equiv \quad \text{by \cref{thm:cp-display-1}} 
          \\
          \tm{\cpPlug{G}{\cpPlug{H_i}{\cpCut{x}{P_i}{\cpPlug{H_j}{P_j}}}}}
          & \equiv \quad \text{by \cref{thm:cp-display-1} and
            \cref{thm:cp-preservation-equiv}} 
          \\
          \tm{\cpPlug{G}{\cpPlug{H_i}{\cpPlug{H_j}{\cpCut{x}{P_i}{P_j}}}}} 
        \end{array}
      \end{gather*}
      We then apply one of the \textbeta-reduction rules.
    \item
      Otherwise (at least) one of the processes acts on an external channel.
      \\
      No process \tm{P_i} is a link.
      No two processes \tm{P_i} and \tm{P_j} act on the same channel \tm{x}.
      Therefore, \tm{P} is canonical.
    \end{itemize}
  \end{proof}
%%% Local Variables:
%%% TeX-master: "main"
%%% End:


\subsubsection{Termination}
\begin{theorem}[Termination]\label{thm:cp-termination}
  If $\seq[{ P }]{ \Gamma }$, then there are no infinite $\Longrightarrow$
  reduction sequences.
\end{theorem}
  \begin{proof}
    Every reduction reduces a single cut to zero, one or two cuts.
    However, each of these cuts is \emph{smaller}, in the sense that the type of
    the channel on which the communication takes place is smaller, as each
    reduction eliminates a connective---see
    \cref{fig:cp-preservation-1,fig:cp-preservation-2a,fig:cp-preservation-2b}.
    Therefore, there cannot be an infinite reduction sequence.
  \end{proof}
%%% Local Variables:
%%% TeX-master: "main"
%%% End:

\subsection{Correspondence with linear logic}
\begin{figure*}[ht]
  \centering
  \begin{tabular}{ll}
    \cpRedKappaTens1
    &
      \begin{prooftree*}
        \AXC{$\seq[{ P }]{ \Gamma, \tmty{x}{A}, \tmty{z}{B} }$}
        \AXC{$\seq[{ Q }]{ \Delta, \tmty{y}{C} }$}
        \SYM{\tens}
        \BIC{$\seq[{ \cpSend{y}{z}{P}{Q} }]{ \Gamma, \Delta, \tmty{y}{B \tens C} }$}
        \AXC{$\seq[{ R }]{ \Theta, \tmty{x}{A^\bot} }$}
        \NOM{Cut}
        \BIC{$\seq[{ \cpCut{x}{\cpSend{y}{z}{P}{Q}}{R} }]{ \Gamma, \Delta, \Theta, \tmty{y}{B \tens C} }$}
      \end{prooftree*}
    \\[30pt]
    $\Longrightarrow$
    &
       \begin{prooftree*}
         \AXC{$\seq[{ P }]{ \Gamma, \tmty{x}{A}, \tmty{z}{B} }$}
         \AXC{$\seq[{ R }]{ \Theta, \tmty{x}{A^\bot} }$}
         \NOM{Cut}
         \BIC{$\seq[{ \cpCut{x}{P}{Q} }]{ \Gamma, \Theta, \tmty{z}{B} }$}
         \AXC{$\seq[{ Q }]{ \Delta, \tmty{y}{C} }$}
         \SYM{\tens}
         \BIC{$\seq[{ \cpSend{y}{z}{\cpCut{x}{P}{Q}}{R} }]{ \Gamma, \Delta, \Theta, \tmty{y}{B \tens C} }$}
       \end{prooftree*}
    \\[30pt]
    \cpRedKappaTens2
    &
      (as above)
    \\[20pt]
    \cpRedKappaParr
    &
      \begin{prooftree*}
        \AXC{$\seq[{ P }]{ \Gamma, \tmty{x}{A}, \tmty{z}{B}, \tmty{y}{C} }$}
        \SYM{\parr}
        \UIC{$\seq[{ \cpRecv{y}{z}{P} }]{ \Gamma, \tmty{x}{A}, \tmty{y}{B \parr C} }$}
        \AXC{$\seq[{ R }]{ \Theta, \tmty{x}{A^\bot} }$}
        \NOM{Cut} 
        \BIC{$\seq[{ \cpCut{x}{\cpRecv{y}{z}{P}}{R} }]{ \Gamma, \Theta, \tmty{y}{B \parr C} }$}
      \end{prooftree*}
    \\[30pt]
    $\Longrightarrow$
    &
      \begin{prooftree*}
        \AXC{$\seq[{ P }]{ \Gamma, \tmty{x}{A}, \tmty{z}{B}, \tmty{y}{C} }$}
        \AXC{$\seq[{ R }]{ \Theta, \tmty{x}{A^\bot} }$}
        \NOM{Cut}
        \BIC{$\seq[{ \cpCut{x}{P}{R} }]{ \Gamma, \Theta, \tmty{z}{B}, \tmty{y}{C} }$}
        \SYM{\parr}
        \UIC{$\seq[{ \cpRecv{y}{z}{\cpCut{x}{P}{R}} }]{ \Gamma, \Theta, \tmty{y}{B \parr C} }$}
      \end{prooftree*}
    \\[40pt]
    \cpRedKappaBot
    &
      \begin{prooftree*}
        \AXC{$\seq[{ P }]{ \Gamma, \tmty{x}{A} }$}
        \SYM{\bot}
        \UIC{$\seq[{ \cpWait{y}{P} }]{ \Gamma, \tmty{x}{A}, \tmty{y}{\bot} }$}
        \AXC{$\seq[{ R }]{ \Theta, \tmty{x}{A^\bot} }$}
        \NOM{Cut} 
        \BIC{$\seq[{ \cpCut{x}{\cpWait{y}{P}}{R} }]{ \Gamma, \Theta, \tmty{y}{\bot} }$}
      \end{prooftree*}
    \\[30pt]
    $\Longrightarrow$
    &
      \begin{prooftree*}
        \AXC{$\seq[{ P }]{ \Gamma, \tmty{x}{A} }$}
        \AXC{$\seq[{ R }]{ \Theta, \tmty{x}{A^\bot} }$}
        \NOM{Cut} 
        \BIC{$\seq[{ \cpCut{x}{P}{R} }]{ \Gamma, \Theta }$}
        \SYM{\bot}
        \UIC{$\seq[{ \cpWait{y}{\cpCut{x}{P}{R}} }]{ \Gamma, \Theta, \tmty{y}{\bot} }$}
      \end{prooftree*}
  \end{tabular}

  \caption{Type preservation for the commutative conversions of \cp}
  \label{fig:cp-preservation-2a}
\end{figure*}

\begin{figure*}[ht]
  \makebox[\textwidth][c]{
    \begin{tabular}{ll}
      \cpRedKappaPlus1
      &
        \begin{prooftree*}
          \AXC{$\seq[{ P }]{ \Gamma, \tmty{x}{A}, \tmty{y}{B} }$}
          \SYM{\plus_1}
          \UIC{$\seq[{ \cpInl{y}{P} }]{ \Gamma, \tmty{x}{A}, \tmty{y}{B \plus C} }$}
          \AXC{$\seq[{ R }]{ \Theta, \tmty{x}{A^\bot} }$}
          \NOM{Cut}
          \BIC{$\seq[{ \cpCut{x}{\cpInl{y}{P}}{R} }]{ \Gamma, \Theta, \tmty{y}{B \plus C} }$}
        \end{prooftree*}
      \\[30pt]
      $\Longrightarrow$
      &
        \begin{prooftree*}
          \AXC{$\seq[{ P }]{ \Gamma, \tmty{x}{A}, \tmty{y}{B} }$}
          \AXC{$\seq[{ R }]{ \Theta, \tmty{x}{A^\bot} }$}
          \NOM{Cut}
          \BIC{$\seq[{ \cpCut{x}{P}{R} }]{ \Gamma, \Theta, \tmty{y}{B} }$}
          \SYM{\plus_1}
          \UIC{$\seq[{ \cpInl{y}{\cpCut{x}{P}{R}} }]{ \Gamma, \Theta, \tmty{y}{B \plus C} }$}
        \end{prooftree*}
      \\[30pt] 
      \cpRedKappaPlus2
      &
        (as above)
      \\[20pt]
      \cpRedKappaWith
      &
        \begin{prooftree*}
          \AXC{$\seq[{ P }]{ \Gamma, \tmty{x}{A}, \tmty{y}{B} }$}
          \AXC{$\seq[{ Q }]{ \Gamma, \tmty{x}{A}, \tmty{y}{C} }$}
          \SYM{\with}
          \BIC{$\seq[{ \cpCase{y}{P}{Q} }]{ \Gamma, \tmty{x}{A}, \tmty{y}{B \with C} }$}
          \AXC{$\seq[{ R }]{ \Theta, \tmty{x}{A^\bot} }$}
          \NOM{Cut}
          \BIC{$\seq[{ \cpCut{x}{\cpCase{y}{P}{Q}}{R} }]{ \Gamma, \Theta, \tmty{y}{B \with C} }$}
        \end{prooftree*}
      \\[30pt]
      $\Longrightarrow$
      &
        \begin{prooftree*}
          \AXC{$\seq[{ P }]{ \Gamma, \tmty{x}{A}, \tmty{y}{B} }$}
          \AXC{$\seq[{ R }]{ \Theta, \tmty{x}{A^\bot} }$}
          \NOM{Cut}
          \BIC{$\seq[{ \cpCut{x}{P}{R} }]{ \Gamma, \Theta, \tmty{y}{B} }$}
          \AXC{$\seq[{ Q }]{ \Gamma, \tmty{x}{A}, \tmty{y}{C} }$}
          \AXC{$\seq[{ R }]{ \Theta, \tmty{x}{A^\bot} }$}
          \NOM{Cut}
          \BIC{$\seq[{ \cpCut{x}{Q}{R} }]{ \Gamma, \Theta, \tmty{y}{C} }$}
          \SYM{\with}
          \BIC{$\seq[{ \cpCase{y}{\cpCut{x}{P}{R}}{\cpCut{x}{Q}{R}} }]{ \Gamma, \Theta, \tmty{y}{B \with C} }$}
        \end{prooftree*}
      \\[40pt] 
      \cpRedKappaTop
      &
        \begin{prooftree*}
          \AXC{}
          \SYM{\top}
          \UIC{$\seq[{ \cpAbsurd{y} }]{ \Gamma, \tmty{x}{A}, \tmty{y}{\top} }$}
          \AXC{$\seq[{ R }]{ \Theta, \tmty{x}{A^\bot} }$}
          \NOM{Cut}
          \BIC{$\seq[{ \cpCut{x}{\cpAbsurd{y}}{R} }]{ \Gamma, \Theta, \tmty{y}{\top} }$}
        \end{prooftree*}
      \\[30pt]
      $\Longrightarrow$
      &
        \begin{prooftree*}
          \AXC{}
          \SYM{\top}
          \UIC{$\seq[{ \cpAbsurd{y} }]{ \Gamma, \Theta, \tmty{y}{\top} }$}
        \end{prooftree*}
    \end{tabular}
  }
  \caption{Type preservation for the commutative conversions of \cp (cont'd)}
  \label{fig:cp-preservation-2b}
\end{figure*}
%%% Local Variables:
%%% TeX-master: "main"
%%% End:

%%% Local Variables:
%%% TeX-master: "main"
%%% End:
