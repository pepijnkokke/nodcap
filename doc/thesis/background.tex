%% Background
\chapter{Background}\label{sec:background}
\section{Classical Processes}\label{sec:cp}
In this section, we will discuss a rudimentary subset of the typed process
calculus \cp~\cite{wadler2012,lindley2015semantics}, which we will refer to as
\rcp. 
We have chosen to discuss only a subset in order to keep our later discussion of
our extension to \cp in \cref{sec:main} as simple as possible.
\rcp is the subset which corresponds to rudimentary linear
logic~\cite[RLL]{girard1992}, also known as multiplicative-applicative linear
logic. 
However, we foresee no problems in extending the proofs from \cref{sec:main} to
cover the remaining features of \cp, polymophism and the exponentials $\ty{!A}$
and $\ty{?A}$. 

This chapter will proceed as follows.
In \cref{sec:cp-terms-and-types}, we will discuss the terms, the structural
congruence, and the types of \rcp. 
In \cref{sec:cp-multiplicatives,sec:cp-additives,sec:cp-duality,sec:cp-commutative-conversions},
we will discuss the terms and their corresponding types, in small groups,
together with their typing and reduction rules.
In \cref{sec:cp-properties}, we will prove preservation, progress and
termination for \rcp.

\subsection{Terms and types}\label{sec:cp-terms-and-types}
The term language for \rcp is a variant of the
\textpi-calculus~\cite{milner1992b}.
Its terms are defined by the following grammar:
\begin{definition}[Terms]\label{def:cp-terms}
  \[\!
    \begin{aligned}
      \tm{P}, \tm{Q}, \tm{R}
           :=& \; \tm{\cpLink{x}{y}}       &&\text{link}
      \\ \mid& \; \tm{\cpCut{x}{P}{Q}}     &&\text{parallel composition, or ``cut'''}
      \\ \mid& \; \tm{\cpSend{x}{y}{P}{Q}} &&\text{``output''}
      \\ \mid& \; \tm{\cpRecv{x}{y}{P}}    &&\text{``input''}
      \\ \mid& \; \tm{\cpHalt{x}}          &&\text{halt}
      \\ \mid& \; \tm{\cpWait{x}{P}}       &&\text{wait}
      \\ \mid& \; \tm{\cpInl{x}{P}}        &&\text{select left choice}
      \\ \mid& \; \tm{\cpInr{x}{P}}        &&\text{select right choice}
      \\ \mid& \; \tm{\cpCase{x}{P}{Q}}    &&\text{offer binary choice}
      \\ \mid& \; \tm{\cpAbsurd{x}}        &&\text{offer nullary choice}
    \end{aligned}
  \]  
\end{definition}
%%% Local Variables:
%%% TeX-master: "main"
%%% End:

The construct \tm{\cpLink{x}{y}} links two
channels~\cite{sangiorgi1996,boreale1998}, forwarding messages received on
\tm{x} to \tm{y} and vice versa.
The construct \tm{\cpCut{x}{P}{Q}} creates a new channel \tm{x}, and composes
two processes, \tm{P} and \tm{Q}, which communicate on \tm{x}, in parallel.
Therefore, in \tm{\cpCut{x}{P}{Q}} the name \tm{x} is bound in both \tm{P} and
\tm{Q}.
In \tm{\cpRecv{x}{y}{P}} and \tm{\cpSend{x}{y}{P}{Q}}, round brackets denote
input, square brackets denote output. 
We use bound output~\cite{sangiorgi1996}.
This means that unlike in the \textpi-calculus, both input and output bind a new
name.
In \tm{\cpRecv{x}{y}{P}} the new name \tm{y} is bound in \tm{P}.
In \tm{\cpSend{x}{y}{P}{Q}}, the new name \tm{y} is only bound in \tm{P}, while
\tm{x} is only bound in \tm{Q}.

Terms in \rcp are identified up to structural congruence, which states that
parallel compositions \tm{\cpCut{x}{P}{Q}} are associative and commutative.
It is defined as follows:
\begin{definition}[Structural congruence]\label{def:cp-equiv}
  We define the structural congruence $\equiv$ as a reflexive, transitive
  congruence over terms which satisfies the following additional axioms:
  \[
    \begin{array}{llll}
      \cpEquivLinkComm
      & \tm{\cpLink{x}{y}}
      & \equiv \;
      & \tm{\cpLink{y}{x}}
      \\
      \cpEquivCutComm
      & \tm{\cpCut{x}{P}{Q}}
      & \equiv \;
      & \tm{\cpCut{x}{Q}{P}}
      \\
      \cpEquivCutAss1
      & \tm{\cpCut{x}{P}{\cpCut{y}{Q}{R}}}
      & \equiv \;
      & \tm{\cpCut{y}{\cpCut{x}{P}{Q}}{R}}
        \quad \text{if} \; \notFreeIn{x}{R} \; \text{and} \; \notFreeIn{y}{P}
    \end{array}
  \]
\end{definition}
%%% Local Variables:
%%% TeX-master: "main"
%%% End:

We do not add an axiom for \cpEquivCutAss2, as it follows from
\cref{def:cp-equiv}, see~\cref{thm:cp-cut-assoc2}.
Note that throughout this dissertation, we will leave uses of the transitivity
and congruence rules implicit.
\begin{lemma}[\cpEquivCutAssNoParen2]\label{thm:cp-cut-assoc2}
  If $\tm{x}\not\in\tm{R}$ and $\tm{y}\not\in\tm{P}$, then 
  \(
    \tm{\cpCut{y}{\cpCut{x}{P}{Q}}{R}} \equiv
    \tm{\cpCut{x}{P}{\cpCut{y}{Q}{R}}}
  \).
\end{lemma}
  \begin{proof}
    \begin{align*}
      \tm{\cpCut{y}{\cpCut{x}{P}{Q}}{R}} &\equiv \qquad \text{by \cpEquivCutComm} \\
      \tm{\cpCut{y}{\cpCut{x}{Q}{P}}{R}} &\equiv \qquad \text{by \cpEquivCutComm} \\
      \tm{\cpCut{y}{R}{\cpCut{x}{Q}{P}}} &\equiv \qquad \text{by \cpEquivCutAss1} \\
      \tm{\cpCut{x}{\cpCut{y}{R}{Q}}{P}} &\equiv \qquad \text{by \cpEquivCutComm} \\
      \tm{\cpCut{x}{P}{\cpCut{y}{R}{Q}}} &\equiv \qquad \text{by \cpEquivCutComm} \\
      \tm{\cpCut{x}{P}{\cpCut{y}{Q}{R}}}
    \end{align*}
    The side conditions for \cpEquivCutAss1 are given.
  \end{proof}
%%% Local Variables:
%%% TeX-master: "main"
%%% End:

Furthermore, structural congruence is a symmetric relation.
\begin{theorem}[Symmetry]\label{thm:cp-symmetry}
  If $\tm{P} \equiv \tm{Q}$, then $\tm{Q} \equiv \tm{P}$.
\end{theorem}
\begin{proof}
  By induction on the structure of the equivalence proof.
\end{proof}
%%% Local Variables:
%%% TeX-master: "main"
%%% End:

%
Channels in \rcp are typed using a session type system which corresponds to RLL,
the multiplicative, additive fragment of linear logic.
These are defined using the following grammar:
\begin{definition}[Types]\label{def:cp-types}
  \[\!
    \begin{aligned}
      \ty{A}, \ty{B}, \ty{C}
           :=& \; \ty{A \tens B} &&\text{pair of independent processes}
      \\ \mid& \; \ty{A \parr B} &&\text{pair of interdependent processes}
      \\ \mid& \; \ty{\one}      &&\text{unit for} \; {\tens}
      \\ \mid& \; \ty{\bot}      &&\text{unit for} \; {\parr}
      \\ \mid& \; \ty{A \plus B} &&\text{internal choice}
      \\ \mid& \; \ty{A \with B} &&\text{external choice}
      \\ \mid& \; \ty{\nil}      &&\text{unit for} \; {\plus}
      \\ \mid& \; \ty{\top}      &&\text{unit for} \; {\with}
    \end{aligned}
  \]  
\end{definition}
%%% Local Variables:
%%% TeX-master: "main"
%%% End:

Duality plays a crucial role in both linear logic and session types.
In \rcp, the two endpoints of a channel are assigned dual types.
This ensures that, for instance, whenever a process \emph{sends} across a
channel, the process on the other end of that channel is waiting to
\emph{receive}.
Each type \ty{A} has a dual, written \ty{A^\bot}, which is defined as follows:
\begin{definition}[Negation]\label{def:cp-negation}
  \[\!
    \begin{array}{lclclcl}
              \ty{(A \tens B)^\bot} &=& \ty{A^\bot \parr B^\bot}
      &\quad& \ty{\one^\bot}        &=& \ty{\bot}
      \\      \ty{(A \parr B)^\bot} &=& \ty{A^\bot \tens B^\bot}
      &\quad& \ty{\bot^\bot}        &=& \ty{\one}
      \\      \ty{(A \plus B)^\bot} &=& \ty{A^\bot \with B^\bot}
      &\quad& \ty{\nil^\bot}        &=& \ty{\top}
      \\      \ty{(A \with B)^\bot} &=& \ty{A^\bot \plus B^\bot}
      &\quad& \ty{\top^\bot}        &=& \ty{\nil}
    \end{array}
  \]
\end{definition}
%%% Local Variables:
%%% TeX-master: "main"
%%% End:

Duality is an involutive function.
\begin{lemma}[Involutive]\label{thm:negation-involutive}
  We have $\ty{A^{\bot\bot}} = \ty{A}$.
\end{lemma}
\begin{proof}
  By induction on the structure of the type $\ty{A}$.
\end{proof}
%%% Local Variables:
%%% TeX-master: "main"
%%% End:

%
Environments associate channels with types. They are defined as follows:
\begin{definition}[Environments]\label{def:cp-environments}
  We define environments as follows:
  \[
    \ty{\Gamma}, \ty{\Delta}, \ty{\Theta}
    ::= \tmty{x_1}{A_1}\dots\tmty{x_n}{A_n}
  \] 
  Names in environments must be unique, and environments \ty{\Gamma} and
  \ty{\Delta} can only be combined as $\ty{\Gamma}, \ty{\Delta}$ if
  $\text{fv}(\ty{\Gamma}) \cap \text{fv}(\ty{\Delta}) = \varnothing$. 
\end{definition}
%%% Local Variables:
%%% TeX-master: "main"
%%% End:

Typing judgements associate processes with collections of typed channels. They
are defined as follows:
\begin{definition}[Typing judgements]\label{def:cp-typing-judgement}
  A typing judgement $\seq[{ P }]{\tmty{x_1}{A_1}\dots\tmty{x_n}{A_n}}$ denotes
  that the process \tm{P} communicates along channels $\tm{x_1}\dots\tm{x_n}$
  following protocols $\ty{A_1}\dots\ty{A_n}$.
  Typing judgements can be constructed using the inference rules in
  \cref{fig:cp-typing-judgement}.
\end{definition}
%%% Local Variables:
%%% TeX-master: "main"
%%% End:

\newcommand{\cpInfAx}{%
  \begin{prooftree*}
    \AXC{$\vphantom{\seq[ Q ]{ \Delta, \tmty{y}{A^\bot} }}$}
    \NOM{Ax}
    \UIC{$\seq[ \cpLink{x}{y} ]{ \tmty{x}{A}, \tmty{y}{A^\bot} }$}
  \end{prooftree*}
}
\newcommand{\cpInfCut}{%
  \begin{prooftree*}
    \AXC{$\seq[ P ]{ \Gamma, \tmty{x}{A} }$}
    \AXC{$\seq[ Q ]{ \Delta, \tmty{y}{A^\bot} }$}
    \NOM{Cut}
    \BIC{$\seq[ \cpCut{x}{P}{Q} ]{ \Gamma, \Delta }$}
  \end{prooftree*}
}
\newcommand{\cpInfTens}{%
  \begin{prooftree*}
    \AXC{$\seq[ P ]{ \Gamma , \tmty{y}{A} }$}
    \AXC{$\seq[ Q ]{ \Delta , \tmty{x}{B} }$}
    \SYM{(\tens)}
    \BIC{$\seq[ \cpSend{x}{y}{P}{Q} ]{ \Gamma , \Delta , \tmty{x}{A \tens B} }$}
  \end{prooftree*}
}
\newcommand{\cpInfParr}{%
  \begin{prooftree*}
    \AXC{$\seq[ P ]{ \Gamma , \tmty{y}{A} , \tmty{x}{B} }$}
    \SYM{(\parr)}
    \UIC{$\seq[ \cpRecv{x}{y}{P} ]{ \Gamma , \tmty{x}{A \parr B} }$}
  \end{prooftree*}
}
\newcommand{\cpInfOne}{%
  \begin{prooftree*}
    \AXC{$\vphantom{\seq[ P ]{ \Gamma }}$}
    \SYM{(\one)}
    \UIC{$\seq[ \cpHalt{x} ]{ \tmty{x}{\one} }$}
  \end{prooftree*}
}
\newcommand{\cpInfBot}{%
  \begin{prooftree*}
    \AXC{$\seq[ P ]{ \Gamma }$}
    \SYM{(\bot)}
    \UIC{$\seq[ \cpWait{x}{P} ]{ \Gamma , \tmty{x}{\bot} }$}
  \end{prooftree*}
}
\newcommand{\cpInfPlus}[1]{%
  \ifdim#1pt=1pt
  \begin{prooftree*}
    \AXC{$\seq[ P ]{ \Gamma , \tmty{x}{A} }$}
    \SYM{(\plus_1)}
    \UIC{$\seq[{ \cpInl{x}{P} }]{ \Gamma , \tmty{x}{A \plus B} }$}
  \end{prooftree*}
  \else%
  \ifdim#1pt=2pt
  \begin{prooftree*}
    \AXC{$\seq[ P ]{ \Gamma , \tmty{x}{B} }$}
    \SYM{(\plus_2)}
    \UIC{$\seq[ \cpInr{x}{P} ]{ \Gamma , \tmty{x}{A \plus B} }$}
  \end{prooftree*}
  \else%
  \fi%
  \fi%
}
\newcommand{\cpInfWith}{%
  \begin{prooftree*}
    \AXC{$\seq[ P ]{ \Gamma , \tmty{x}{A} }$}
    \AXC{$\seq[ Q ]{ \Delta , \tmty{x}{B} }$}
    \SYM{(\with)}
    \BIC{$\seq[ \cpCase{x}{P}{Q} ]{ \Gamma , \Delta , \tmty{x}{A \with B} }$}
  \end{prooftree*}
}
\newcommand{\cpInfNil}{%
  (no rule for \ty{\nil})
}
\newcommand{\cpInfTop}{%
  \begin{prooftree*}
    \AXC{}
    \SYM{(\top)}
    \UIC{$\seq[ \cpAbsurd{x} ]{ \Gamma, \tmty{x}{\top} }$}
  \end{prooftree*}
}
\begin{figure*}[b]
  \begin{center}
    \cpInfAx
    \cpInfCut
  \end{center}
  \begin{center}
    \cpInfTens
    \cpInfParr
  \end{center}
  \begin{center}
    \cpInfOne
    \cpInfBot
  \end{center}
  \begin{center}
    \cpInfPlus1
    \cpInfPlus2
  \end{center}
  \begin{center}
    \cpInfWith
  \end{center}
  \begin{center}
    \cpInfNil
    \cpInfTop
  \end{center}
  \caption{Typing judgement for the multiplicative applicative subset of \rcp.}
  \label{fig:cp-typing-judgement}
\end{figure*}
%%% Local Variables:
%%% TeX-master: "main"
%%% End:

Reductions relate processes with their reduced forms.
They are defined as follows:
\begin{definition}[Term reduction]\label{def:cp-term-reduction-1}
  A reduction $\reducesto{P}{P'}$ denotes that the process \tm{P} can reduce to
  the process \tm{P'} in a single step. Reductions can be constructed using the
  rules in~\cref{fig:cp-term-reduction-1,fig:cp-term-reduction-2}. 
\end{definition}
%%% Local Variables:
%%% TeX-master: "main"
%%% End:

\begin{figure*}[b]
  \[
    \begin{array}{llll}
      \cpRedAxCut1
      & \tm{\cpCut{x}{\cpLink{w}{x}}{P}}
      & \Longrightarrow \;
      & \tm{\cpSub{w}{x}{P}} 
      \\
      \cpRedAxCut2
      & \tm{\cpCut{x}{\cpLink{x}{w}}{P}}
      & \Longrightarrow \;
      & \tm{\cpSub{w}{x}{P}} 
      \\
      \cpRedBetaTensParr
      & \tm{\cpCut{x}{\cpSend{x}{y}{P}{Q}}{\cpRecv{x}{z}{R}}}
      & \Longrightarrow \;
      & \tm{\cpCut{y}{P}{\cpCut{x}{Q}{\cpSub{y}{z}{R}}}}
      \\
      \cpRedBetaOneBot
      & \tm{\cpCut{x}{\cpHalt{x}}{\cpWait{x}{P}}}
      & \Longrightarrow \;
      & \tm{P}
      \\
      \cpRedBetaPlusWith1
      & \tm{\cpCut{x}{\cpInl{x}{P}}{\cpCase{x}{Q}{R}}}
      & \Longrightarrow \;
      & \tm{\cpCut{x}{P}{Q}}
      \\
      \cpRedBetaPlusWith2
      & \tm{\cpCut{x}{\cpInr{x}{P}}{\cpCase{x}{Q}{R}}}
      & \Longrightarrow \;
      & \tm{\cpCut{x}{P}{R}}
    \end{array}
  \]
  \begin{prooftree}
    \AXC{\reducesto{P}{P^\prime}}
    \SYM{\cpRedGammaCut}
    \UIC{\reducesto{\cpCut{x}{P}{Q}}{\cpCut{x}{P^\prime}{Q}}}
  \end{prooftree}
  \begin{prooftree}
    \AXC{$\tm{P}\equiv\tm{Q}$}
    \AXC{\reducesto{Q}{Q^\prime}}
    \AXC{$\tm{Q^\prime}\equiv\tm{P^\prime}$}
    \SYM{\cpRedGammaEquiv}
    \TIC{\reducesto{P}{P^\prime}}
  \end{prooftree}
  \caption{Term reduction rules for \rcp.}
  \label{fig:cp-term-reduction-1}
\end{figure*}
%%% Local Variables:
%%% TeX-master: "main"
%%% End:

\begin{figure*}[b]
  \[
    \begin{array}{llll}
      \cpRedKappaTens1
      & \tm{\cpCut{x}{\cpSend{y}{z}{P}{Q}}{R}}
      & \Longrightarrow \;
      & \tm{\cpSend{y}{z}{\cpCut{x}{P}{R}}{Q}}
        \quad \text{if} \; \notFreeIn{x}{Q}
      \\
      \cpRedKappaTens2
      & \tm{\cpCut{x}{\cpSend{y}{z}{P}{Q}}{R}}
      & \Longrightarrow \;
      & \tm{\cpSend{y}{z}{P}{\cpCut{x}{Q}{R}}}
        \quad \text{if} \; \notFreeIn{x}{P}
      \\
      \cpRedKappaParr
      & \tm{\cpCut{x}{\cpRecv{y}{z}{P}}{R}}
      & \Longrightarrow \;
      & \tm{\cpRecv{y}{z}{\cpCut{x}{P}{R}}}
      \\
      \cpRedKappaBot
      & \tm{\cpCut{x}{\cpWait{y}{P}}{R}}
      & \Longrightarrow \;
      & \tm{\cpWait{y}{\cpCut{x}{P}{R}}}
      \\
      \cpRedKappaPlus1
      & \tm{\cpCut{x}{\cpInl{y}{P}}{R}}
      & \Longrightarrow \;
      & \tm{\cpInl{y}{\cpCut{x}{P}{R}}}
      \\
      \cpRedKappaPlus2
      &\tm{\cpCut{x}{\cpInr{y}{P}}{R}}
      & \Longrightarrow \;
      & \tm{\cpInr{y}{\cpCut{x}{P}{R}}}
      \\
      \cpRedKappaWith
      & \tm{\cpCut{x}{\cpCase{y}{P}{Q}}{R}}
      & \Longrightarrow \;
      & \tm{\cpCase{y}{\cpCut{x}{P}{R}}{\cpCut{x}{Q}{R}}}
      \\
      \cpRedKappaTop
      & \tm{\cpCut{x}{\cpAbsurd{y}}{R}}
      & \Longrightarrow \;
      & \tm{\cpAbsurd{y}}
    \end{array}
  \]
  \caption{Commutative conversions for \cp.}
  \label{fig:cp-term-reduction-2}
\end{figure*}
%%% Local Variables:
%%% TeX-master: "main"
%%% End:

We will discuss the interpretations of each connective, together with their
typing and reduction rules, in \cref{sec:cp-multiplicatives,sec:cp-additives,sec:cp-duality}.

\subsection{Multiplicatives and in- and interdependence}
\label{sec:cp-multiplicatives}
The multiplicatives ($\ty{\tens}, \ty{\parr}$) deal with independence and
interdependence:
\begin{itemize}
\item
  A channel of type \ty{A \tens B} represents a pair of channels, which
  communicate with two \emph{independent} processes---that is to say, two
  processes who share no channels.
  A process acting on a channel of type \ty{A \tens B} will send one endpoint of
  a fresh channel, and then split into a pair of independent processes.
  One of these processes will be responsible for an interaction of type \ty{A}
  over the fresh channel, while the other process continues to interact as
  \ty{B}.
\item
  A channel of type \ty{A \parr B} represents a pair of interdependent channels,
  which are used within a single process.
  A process acting on a channel of type \ty{A \parr B} will receive a channel to
  act on, and communicate on its channels in whatever order it pleases.
  This means that the usage of one channel can depend on that of
  another---e.g.\ the interaction of type \ty{B} could depend on the result of
  the interaction of type \ty{A}, or vise versa, and if \ty{A} and \ty{B} are
  complex types, their interactions could likewise interweave in complex ways.
\end{itemize}
While the rules for \ty{\tens} and \ty{\parr} introduce input and output
operations, these are inessential---the essential distinction lies two in the
fact that (\tens) composes two independent processes, and therefore \emph{must}
split the environment between them, whereas (\parr) uses a single process, which
then can---and must---use all the channels in the environment.
\begin{center}
  \cpInfTens
  \cpInfParr
\end{center}
The \textbeta-reduction rule for terms introduced by $(\tens)$ and $(\parr)$
implements the behaviour outlined above:
\[
  \tm{\cpCut{x}{\cpSend{x}{y}{P}{Q}}{\cpRecv{x}{z}{R}}}
  \Longrightarrow
  \tm{\cpCut{y}{P}{\cpCut{x}{Q}{\cpSub{y}{z}{R}}}}
\]
%
The rules for the multiplicative units ($\ty{\one}, \ty{\bot}$) follow the same
pattern, except for the nullary instead of the binary case:
\begin{itemize}
\item
  A term constructed by $(\one)$ must composes \emph{zero} independent
  processes, and thus must halt. Furthermore, it must be able to split its
  environment between zero processes, and thus its environment must be empty.
\item
  A term constructed by $(\bot)$, on the other hand, uses a single process,
  which is not further restricted.
\end{itemize}
Note that the rules for $\ty{\one}$ and $\ty{\bot}$ introduce a nullary send and
receive operation, such as those found in the polyadic \textpi-calculus.
\begin{center}
  \cpInfOne
  \cpInfBot
\end{center}
The \textbeta-reduction rule for terms introduced by $(\one)$ and $(\bot)$
implements the behaviour outlined above:
\[
  \tm{\cpCut{x}{\cpHalt{x}}{\cpWait{x}{P}}}
  \Longrightarrow
  \tm{P}
\]

\subsection{Additives and choice}
\label{sec:cp-additives}
The additives ($\ty{\plus}, \ty{\with}$) deal with choice:
\begin{itemize}
\item
  A process acting on a channel of type \ty{A \plus B} either sends the value
  \tm{inl} to select an interaction of type \ty{A} or the value \tm{inr} to
  select one of type \ty{B}.
\item
  A process acting on a channel of type \ty{A \with B} receives such a value,
  and then offers an interaction of either type \ty{A} or \ty{B},
  correspondingly.
\end{itemize}
Note that, in essence, the additive operations implement the sending and
receiving of a single bit of information, \tm{inl} or \tm{inr}, and branching
based on the value of that bit.
The rule for constructing a process which sends \tm{inr}, $(\plus_2)$, has been
omitted, but can be found in~\cref{fig:cp-typing-judgement}.
\begin{center}
  \cpInfPlus1
  \cpInfWith
\end{center}
The \textbeta-reduction rules for terms introduced by $(\plus_1)$, $(\plus_2)$
and $(\with)$ implement the behaviour outlined above.
\[
  \begin{array}{c}
    \tm{\cpCut{x}{\cpInl{x}{P}}{\cpCase{x}{Q}{R}}} \Longrightarrow \tm{\cpCut{x}{P}{Q}}
    \\
    \tm{\cpCut{x}{\cpInr{x}{P}}{\cpCase{x}{Q}{R}}} \Longrightarrow \tm{\cpCut{x}{P}{R}}
  \end{array}
\]
%
The rules for the additive units ($\ty{\nil}, \ty{\top}$) follow the same
pattern, except for a nullary choice:
\begin{itemize}
\item
  There is \emph{no} rule for \ty{\nil}, as a process acting on a channel of
  that type would have to select one of \emph{zero} options, which is clearly
  impossible.
\item
  A process acting on a channel of type \ty{\top} will wait to receive a choice
  of out \emph{zero} options. Since this will clearly never arrive, we have two
  options: either we block, waiting forever, or we simply crash.
\end{itemize}
It may seem odd at first to include a type for the process which cannot possibly
exist, and for the process which waits forever, but these make sensible units
for choice.
When offered a choice of type \ty{A \plus \nil}, one can either choose to
interact as \ty{A}, or choose to commit to doing the impossible.
Similarly, when offering a choice of type \ty{A \with \top}, one can safely
implement the right branch with a process which waits forever, as no sound
process will ever be able to select that branch anyway.
\begin{center}
  \cpInfNil
  \cpInfTop
\end{center}
As there is no way to construct a process of type \ty{\nil}, there is no
reduction rule for the additive units.


\subsection{Structural rules and duality}\label{sec:cp-duality}
Duality plays a crucial role in session type systems.
In~\cref{sec:cp-additives}, we saw that duality ensures a process offering a choice
is always matched with a process making a choice.
In~\cref{sec:cp-multiplicatives}, we saw that it is also crucial to deadlock freedom,
as it ensures that, for instance, a process which uses communication on \tm{x}
to decide what to send on \tm{y} is matched with a pair of independent processes
on \tm{x} and \tm{y}, preventing circular dependencies.

Duality appears in the typing rules for two \rcp term constructs.
Forwarding, \tm{\cpLink{x}{y}}, connects two dual channels with dual endpoints,
while composition, \tm{\cpCut{x}{P}{Q}}, composes two processes \tm{P} and
\tm{Q} with a shared channel \tm{x}, requiring that they follow dual protocols
on \tm{x}.
\begin{center}
  \cpInfAx
  \cpInfCut
\end{center}
There are two reduction rules which deal with the interactions between
forwarding and compositions. These implement the intuition that if a process is
meant to communicate on \tm{x}, \tm{x} is forwarding to \tm{y}, and nobody else
is listening on \tm{x}, then the process might as well start communicating on
\tm{y}.
\[
  \begin{array}{c}
    \tm{\cpCut{x}{\cpLink{w}{x}}{P}} \Longrightarrow \tm{\cpSub{w}{x}{P}} 
    \\
    \tm{\cpCut{x}{\cpLink{x}{w}}{P}} \Longrightarrow \tm{\cpSub{w}{x}{P}}  
  \end{array}
\]
Note that we can do this \emph{solely} because \rcp implements a binary session
type system, meaning that each communication has only two participants, and
therefore we know that no other process is communicating on \tm{x}.

\subsection{Commutative conversions}\label{sec:cp-commutative-conversions}
\wen{The commutative conversions are not rules typically found in the
  \textpi-calculus. Instead, they come from classical linear logic, where they
  play a crucial role in the procedure for cut elimination. These rules allow us
  to push a cut deeper into a term. Viewed from the other perspective, they
  allow us to pull actions upwards, past cuts, but never past the cut which
  introduced the channel on which they act.}

\subsection{Properties of \rcp}\label{sec:cp-properties}
In this section, we will prove three important properties of \rcp, namely
preservation, progress and termination.

\subsubsection{Preservation}
Preservation is the fact that term reduction preserves typing. In order to prove
this, we will first need to prove that equivalence preserves typing.
\begin{theorem}[Preservation for $\equiv$]\label{thm:cp-preservation-equiv}
  If $\seq[{ P }]{ \Gamma }$ and $\tm{P} \equiv \tm{Q}$,
  then $\seq[{ Q }]{\Gamma }$.
\end{theorem}
\begin{proof}
  By induction on the structure of the equivalence. The cases for reflexivity,
  transitivity and congruence are trivial. The two interesting cases, for
  \cpEquivCutComm and \cpEquivCutAss1 are given in \cref{fig:cp-preservation-equiv}
\end{proof}
%%% Local Variables:
%%% TeX-master: "main"
%%% End:

\begin{figure*}[ht]
  \centering
  \begin{tabular}{ll}
    \cpEquivCutComm
    &
      \begin{prooftree*}
        \AXC{$\seq[{ P }]{ \Gamma, \tmty{x}{A} }$}
        \AXC{$\seq[{ Q }]{ \Delta, \tmty{x}{A^\bot} }$}
        \NOM{Cut}
        \BIC{$\seq[{ \cpCut{x}{P}{Q} }]{ \Gamma, \Delta }$}
      \end{prooftree*}
    \\[30pt]
    $\equiv$
    &
      \begin{prooftree*}
        \AXC{$\seq[{ Q }]{ \Delta, \tmty{x}{A^\bot} }$}
        \AXC{$\seq[{ P }]{ \Gamma, \tmty{x}{A} }$}
        \NOM{\cref{thm:cp-negation-involutive}}
        \UIC{$\seq[{ P }]{ \Gamma, \tmty{x}{A^{\bot\bot}} }$}
        \NOM{Cut}
        \BIC{$\seq[{ \cpCut{x}{Q}{P} }]{ \Gamma, \Delta }$}
      \end{prooftree*}
    \\[40pt]
    \cpEquivCutAss1
    &
      \begin{prooftree*}
        \AXC{$\seq[{ P }]{ \Gamma, \tmty{x}{A} }$}
        \AXC{$\seq[{ Q }]{ \Delta, \tmty{x}{A^\bot}, \tmty{y}{B} }$}
        \AXC{$\seq[{ R }]{ \Theta, \tmty{y}{B^\bot} }$}
        \NOM{Cut}
        \BIC{$\seq[{ \cpCut{y}{Q}{R} }]{ \Delta, \Theta, \tmty{x}{A^\bot} }$}
        \NOM{Cut}
        \BIC{$\seq[{ \cpCut{x}{P}{\cpCut{y}{Q}{R}} }]{ \Gamma, \Delta, \Theta }$}
      \end{prooftree*}
    \\[30pt]
    $\equiv$
    &
      \begin{prooftree*}
        \AXC{$\seq[{ P }]{ \Gamma, \tmty{x}{A} }$}
        \AXC{$\seq[{ Q }]{ \Delta, \tmty{x}{A^\bot}, \tmty{y}{B} }$}
        \NOM{Cut}
        \BIC{$\seq[{ \cpCut{x}{P}{Q} }]{ \Gamma, \Delta, \tmty{y}{B} }$}
        \AXC{$\seq[{ R }]{ \Theta, \tmty{y}{B^\bot} }$}
        \NOM{Cut}
        \BIC{$\seq[{ \cpCut{y}{\cpCut{x}{P}{Q}}{R} }]{ \Gamma, \Delta, \Theta }$}
      \end{prooftree*}
  \end{tabular}
  \caption{Type preservation for the structural congruence of \rcp}
  \label{fig:cp-preservation-equiv}
\end{figure*}
%%% Local Variables:
%%% TeX-master: "main"
%%% End:

Then, we can prove preservation.
\begin{theorem}[Preservation]\label{thm:cp-preservation}
  If \reducesto{P}{Q} and $\seq[{ P }]{ \Gamma }$, then $\seq[{ Q }]{ \Gamma }$.
\end{theorem}
\begin{proof}
  By induction on the structure of the reduction. See
  \cref{fig:cp-preservation-1} for the \cpRedAxCut{i} and \textbeta-reduction
  rules, and \cref{fig:cp-preservation-2a,fig:cp-preservation-2b} for the
  commutative conversions.
  %
  The case for \cpRedGammaCut is trivial by the induction hypothesis, and the
  case for \cpRedGammaEquiv is trivial by the induction hypothesis and
  \cref{thm:cp-preservation-equiv}. 
\end{proof}
%%% Local Variables:
%%% TeX-master: "main"
%%% End:

\begin{figure*}[ht]
  \begin{tabular}{ll}
    \cpRedAxCut1
    &
      \begin{prooftree*}
        \AXC{}
        \NOM{Ax}
        \UIC{$\seq[{ \cpLink{w}{x} }]{ \tmty{w}{A}, \tmty{x}{A^\bot} }$}
        \AXC{$\seq[{ R }]{ \tmty{x}{A}, \Gamma }$}
        \NOM{Cut}
        \BIC{$\seq[{ \cpCut{x}{\cpLink{w}{x}}{R} }]{ \tmty{w}{A}, \Gamma }$}
      \end{prooftree*}
    \\[30pt]
    $\Longrightarrow$
    &
      \begin{prooftree*}
        \AXC{$\seq[{ \cpSub{w}{x}{R} }]{ \tmty{w}{A}, \Gamma }$}
      \end{prooftree*}
    \\[30pt]
    
    \cpRedAxCut2
    &
      (as above)
    \\[30pt]
    
    \cpRedBetaTensParr
    &
      \begin{prooftree*}
        \AXC{$\seq[{ P }]{ \Gamma, \tmty{x}{A^\bot}, \tmty{y}{B^\bot} }$}
        \SYM{\parr}
        \UIC{$\seq[{ \cpRecv{y}{x}P }]{ \Gamma, \tmty{y}{A^\bot \parr B^\bot} }$}
        \AXC{$\seq[{ Q }]{ \Delta, \tmty{x}{A} }$}
        \AXC{$\seq[{ R }]{ \Theta, \tmty{y}{B} }$}
        \SYM{\tens}
        \BIC{$\seq[{ \cpSend{y}{x}(Q \mid R) }]{ \Delta, \Theta, \tmty{y}{A \tens B} }$}
        \NOM{Cut}
        \BIC{$\seq[{ \cpCut{y}{\cpRecv{y}{x}{P}}{\cpSend{y}{x}{Q}{R}} }]{ \Gamma, \Delta, \Theta }$}
      \end{prooftree*}
    \\[30pt]
    $\Longrightarrow$
    &
      \begin{prooftree*}
        \AXC{$\seq[{ P }]{ \Gamma, \tmty{x}{A^\bot}, \tmty{y}{B^\bot} }$}
        \AXC{$\seq[{ Q }]{ \Delta, \tmty{x}{A} }$}
        \NOM{Cut}
        \BIC{$\seq[{ \cpCut{x}{P}{Q} }]{ \Gamma, \Delta, \tmty{y}{B^\bot} }$}
        \AXC{$\seq[{ R }]{ \Theta, \tmty{y}{B} }$}      
        \NOM{Cut}
        \BIC{$\seq[{ \cpCut{y}{\cpCut{x}{P}{Q}}{R} }]{ \Gamma, \Delta, \Theta }$}
      \end{prooftree*}
    \\[40pt]
    
    \cpRedBetaOneBot
    &
      \begin{prooftree*}
        \AXC{$\seq[{ P }]{ \Gamma }$}
        \SYM{\bot}
        \UIC{$\seq[{ \cpWait{x}{P} }]{ \Gamma, \tmty{x}{\bot} }$}
        \AXC{}
        \SYM{\one}
        \UIC{$\seq[{ \cpHalt{x} }]{ \tmty{x}{\one} }$}
        \NOM{Cut}
        \BIC{$\seq[{ \cpCut{x}{\cpHalt{x}}{\cpWait{x}{P}} }]{ \Gamma }$}
      \end{prooftree*}
    \\[30pt]
    $\Longrightarrow$
    &
      \begin{prooftree*}
        \AXC{$\seq[{ P }]{ \Gamma }$}
      \end{prooftree*}
    \\[40pt]
    
    \cpRedBetaPlusWith1
    &
      \begin{prooftree*}
        \AXC{$\seq[{ P }]{ \Gamma, \tmty{x}{A^\bot} }$}
        \AXC{$\seq[{ Q }]{ \Gamma, \tmty{x}{B^\bot} }$}
        \SYM{\with}
        \BIC{$\seq[{ \cpCase{x}{P}{Q} }]{ \Gamma, \tmty{x}{A^\bot \with B^\bot} }$}
        \AXC{$\seq[{ R }]{ \Delta, \tmty{x}{A} }$}
        \SYM{\plus_1}
        \UIC{$\seq[{ \cpInl{x}{R}}]{ \Delta, \tmty{x}{A \plus B} }$}
        \NOM{Cut}
        \BIC{$\seq[{ \cpCut{x}{\cpCase{x}{P}{Q}}{\cpInl{x}{R}} }]{ \Gamma, \Delta }$}
      \end{prooftree*}
    \\[30pt]
    $\Longrightarrow$
    &
      \begin{prooftree*}
        \AXC{$\seq[{ P }]{ \Gamma, \tmty{x}{A^\bot} }$}
        \AXC{$\seq[{ R }]{ \Delta, \tmty{x}{A} }$}
        \NOM{Cut}
        \BIC{$\seq[{ \cpCut{x}{P}{R} }]{ \Gamma, \Delta }$} 
      \end{prooftree*}
    \\[40pt]
    
    \cpRedBetaPlusWith2
    &
      (as above)
  \end{tabular}
  
  \caption{Type preservation for the \cpRedAxCut{i} and \textbeta-reduction rules of \cp}
  \label{fig:cp-preservation-1}
\end{figure*}
%%% Local Variables:
%%% TeX-master: "main"
%%% End:
\begin{figure*}[ht]
  \centering
  \begin{tabular}{ll}
    \cpRedKappaTens1
    &
      \begin{prooftree*}
        \AXC{$\seq[{ P }]{ \Gamma, \tmty{x}{A}, \tmty{z}{B} }$}
        \AXC{$\seq[{ Q }]{ \Delta, \tmty{y}{C} }$}
        \SYM{\tens}
        \BIC{$\seq[{ \cpSend{y}{z}{P}{Q} }]{ \Gamma, \Delta, \tmty{y}{B \tens C} }$}
        \AXC{$\seq[{ R }]{ \Theta, \tmty{x}{A^\bot} }$}
        \NOM{Cut}
        \BIC{$\seq[{ \cpCut{x}{\cpSend{y}{z}{P}{Q}}{R} }]{ \Gamma, \Delta, \Theta, \tmty{y}{B \tens C} }$}
      \end{prooftree*}
    \\[30pt]
    $\Longrightarrow$
    &
       \begin{prooftree*}
         \AXC{$\seq[{ P }]{ \Gamma, \tmty{x}{A}, \tmty{z}{B} }$}
         \AXC{$\seq[{ R }]{ \Theta, \tmty{x}{A^\bot} }$}
         \NOM{Cut}
         \BIC{$\seq[{ \cpCut{x}{P}{Q} }]{ \Gamma, \Theta, \tmty{z}{B} }$}
         \AXC{$\seq[{ Q }]{ \Delta, \tmty{y}{C} }$}
         \SYM{\tens}
         \BIC{$\seq[{ \cpSend{y}{z}{\cpCut{x}{P}{Q}}{R} }]{ \Gamma, \Delta, \Theta, \tmty{y}{B \tens C} }$}
       \end{prooftree*}
    \\[30pt]
    \cpRedKappaTens2
    &
      (as above)
    \\[20pt]
    \cpRedKappaParr
    &
      \begin{prooftree*}
        \AXC{$\seq[{ P }]{ \Gamma, \tmty{x}{A}, \tmty{z}{B}, \tmty{y}{C} }$}
        \SYM{\parr}
        \UIC{$\seq[{ \cpRecv{y}{z}{P} }]{ \Gamma, \tmty{x}{A}, \tmty{y}{B \parr C} }$}
        \AXC{$\seq[{ R }]{ \Theta, \tmty{x}{A^\bot} }$}
        \NOM{Cut} 
        \BIC{$\seq[{ \cpCut{x}{\cpRecv{y}{z}{P}}{R} }]{ \Gamma, \Theta, \tmty{y}{B \parr C} }$}
      \end{prooftree*}
    \\[30pt]
    $\Longrightarrow$
    &
      \begin{prooftree*}
        \AXC{$\seq[{ P }]{ \Gamma, \tmty{x}{A}, \tmty{z}{B}, \tmty{y}{C} }$}
        \AXC{$\seq[{ R }]{ \Theta, \tmty{x}{A^\bot} }$}
        \NOM{Cut}
        \BIC{$\seq[{ \cpCut{x}{P}{R} }]{ \Gamma, \Theta, \tmty{z}{B}, \tmty{y}{C} }$}
        \SYM{\parr}
        \UIC{$\seq[{ \cpRecv{y}{z}{\cpCut{x}{P}{R}} }]{ \Gamma, \Theta, \tmty{y}{B \parr C} }$}
      \end{prooftree*}
    \\[40pt]
    \cpRedKappaBot
    &
      \begin{prooftree*}
        \AXC{$\seq[{ P }]{ \Gamma, \tmty{x}{A} }$}
        \SYM{\bot}
        \UIC{$\seq[{ \cpWait{y}{P} }]{ \Gamma, \tmty{x}{A}, \tmty{y}{\bot} }$}
        \AXC{$\seq[{ R }]{ \Theta, \tmty{x}{A^\bot} }$}
        \NOM{Cut} 
        \BIC{$\seq[{ \cpCut{x}{\cpWait{y}{P}}{R} }]{ \Gamma, \Theta, \tmty{y}{\bot} }$}
      \end{prooftree*}
    \\[30pt]
    $\Longrightarrow$
    &
      \begin{prooftree*}
        \AXC{$\seq[{ P }]{ \Gamma, \tmty{x}{A} }$}
        \AXC{$\seq[{ R }]{ \Theta, \tmty{x}{A^\bot} }$}
        \NOM{Cut} 
        \BIC{$\seq[{ \cpCut{x}{P}{R} }]{ \Gamma, \Theta }$}
        \SYM{\bot}
        \UIC{$\seq[{ \cpWait{y}{\cpCut{x}{P}{R}} }]{ \Gamma, \Theta, \tmty{y}{\bot} }$}
      \end{prooftree*}
  \end{tabular}

  \caption{Type preservation for the commutative conversions of \cp}
  \label{fig:cp-preservation-2a}
\end{figure*}

\begin{figure*}[ht]
  \makebox[\textwidth][c]{
    \begin{tabular}{ll}
      \cpRedKappaPlus1
      &
        \begin{prooftree*}
          \AXC{$\seq[{ P }]{ \Gamma, \tmty{x}{A}, \tmty{y}{B} }$}
          \SYM{\plus_1}
          \UIC{$\seq[{ \cpInl{y}{P} }]{ \Gamma, \tmty{x}{A}, \tmty{y}{B \plus C} }$}
          \AXC{$\seq[{ R }]{ \Theta, \tmty{x}{A^\bot} }$}
          \NOM{Cut}
          \BIC{$\seq[{ \cpCut{x}{\cpInl{y}{P}}{R} }]{ \Gamma, \Theta, \tmty{y}{B \plus C} }$}
        \end{prooftree*}
      \\[30pt]
      $\Longrightarrow$
      &
        \begin{prooftree*}
          \AXC{$\seq[{ P }]{ \Gamma, \tmty{x}{A}, \tmty{y}{B} }$}
          \AXC{$\seq[{ R }]{ \Theta, \tmty{x}{A^\bot} }$}
          \NOM{Cut}
          \BIC{$\seq[{ \cpCut{x}{P}{R} }]{ \Gamma, \Theta, \tmty{y}{B} }$}
          \SYM{\plus_1}
          \UIC{$\seq[{ \cpInl{y}{\cpCut{x}{P}{R}} }]{ \Gamma, \Theta, \tmty{y}{B \plus C} }$}
        \end{prooftree*}
      \\[30pt] 
      \cpRedKappaPlus2
      &
        (as above)
      \\[20pt]
      \cpRedKappaWith
      &
        \begin{prooftree*}
          \AXC{$\seq[{ P }]{ \Gamma, \tmty{x}{A}, \tmty{y}{B} }$}
          \AXC{$\seq[{ Q }]{ \Gamma, \tmty{x}{A}, \tmty{y}{C} }$}
          \SYM{\with}
          \BIC{$\seq[{ \cpCase{y}{P}{Q} }]{ \Gamma, \tmty{x}{A}, \tmty{y}{B \with C} }$}
          \AXC{$\seq[{ R }]{ \Theta, \tmty{x}{A^\bot} }$}
          \NOM{Cut}
          \BIC{$\seq[{ \cpCut{x}{\cpCase{y}{P}{Q}}{R} }]{ \Gamma, \Theta, \tmty{y}{B \with C} }$}
        \end{prooftree*}
      \\[30pt]
      $\Longrightarrow$
      &
        \begin{prooftree*}
          \AXC{$\seq[{ P }]{ \Gamma, \tmty{x}{A}, \tmty{y}{B} }$}
          \AXC{$\seq[{ R }]{ \Theta, \tmty{x}{A^\bot} }$}
          \NOM{Cut}
          \BIC{$\seq[{ \cpCut{x}{P}{R} }]{ \Gamma, \Theta, \tmty{y}{B} }$}
          \AXC{$\seq[{ Q }]{ \Gamma, \tmty{x}{A}, \tmty{y}{C} }$}
          \AXC{$\seq[{ R }]{ \Theta, \tmty{x}{A^\bot} }$}
          \NOM{Cut}
          \BIC{$\seq[{ \cpCut{x}{Q}{R} }]{ \Gamma, \Theta, \tmty{y}{C} }$}
          \SYM{\with}
          \BIC{$\seq[{ \cpCase{y}{\cpCut{x}{P}{R}}{\cpCut{x}{Q}{R}} }]{ \Gamma, \Theta, \tmty{y}{B \with C} }$}
        \end{prooftree*}
      \\[40pt] 
      \cpRedKappaTop
      &
        \begin{prooftree*}
          \AXC{}
          \SYM{\top}
          \UIC{$\seq[{ \cpAbsurd{y} }]{ \Gamma, \tmty{x}{A}, \tmty{y}{\top} }$}
          \AXC{$\seq[{ R }]{ \Theta, \tmty{x}{A^\bot} }$}
          \NOM{Cut}
          \BIC{$\seq[{ \cpCut{x}{\cpAbsurd{y}}{R} }]{ \Gamma, \Theta, \tmty{y}{\top} }$}
        \end{prooftree*}
      \\[30pt]
      $\Longrightarrow$
      &
        \begin{prooftree*}
          \AXC{}
          \SYM{\top}
          \UIC{$\seq[{ \cpAbsurd{y} }]{ \Gamma, \Theta, \tmty{y}{\top} }$}
        \end{prooftree*}
    \end{tabular}
  }
  \caption{Type preservation for the commutative conversions of \cp (cont'd)}
  \label{fig:cp-preservation-2b}
\end{figure*}
%%% Local Variables:
%%% TeX-master: "main"
%%% End:

\subsubsection{Progress}
Progress is the fact that every term is either in some canonical form, or can be
reduced further. In order for a statement of progress to make sense, we need a
definition of canonical form. The canonical form used by \cp is ``any term which
is not a cut.'' We will refer to this canonical form as \emph{weak head normal
form}, for its relation to the eponymous \textlambda-calculus normal form. 
\begin{definition}[Weak head normal form]\label{def:cp-weak-head-normal-form}
  A process \tm{P} is in weak head normal form if it is not a cut.
\end{definition}
%%% Local Variables:
%%% TeX-master: "main"
%%% End:

As \cp has a tight correspondence with classical linear logic, so does its proof
of progress have a tight correspondence with (part of) the procedure proof
normalisation for classical linear logic for classical linear
logic~\cite{girard1987}.
\begin{theorem}[Progress, WHNF]\label{thm:cp-progress-1}
  If $\seq[{ P }]{ \Gamma }$, then either $\tm{P}$ is in weak head normal form,
  or there exists a $\tm{P'}$ such that $\reducesto{P}{P'}$.
\end{theorem}
\begin{proof}
  By induction on the structure of \tm{P}. The only interesting case is the
  case where \tm{P} is a cut \tm{\cpCut{x}{P'}{Q'}}. In every other case,
  \tm{P} is in weak head normal form. There are three possibilities:
  \begin{itemize}
  \item
    Both \tm{P'} and \tm{Q'} act on \tm{x}.
    \\
    We apply one of the \textbeta-reduction rules.
  \item
    Either \tm{P'} or \tm{Q'} acts on an external channel.
    \\
    We apply one of the commutative conversions.
  \item
    Otherwise one or both of \tm{P'} and \tm{Q'} are themselves cuts.
    \\
    We apply the induction hypothesis.
  \end{itemize}
\end{proof}
%%% Local Variables:
%%% TeX-master: "main"
%%% End:
  
If we extend the reduction system with all congruence rules---not just
\cpRedGammaCut for reduction under cuts, but for reduction under any term
context---then we can strengthen our canonical form, and extend our proof for
progress to the \emph{full} proof normalisation procedure.
\begin{definition}[Normal form]\label{def:cp-normal-form}
  A process \tm{P} is in normal form if it does not contain any cuts.
\end{definition}
%%% Local Variables:
%%% TeX-master: "main"
%%% End:

\begin{theorem}[Progress, NF]\label{thm:cp-progress-2}
  If $\seq[{ P }]{ \Gamma }$, then either \tm{P} is in normal form, or there
  exists a \tm{P'} such that \reducesto{P}{P'}.
\end{theorem}
\begin{proof}
  Either \tm{P} is in normal form, or there is a cut \emph{somewhere} in \tm{P}.
  If there is a cut, then we obtain a reduction by \cref{thm:cp-progress-1} and
  congruence.
\end{proof}
%%% Local Variables:
%%% TeX-master: "main"
%%% End:
 
\citenat{wadler2012} opts to leave these additional congruence rules out,
because ``such rules do not correspond well to our notion of computation on
processes'', and his choice is analogous to a common practice in the
\textlambda-calculus to not allow reduction under lambdas.

\wen{Perhaps, however, it would not be so odd to allow these reductions in the
  context of \cp, as the rigid structure of communication disallows the strange
  behaviours which could surface if we were to allow such reductions in the
  \textpi-calculus. In fact, it does not seem at all unreasonable, when one
  conceives of the two sides of a parallel composition \tm{\cpCut{x}{P}{Q}} as
  separate processes, to allow the communication on \tm{x} to happen even when
  the parent process, e.g.\ \tm{\cpRecv{y}{z}{\cpCut{x}{P}{Q}}}, is still
  waiting for an external communication to come in. It is simply eager a form of
  evaluation.}

\subsubsection{Termination}
Termination is the fact that if we iteratively apply progress to obtain a
reduction, and apply that reduction, we will eventually end up with a term in
canonical form.
Its proof is quite simple, owing to the fact that our reduction rules were all
inspired by cut reductions from classical linear logic.
\begin{theorem}[Termination]\label{thm:cp-termination}
  If $\seq[{ P }]{ \Gamma }$, then there are no infinite $\Longrightarrow$
  reduction sequences.
\end{theorem}
  \begin{proof}
    Every reduction reduces a single cut to zero, one or two cuts.
    However, each of these cuts is \emph{smaller}, in the sense that the type of
    the channel on which the communication takes place is smaller, as each
    reduction eliminates a connective---see
    \cref{fig:cp-preservation-1,fig:cp-preservation-2a,fig:cp-preservation-2b}.
    Therefore, there cannot be an infinite reduction sequence.
  \end{proof}
%%% Local Variables:
%%% TeX-master: "main"
%%% End:

%%% Local Variables:
%%% TeX-master: "main"
%%% End:
