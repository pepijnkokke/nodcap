\chapter{\cp as a type system for the \textpi-calculus}\label{sec:cppi}
\cp has a tight correspondence with classical linear logic.
This has many advantages.
It is deadlock free and terminating, and as seen in \cref{sec:cp-properties},
the proofs of its meta-theoretical properties are concise and elegant.
The price paid for this is a weaker correspondence with the \textpi-calculus.
It would be useful to be able to think of \cp as a type system for the
\textpi-calculus.
However, as it stands there are many differences between these systems.
For instance, in \cp name restriction and parallel composition are considered a
single, atomic construct, and it uses case statements instead of input-guarded
choice.
Most prominent among the differences, however, are the commuting conversions.
These reduction rules are taken directly from the proof normalisation procedure
of classical linear logic, and do not correspond to any reductions in the
\textpi-calculus. 

\textcite{lindley2015semantics} observed that, using a different reduction
strategy, which more closely resembles that of the \textpi-calculus, we can
ensure that the commuting conversions are always applied \emph{last}.
That is to say, they define two separate reduction relations:
$\longrightarrow_{C}$ for \cpRedAxCut1 and \textbeta-reductions,
and $\longrightarrow_{CC}$ for commuting conversions, and show that any
sequence of reductions is equivalent to one of the form:
\[
  P \longrightarrow_{C} \dots \longrightarrow_{C} Q \longrightarrow_{CC} \dots \longrightarrow_{CC} R
\]
In this dissertation, we use a reduction strategy which follows
\textcite{lindley2015semantics}, but drop the suffix of commutative
conversions.
Consequently, we can drop the commuting conversions from our reduction system,
which therefore more tightly corresponds to that of the \textpi-calculus.
The price we pay for this is a weaker correspondence to classical linear logic.
This shows in our notion of canonical form, which is weaker, and in our proof of
progress, which is much more involved.

Whenever we refer to \cp or \rcp, for the remainder of this dissertation, we
refer to the variant \emph{without} commuting conversions, i.e.\ which uses
the following definition of term reduction.
\begin{definition}[Term reduction]\label{def:cp-term-reduction-2}
  A reduction $\reducesto{P}{P'}$ denotes that the process \tm{P} can reduce to
  the process \tm{P'} in a single step. Reductions can be constructed using the
  rules in~\cref{fig:cp-term-reduction-2}.
\end{definition}
%%% Local Variables:
%%% TeX-master: "main"
%%% End:


This chapter will proceed as follows.
In \cref{sec:cp-canonical-forms}, we will describe what it means for a term to
be in canonical form.
In \cref{sec:cp-evaluation-contexts}, we will define evaluation contexts.
In \cref{sec:cp-progress}, we will give a new proof of progress, which follows
\textcite{lindley2015semantics}.

\section{Canonical forms}\label{sec:cp-canonical-forms}
The reduction strategy described by \citeauthor{lindley2015semantics} applies
\cpRedAxCut1 and \textbeta-reductions until the process blocks on
one or more external communications, and then applies the commuting conversions
to bubble one of those external communications to the front of the term.
This allows them to define canonical forms as any term which is not a cut.
For us, terms in canonical form will be those terms which are blocked on an
external communication, before any commuting conversions are applied.
In this section, we will describe the form of such terms.

We have informally used the phrase ``act on'' in previous sections. It is time
to formally define what it means when we say a process \emph{acts on} some
channel.
\begin{definition}[Action]\label{def:cp-action}
  A process \tm{P} \emph{acts on} a channel \tm{x} if it is in one of the
  following forms:
  \begin{multicols}{3}
    \begin{itemize}[noitemsep,topsep=0pt,parsep=0pt,partopsep=0pt]
    \item \tm{\cpLink{x}{y}}
    \item \tm{\cpLink{y}{x}}
    \item \tm{\cpSend{x}{y}{P'}{Q'}}
    \item \tm{\cpRecv{x}{y}{P'}}
    \item \tm{\cpHalt{x}}
    \item \tm{\cpWait{x}{P'}}
    \item \tm{\cpInl{x}{P'}}
    \item \tm{\cpInr{x}{P'}}
    \item \tm{\cpCase{x}{P'}{Q'}}
    \item \tm{\cpAbsurd{x}}
    \end{itemize}
  \end{multicols}
\end{definition}
%%% Local Variables:
%%% TeX-master: "main"
%%% End:

Furthermore, we will need the notion of an \emph{evaluation prefix}.
Intuitively, evaluation prefixes are multi-holed contexts consisting solely of
cuts. We will use evaluation prefixes in order to have a view of every
\emph{action} in a process at once.
\begin{definition}[Evaluation prefixes]\label{def:cp-evaluation-prefixes}
  We define evaluation prefixes as:
  \begin{align*}
    \tm{G}, \tm{H} := \tm{\Box} \mid \tm{\cpCut{x}{G}{H}}
  \end{align*}
\end{definition}
\begin{definition}[Plugging]\label{def:cp-evaluation-prefix-plugging}
  We define plugging for an evaluation prefix with $n$ holes as:
  \[
    \begin{array}{ll}
      \tm{\cpPlug{\Box}{R}} & := \; \tm{R} \\
      \tm{\cpPlug{\cpCut{x}{G}{H}}{R_1 \dots R_m, R_{m+1} \dots R_{n}}}
                            & := \; \tm{\cpCut{x}{\cpPlug{G}{R_1 \dots R_m}}{\cpPlug{H}{R_{m+1} \dots R_n}}}
    \end{array}
  \]
  Note that in the second case, \tm{G} is an evaluation prefix with $m$ holes,
  and \tm{H} is an evaluation prefix with $(n-m)$ holes.
\end{definition}
%%% Local Variables:
%%% TeX-master: "main"
%%% End:

Intuitively, we can say that every term of the form
\tm{\cpPlug{G}{P_1 \dots P_n}} is equivalent to some term of the form
\tm{\cpCut{x_1}{P_1}{\cpCut{x_2}{P_2}{\dots \cpCut{x_n}{P_{n-1}}{P_n} \dots}}} 
where $\tm{x_1} \dots \tm{x_{n-1}}$ are the channels bound in \tm{G}.
In fact, a similar equivalence was used by \citeauthor{lindley2015semantics}
\parencite{lindley2015semantics} in their semantics for \cp. 
\begin{definition}[Maximum evaluation prefix]\label{def:cp-maximum-evaluation-prefix}
  We say that \tm{G} is the evaluation prefix of \tm{P} when there exist terms
  $\tm{P_1} \dots \tm{P_n}$ such that $\tm{P} = \tm{\cpPlug{G}{P_1 \dots P_n}}$.
  We say that \tm{G} is the maximum evaluation prefix if each \tm{P_i} is an
  action. 
\end{definition}
\begin{lemma}\label{thm:cp-maximum-evaluation-prefix}
  Every term \tm{P} has a maximum evaluation prefix.
\end{lemma}
\begin{proof}
  By induction on the structure of \tm{P}.
\end{proof}
%%% Local Variables:
%%% TeX-master: "main"
%%% End:

We can now define what it means for a term to be in canonical form. Intuitively,
a process is in canonical form either when there is no top-level cut, or when it
is blocked on an external communication. We state this formally as follows:
\begin{definition}[Canonical forms]\label{def:cp-canonical-forms}
  We define the following forms to be canonical:
  \begin{center}
    \begin{prooftree*}
      \AXC{}
      \UIC{\canonical{\cpLink{x}{y}}}
    \end{prooftree*}
    \begin{prooftree*}
      \AXC{}
      \UIC{\canonical{\cpSend{x}{y}{P}{Q}}}
    \end{prooftree*}
    \begin{prooftree*}
      \AXC{}
      \UIC{\canonical{\cpRecv{x}{y}{P}}}
    \end{prooftree*}
  \end{center}
  \begin{center}
    \begin{prooftree*}
      \AXC{}
      \UIC{\canonical{\cpHalt{x}}}
    \end{prooftree*}
    \begin{prooftree*}
      \AXC{}
      \UIC{\canonical{\cpWait{x}{P}}}
    \end{prooftree*}
    \begin{prooftree*}
      \AXC{}
      \UIC{\canonical{\cpAbsurd{x}}}
    \end{prooftree*}
  \end{center}
  \begin{center}
    \begin{prooftree*}
      \AXC{}
      \UIC{\canonical{\cpInl{x}{P}}}
    \end{prooftree*}
    \begin{prooftree*}
      \AXC{}
      \UIC{\canonical{\cpInr{x}{P}}}
    \end{prooftree*}
    \begin{prooftree*}
      \AXC{}
      \UIC{\canonical{\cpCase{x}{P}{Q}}}
    \end{prooftree*}
  \end{center}
\end{definition}
%%% Local Variables:
%%% TeX-master: "main"
%%% End:


\section{Evaluation contexts}\label{sec:cp-evaluation-contexts}
Intuitively, evaluation contexts are one-holed term contexts under which
reduction can take place. For \rcp, these consist solely of cuts.
\begin{definition}[Evaluation contexts]\label{def:cp-evaluation-contexts}
  We define evaluation contexts as:
  \begin{align*}
    \tm{G}, \tm{H} := \tm{\Box}
    \mid \tm{\cpCut{x}{G}{P}}
    \mid \tm{\cpCut{x}{P}{G}}
  \end{align*}
\end{definition}
\begin{definition}[Plugging]\label{def:cp-evaluation-context-plugging}
  We define plugging for evaluation contexts as:
  \begin{gather*}
    \begin{array}{ll}
      \tm{\cpPlug{\Box}{R}}            
      & := \; \tm{R}
      \\
      \tm{\cpPlug{\cpCut{x}{G}{P}}{R}}
      & := \; \tm{\cpCut{x}{\cpPlug{G}{R}}{P}}
      \\
      \tm{\cpPlug{\cpCut{x}{P}{G}}{R}}
      & := \; \tm{\cpCut{x}{P}{\cpPlug{G}{R}}}
    \end{array}
  \end{gather*}
\end{definition}
%%% Local Variables:
%%% TeX-master: "main"
%%% End:

We can prove that we can push any cut downwards under an evaluation context, as
long as the channel it binds does not occur in the context itself.
In proofs throughout this dissertation, we will leave uses of
\cref{thm:cp-preservation-equiv} and \cref{thm:cp-preservation} implicit.
\begin{lemmaB}\label{thm:cp-display-cut-1}
  If $\seq[{ \tm{\cpCut{x}{\cpPlug{E}{P}}{Q}} }]{ \Gamma }$ and
  $\notFreeIn{x}{E}$, then $\tm{\cpCut{x}{\cpPlug{E}{P}}{Q}} \equiv
  \tm{\cpPlug{E}{\cpCut{x}{P}{Q}}}$. 
\end{lemmaB}
  \begin{proof}
    By induction on the structure of the evaluation context \tm{E}.
    \begin{itemize}
    \item
      Case $\tm{\Box}$. By reflexivity.
    \item
      Case $\tm{\cpCut{y}{E}{R}}$.
      \[\!
        \begin{array}{ll}
          \tm{\cpCut{x}{\cpCut{y}{\cpPlug{E}{P}}{R}}{Q}} & \equiv \quad \text{by \cpEquivCutComm}\\
          \tm{\cpCut{x}{\cpCut{y}{R}{\cpPlug{E}{P}}}{Q}} & \equiv \quad \text{by \cpEquivCutAss2}\\
          \tm{\cpCut{y}{R}{\cpCut{x}{\cpPlug{E}{P}}{Q}}} & \equiv \quad \text{by \cpEquivCutComm}\\
          \tm{\cpCut{y}{\cpCut{x}{\cpPlug{E}{P}}{Q}}{R}} & \equiv \quad \text{by the induction hypothesis}\\ 
          \tm{\cpCut{y}{\cpPlug{E}{\cpCut{x}{P}{Q}}}{R}} &
        \end{array}
      \]
    \item
      Case $\tm{\cpCut{y}{R}{E}}$.
      \[\!
        \begin{array}{ll}
          \tm{\cpCut{x}{\cpCut{y}{R}{\cpPlug{E}{P}}}{Q}} & \equiv \quad \text{by \cpEquivCutAss2}\\
          \tm{\cpCut{y}{R}{\cpCut{x}{\cpPlug{E}{P}}{Q}}} & \equiv \quad \text{by the induction hypothesis}\\
          \tm{\cpCut{y}{R}{\cpPlug{E}{\cpCut{x}{P}{Q}}}}
        \end{array}
      \]
    \end{itemize}
    In each case, the side conditions for \cpEquivCutAss2, $\notFreeIn{x}{R}$ and
    $\notFreeIn{y}{Q}$, can be inferred from $\notFreeIn{x}{E}$ and the fact that
    $\tm{\cpCut{x}{\cpPlug{E}{P}}{Q}}$ is well-typed.
  \end{proof}
%%% Local Variables:
%%% TeX-master: "main"
%%% End:
And vice versa. However, we will not use the following lemma in this
dissertation, and leave its proof as an exercise to the reader.
\begin{lemmaB}\label{thm:cp-display-cut-2}
  If $\seq[{ \cpPlug{E}{\cpCut{x}{P}{Q}} }]{ \Gamma }$ and $\notFreeIn{x}{E}$,
  then 
  $\tm{\cpPlug{E}{\cpCut{x}{P}{Q}}} \equiv \tm{\cpCut{x}{\cpPlug{E}{P}}{Q}}$. 
\end{lemmaB}
%%% Local Variables:
%%% TeX-master: "main"
%%% End:

\section{Progress}\label{sec:cp-progress}
Progress is the fact that every term is either in some canonical form, or can be
reduced further.
%
There are two important lemmas which we will need in order to prove progress.
These relate evaluation prefixes to evaluation contexts.
%
Specifically, if a process under an evaluation prefix is a link, we can rewrite
the entire process in such a way as to reveal the cut which introduced one of
the channels acted upon by that link.
\begin{lemmaB}\label{thm:cp-progress-link}
  If $\seq[{ \cpPlug{G}{P_1 \dots P_n} }]{ \Gamma }$, and some \tm{P_i} is a
  link \tm{\cpLink{x}{y}}, then either \tm{x} and \tm{y} are not bound by
  \tm{G}, or there exist \tm{H}, \tm{H'} and \tm{Q} such that either
  $\tm{\cpPlug{G}{P_1 \dots P_n}} \equiv
  \tm{\cpPlug{H}{\cpCut{x}{\cpPlug{H'}{\cpLink{x}{y}}}{Q}}}$ or 
  $\tm{\cpPlug{G}{P_1 \dots P_n}} \equiv
  \tm{\cpPlug{H}{\cpCut{y}{\cpPlug{H'}{\cpLink{x}{y}}}{Q}}}$. 
\end{lemmaB}
\begin{proof}
  By induction on the structure of \tm{G}.
  \begin{itemize}
  \item
    Case \tm{\Box}. Clearly \tm{x} and \tm{y} are not bound.
  \item
    Case \tm{\cpCut{z}{\cpPlug{G'}{P_1 \dots P_i \dots P_m}}{\cpPlug{G''}{P_{m+1} \dots P_n}}}.\\
    Case \tm{\cpCut{z}{\cpPlug{G'}{P_1 \dots P_m}}{\cpPlug{G''}{P_{m+1} \dots P_i \dots P_n}}}.\\
    We apply the induction hypothesis. There are two cases:
    \begin{itemize}
    \item
      If \tm{x} and \tm{y} were not bound, they remain unbound.
    \item
      If \tm{x} or \tm{y} is bound deeper in \tm{G}, we prepend one of
      \tm{\cpCut{z}{\Box}{\cpPlug{G''}{P_{m+1} \dots P_n}}},
      \tm{\cpCut{z}{\cpPlug{G'}{P_1 \dots P_m}}{\Box}},
      \tm{\ncPool{\Box}{\cpPlug{G''}{P_{m+1} \dots P_n}}}, or
      \tm{\ncPool{\cpPlug{G'}{P_1 \dots P_m}}{\Box}} to \tm{H}.
      \\
      The desired equality follows by congruence.
    \end{itemize}
  \item
    Case \tm{\cpCut{x}{\cpPlug{G'}{P_1 \dots P_i \dots P_m}}{\cpPlug{G''}{P_{m+1} \dots P_n}}}.\\
    Case \tm{\cpCut{y}{\cpPlug{G'}{P_1 \dots P_i \dots P_m}}{\cpPlug{G''}{P_{m+1} \dots P_n}}}.\\
    Case \tm{\cpCut{x}{\cpPlug{G'}{P_1 \dots P_m}}{\cpPlug{G''}{P_{m+1} \dots P_i \dots P_n}}}.\\
    Case \tm{\cpCut{y}{\cpPlug{G'}{P_1 \dots P_m}}{\cpPlug{G''}{P_{m+1} \dots P_i \dots P_n}}}.
    \\
    Let $\tm{H} := \tm{\Box}$ and 
    \(\arraycolsep=0pt\begin{array}[t]{ll}
      \text{let}
      & \; \tm{H_i} := \tm{\cpPlug{G'}{P_1 \dots P_{i-1}, \Box, P_{i+1}, \dots P_m}}
      \\
      \text{or}
      & \; \tm{H_i} := \tm{\cpPlug{G''}{P_{m+1} \dots P_{i-1}, \Box, P_{i+1}, \dots P_n}}
    \end{array}\)
    \\[1ex]
    By reflexivity and \cpEquivCutComm.
  \end{itemize}
\end{proof}
%%% Local Variables:
%%% TeX-master: "main"
%%% End:
If two processes under an evaluation prefix act on the same channel, then we
can rewrite the entire process in such a way as to reveal the cut that 
introduced the channel. 
\begin{lemmaB}\label{thm:cp-progress-beta}
  If $\seq[{ \cpPlug{G}{P_1 \dots P_n} }]{ \Gamma }$, and some \tm{P_i} and
  \tm{P_j}, on different sides of at least one cut, act on the same bound
  channel \tm{x}, then there exist \tm{H}, \tm{H_i} and \tm{H_j} such that 
  \(
  \tm{\cpPlug{G}{P_1 \dots P_n}} =
  \tm{\cpPlug{H}{\cpCut{x}{\cpPlug{H_i}{P_i}}{\cpPlug{H_j}{P_j}}}}
  \).
\end{lemmaB}
\begin{proof}
  By induction on the structure of \tm{G}.
  \begin{itemize}
  \item
    Case \tm{\cpCut{x}{\cpPlug{G'}{P_1\dots P_i\dots P_m}}{\cpPlug{G''}{P_{m+1}\dots P_j\dots P_n}}}. 
    \\
    \(\arraycolsep=0pt\begin{array}[t]{lll}
      \text{Let}&\ \tm{H}  &\ :=\ \tm{\Box}, \\
                &\ \tm{H_i}&\ :=\ \tm{\cpPlug{G'}{P_1\dots P_{i-1},\Box,P_{i+1}\dots P_m}}, \\
                &\ \tm{H_j}&\ :=\ \tm{\cpPlug{G''}{P_{m+1}\dots P_{j-1},\Box,P_{j+1}\dots P_n}},
    \end{array}\)
    \\[1ex]
    By reflexivity.
  \item
    Case \tm{\cpCut{x}{\cpPlug{G'}{P_1\dots P_j\dots P_m}}{\cpPlug{G''}{P_{m+1}\dots P_i\dots P_n}}}.
    \\
    As above.
  \item
    Case \tm{\cpCut{y}{\cpPlug{G'}{P_1\dots P_i\dots P_j\dots P_m}}{\cpPlug{G''}{P_{m+1}\dots P_n}}}. \\
    Case \tm{\cpCut{y}{\cpPlug{G'}{P_1\dots P_m}}{\cpPlug{G''}{P_{m+1}\dots P_i\dots P_j\dots P_n}}}.
    \\
    We obtain \tm{H}, \tm{H_1} and \tm{H_2} from the induction hypothesis and
    \cref{thm:cp-preservation-equiv}, and then prepend either
    \tm{\cpCut{y}{\Box}{\cpPlug{G''}{P_{m+1}\dots P_n}}} or
    \tm{\cpCut{y}{\cpPlug{G'}{P_1\dots P_m}}{\Box}} to \tm{H}.
    The desired equality follows by congruence.
  \end{itemize}
  The case for \tm{\Box} is excluded because $n > 1$.
  The cases in which \tm{P_i} and \tm{P_j} are on the \emph{same} side of the
  cut, but the cut binds \tm{x}, and the cases in which \tm{P_i} and \tm{P_j}
  are on different sides of the cut, but the cut binds some other channel
  \tm{y}, are excluded by the type system.
\end{proof}
%%% Local Variables:
%%% TeX-master: "main"
%%% End:


Finally, we are ready to prove progress.
In proofs throughout this dissertation, we will leave uses of
\cpRedGammaEquiv and the congruence rules for reductions implicit. 
\begin{theorem}[Progress]\label{thm:cp-progress-3}
  If $\seq[{ P }]{ \Gamma }$, then either $\tm{P}$ is in canonical form, or
  there exists a $\tm{P'}$ such that $\reducesto{P}{P'}$. 
\end{theorem}
\begin{proof}
  By induction on the structure of derivation for $\seq[{ P }]{ \Gamma }$.
  The only interesting case is when the last rule of the derivation is
  \textsc{Cut}---in every other case, the typing rule constructs a term in which 
  is in canonical form. 
  \\
  We consider the maximum evaluation prefix \tm{G} of \tm{P}, such that $\tm{P}
  = \tm{\cpPlug{G}{P_1 \dots P_{n+1}}}$ and each \tm{P_i} is an action.
  The prefix \tm{G} consists of $n$ cuts, and introduces $n$ channels, but
  composes $n+1$ actions. Therefore, one of the following must be true:
  \begin{itemize}
  \item
    One of the processes is a link \tm{\cpLink{x}{y}} acting on a bound channel.
    We have:
    \begin{gather*}
      \begin{array}{ll}
        \tm{\cpPlug{G}{P_1 \dots \cpLink{x}{y} \dots P_{n+1}}}
        & \equiv \quad \text{by \cref{thm:cp-progress-link}}
        \\
        \tm{\cpPlug{E}{\cpCut{z}{\cpPlug{E'}{\cpLink{x}{y}}}{Q}}}
        & \equiv \quad \text{by \cref{thm:cp-display-cut-1}}
        \\
        \tm{\cpPlug{E}{\cpPlug{E'}{\cpCut{z}{\cpLink{x}{y}}{Q}}}}
      \end{array}
    \end{gather*}
    Where $\tm{z} = \tm{x}$ or $\tm{z} = \tm{y}$.
    We then apply \cpRedAxCut1.
  \item
    Two of the processes, \tm{P_i} and \tm{P_j}, act on the same bound channel
    \tm{x}. We have:
    \begin{gather*}
      \begin{array}{ll}
        \tm{\cpPlug{G}{P_1 \dots P_i \dots P_j \dots P_{n+1}}}
        & = \quad \text{by \cref{thm:cp-progress-beta}}
        \\
        \tm{\cpPlug{G}{\cpCut{x}{\cpPlug{E_i}{P_i}}{\cpPlug{E_j}{P_j}}}}
        & \equiv \quad \text{by \cref{thm:cp-display-cut-1}} 
        \\
        \tm{\cpPlug{G}{\cpPlug{E_i}{\cpCut{x}{P_i}{\cpPlug{E_j}{P_j}}}}}
        & \equiv \quad \text{by \cref{thm:cp-display-cut-1}} 
        \\
        \tm{\cpPlug{G}{\cpPlug{E_i}{\cpPlug{E_j}{\cpCut{x}{P_i}{P_j}}}}} 
      \end{array}
    \end{gather*}
    We then apply one of the \textbeta-reduction rules.
  \item
    Otherwise (at least) one of the processes acts on an external channel.
    \\
    No process \tm{P_i} is a link.
    No two processes \tm{P_i} and \tm{P_j} act on the same channel \tm{x}.
    Therefore, \tm{P} is canonical.
  \end{itemize}
\end{proof}
%%% Local Variables:
%%% TeX-master: "main"
%%% End:

The proof of progress described in this section is novel, though it takes
inspiration from the reduction system for \cp described by \textcite{lindley2015semantics}. 
The proof itself is somewhat involved. The reason for this is that we want our
reduction strategy to match that of the \textpi-calculus as closely as possible.
It is also for this reason that our proof of progress involves some
non-determinism. For instance, in the second case of the proof, we do not
specify how to select the two processes \tm{P_i} and \tm{P_j} if there are
multiple options available.

\section{Rewriting versus commuting}
Let's have a look at the differences between the reduction strategy we have
defined in this chapter, and the reduction strategy which \textcite{wadler2012}
defines. Let's imagine the following scenario:
\begin{quote}
  Alice, John, and Mary went on a lovely trip together.
  However, John and Mary can be a bit scatterbrained sometimes, and it just so
  happens that they both forgot to bring their wallets.
  They both owe Alice some money, which they're now trying to pay back at the
  same time.
\end{quote}
We can model this interaction in \cp as the following term, assuming \alice,
\john, and \mary are three processes representing Alice, John and Mary, and
\bankjohn and \bankmary are two processes representing John and Mary's
respective banks.
To simplify notation, we will adopt an ``anti-Barendregt'' convention in our
examples. 
This means that if two channel names will be identified after a reduction, we
will preemptively give them the same name, for instance, \tm{z} and \tm{w} in
the example below. 
\[
  \tm{\cpCut{x}{\cpCut{y}{
  \cpRecv{x}{z}{\cpRecv{y}{w}{\alice}}}{
  \cpSend{y}{w}{\bankjohn}{\john}}}{
  \cpSend{x}{z}{\bankmary}{\mary}}}
\]
In the above interaction, no \textbeta-reduction rule applies
\emph{immediately}. This means that using the reduction strategy described by
\textcite{wadler2012}, we apply a commuting conversion.
\[
  \begin{array}{cll}
    \tm{\cpCut{x}{\cpCut{y}{
    \cpRecv{x}{z}{\cpRecv{y}{w}{\alice}}}{
    \cpSend{y}{w}{\bankjohn}{\john}}}{
    \cpSend{x}{z}{\bankmary}{\mary}}}
    & \Longrightarrow & \text{by \cpRedKappaParr}
    \\
    \tm{\cpCut{x}{\cpRecv{x}{z}{\cpCut{y}{
    \cpRecv{y}{w}{\alice}}{
    \cpSend{y}{w}{\bankjohn}{\john}}}}{
    \cpSend{x}{z}{\bankmary}{\mary}}}
    & \Longrightarrow & \text{by \cpRedBetaTensParr}
    \\
    \tm{\cpCut{z}{\cpCut{x}{\cpCut{y}{
    \cpRecv{y}{w}{\alice}}{
    \cpSend{y}{w}{\bankjohn}{\john}}}{
    \mary}}{
    \bankmary}}
    & \Longrightarrow & \text{by \cpRedBetaTensParr}
    \\
    \tm{\cpCut{z}{\cpCut{x}{\cpCut{w}{\cpCut{y}{
    \alice}{
    \john}}{
    \bankjohn}}{
    \mary}}{
    \bankmary}}
  \end{array}
\]
On the other hand, using the reduction strategy described in this chapter, we will
rewrite using structural congruence.
\[
  \begin{array}{cll}
    \tm{\cpCut{x}{\cpCut{y}{
    \cpRecv{x}{z}{\cpRecv{y}{w}{\alice}}}{
    \cpSend{y}{w}{\bankjohn}{\john}}}{
    \cpSend{x}{z}{\bankmary}{\mary}}}
    & \equiv          & \text{by \cref{thm:cp-display-cut-1}}
    \\
    \tm{\cpCut{y}{\cpCut{x}{
    \cpRecv{x}{z}{\cpRecv{y}{w}{\alice}}}{
    \cpSend{x}{z}{\bankmary}{\mary}}}{
    \cpSend{y}{w}{\bankjohn}{\john}}}
    & \Longrightarrow & \text{by \cpRedBetaTensParr}
    \\
    \tm{\cpCut{y}{\cpCut{z}{\cpCut{x}{
    \cpRecv{y}{w}{\alice}}{
    \mary}}{
    \bankmary}}{
    \cpSend{y}{w}{\bankjohn}{\john}}} 
    & \equiv          & \text{by \cref{thm:cp-display-cut-1}}
    \\
    \tm{\cpCut{z}{\cpCut{x}{\cpCut{y}{
    \cpRecv{y}{w}{\alice}}{
    \cpSend{y}{w}{\bankjohn}{\john}}}{
    \mary}}{
    \bankmary}}
    & \Longrightarrow & \text{by \cpRedBetaTensParr}
    \\
    \tm{\cpCut{z}{\cpCut{x}{\cpCut{w}{\cpCut{y}{
    \alice}{
    \john}}{
    \bankjohn}}{
    \mary}}{
    \bankmary}}
  \end{array}
\]
Note that the second reduction sequence is a \emph{valid} reduction sequence in
the either reduction system---it is simply not the sequence \emph{chosen} by the
reduction strategy described by \textcite{wadler2012}.
In fact, the structural congruence is only used in a \emph{single} instance in
the original reduction strategy for \cp---to derive the flipped version of
\cpRedGammaCut.

%%% Local Variables:
%%% TeX-master: "main"
%%% End:
