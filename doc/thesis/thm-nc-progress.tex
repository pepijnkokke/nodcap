\begin{theorem}[Progress]\label{thm:nc-progress}
  If $\seq[{ P }]{ \Gamma }$, then either $\tm{P}$ is in canonical form, or
  there exists a $\tm{P'}$ such that $\reducesto{P}{P'}$. 
\end{theorem}
\begin{proof}
  By induction on the structure of derivation for $\seq[{ P }]{ \Gamma }$.
  The only interesting cases are when the last rule of the derivation is
  \textsc{Cut} or \textsc{Pool}. In every other case, the typing rule constructs
  a term in which is in canonical form. 
  \\
  If the last rule in the derivation is \textsc{Cut} or \textsc{Pool}, we
  consider the maximum evaluation prefix \tm{G} of \tm{P}, such that $\tm{P} =
  \tm{\cpPlug{G}{P_1 \dots P_{m+n+1}}}$ and each \tm{P_i} is an action.
  The prefix \tm{G} consists of $m$ pools, $n$ cuts, and introduces $n$
  channels, but composes $m+n+1$ actions, at most $m+1$ of which are on the same
  side of all cuts.
  Therefore, one of the following must be true:
  \begin{itemize}
  \item
    One of the processes is a link \tm{\cpLink{x}{y}} acting on a bound channel.
    \\
    We proceed as in \cref{thm:cp-progress-3},
    replacing \cref{thm:cp-display-cut-1} and \cref{thm:cp-progress-link}
    with \cref{thm:nc-display-cut-1} and \cref{thm:nc-progress-link}.
  \item
    Two of the processes, \tm{P_i} and \tm{P_j}, on different sides of at least
    one cut, act on the same bound channel \tm{x}. We have:
    \begin{gather*}
      \begin{array}{ll}
        \tm{\cpPlug{G}{P_1 \dots P_i \dots P_j \dots P_{m+n+1}}}
        & \equiv \quad \text{by \cref{thm:nc-progress-beta}} \\
        \tm{\cpPlug{H}{\cpCut{x}{\cpPlug{H_i}{P_i}}{\cpPlug{H_j}{P_j}}}}
      \end{array}
    \end{gather*}
    There are two cases:
    \begin{itemize}
    \item
      If \tm{x} is a shared channel, we have \tmty{x}{\take[n]{A}} with $n > 1$ 
      in either \tm{\cpPlug{H_i}{P_i}} or \tm{\cpPlug{H_j}{P_j}}.
      Assume the former. 
      We can infer $\notFreeIn{x}{H_j}$.
      We have:
      \begin{gather*}
        \begin{array}{ll}
          \tm{\cpPlug{H}{\cpCut{x}{\cpPlug{H_i}{P_i}}{\cpPlug{H_j}{P_j}}}}
          & \equiv \quad \text{by \cref{thm:nc-display-pool-1}}
          \\
          \tm{\cpPlug{H}{\cpPlug{H_j}{\cpCut{x}{\cpPlug{H_i}{P_i}}{P_j}}}}
          & \equiv \quad \text{by \cref{thm:nc-progress-shared}}
          \\
          \multicolumn{2}{l}{
          \tm{\cpPlug{H}{\cpPlug{H_j}{\cpPlug{H_i^\prime}{\cpCut{x}{\ncPool{P_i}{
          \ncPool{R_1}{\ncPool{\dots}{R_{n-1}} \dots }}}{P_j}}}}}}
        \end{array}
      \end{gather*}
      We apply \ncRedBetaStar{n+1}.
      Similarly if \tmty{x}{\take[n]{A}} in \tm{\cpPlug{H_j}{P_j}}.
    \item
      Otherwise, we can infer $\notFreeIn{y}{H_i}$ and $\notFreeIn{y}{H_j}$.
      \\
      We proceed as in \cref{thm:cp-progress-3}, including \ncRedBetaStar1 in
      the \textbeta-reduction rules, and replacing \cref{thm:cp-display-cut-1}
      and \cref{thm:cp-progress-beta} with \cref{thm:nc-display-cut-1} and
      \cref{thm:nc-progress-beta}.
    \end{itemize}
  \item 
    Otherwise (at least) one of the actions acts on a free variable.
    \\
    No process \tm{P_i} is a link acting on a bound channel.
    No two processes \tm{P_i} and \tm{P_j} act on the same channel \tm{x}.
    Therefore, \tm{P} is canonical.
  \end{itemize}
\end{proof}
%%% Local Variables:
%%% TeX-master: "main"
%%% End:
