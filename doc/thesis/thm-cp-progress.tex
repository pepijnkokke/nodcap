\begin{theorem}[Progress]\label{thm:cp-progress}
  If $\seq[{ P }]{ \Gamma }$,
  then either $\tm{P}$ is in canonical form,
  or there exists some $\tm{P'}$ for which $\reducesto{P}{P'}$.
\end{theorem}
\begin{proof}
  By induction on the structure of derivation for $\seq[{ P }]{ \Gamma }$.
  The only interesting case is when the last rule of the derivation is
  \textsc{Cut}. In every other case, the typing rule constructs a term in which
  is in canonical form. 
  %
  If the last rule in the derivation is \textsc{Cut}, we have:
  \begin{prooftree}
    \AXC{$\seq[{ P' }]{ \Gamma, \tmty{x}{A} }$}
    \AXC{$\seq[{ Q' }]{ \Delta, \tmty{x}{A^\bot} }$}
    \NOM{Cut}
    \BIC{$\seq[{ \cpCut{x}{P'}{Q'} }]{ \Gamma, \Delta }$}
  \end{prooftree}
  We consider three important cases:
  \begin{itemize}
  \item
    If either \tm{P'} or \tm{Q'} was introduced using the \textsc{Ax} rule, we
    can apply one of \cpRedAxCut1 or \cpRedAxCut2; 
  \item
    If both \tm{P'} and \tm{Q'} act on \tm{x}, we can apply one of the
    \textbeta-reduction rules;
  \item
    % TODO introduce the notion of an evaluation context
    %
    %   G := [] | vx.(G | P) | vx.(P | G) 
    %
    % s.t. we can rewrite any vx.(G[P]|Q) to G[vx.(P|Q)]
    % then state that there are two cases
    % either P is of the form 
    %
    %   vx.(G[P]|H[Q]) s.t. both P and Q act on x
    %
    % or there is a G s.t. vx.(G[P]|Q) or vx.(P|G[Q])
    % and P (or Q) acts on a free variable
    % because n cuts are composed of n+1 actions, one of these must be true
    % the second case is now superfluous
  \end{itemize}
\end{proof}
%%% Local Variables:
%%% TeX-master: "main"
%%% End:
