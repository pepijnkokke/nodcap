\begin{theorem}[Progress]\label{thm:cp-progress}
  If $\seq[{ P }]{ \Gamma }$, then $\tm{P}$ is in canonical form, or there
  exists a $\tm{P'}$ s.t.\ $\reducesto{P}{P'}$. 
\end{theorem}
\begin{proof}
  By induction on the structure of derivation for $\seq[{ P }]{ \Gamma }$.
  The only interesting case is when the last rule of the derivation is
  \textsc{Cut}. In every other case, the typing rule constructs a term in which
  is in canonical form. 
  \\
  If the last rule in the derivation is \textsc{Cut}, we consider the maximum
  evaluation prefix \tm{G} of \tm{P}, such that $\tm{P} = \tm{\cpPlug{G}{P_1
      \dots P_{n+1}}}$ and each $P_i$ is an action.
  The prefix \tm{G} consists of $n$ cuts, and introduces $n$ variables, but
  composes $n+1$ actions. Therefore, one of the following must be true:
  \begin{itemize}
  \item
    One of the processes is a link \tm{\cpLink{x}{y}}.
    We have:
    \begin{gather*}
      \begin{array}{ll}
        \tm{\cpPlug{G}{P_1 \dots \cpLink{x}{y} \dots P_{n+1}}}
        & = \quad \text{by \cref{thm:cp-display-3}}
        \\
        \tm{\cpPlug{H}{\cpCut{z}{\cpPlug{H'}{\cpLink{x}{y}}}{Q}}}
        & \equiv \quad \text{by \cref{thm:cp-display-1}}
        \\
        \tm{\cpPlug{H}{\cpPlug{H'}{\cpCut{z}{\cpLink{x}{y}}{Q}}}}
      \end{array}
    \end{gather*}
    Where $\tm{z} = \tm{x}$ or $\tm{z} = \tm{y}$.
    We then apply one of \cpRedAxCut1 or \cpRedAxCut2.
  \item
    Two of the processes, \tm{P_i} and \tm{P_j}, act on the same channel \tm{x}.
    \begin{gather*}
      \begin{array}{ll}
        \tm{\cpPlug{G}{P_1 \dots P_i \dots P_j \dots P_{n+1}}}
        & = \quad \text{by \cref{thm:cp-display-4}}
        \\
        \tm{\cpPlug{G}{\cpCut{x}{\cpPlug{H_i}{P_i}}{\cpPlug{H_j}{P_j}}}}
        & \equiv \quad \text{by \cref{thm:cp-display-1}} 
        \\
        \tm{\cpPlug{G}{\cpPlug{H_i}{\cpCut{x}{P_i}{\cpPlug{H_j}{P_j}}}}}
        & \equiv \quad \text{by \cref{thm:cp-display-1} and
          \cref{thm:cp-preservation-equiv}} 
        \\
        \tm{\cpPlug{G}{\cpPlug{H_i}{\cpPlug{H_j}{\cpCut{x}{P_i}{P_j}}}}} 
      \end{array}
    \end{gather*}
    We then apply one of the \textbeta-reduction rules.
  \item
    Otherwise (at least) one of the actions acts on a free variable.
    \\
    No process \tm{P_i} is a link.
    No two processes \tm{P_i} and \tm{P_j} act on the same channel \tm{x}.
    Therefore, \tm{P} is canonical.
  \end{itemize}
\end{proof}
%%% Local Variables:
%%% TeX-master: "main"
%%% End:
