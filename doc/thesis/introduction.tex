%% Introduction
\chapter{Introduction}\label{sec:introduction}
%% - Motivating examples
Consider the following scenario:
\begin{quote}
  John and Mary are working from home one morning when they get a craving for a
  slice of cake. Being denizens of the web, they quickly find the nearest store
  which does home deliveries.
  Unfortunately for them, they both order their cake at the \emph{same} store,
  which has only one slice left. After that, all it can deliver is
  disappointment.
\end{quote}
This is an example of a \emph{race condition}. We can model this scenario in the
\textpi-calculus, assuming \john, \mary and \store are three processes
modeling John, Mary and the store, and \sliceofcake and \nope are two channels
giving access to a slice of cake and disappointment, respectively.
As expected, this process has two possible outcomes: either John gets the cake,
and Mary gets disappointment, or vice versa.
\[
  \begin{array}{c}
    \tm{(\piPar{%
    \piSend{x}{\sliceofcake}{\piSend{x}{\nope}{\store}}
    }{%
    \piPar{\piRecv{x}{y}{\john}}{\piRecv{x}{z}{\mary}}
    })}
    \\[1ex]
    \rotatebox[origin=c]{270}{$\Longrightarrow^{\star}$}
    \\[1ex]
    \tm{(\piPar{\store}{\piPar{\piSub{\sliceofcake}{y}{\john}}{\piSub{\nope}{z}{\mary}}})}
    \quad
    \text{or}
    \quad
    \tm{(\piPar{\store}{\piPar{\piSub{\nope}{y}{\john}}{\piSub{\sliceofcake}{z}{\mary}}})}
  \end{array}
\]
While John or Mary may not like all of the outcomes, it is the store which is
responsible for implementing the online delivery service, and the store is happy
with either outcome. Thus, the above is an interaction we would like to be able to
model.

Now consider another scenario, which takes place \emph{after} John has already
bought the cake:
\begin{quote}
  Mary is \emph{really} disappointed when she finds out the cake has sold out.
  John, always looking to make some money, offers to sell the slice to her for a
  profit. Mary agrees to engage in a little bit of back-alley cake resale, but
  sadly there is no trust between the two.
  John demands payment first.
  Mary would rather get her slice of cake before she gives John the money.
\end{quote}
This is an example of a \emph{deadlock}. We can also model this scenario in the
\textpi-calculus, assuming that \bill\ is a channel giving access to some
adequate amount of money.
\[
  \begin{array}{c}
    \tm{(\piPar{%
    \piRecv{x}{z}{\piSend{y}{\sliceofcake}{\john}}
    }{%
    \piRecv{y}{w}{\piSend{x}{\bill}{\mary}}
    })}
    \quad
    \centernot\Longrightarrow^{\star}
  \end{array}  
\]
The above process does not reduce. As both John and Mary would prefer the
exchange to be made, this interaction is desired by \emph{neither}. Thus, the
above is an interaction we would \emph{somehow} like to exclude.

Session types~\cite{honda1993} statically guarantee that concurrent
programs, such as those above, respect communication protocols.
Session-typed calculi with logical foundations, such as
\piDILL~\cite{caires2010} and CP~\cite{wadler2012}, obtain deadlock freedom as a
result of a close correspondence with logic.
The same correspondence, however, also rules out non-determinism and race
conditions.

In this dissertation, we present \nodcap (nodcap), an extension of
CP~\cite{wadler2012} with a novel account of non-determinism and races.
Inspired by bounded linear logic~\cite{girard1992}, we introduce a form of
shared channels, in which the type of a shared channel tracks how many times it
is reused.
As in the untyped $\pi$-calculus, sharing introduces the potential for
non-determinism.
We show that our approach is sufficient to capture practical examples of races,
such as the web store, as well as other formal characterizations of
non-determinism, such as non-deterministic choice.  However, \nodcap does not
lose the metatheoretical benefits of CP: we show that it enjoys termination and
deadlock-freedom.

This dissertation proceeds as follows.
In \cref{sec:background}, we introduce \cp and reproduce the proofs of some of
its meta-theoretical results, and discuss related work which also addresses the
addition of non-determinism to logic-inspired session typed process calculi.
In \cref{sec:cppi}, we introduce an alternative reduction strategy for \cp,
which more closely resembles reduction in the \textpi-calculus.
In \cref{sec:main}, we introduce \nodcap.
Finally, in \cref{sec:discussion}, we conclude with a discussion of the work
done in this dissertation and potential avenues for future work.
%%% Local Variables:
%%% TeX-master: "main"
%%% End:
