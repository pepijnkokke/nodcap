\documentclass[a4paper,UKenglish]{article}
\usepackage[margin=1.5cm, includefoot, footskip=30pt]{geometry}
\usepackage{cmap}
\usepackage{alphabeta}
\usepackage[greek,english]{babel}
\languageattribute{greek}{polutoniko}
\usepackage{hyperref}
\usepackage{comment}
\usepackage{lmodern}

%%% TikZ and QTree

\usepackage{tikz}
\usepackage{tikz-qtree}

%%% Term Language
\newcommand{\subst}[3]{\ensuremath{#1 \{ #2 / #3 \}}}
\newcommand{\case}[3]{%
\toks0={#2#3}%
\edef\param{\the\toks0}%
\ifx\param\empty
  \ensuremath{\text{case} \; #1 \; \{\}}
\else
  \ensuremath{\text{case} \; #1 \; \{ #2 ; #3 \}}
\fi}
\newcommand{\inl}[2]{%
  \ensuremath{#1[\text{inl}].#2}}
\newcommand{\inr}[2]{%
  \ensuremath{#1[\text{inr}].#2}}

%%% Local Variables:
%%% TeX-master: "main"
%%% End:
\title{Races in Classical Linear Logic}
\author{%
  Wen Kokke\\
  Supervised by Philip Wadler and Garrett J.\ Morris\\
  Graduate student (8450698)}
\affil{%
  LFCS, University of Edinburgh\protect\\
  Informatics Forum, 10 Crichton St, Edinburgh EH8 9AB, UK\protect\\
  \email{wen.kokke@ed.ac.uk}
  }
\date{}
%
\begin{document}
\maketitle

\section*{Problem and motivation}
Consider the following scenario:
\begin{quote}
  John and Mary are working from home one morning when they get a craving for a
  slice of cake. Being denizens of the web, they quickly find the nearest store
  which does home deliveries.
  Unfortunately for them, they both order their cake at the \emph{same} store,
  which has only one slice left. After that, all it can deliver is
  disappointment.
\end{quote}
This is an example of a race condition. We can model the scenario in the
\textpi-calculus, assuming \john, \mary\ and \store\ are three processes
modeling John, Mary and the store, and \sliceofcake\ and \nope\ are two channels
giving access to a slice of cake and disappointment, respectively.
As expected, this process has two possible outcomes: either John gets the cake,
and Mary gets disappointment, or vice versa.
\[
  (\piPar{%
    \piSnd{x}{\sliceofcake}{\piSnd{x}{\nope}{\store}}
  }{%
    \piPar{\piRcv{x}{y}{\john}}{\piRcv{x}{z}{\mary}}
  })
  \quad
  \Longrightarrow
  \quad
  (\piPar{\store}{\piPar{\piSub{\sliceofcake}{y}{\john}}{\piSub{\nope}{z}{\mary}}})
  \quad
  \text{or}
  \quad
  (\piPar{\store}{\piPar{\piSub{\nope}{y}{\john}}{\piSub{\sliceofcake}{z}{\mary}}})
\]
While John or Mary may not like all of the outcomes, it is the store which is
responsible for implementing the online delivery service, and the store is happy
with either outcome. Thus, the above is program we would like to be able to
write.
\\[1ex]
\noindent
Now consider another scenario, which takes place \emph{after} John has already
bought the cake:
\begin{quote}
  Mary is \emph{really} disappointed when she finds out the cake has sold out.
  John, always looking to make some money, offers to sell the slice to her for a
  profit. Mary agrees to engage in a little bit of back-alley cake resale, but
  sadly there is no trust between the two.
  John demands payment first.
  Mary would rather get her slice of cake before she gives John the money.
\end{quote}
This is an example of a deadlock. We can also model this scenario in the
\textpi-calculus, assuming that \bill\ is a channel giving access to some
adequate amount of money.
\[
  (\piPar{%
    \piRcv{x}{z}{\piSnd{y}{\sliceofcake}{\john}}
  }{%
    \piRcv{y}{w}{\piSnd{x}{\bill}{\mary}}
  })
\]
The above process does not reduce. As both John and Mary would prefer the
exchange to be made, this program is desired by \emph{neither}. Thus, the above
is a program we would \emph{somehow} like to exclude.
\\[1ex]\noindent
Session types can provide a static guarantee that concurrent programs, such as
those above, respect communication protocols.
Session-typed calculi with logical foundations, such as
\pidill~\cite{caires2010} and CP~\cite{wadler2012}, obtain deadlock freedom as a
result of the close correspondence with logic.
The same correspondence, however, also rules out non-determinism and race conditions.

There is some work on reintroducing non-determinism~\cite{caires2014, atkey2016,
  caires2017}. 
This work all relies on some variant of non-deterministic local choice, an
operator which joins two processes of the same type and non-deterministically
chooses to run \emph{one} of them:
\begin{center}
  \begin{minipage}{0.5\linewidth}
    \begin{prooftree}
      \AXC{$\seq[ P ]{ \Gamma }$}
      \AXC{$\seq[ Q ]{ \Gamma }$}
      \SYM{$+$}
      \BIC{$\seq[ P + Q ]{ \Gamma }$}
    \end{prooftree}  
  \end{minipage}%
  \begin{minipage}{0.5\linewidth}
    \[\!
      \begin{aligned}
        P + Q &\Longrightarrow P\\
        P + Q &\Longrightarrow Q
      \end{aligned}
    \]
  \end{minipage}
\end{center}
While this allows us to describe any non-deterministic program we wish, we have
to do so by enumerating the possible outcomes. For example, we have to write our
first example as the program in which either John or Mary gets the cake:
\[
  (\piPar{\store}{\piPar{\piSub{\sliceofcake}{y}{\john}}{\piSub{\nope}{z}{\mary}}})
  +
  (\piPar{\store}{\piPar{\piSub{\nope}{y}{\john}}{\piSub{\sliceofcake}{z}{\mary}}})
\]
In general, rewriting processes with races in this style results in an
exponential blowup the size of the program.
In this abstract, we propose a novel method of assigning types to racy processes
with channels shared between more than two processes, based on bounded linear
logic~\cite{girard1992} with exact instead of affine bounds. 

\section*{Background and related work}
The \textpi-calculus is a process calculus which models concurrent computation,
which we informally describe below:
\[\!
  \begin{aligned}
    \text{Term} \; P, Q, R
    ::=   &\;\piSnd{x}{y}{P} &&\text{send a value $y$ over $x$, then run $P$}
     &\mid&\;(\piPar{P}{Q})  &&\text{run $P$ and $Q$ in parallel}
    \\\mid&\;\piRcv{x}{y}{P} &&\text{receive a value $y$ over $x$, then run $P$}
     &\mid&\;\piNew{x}{P}    &&\text{create a new channel $x$, then run $P$}
    \\\mid&\;\piEnd          &&\text{halt}
  \end{aligned}
\]
CP~\cite{wadler2012} is a session-typed process calculus which takes the rules
of classical linear logic as prescribing the typing rules of a process
calculus.\footnote{%
  As more readers may be familiar with the \textpi-calculus, we will present CP
  as a type system for \textpi-calculus terms, instead of using the notation
  used by Wadler~\cite{wadler2012}.
}
For instance, the rule for \llTens\ types two parallel, independent processes --
meaning they do not share any communication channels -- and the rule for
\llParr\ types a single process which may communicate with two processes in any
order it pleases: 
\begin{center}
  \begin{prooftree*}
    \AXC{$\seq[ P ]{ \Gamma , \tm[y]{A} }$}
    \AXC{$\seq[ Q ]{ \Delta , \tm[x]{B} }$}
    \SYM{\llTens}
    \BIC{$\seq[{
        \piNew{y}{( \piSnd{x}{y}{( \piPar{P}{Q} ) )}}
      }]{ \Gamma , \Delta , \tm[x]{A \llTens B} }$}
  \end{prooftree*}
  \begin{prooftree*}
    \AXC{$\seq[ P ]{ \Gamma , \tm[y]{A} , \tm[x]{B} }$}
    \SYM{\llParr}
    \UIC{$\seq[ \piRcv{x}{y}{P} ]{ \Gamma , \tm[x]{A \llParr B} }$}
  \end{prooftree*}
\end{center}
The rules for \llWith\ and \llPlus\ type a process offering and making a choice,
respectively. The axiom and cut rules can both be eliminated, and doing so
corresponds to reduction in the \textpi-calculus, with the axiom elimination
corresponding to forwarding~\cite{boreale1998,wadler2012} and cut elimination
corresponding to communication~\cite{wadler2012}.

\section*{Approach and uniqueness}
We extend CP with two new types, inspired by bounded linear
logic~\cite{girard1992}, written \llTake[n]{A} and \llGive[n]{A}. The typing
rules for \llTake[n]{A} construct a pool of a known size, filled with parallel,
independent client-like processes. Logically, \llTake[n]{A} is akin to an
$n$-length vector of the form $A \llTens\dots\llTens A$:
\begin{center}
  \begin{prooftree*}
    \AXC{$\seq[{ P }]{ \Gamma , \tm[y]{A} }$}
    \SYM{\llTake[1]{}}
    \UIC{$\seq[{ \piRcv{x}{y}{P} }]{ \Gamma , \tm[x]{\llTake[1]{A}} }$}
  \end{prooftree*}%
  \begin{prooftree*}
    \AXC{$\seq[{ P }]{ \Gamma , \tm[x]{\llTake[m]{A}} }$}
    \AXC{$\seq[{ Q }]{ \Delta , \tm[x]{\llTake[n]{A}} }$}
    \NOM{Pool}
    \BIC{$\seq[{ (\piPar{P}{Q}) }]{ \Gamma , \Delta , \tm[x]{\llTake[m+n]{A}} }$}
  \end{prooftree*}
\end{center}
The typing rules for \llGive[n]{A} construct a single, server-like process,
which communicates with $n$ processes in any order it pleases. Logically,
\llGive[n]{A} is akin to an $n$-length vector of the form $A \llParr\dots\llParr A$: 
\begin{center}
  \begin{prooftree*}
    \AXC{$\seq[{P}]{ \Gamma , \tm[y]{A} }$}
    \SYM{\llGive[1]{}}
    \UIC{$\seq[{
        \piNew{x}{( \piSnd{x}{y}{P} )}
      }]{ \Gamma , \tm[x]{\llGive[1]{A}} }$}
  \end{prooftree*}%
  \begin{prooftree*}
    \AXC{$\seq[{ P }]{ \Gamma , \tm[x]{\llGive[m]{A}} , \tm[y]{\llGive[n]{A}} }$}
    \NOM{Cont}
    \UIC{$\seq[{ \piSub{x}{y}{P} }]{ \Gamma , \tm[x]{\llGive[m+n]{A}} }$}
  \end{prooftree*}
\end{center}
The novel insight is that because all elements of these vectors have the same
type, we can let any client communicate with any of the server interactions, and
we can let this choice be made non-deterministically while running the process
-- or, similarly, while eliminating the cuts.
On the term level, this means we can now assign types to some processes in which
channels are used by more than two processes concurrently.

\section*{Results and contributions}
The system \cpnd -- CP extended with \llTake[n]{A} and \llGive[n]{A} -- offers
the same guarantees of deadlock-freedom as CP. 
The new system, however, has some limited form of non-determinism.

When compared with CP extended with non-deterministic local choice ($+$), we
find that translations exists in both ways, implying the systems are equally
expressive -- as long as we are allowed to perform whole-program
transformations.
However, as mentioned before, translating from \cpnd\ into CP with `$+$' results
in an exponential blowup in the size of the program.
Furthermore, \cpnd\ allows us to capture our example program with as-is.
This implies that while \cpnd\ and CP with `$+$' are equally expressive in the
strict sense, \cpnd\ is locally more expressive, and thus allows us to capture
programs more naturally and more succinctly.
 
The type system of \cpnd, together with all the aforementioned properties,
has been mechanized using the proof assistant Agda~\cite{norell2009}.
As a consequence of that, there is a verified interpreter for the language.

\clearpage
\bibliographystyle{plain}
\bibliography{main}
\end{document}
%%% Local Variables:
%%% TeX-master: "main"
%%% fill-column: 80
%%% End:
