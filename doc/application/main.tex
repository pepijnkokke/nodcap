\documentclass{scrartcl}

\title{Between Practice and Purity}
\subtitle{Extending GV for practical concurrent programming}
\author{Pepijn Kokke}

\usepackage{natbib,hyperref}
\usepackage{mathrsfs}
\usepackage{textcomp,textgreek}

\begin{document}

\maketitle

\begin{abstract}
  The aim of this research project is to find extensions to the
  functional language GV~\citep{wadler2012}, which facilitate practical
  and (where possible) preexisting paradigms of concurrent
  programming, without breaking the strong guarantees about deadlock
  and race freeness offered by the correspondence with classical
  linear logic.
\end{abstract}

\vspace{5ex}

The Curry-Howard correspondence is an incredibly powerful principle
for programming language design; a language constructed entirely
within the boundaries of such a correspondence can eliminate huge
classes of errors.
One of these classes is non-termination---infinite recursion. This
turns out to be both a blessing and a curse. Infinite recursion is
notoriously hard to debug, and therefore we are more than glad to part
with it, but it takes with it something that has an integral part of
the functional programmer's toolbox: general recursion.

Haskell addresses this problem by simply using general recursion
anyway. This allows for non-termination, making the type system
unsound. In practice, however, Haskell still makes for a safe and
practical programming language, because programs will either be type
correct, or won't terminate.

On the other hand, there are many ways to approach the look and feel
of recursive programming in a way that still guarantees all programs
terminate. Examples of this are primitive recursion, structural
recursion and, more recently, sized types~\citep{lee2001}.

\vspace{5mm}

Today, concurrency is one of the most pressing problems in computing
science. In recent times, the growth in the speed of processor cores
has stagnated, but amount of processor cores in a single processor is
rapidly increasing. This makes concurrency essential. However, there
is very little support in contemporary programming languages that
helps the programmer reliably write correct concurrent code.

Recently, research by~\citet{wadler2012,caires2010,gay2009} gave us a
new correspondence between CP, a process calculus typed by classical
linear logic \citep{girard1987}, and GV, a session-typed $\pi$-calculus
\citep{honda1993,milner1992}.
Where the Curry-Howard correspondence guides us in the design of
programming languages free of type errors and non-termination, this
new correspondence promises to guide us towards concurrent programs
free of deadlocks and race conditions.

However, as with the Curry-Howard correspondence, the freedom from
deadlocks seems to come at a cost.
\citet{lindley2015lightweight} note that the session-typed functional language
GV is quite inflexible in the ways that it can construct concurrent programs.
When faced with this dilemma---a well-behaved or a practical
language?---\citet{lindley2015lightweight} opt to go the way of Haskell.
Their implementation of session-typed Links is based on the canonical
language GV. To remedy its inflexibility, they adopt `access points',
a language construct inspired by service-oriented programming which
they describe as ``a matchmaking service for processes.'' While the
resulting language is certainly much more flexible, and---one
assumes---more intuitive to programmers, it also loses almost all of
the pleasant formal properties that GV enjoys.

\citet{dardha2015} make this notion of inflexibility more precise by
showing that there is a class of deadlock-free programs $\mathscr{K}$,
which contains strictly more programs than typed by GV. These programs
can be assigned a type in other session-typing systems, but, as they
note, GV has one strong advantage over these other systems: its
canonicity, which it gets through its correspondence with linear
logic. Not keen on losing canonicity for expressivity, they define a
procedure to systematically rewrite programs in $\mathscr{K}$ to
programs in GV.

\paragraph{What is my research purpose?}
For this research project, I propose to bridge the two branches of
research formed by \citet{lindley2015lightweight}, and \citet{dardha2015}, and
develop techniques for concurrent programming analogous to size-based
recursion for recursive programming.
This means that I will aim to find find extensions to the functional
language GV (or CP) which:
\begin{enumerate}
\item[a)]%
  support a wide range of concurrent programs, ideally based on
  existing language constructs and best practices in concurrent
  programming; and
\item[b)]%
  can be systematically rewritten into the core language GV, either at
  compile time or in run time, in a way that preserves the formal
  properties of GV.
\end{enumerate}
There are good reasons for wanting to do this. The calculus GV is
a very promising lead in one of the most pressing problems in
computing science, but its current incarnation is too inexpressive for
practical programming.

Whereas using general recursion, and accepting the consequences worked
in the case of Haskell, I would argue that in the case of access points
such a compromise is unacceptable.
Debugging an infinite recursion can be troublesome, but it at least
occurs deterministically for certain inputs.
Due to the non-determinism inherent in access points, programs written
using the construct may contain deadlocks that only occur with a small
probability. This makes debugging them that much more maddening.

As an additional goal, I intend to formalise the GV and CP, their
correspondence, eventual extension, and the rewrite procedure, using a
proof assistant such as Agda~\citep{norell2009}.
This will give me confidence in the correctness of my work, and in
effect will result in a verified compiler, which can be made available
to the community.

\paragraph{What is my methodology?}
I propose to start out by formalising GV and CP, using a proof
assistant such as Agda. This will allow me to get more acquainted with
the calculi and the correspondence.
I have experience in formalising substructural logics and calculi, and
cut-elimination \citep{kokke2015}, and recent work by \citet{allais2015}
has paved the way for formalising linear logic specifically.

Simultaneously, I would also conduct a review of the literature, in
order to identify roughly what is missing from GV, and what kinds of
language constructs---preferably already  used in concurrent
programming---could be added to remedy this.
During this literature review, I would single out a strong candidate
for addition to GV, based on its expressivity and whether or not
it has a basis in the concurrent programming literature.
For the remainder of this section, let us assume that I have chosen
access points.

After having obtained an encoding and session-typing of the new
language construct, and examining its effect on the formal properties
of GV and the correspondence with CP, I would attempt to break it down
to or more primitive operations.
For instance, in \citet{lindley2015lightweight} it is mentioned that access
points alone are sufficiently expressive to derive concurrent state,
non-determination, and recursion. It seems plausible, reading over the
definition of access points, that they could be implemented in terms
of just these three operations.
If this is indeed the case, then the cycle continues: I would examine
the effects of the new primitives on the formal properties of GV, and
attempt to break these down into more primitive operations.

However, if it is not possible to implement access points in terms of
these operations, new questions arise. Is there a restricted form of
access points that can be implemented, and is this still sufficiently
flexible? If there are such restrictions of access points that are
more expressive, but better behaved, I will gladly take them.
Otherwise, I would have to ask what kind of expressivity is missing
that is needed to encode access points---it is the answer to that
question that will lead to progress.

\vspace{5mm}

I thank you in advance for taking the time to consider this proposal.

\bibliographystyle{abbrvnat}%
\bibliography{main}%

\end{document}
