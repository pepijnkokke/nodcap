% color
\usepackage[dvipsnames]{xcolor}
\colorlet{tm}{red}
\colorlet{ty}{blue}

% wider
\newcommand{\wide}[2][4cm]%
{\makebox[\linewidth][c]{\begin{minipage}{\dimexpr\textwidth+#1\relax}#2\end{minipage}}}

% toplabel
\usepackage{ifthen}
\newboolean{toplabel}
\setboolean{toplabel}{false}
\newcommand{\toplabel}[2]{%
  \ifthenelse{\boolean{toplabel}}{%
    \stackrel{\makebox[0pt]{\mbox{\normalfont\tiny #1}}}{#2}
  }{%
    #2
  }}

% emoji
\DeclareGraphicsRule{.ai}{pdf}{.ai}{}
\newcommand{\emoji}[1]{\ensuremath{\vcenter{\hbox{\includegraphics[width=1em]{twemoji/assets/#1.ai}}}}}
\def\mary{\toplabel{Mary}{\emoji{1f469}}}
\def\john{\toplabel{John}{\emoji{1f466}}}
\def\cake{\toplabel{Cake}{\emoji{1f370}}}
\def\plato{\toplabel{Cake}{\emoji{1f382}}}
\def\nocake{\toplabel{No Cake}{\emoji{1f64c}}}
\def\money{\toplabel{Money}{\emoji{1f4b0}}}
\def\store{\toplabel{Store}{\emoji{1f3e0}}}
\def\good{\emoji{2714}}
\def\bad{\emoji{2716}}

% pi-calculus syntax
\newcommand{\PP}{{\color{tm}P}}
\newcommand{\QQ}{{\color{tm}Q}}
\newcommand{\RR}{{\color{tm}R}}
\newcommand{\new}[1]{{\color{tm}\nu #1.}}
\newcommand{\recv}[2]{{\color{tm}#1(#2).}}
\newcommand{\send}[3][]{{%
    \color{tm}
    \toks0={#1}%
    \edef\param{\the\toks0}%
    \edef\bound{b}%
    \edef\unbound{u}%
    \ifx\param\bound%
    \new{#3}\send[u]{#2}{#3}
    \else%
      \ifx\param\unbound%
      #2\langle #3 \rangle.
      \else%
      #2[#3].
      \fi
    \fi
  }}
\newcommand{\halt}[0]{{\color{tm}0}}
\newcommand{\link}[2]{{\color{tm}#1{\leftrightarrow}#2}}
\newcommand{\expn}[2]{{\color{tm}#1 \uparrow #2.}}
\newcommand{\intl}[2]{{\color{tm}#1 \downarrow #2.}}
\newcommand{\conc}[0]{\mathbin{\color{tm}\mid}}
\newcommand{\paren}[1]{{\color{tm}(}#1{\color{tm})}}
\newcommand{\subst}[3]{{\color{tm}#1\{#2/#3\}}}
\newcommand{\tm}[2][]{%
\toks0={#1}%
\edef\param{\the\toks0}%
\ifx\param\empty
  \ensuremath{{\color{ty}#2}}%
\else
  \ensuremath{{\color{tm}#1}\color{black}:{\color{ty}#2}}%
\fi}
\newcommand{\seq}[2][]{\ensuremath{{\color{tm}#1}\vdash{\color{ty}#2}}}

% ambiguous reduction
\usepackage{tikz}
\usepackage{tikz-qtree}
\newcommand{\amb}[3]{%
  \begin{tikzpicture}[sibling distance=1cm,level distance=3cm]
    \Tree
    [.{$#1$}
    \edge[->]; {$#2$}
    \edge[->]; {$#3$}
    ]
  \end{tikzpicture}
}

% type syntax
\def\parr{\ensuremath{\mathbin{\rotatebox[origin=c]{180}{\&}}}}
\def\with{\ensuremath{\mathbin{\text{\&}}}}
\def\plus{\ensuremath{\oplus}}
\def\tens{\ensuremath{\otimes}}
\def\limp{\ensuremath{\multimap}}
\def\one{\ensuremath{\mathbf{1}}}
\def\nil{\ensuremath{\mathbf{0}}}
\def\lequiv{\ensuremath{\multimapboth}}
\newcommand{\give}[2][]{\ensuremath{{ ? }_{#1}{#2}}}
\newcommand{\take}[2][]{\ensuremath{{ ! }_{#1}{#2}}}

% proof syntax
\usepackage{bussproofs}
\def\fCenter{\ensuremath{\;\vdash\;}}
\newcommand{\NOM}[1]{\RightLabel{\textsc{#1}}}
\newcommand{\SYM}[1]{\RightLabel{\ensuremath{#1}}}
\EnableBpAbbreviations
\newenvironment{proofbox}[1][0.9]%
  {\gdef\scalefactor{#1} \leavevmode\hbox\bgroup}
  {\scalebox{\scalefactor}{\DisplayProof} \egroup}
\newenvironment{proofblock}[1][0.9]%
  {\gdef\scalefactor{#1}\center\proofSkipAmount \leavevmode}%
  {\scalebox{\scalefactor}{\DisplayProof}\proofSkipAmount \endcenter }


%%% Local Variables:
%%% TeX-master: "main"
%%% End:
