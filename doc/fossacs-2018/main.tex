\documentclass[UKenglish]{llncs}
\usepackage{makeidx}
\usepackage{cmap}
\usepackage{alphabeta}
\usepackage[greek,english]{babel}
\languageattribute{greek}{polutoniko}
\usepackage{hyperref}
\usepackage{comment}
\usepackage{lmodern}

%%% TikZ and QTree

\usepackage{tikz}
\usepackage{tikz-qtree}

%%% Term Language
\newcommand{\subst}[3]{\ensuremath{#1 \{ #2 / #3 \}}}
\newcommand{\case}[3]{%
\toks0={#2#3}%
\edef\param{\the\toks0}%
\ifx\param\empty
  \ensuremath{\text{case} \; #1 \; \{\}}
\else
  \ensuremath{\text{case} \; #1 \; \{ #2 ; #3 \}}
\fi}
\newcommand{\inl}[2]{%
  \ensuremath{#1[\text{inl}].#2}}
\newcommand{\inr}[2]{%
  \ensuremath{#1[\text{inr}].#2}}

%%% Local Variables:
%%% TeX-master: "main"
%%% End:
\usepackage{graphicx}
\usepackage{tikz}

%%% Emoji:
\DeclareGraphicsRule{.ai}{pdf}{.ai}{}
\newcommand{\emoji}[2][1em]{\ensuremath{\vcenter{\hbox{\includegraphics[width=#1]{twemoji/assets/#2.ai}}}}}
\newcommand{\mary}[1][1em]{\emoji[#1]{1f469}}
\newcommand{\john}[1][1em]{\emoji[#1]{1f466}}
\newcommand{\sliceofcake}[1][1em]{\emoji[#1]{1f370}}
\newcommand{\cake}[1][1em]{\emoji[#1]{1f382}}
\newcommand{\nope}[1][1em]{\emoji[#1]{1f64c}}
\newcommand{\money}[1][1em]{\emoji[#1]{1f4b0}}
\newcommand{\bill}[1][1em]{\emoji[#1]{1f4b5}}
\newcommand{\store}[1][1em]{\emoji[#1]{1f3e0}}
%%% Local Variables:
%%% TeX-master: "main"
%%% End:

%%% The π-calculus term syntax:
\newcommand{\piCalc}[0]{\textpi-calculus\xspace}
\newcommand{\piSend}[3]{\ensuremath{\overline{#1}\langle #2 \rangle.#3}}
\newcommand{\piRecv}[3]{\ensuremath{#1(#2).#3}}
\newcommand{\piPar}[2]{\ensuremath{#1 \mid #2}}
\newcommand{\piNew}[2]{\ensuremath{(\nu #1)#2}}
\newcommand{\piRepl}[1]{\ensuremath{!#1}}
\newcommand{\piHalt}[0]{\ensuremath{0}}
\newcommand{\piSub}[3]{\ensuremath{#3\{#1/#2\}}}

%%% The CP term syntax:
\newcommand{\cp}{CP\xspace}
\newcommand{\cpLink}[2]{\ensuremath{#1{\leftrightarrow}#2}}
\newcommand{\cpCut}[3]{\ensuremath{\nu #1.(\piPar{#2}{#3})}}
\newcommand{\cpSend}[4]{\ensuremath{#1[#2].(\piPar{#3}{#4})}}
\newcommand{\cpRecv}[3]{\ensuremath{\piRecv{#1}{#2}{#3}}}
\newcommand{\cpWait}[2]{\ensuremath{#1().#2}}
\newcommand{\cpHalt}[1]{\ensuremath{#1[].0}}
\newcommand{\cpInl}[2]{\ensuremath{#1[\text{inl}].#2}}
\newcommand{\cpInr}[2]{\ensuremath{#1[\text{inr}].#2}}
\newcommand{\cpCase}[3]{\ensuremath{\text{case}\;#1\;\{#2;#3\}}}
\newcommand{\cpAbsurd}[1]{\ensuremath{\text{case}\;#1\;\{\}}}
\newcommand{\cpSub}[3]{\ensuremath{\piSub{#1}{#2}{#3}}}

%%% The CP reduction rules:
\newcommand{\cpEquivCutComm}{\ensuremath{(\nu\text{-comm})}\xspace}
\newcommand{\cpEquivCutAssNoParen}[1]{\ensuremath{\nu\text{-assoc}_{#1}}\xspace}
\newcommand{\cpEquivCutAss}[1]{\ensuremath{(\cpEquivCutAssNoParen{#1})}\xspace}
\newcommand{\cpRedAxCut}[1]{\ensuremath{(\cpLink{}{}_{#1})}\xspace}
\newcommand{\cpRedBetaTensParr}{\ensuremath{(\beta{\tens}{\parr})}\xspace}
\newcommand{\cpRedBetaOneBot}{\ensuremath{(\beta{\one}{\bot})}\xspace}
\newcommand{\cpRedBetaPlusWith}[1]{\ensuremath{(\beta{\plus}{\with}_{#1})}\xspace}
\newcommand{\cpRedKappaTens}[1]{\ensuremath{(\kappa{\tens}_{#1})}\xspace}
\newcommand{\cpRedKappaParr}{\ensuremath{(\kappa{\parr})}\xspace}
\newcommand{\cpRedKappaBot}{\ensuremath{(\kappa{\bot})}\xspace}
\newcommand{\cpRedKappaPlus}[1]{\ensuremath{(\kappa{\plus}_{#1)}}\xspace}
\newcommand{\cpRedKappaWith}{\ensuremath{(\kappa{\with})}\xspace}
\newcommand{\cpRedKappaTop}{\ensuremath{(\kappa{\top})}\xspace}
\newcommand{\cpRedGammaCut}{\ensuremath{(\gamma\nu)}\xspace}
\newcommand{\cpRedGammaEquiv}{\ensuremath{(\gamma{\equiv})}\xspace}

%%% The nodcap term syntax:
\newcommand{\nc}{\ensuremath{\text{CP}_{\text{ND}}}\xspace}
\newcommand{\ncHalt}{\ensuremath{0}}
\newcommand{\ncExpn}[3]{\ensuremath{#1\uparrow #2.#3}}
\newcommand{\ncIntl}[3]{\ensuremath{#1\downarrow #2.#3}}
\newcommand{\ncSrv}[3]{\ensuremath{{\star}{#1}(#2).#3}}
\newcommand{\ncCnt}[3]{\ensuremath{{\star}{#1}[#2].#3}}
\newcommand{\ncPool}[2]{\ensuremath{(\piPar{#1}{#2})}}
\newcommand{\ncCont}[3]{\ensuremath{\cpSub{#1}{#2}{#3}}}

%%% The nodcap reduction rules:
\newcommand{\ncEquivPoolId}{\ensuremath{(|\text{-ident})}\xspace}
\newcommand{\ncEquivPoolComm}{\ensuremath{(|\text{-comm})}\xspace}
\newcommand{\ncEquivPoolAssNoParen}[1]{\ensuremath{|\text{-assoc}_#1}\xspace}
\newcommand{\ncEquivPoolAss}[1]{\ensuremath{(\ncEquivPoolAssNoParen{#1})}\xspace}
\newcommand{\ncEquivCutCong}{\ensuremath{(\nu\text{-cong})}\xspace}
\newcommand{\ncRedBetaStar}[1]{\ensuremath{(\beta{\star}_{#1})}\xspace}
\newcommand{\ncRedBetaTakeGive}[1]{\ensuremath{(\beta\take{}\give{})}\xspace}
\newcommand{\ncRedKappaTake}{\ensuremath{(\kappa{\take[1]{}})}\xspace}
\newcommand{\ncRedKappaGive}{\ensuremath{(\kappa{\give[1]{}})}\xspace}
\newcommand{\ncRedKappaPool}{\ensuremath{(\kappa{|})}\xspace}
\newcommand{\ncRedGammaPool}{\ensuremath{(\gamma{|})}\xspace}

%%% Local Variables:
%%% TeX-master: "main"
%%% End:

%%% Type Syntax:
\newcommand*{\parr}{\ensuremath{\mathbin{\rotatebox[origin=c]{180}{\&}}}}
\newcommand*{\with}{\ensuremath{\mathbin{\text{\&}}}}
\newcommand*{\plus}{\ensuremath{\oplus}}
\newcommand*{\tens}{\ensuremath{\otimes}}
\newcommand*{\one}{\ensuremath{\mathbf{1}}}
\newcommand*{\nil}{\ensuremath{\mathbf{0}}}
\newcommand*{\give}[2][]{\ensuremath{{ ? }_{#1}{#2}}}
\newcommand*{\take}[2][]{\ensuremath{{ ! }_{#1}{#2}}}
\newcommand*{\limp}{\ensuremath{\multimap}}

%%% Local Variables:
%%% TeX-master: "main"
%%% End:

%% Theorem/Proof Style:
\usepackage{amsthm}
\usepackage{centernot}
\usepackage{multicol}
\usepackage{thmtools}
\usepackage{nameref,hyperref,cleveref}
\declaretheoremstyle[headfont=\normalfont]{normalhead}
\declaretheorem[name=Theorem,
  refname={theorem,theorems},
  Refname={Theorem,Theorems},
  parent=chapter,]{theorem}
\declaretheorem[name=Lemma,
  refname={lemma,lemmas},
  Refname={Lemma,Lemmas},
  sibling=theorem,]{lemma}
\declaretheorem[name=Definition,
  refname={definition,definitions},
  Refname={Definition,Definitions},
  style=normalhead,
  sibling=theorem,]{definition}
\declaretheorem[name=Corollary,
  refname={corollary,corollaries},
  Refname={corollary,Corollaries},
  sibling=theorem,]{corollary}
  
%% Proof Tree Syntax:
\usepackage{bussproofs}
\EnableBpAbbreviations

\newcommand{\tm}[1]{\textcolor{red}{\ensuremath{#1}}}
\newcommand{\ty}[1]{\textcolor{blue}{\ensuremath{#1}}}
\newcommand{\seq}[2][]{\ensuremath{\tm{#1}\vdash\ty{#2}}}
\newcommand{\actson}[2]{\ensuremath{\tm{#1} \; \text{acts on} \; \tm{#2}}}
\newcommand{\canonical}[1]{\ensuremath{\text{canonical} \; \tm{#1}}}
\newcommand{\canonicalpool}[1]{\ensuremath{\text{canonical}^\prime \; \tm{#1}}}
\newcommand{\reducesto}[2]{\ensuremath{\tm{#1}\Longrightarrow\tm{#2}}}
\newcommand{\tmty}[2]{\ensuremath{\tm{#1}\colon\!\ty{#2}}}
\newcommand{\NOM}[1]{\RightLabel{\textsc{#1}}}
\newcommand{\SYM}[1]{\RightLabel{\ensuremath{#1}}}
\newenvironment{prooftree*}{\leavevmode\hbox\bgroup}{\DisplayProof\egroup}
\newenvironment{scprooftree}[1][1]%
  {\gdef\scalefactor{#1}\begin{center}\proofSkipAmount \leavevmode}%
  {\scalebox{\scalefactor}{\DisplayProof}\proofSkipAmount \end{center} }
\newenvironment{scprooftree*}[1][1]%
  {\gdef\scalefactor{#1} \leavevmode\hbox\bgroup}

%%% Local Variables:
%%% TeX-master: "main"
%%% End:

\addbibresource{main.bib}
\arraycolsep=1ex
\mainmatter % start of the contributions
\title{Races in Classical Linear Logic}
\titlerunning{Races in Classical Linear Logic} 
\author{Wen Kokke \and J.\ Garrett Morris \and Philip Wadler}
\tocauthor{Wen Kokke, J.\ Garrett Morris, and Philip Wadler}
\institute{University of Edinburgh, Edinburgh, UK,\\
\email{wen.kokke@ed.ac.uk}}
\begin{document}

\maketitle % typeset the title of the contribution

\begin{abstract}
  Process calculi based in logic, such as \piDILL and CP, provide a foundation
  for deadlock-free concurrent programming, but at the cost of excluding
  non-determinism and races.
  We introduce \nodcap (nodcap), which extends CP with a novel account of
  non-determinism.
  Our approach draws on bounded linear logic to provide a strongly-typed
  account of standard process calculus expressions of non-determinism.
  We show that our extension is expressive enough to capture many uses of
  non-determinism in untyped calculi, such as non-deterministic choice, while
  preserving CP's meta-theoretic properties, including deadlock freedom.  
\end{abstract}

%% Introduction
\section{Introduction}\label{sec:introduction}

Consider the following scenario:
\begin{quote}
  John and Mary are working from home one morning when they get a craving for a
  slice of cake. Being denizens of the web, they quickly find the nearest store
  which does home deliveries.
  Unfortunately for them, they both order their cake at the \emph{same} store,
  which has only one slice left. After that, all it can deliver is
  disappointment.
\end{quote}
This is an example of a \emph{race condition}. We can model this scenario in the
\textpi-calculus, assuming \john, \mary and \store are three processes
modeling John, Mary and the store, and \sliceofcake and \nope are two channels
giving access to a slice of cake and disappointment, respectively.
As expected, this process has two possible outcomes: either John gets the cake,
and Mary gets disappointment, or vice versa.
\[
  \begin{array}{c}
    \tm{(\piPar{%
    \piSend{x}{\sliceofcake}{\piSend{x}{\nope}{\store}}
    }{%
    \piPar{\piRecv{x}{y}{\john}}{\piRecv{x}{z}{\mary}}
    })}
    \\[1ex]
    \rotatebox[origin=c]{270}{$\Longrightarrow^{\star}$}
    \\[1ex]
    \tm{(\piPar{\store}{\piPar{\piSub{\sliceofcake}{y}{\john}}{%
    \piSub{\nope}{z}{\mary}}})}
    \quad
    \text{or}
    \quad
    \tm{(\piPar{\store}{\piPar{\piSub{\nope}{y}{\john}}{%
    \piSub{\sliceofcake}{z}{\mary}}})}
  \end{array}
\]
While John or Mary may not like all of the outcomes, it is the store which is
responsible for implementing the online delivery service, and the store is happy
with either outcome. Thus, the above is an interaction we would like to be able to
model.

Now consider another scenario, which takes place \emph{after} John has already
bought the cake:
\begin{quote}
  Mary is \emph{really} disappointed when she finds out the cake has sold out.
  John, always looking to make some money, offers to sell the slice to her for a
  profit. Mary agrees to engage in a little bit of back-alley cake resale, but
  sadly there is no trust between the two.
  John demands payment first.
  Mary would rather get her slice of cake before she gives John the money.
\end{quote}
This is an example of a \emph{deadlock}. We can also model this scenario in the
\textpi-calculus, assuming that \bill\ is a channel giving access to some
adequate amount of money.
\[
  \begin{array}{c}
    \tm{(\piPar{%
    \piRecv{x}{z}{\piSend{y}{\sliceofcake}{\john}}
    }{%
    \piRecv{y}{w}{\piSend{x}{\bill}{\mary}}
    })}
    \quad
    \centernot\Longrightarrow^{\star}
  \end{array}  
\]
The above process does not reduce. As both John and Mary would prefer the
exchange to be made, this interaction is desired by \emph{neither}. Thus, the
above is an interaction we would \emph{somehow} like to exclude.

Session types~\cite{honda1993} statically guarantee that concurrent
programs, such as those above, respect communication protocols.
Session-typed calculi with logical foundations, such as
\piDILL~\cite{caires2010} and CP~\cite{wadler2012}, obtain deadlock freedom as a
result of a close correspondence with logic.
The same correspondence, however, also rules out non-determinism and race
conditions.

In this paper, we present \nodcap (nodcap), an extension of
CP~\cite{wadler2012} with a novel account of non-determinism and races.
Inspired by bounded linear logic~\cite{girard1992}, we introduce a form of
shared channels, in which the type of a shared channel tracks how many times it
is reused.
As in the untyped $\pi$-calculus, sharing introduces the potential for
non-determinism.
We show that our approach is sufficient to capture practical examples of races,
such as the web store, as well as other formal characterizations of
non-determinism, such as non-deterministic choice.  However, \nodcap does not
lose the metatheoretical benefits of CP: we show that it enjoys termination and
deadlock-freedom.

This paper proceeds as follows. In \cref{sec:cp-revisited}, we introduce a variant of \cp and prove progress and preservation. In \cref{sec:cpnd}, we introduce \nodcap. Finally, in \cref{sec:discussion}, we conclude with a discussion of the work
done in this paper and potential avenues for future work.

\section{Classical Processes Revisited}\label{sec:cp-revisited}

In this section, we introduce of \cp~\cite{wadler2012,lindley2015semantics}.
To keep the discussion simply, we will only discuss the subset which corresponds
to multiplicative-additive linear logic~\cite[MALL]{girard1987} or rudimentary
linear logic~\cite[RLL]{girard1992}.
We foresee no problems in extending the proofs from \cref{sec:cpnd} to
cover the remaining features of \cp, polymophism and the exponentials $\ty{!A}$
and $\ty{?A}$. 

We deviate from the original presentation of \cp~\cite{wadler2012} in two ways:
we include the commutativity of links in our structural congruence, as done by
\textbf{paykin2016}; and we exclude the commutative conversions from our
reduction system, inspired by \citenat{lindley2015semantics}, which leads to a
different notion of canonical form and a reduction strategy closer to that of
the \textpi-calculus.

\subsection{Terms and types}
\label{sec:cp-terms-and-types}
The term language for \cp is a variant of the
\textpi-calculus~\cite{milner1992b}.
Its terms are defined by the following grammar:
\begin{definition}[Terms]\label{def:cp-terms}
  \[\!
    \begin{aligned}
      \tm{P}, \tm{Q}, \tm{R}
           :=& \; \tm{\cpLink{x}{y}}       &&\text{link}
      \\ \mid& \; \tm{\cpCut{x}{P}{Q}}     &&\text{parallel composition, or ``cut'''}
      \\ \mid& \; \tm{\cpSend{x}{y}{P}{Q}} &&\text{``output''}
      \\ \mid& \; \tm{\cpRecv{x}{y}{P}}    &&\text{``input''}
      \\ \mid& \; \tm{\cpHalt{x}}          &&\text{halt}
      \\ \mid& \; \tm{\cpWait{x}{P}}       &&\text{wait}
      \\ \mid& \; \tm{\cpInl{x}{P}}        &&\text{select left choice}
      \\ \mid& \; \tm{\cpInr{x}{P}}        &&\text{select right choice}
      \\ \mid& \; \tm{\cpCase{x}{P}{Q}}    &&\text{offer binary choice}
      \\ \mid& \; \tm{\cpAbsurd{x}}        &&\text{offer nullary choice}
    \end{aligned}
  \]  
  \qed
\end{definition}
The variables \tm{x}, \tm{y}, \tm{z} and \tm{w} range over channel names.
The construct \tm{\cpLink{x}{y}} links two
channels~\cite{sangiorgi1996,boreale1998}, forwarding messages received on
\tm{x} to \tm{y} and vice versa.
The construct \tm{\cpCut{x}{P}{Q}} creates a new channel \tm{x}, and composes
two processes, \tm{P} and \tm{Q}, which communicate on \tm{x}, in parallel.
Therefore, in \tm{\cpCut{x}{P}{Q}} the name \tm{x} is bound in both \tm{P} and
\tm{Q}.
In \tm{\cpRecv{x}{y}{P}} and \tm{\cpSend{x}{y}{P}{Q}}, round brackets denote
input, square brackets denote output. 
We use bound output~\cite{sangiorgi1996}, meaning that both input and output
bind a new name. 
In \tm{\cpRecv{x}{y}{P}} the new name \tm{y} is bound in \tm{P}.
In \tm{\cpSend{x}{y}{P}{Q}}, the new name \tm{y} is only bound in \tm{P}, while
\tm{x} is only bound in \tm{Q}.

Terms in \cp are identified up to structural congruence, which states that
parallel compositions \tm{\cpCut{x}{P}{Q}} are associative and commutative.
It is defined as follows:
\begin{definition}[Structural congruence]\label{def:cp-equiv}
  We define the structural congruence $\equiv$ as a reflexive, transitive
  congruence over terms which satisfies the following additional axioms:
  \[
    \begin{array}{llll}
      \cpEquivLinkComm
      & \tm{\cpLink{x}{y}}
      & \equiv \;
      & \tm{\cpLink{y}{x}}
      \\
      \cpEquivCutComm
      & \tm{\cpCut{x}{P}{Q}}
      & \equiv \;
      & \tm{\cpCut{x}{Q}{P}}
      \\
      \cpEquivCutAss1
      & \tm{\cpCut{x}{P}{\cpCut{y}{Q}{R}}}
      & \equiv \;
      & \tm{\cpCut{y}{\cpCut{x}{P}{Q}}{R}}
        \quad \text{if} \; \notFreeIn{x}{R} \; \text{and} \; \notFreeIn{y}{P}
    \end{array}
  \]
  \qed
\end{definition}
We do not add an axiom for \cpEquivCutAss2, as it follows from
\cref{def:cp-equiv}.
Throughout this dissertation, we will leave uses of the transitivity and
congruence rules implicit.
Structural congruence is a symmetric relation.
%
Channels in \cp are typed using a session type system which corresponds to RLL,
the multiplicative, additive fragment of linear logic.
The types are defined by the following grammar:
\begin{definition}[Types]\label{def:cp-types}
  \[\!
    \begin{aligned}
      \ty{A}, \ty{B}, \ty{C}
           :=& \; \ty{A \tens B} &&\text{pair of independent channels}
      \\ \mid& \; \ty{A \parr B} &&\text{pair of interdependent channels}
      \\ \mid& \; \ty{\one}      &&\text{unit for} \; {\tens}
      \\ \mid& \; \ty{\bot}      &&\text{unit for} \; {\parr}
      \\ \mid& \; \ty{A \plus B} &&\text{internal choice}
      \\ \mid& \; \ty{A \with B} &&\text{external choice}
      \\ \mid& \; \ty{\nil}      &&\text{unit for} \; {\plus}
      \\ \mid& \; \ty{\top}      &&\text{unit for} \; {\with}
    \end{aligned}
  \]  
  \qed
\end{definition}
Duality plays a crucial role in both linear logic and session types.
In \cp, the two endpoints of a channel are assigned dual types.
This ensures that, for instance, whenever a process \emph{sends} across a
channel, the process on the other end of that channel is waiting to
\emph{receive}.
Each type \ty{A} has a dual, written \ty{A^\bot}.
Duality is an involutive function.
\begin{definition}[Duality]\label{def:cp-negation}
  \[\!
    \begin{array}{lclclcl}
              \ty{(A \tens B)^\bot} &=& \ty{A^\bot \parr B^\bot}
      &\quad& \ty{\one^\bot}        &=& \ty{\bot}
      \\      \ty{(A \parr B)^\bot} &=& \ty{A^\bot \tens B^\bot}
      &\quad& \ty{\bot^\bot}        &=& \ty{\one}
      \\      \ty{(A \plus B)^\bot} &=& \ty{A^\bot \with B^\bot}
      &\quad& \ty{\nil^\bot}        &=& \ty{\top}
      \\      \ty{(A \with B)^\bot} &=& \ty{A^\bot \plus B^\bot}
      &\quad& \ty{\top^\bot}        &=& \ty{\nil}
    \end{array}
  \]
  \qed
\end{definition}
Environments associate channels with types.
\begin{definition}[Environments]\label{def:cp-environments}
  We define environments as follows:
  \[
    \ty{\Gamma}, \ty{\Delta}, \ty{\Theta}
    ::= \tmty{x_1}{A_1}\dots\tmty{x_n}{A_n}
  \] 
  Names in environments must be unique, and environments \ty{\Gamma} and
  \ty{\Delta} can only be combined as $\ty{\Gamma}, \ty{\Delta}$ if
  $\text{fv}(\ty{\Gamma}) \cap \text{fv}(\ty{\Delta}) = \varnothing$. 
  \qed
\end{definition}
Typing judgements associate processes with collections of typed channels.
\begin{definition}[Typing judgements]\label{def:cp-typing-judgement}
  A typing judgement $\seq[{ P }]{\tmty{x_1}{A_1}\dots\tmty{x_n}{A_n}}$ denotes
  that the process \tm{P} communicates along channels $\tm{x_1}\dots\tm{x_n}$
  following protocols $\ty{A_1}\dots\ty{A_n}$.
  Typing judgements can be constructed using the following rules:
  {\normalfont
    \begin{center} \cpInfAx    \cpInfCut   \end{center}
    \begin{center} \cpInfTens  \cpInfParr  \end{center}
    \begin{center} \cpInfOne   \cpInfBot   \end{center}
    \begin{center} \cpInfPlus1 \cpInfPlus2 \end{center}
    \begin{center} \cpInfWith              \end{center}
    \begin{center} \cpInfNil   \cpInfTop   \end{center}
  }
  \qed
\end{definition}
Reductions relate processes with their reduced forms.
They are defined as follows:
\begin{definition}[Term reduction]\label{def:cp-term-reduction-1}
  A reduction $\reducesto{P}{P'}$ denotes that the process \tm{P} can reduce to
  the process \tm{P'} in a single step. Reductions can only be constructed using
  the following rules:
  \[
    \begin{array}{llll}
      \cpRedAxCut1
      & \tm{\cpCut{x}{\cpLink{w}{x}}{P}}
      & \Longrightarrow \;
      & \tm{\cpSub{w}{x}{P}} 
      \\
      \cpRedBetaTensParr
      & \tm{\cpCut{x}{\cpSend{x}{y}{P}{Q}}{\cpRecv{x}{z}{R}}}
      & \Longrightarrow \;
      & \tm{\cpCut{y}{P}{\cpCut{x}{Q}{\cpSub{y}{z}{R}}}}
      \\
      \cpRedBetaOneBot
      & \tm{\cpCut{x}{\cpHalt{x}}{\cpWait{x}{P}}}
      & \Longrightarrow \;
      & \tm{P}
      \\
      \cpRedBetaPlusWith1
      & \tm{\cpCut{x}{\cpInl{x}{P}}{\cpCase{x}{Q}{R}}}
      & \Longrightarrow \;
      & \tm{\cpCut{x}{P}{Q}}
      \\
      \cpRedBetaPlusWith2
      & \tm{\cpCut{x}{\cpInr{x}{P}}{\cpCase{x}{Q}{R}}}
      & \Longrightarrow \;
      & \tm{\cpCut{x}{P}{R}}
    \end{array}
  \]
  \begin{prooftree}
    \AXC{\reducesto{P}{P^\prime}}
    \SYM{\cpRedGammaCut}
    \UIC{\reducesto{\cpCut{x}{P}{Q}}{\cpCut{x}{P^\prime}{Q}}}
  \end{prooftree}
  \begin{prooftree}
    \AXC{$\tm{P}\equiv\tm{Q}$}
    \AXC{\reducesto{Q}{Q^\prime}}
    \AXC{$\tm{Q^\prime}\equiv\tm{P^\prime}$}
    \SYM{\cpRedGammaEquiv}
    \TIC{\reducesto{P}{P^\prime}}
  \end{prooftree}
  %
  The relation $\Longrightarrow^\star$ is the reflexive, transitive closure of
  $\Longrightarrow$.
  \qed
\end{definition}
We will discuss the interpretations of each connective, together with their
typing and reduction rules, in \cref{sec:cp-multiplicatives,sec:cp-additives,sec:cp-duality}.

\subsection{Multiplicatives and in- and interdependence}
\label{sec:cp-multiplicatives}
The multiplicatives ($\ty{\tens}, \ty{\parr}$) deal with independence and
interdependence:
\begin{itemize}
\item
  A channel of type \ty{A \tens B} represents a pair of channels, which
  communicate with two independent processes---that is to say, two
  processes who share no channels.
  A process acting on a channel of type \ty{A \tens B} will send one endpoint of
  a fresh channel, and then split into a pair of independent processes.
  One of these processes will be responsible for an interaction of type \ty{A}
  over the fresh channel, while the other process continues to interact as
  \ty{B}.
\item
  A channel of type \ty{A \parr B} represents a pair of interdependent channels,
  which are used within a single process.
  A process acting on a channel of type \ty{A \parr B} will receive a channel to
  act on, and communicate on its channels in whatever order it pleases.
  This means that the usage of one channel can depend on that of
  another---e.g.\ the interaction of type \ty{B} could depend on the result of
  the interaction of type \ty{A}, or vise versa, and if \ty{A} and \ty{B} are
  complex types, their interactions could likewise interweave in complex ways.
\end{itemize}
While the rules for \ty{\tens} and \ty{\parr} introduce input and output
operations, these are inessential---the essential distinction lies two in the
fact that (\tens) composes two independent processes, and therefore \emph{must}
split the environment between them, whereas (\parr) uses a single process, which
then can---and must---use all the channels in the environment.
\begin{center}
  \cpInfTens
  \cpInfParr
\end{center}
The \textbeta-reduction rule for terms introduced by $(\tens)$ and $(\parr)$
implements the behaviour outlined above:
\[
  \tm{\cpCut{x}{\cpSend{x}{y}{P}{Q}}{\cpRecv{x}{z}{R}}}
  \Longrightarrow
  \tm{\cpCut{y}{P}{\cpCut{x}{Q}{\cpSub{y}{z}{R}}}}
\]
The function \tm{\cpSub{x}{y}{P}} denotes the substitution of the name \tm{x}
for the name \tm{y} in the term \tm{P}.
%
The rules for the multiplicative units ($\ty{\one}, \ty{\bot}$) follow the same
pattern, except for the nullary instead of the binary case:
\begin{itemize}
\item
  A term constructed by $(\one)$ must composes \emph{zero} independent
  processes, and thus must halt. Furthermore, it must be able to split its
  environment between zero processes, and thus its environment must be empty.
\item
  A term constructed by $(\bot)$, on the other hand, consists of a single
  process, which is not further restricted.
\end{itemize}
The rules for $\ty{\one}$ and $\ty{\bot}$ introduce a nullary send and receive
operation, such as those found in the polyadic \textpi-calculus~\cite{milner1993}. 
\begin{center}
  \cpInfOne
  \cpInfBot
\end{center}
The \textbeta-reduction rule for terms introduced by $(\one)$ and $(\bot)$
implements the behaviour outlined above:
\[
  \tm{\cpCut{x}{\cpHalt{x}}{\cpWait{x}{P}}}
  \Longrightarrow
  \tm{P}
\]

\subsection{Additives and choice}
\label{sec:cp-additives}
The additives ($\ty{\plus}, \ty{\with}$) deal with choice:
\begin{itemize}
\item
  A process acting on a channel of type \ty{A \plus B} either sends the value
  \tm{\text{inl}} to select an interaction of type \ty{A} or the value
  \tm{\text{inr}} to select one of type \ty{B}. 
\item
  A process acting on a channel of type \ty{A \with B} receives such a value,
  and then offers an interaction of either type \ty{A} or \ty{B},
  correspondingly.
\end{itemize}
In essence, the additive operations implement sending and receiving of a single
bit of information (\tm{\text{inl}} or \tm{\text{inr}}) and branching based on
the value of that bit. 
\begin{center}
  \cpInfPlus1
  \cpInfWith
\end{center}
The rule for constructing a process which sends \tm{\text{inr}}, $(\plus_2)$,
has been omitted, but can be found in~\cref{def:cp-typing-judgement}.
The \textbeta-reduction rules for terms introduced by $(\plus_1)$, $(\plus_2)$
and $(\with)$ implement the behaviour outlined above.
\[
  \begin{array}{c}
    \tm{\cpCut{x}{\cpInl{x}{P}}{\cpCase{x}{Q}{R}}} \Longrightarrow \tm{\cpCut{x}{P}{Q}}
    \\
    \tm{\cpCut{x}{\cpInr{x}{P}}{\cpCase{x}{Q}{R}}} \Longrightarrow \tm{\cpCut{x}{P}{R}}
  \end{array}
\]
%
The rules for the additive units ($\ty{\nil}, \ty{\top}$) follow the same
pattern, except for a nullary choice:
\begin{itemize}
\item
  There is \emph{no} rule for \ty{\nil}, as a process acting on a channel of
  that type would have to select one of \emph{zero} options, which is clearly
  impossible.
\item
  A process acting on a channel of type \ty{\top} will wait to receive a choice
  of out \emph{zero} options. Since this will clearly never arrive, we have two
  options: either we block, waiting forever, or we simply crash.
\end{itemize}
It may seem odd at first to include a type for the process which cannot possibly
exist, and for the process which waits forever, but these make sensible units
for choice.
When offered a choice of type \ty{A \plus \nil}, one can either choose to
interact as \ty{A}, or choose to commit to doing the impossible.
Similarly, when offering a choice of type \ty{A \with \top}, one can safely
implement the right branch with a process which waits forever, as no sound
process will ever be able to select that branch anyway.
\begin{center}
  \cpInfNil
  \cpInfTop
\end{center}
As there is no way to construct a process of type \ty{\nil}, there is no
reduction rule for the additive units.


\subsection{Structural rules and duality}
\label{sec:cp-duality}
Duality plays a crucial role in session type systems.
In~\cref{sec:cp-additives}, we saw that duality ensures a process offering a choice
is always matched with a process making a choice.
In~\cref{sec:cp-multiplicatives}, we saw that it also ensures that, for instance
a process which uses communication on \tm{x} to decide what to send on \tm{y} is
matched with a pair of independent processes on \tm{x} and \tm{y}, a property
which is crucial to deadlock freedom, as it prevents circular dependencies. 

Duality appears in the typing rules for two \cp term constructs.
Forwarding, \tm{\cpLink{x}{y}}, connects two dual channels with dual endpoints,
while composition, \tm{\cpCut{x}{P}{Q}}, composes two processes \tm{P} and
\tm{Q} with a shared channel \tm{x}, requiring that they follow dual protocols
on \tm{x}.
\begin{center}
  \cpInfAx
  \cpInfCut
\end{center}
There are two reduction rules which deal with the interactions between
forwarding and compositions. These implement the intuition that if a process is
meant to communicate on \tm{x}, \tm{x} is forwarding to \tm{y}, and nobody else
is listening on \tm{x}, then the process might as well start communicating on
\tm{y}.
\[
  \tm{\cpCut{x}{\cpLink{w}{x}}{P}} \Longrightarrow \tm{\cpSub{w}{x}{P}} 
\]
We can do this because \cp implements a binary session type system, meaning
that each communication has only two participants, and therefore we know that no
other process is communicating on \tm{x}.

\subsection{Example}
\label{sec:cp-example}
The multiplicatives are responsible for structuring communication, and
it is this structure which rules out deadlocked interactions.
Let us go back to our example of a deadlocked interaction from
\cref{sec:introduction}:
\[
  \tm{(\piPar{%
      \piRecv{x}{z}{\piSend{y}{\sliceofcake}{\john}}
    }{%
      \piRecv{y}{w}{\piSend{x}{\bill}{\mary}}
    })}
\]
If we want to translate this interaction to \cp, we run into a problem: there is
no \emph{plain} sending construct in \cp---we only have \tm{\cpSend{x}{y}{P}{Q}},
which requires that the remainder of the interaction is split in two independent
processes.
This enforces a certain structure on the program. Either John will already have
to have the cake in his hands, or Mary will already have to have the money in
the bank.
We model the second scenario below, assuming \john, \mary and \bank are
processes modeling John, Mary, and Mary's bank, and \cake and \money are the
types of two channels which give access to a slice of cake and appropriate
payment.
\begin{prooftree}
  \AXC{$\seq[{ \john }]{ \Gamma, \tmty{y}{\money^\bot}, \tmty{x}{\cake}}$}
  \SYM{\parr}
  \UIC{$\seq[{ \cpRecv{x}{y}{\john} }]{ \Gamma, \tmty{x}{\money^\bot \parr \cake} }$}
  \AXC{$\seq[{ \bank }]{ \Delta, \tmty{z}{\money} }$}
  \AXC{$\seq[{ \mary }]{ \Theta, \tmty{x}{\cake^\bot} }$}
  \SYM{\tens}
  \BIC{$\seq[{\cpSend{x}{z}{\bank}{\mary} }]{ \Delta, \tmty{x}{\money \tens \cake^\bot} }$}
  \NOM{Cut}
  \BIC{$\seq[{ \cpCut{x}{\cpRecv{x}{y}{\john}}{\cpSend{x}{z}{\bank}{\mary}} }]{\Gamma, \Delta, \Theta }$}
\end{prooftree}
The resulting process reduces, as expected:
\[
  \reducesto
  {\cpCut{x}{\cpRecv{x}{y}{\john}}{\cpSend{x}{z}{\bank}{\mary}}}
  {\cpCut{y}{\bank}{\cpCut{x}{\john}{\mary}}}
\]

\subsection{Preservation}
Preservation is the fact that term reduction preserves typing. In order to prove
this, we will first need to prove that structural congruence preserves typing.
\begin{theorem}[Preservation for $\equiv$]\label{thm:cp-preservation-equiv}
  If $\seq[{ P }]{ \Gamma }$ and $\tm{P} \equiv \tm{Q}$,
  then $\seq[{ Q }]{\Gamma }$.
  \qed
\end{theorem}
\begin{proof}
  By induction on the structure of the equivalence. The cases for reflexivity,
  transitivity and congruence are trivial. The two interesting cases, for
  \cpEquivCutComm and \cpEquivCutAss1 are given in \cref{fig:cp-preservation-equiv}
  \qed
\end{proof}
Then, we can prove preservation.
\begin{theorem}[Preservation]\label{thm:cp-preservation}
  If $\seq[{ P }]{ \Gamma }$ and \reducesto{P}{Q},
  then $\seq[{ Q }]{ \Gamma }$.
  \qed
\end{theorem}
\begin{proof}
  By induction on the structure of the reduction.
  See \cref{fig:cp-preservation}.
  % 
  The case for \cpRedGammaCut is trivial by the induction hypothesis, and the
  case for \cpRedGammaEquiv is trivial by the induction hypothesis and
  \cref{thm:cp-preservation-equiv}.
  \qed
\end{proof}

\begin{figure*}[ht]
  \centering
  \begin{tabular}{ll}
    \cpEquivCutComm
    &
      \begin{prooftree*}
        \AXC{$\seq[{ P }]{ \Gamma, \tmty{x}{A} }$}
        \AXC{$\seq[{ Q }]{ \Delta, \tmty{x}{A^\bot} }$}
        \NOM{Cut}
        \BIC{$\seq[{ \cpCut{x}{P}{Q} }]{ \Gamma, \Delta }$}
      \end{prooftree*}
    \\[30pt]
    $\equiv$
    &
      \begin{prooftree*}
        \AXC{$\seq[{ Q }]{ \Delta, \tmty{x}{A^\bot} }$}
        \AXC{$\seq[{ P }]{ \Gamma, \tmty{x}{A} }$}
        \NOM{\cref{thm:cp-negation-involutive}}
        \UIC{$\seq[{ P }]{ \Gamma, \tmty{x}{A^{\bot\bot}} }$}
        \NOM{Cut}
        \BIC{$\seq[{ \cpCut{x}{Q}{P} }]{ \Gamma, \Delta }$}
      \end{prooftree*}
    \\[40pt]
    \cpEquivCutAss1
    &
      \begin{prooftree*}
        \AXC{$\seq[{ P }]{ \Gamma, \tmty{x}{A} }$}
        \AXC{$\seq[{ Q }]{ \Delta, \tmty{x}{A^\bot}, \tmty{y}{B} }$}
        \AXC{$\seq[{ R }]{ \Theta, \tmty{y}{B^\bot} }$}
        \NOM{Cut}
        \BIC{$\seq[{ \cpCut{y}{Q}{R} }]{ \Delta, \Theta, \tmty{x}{A^\bot} }$}
        \NOM{Cut}
        \BIC{$\seq[{ \cpCut{x}{P}{\cpCut{y}{Q}{R}} }]{ \Gamma, \Delta, \Theta }$}
      \end{prooftree*}
    \\[30pt]
    $\equiv$
    &
      \begin{prooftree*}
        \AXC{$\seq[{ P }]{ \Gamma, \tmty{x}{A} }$}
        \AXC{$\seq[{ Q }]{ \Delta, \tmty{x}{A^\bot}, \tmty{y}{B} }$}
        \NOM{Cut}
        \BIC{$\seq[{ \cpCut{x}{P}{Q} }]{ \Gamma, \Delta, \tmty{y}{B} }$}
        \AXC{$\seq[{ R }]{ \Theta, \tmty{y}{B^\bot} }$}
        \NOM{Cut}
        \BIC{$\seq[{ \cpCut{y}{\cpCut{x}{P}{Q}}{R} }]{ \Gamma, \Delta, \Theta }$}
      \end{prooftree*}
  \end{tabular}
  \caption{Type preservation for the structural congruence of \rcp}
  \label{fig:cp-preservation-equiv}
\end{figure*}
%%% Local Variables:
%%% TeX-master: "main"
%%% End:

\begin{figure*}[ht]
  \begin{tabular}{ll}
    \cpRedAxCut1
    &
      \begin{prooftree*}
        \AXC{}
        \NOM{Ax}
        \UIC{$\seq[{ \cpLink{w}{x} }]{ \tmty{w}{A}, \tmty{x}{A^\bot} }$}
        \AXC{$\seq[{ R }]{ \tmty{x}{A}, \Gamma }$}
        \NOM{Cut}
        \BIC{$\seq[{ \cpCut{x}{\cpLink{w}{x}}{R} }]{ \tmty{w}{A}, \Gamma }$}
      \end{prooftree*}
    \\[20pt]
    $\Longrightarrow$
    &
      \begin{prooftree*}
        \AXC{$\seq[{ \cpSub{w}{x}{R} }]{ \tmty{w}{A}, \Gamma }$}
      \end{prooftree*}
    \\[20pt]
    
    \cpRedBetaTensParr
    &
      \begin{prooftree*}
        \AXC{$\seq[{ P }]{ \Gamma, \tmty{x}{A^\bot}, \tmty{y}{B^\bot} }$}
        \SYM{\parr}
        \UIC{$\seq[{ \cpRecv{y}{x}P }]{ \Gamma, \tmty{y}{A^\bot \parr B^\bot} }$}
        \AXC{$\seq[{ Q }]{ \Delta, \tmty{x}{A} }$}
        \AXC{$\seq[{ R }]{ \Theta, \tmty{y}{B} }$}
        \SYM{\tens}
        \BIC{$\seq[{ \cpSend{y}{x}(Q \mid R) }]{ \Delta, \Theta, \tmty{y}{A \tens B} }$}
        \NOM{Cut}
        \BIC{$\seq[{ \cpCut{y}{\cpRecv{y}{x}{P}}{\cpSend{y}{x}{Q}{R}} }]{ \Gamma, \Delta, \Theta }$}
      \end{prooftree*}
    \\[20pt]
    $\Longrightarrow$
    &
      \begin{prooftree*}
        \AXC{$\seq[{ P }]{ \Gamma, \tmty{x}{A^\bot}, \tmty{y}{B^\bot} }$}
        \AXC{$\seq[{ Q }]{ \Delta, \tmty{x}{A} }$}
        \NOM{Cut}
        \BIC{$\seq[{ \cpCut{x}{P}{Q} }]{ \Gamma, \Delta, \tmty{y}{B^\bot} }$}
        \AXC{$\seq[{ R }]{ \Theta, \tmty{y}{B} }$}      
        \NOM{Cut}
        \BIC{$\seq[{ \cpCut{y}{\cpCut{x}{P}{Q}}{R} }]{ \Gamma, \Delta, \Theta }$}
      \end{prooftree*}
    \\[25pt]
    
    \cpRedBetaOneBot
    &
      \begin{prooftree*}
        \AXC{$\seq[{ P }]{ \Gamma }$}
        \SYM{\bot}
        \UIC{$\seq[{ \cpWait{x}{P} }]{ \Gamma, \tmty{x}{\bot} }$}
        \AXC{}
        \SYM{\one}
        \UIC{$\seq[{ \cpHalt{x} }]{ \tmty{x}{\one} }$}
        \NOM{Cut}
        \BIC{$\seq[{ \cpCut{x}{\cpHalt{x}}{\cpWait{x}{P}} }]{ \Gamma }$}
      \end{prooftree*}
    \\[20pt]
    $\Longrightarrow$
    &
      \begin{prooftree*}
        \AXC{$\seq[{ P }]{ \Gamma }$}
      \end{prooftree*}
    \\[25pt]
    
    \cpRedBetaPlusWith1
    &
      \begin{prooftree*}
        \AXC{$\seq[{ P }]{ \Gamma, \tmty{x}{A^\bot} }$}
        \AXC{$\seq[{ Q }]{ \Gamma, \tmty{x}{B^\bot} }$}
        \SYM{\with}
        \BIC{$\seq[{ \cpCase{x}{P}{Q} }]{ \Gamma, \tmty{x}{A^\bot \with B^\bot} }$}
        \AXC{$\seq[{ R }]{ \Delta, \tmty{x}{A} }$}
        \SYM{\plus_1}
        \UIC{$\seq[{ \cpInl{x}{R}}]{ \Delta, \tmty{x}{A \plus B} }$}
        \NOM{Cut}
        \BIC{$\seq[{ \cpCut{x}{\cpCase{x}{P}{Q}}{\cpInl{x}{R}} }]{ \Gamma, \Delta }$}
      \end{prooftree*}
    \\[20pt]
    $\Longrightarrow$
    &
      \begin{prooftree*}
        \AXC{$\seq[{ P }]{ \Gamma, \tmty{x}{A^\bot} }$}
        \AXC{$\seq[{ R }]{ \Delta, \tmty{x}{A} }$}
        \NOM{Cut}
        \BIC{$\seq[{ \cpCut{x}{P}{R} }]{ \Gamma, \Delta }$} 
      \end{prooftree*}
    \\[25pt]
    
    \cpRedBetaPlusWith2
    &
      (as above)
  \end{tabular}
  
  \caption{Type preservation for the \cpRedAxCut1 and \textbeta-reduction rules of \cp}
  \label{fig:cp-preservation}
\end{figure*}
%%% Local Variables:
%%% TeX-master: "main"
%%% End:

\subsection{Canonical Forms and Progress}
There are several reduction strategies for \cp, all of which use as their
canonical form terms which are not a cut.

The reduction strategy described by \citenat{wadler2012} matches precisely the
proof of cut-elimination for linear logic. It applies commutative conversions to
bubble actions over unrelated \textnu-binders to the \textnu-binder which
introduces the channel on which they act. It does this until it creates a
\textbeta-redex, upon which it applies \textbeta-reduction, and continues.

The reduction strategy described by \citeauthor{lindley2015semantics} creates
redexes by rewriting using the structural congruence, then applies
\cpRedAxCut1 or \textbeta-reduction. It does this until the process blocks on
one or more external communications. Then, to convert the term to canonical
form, it applies the commuting conversions to bubble one of those external
communications to the front of the term.

In this paper, we follow the reduction strategy detailed by
\citeauthor{lindley2015semantics}, but instead define canonical forms as terms
which are blocked on an external communication.
This allows us to exclude the commutative conversions from our reduction system,
which makes our reduction system more closely resemble that of the
\textpi-calculus.

We start out by formally defining actions, and what we mean when we say a
process ``acts on'' a channel.
\begin{definition}[Actions]\label{def:cp-action}
  A process \tm{P} \emph{acts on} a channel \tm{x} if it is in one of the
  following forms:
  \begin{multicols}{3}
    \begin{itemize}[noitemsep,topsep=0pt,parsep=0pt,partopsep=0pt]
    \item \tm{\cpLink{x}{y}}
    \item \tm{\cpLink{y}{x}}
    \item \tm{\cpSend{x}{y}{P'}{Q'}}
    \item \tm{\cpRecv{x}{y}{P'}}
    \item \tm{\cpHalt{x}}
    \item \tm{\cpWait{x}{P'}}
    \item \tm{\cpInl{x}{P'}}
    \item \tm{\cpInr{x}{P'}}
    \item \tm{\cpCase{x}{P'}{Q'}}
    \item \tm{\cpAbsurd{x}}
    \end{itemize}
  \end{multicols}
  \noindent
  We say a process \tm{P} is an \emph{action} if it acts on some channel \tm{x}.
  \qed
\end{definition}
Furthermore, we will need the notion of an \emph{evaluation prefix}.
Intuitively, evaluation prefixes are multi-holed contexts consisting solely of
cuts. We will use evaluation prefixes in order to have a view of every
\emph{action} in a process at once.
\begin{definition}[Evaluation prefixes]\label{def:cp-evaluation-prefixes}
  We define evaluation prefixes as:
  \begin{align*}
    \tm{G}, \tm{H} := \tm{\Box} \mid \tm{\cpCut{x}{G}{H}}
  \end{align*}
  The \tm{\Box} construct represents a hole.
  \qed
\end{definition}
\begin{definition}[Plugging]\label{def:cp-evaluation-prefix-plugging}
  We define plugging for an evaluation prefix with $n$ holes as:
  \[
    \begin{array}{ll}
      \tm{\cpPlug{\Box}{R}} & := \; \tm{R} \\
      \tm{\cpPlug{\cpCut{x}{G}{H}}{R_1 \dots R_m, R_{m+1} \dots R_{n}}}
                            & := \; \tm{\cpCut{x}{\cpPlug{G}{R_1 \dots R_m}}{\cpPlug{H}{R_{m+1} \dots R_n}}}
    \end{array}
  \]
  Note that in the second case, \tm{G} is an evaluation prefix with $m$ holes,
  and \tm{H} is an evaluation prefix with $(n-m)$ holes.
  \qed
\end{definition}
Intuitively, we can say that every term of the form
\tm{\cpPlug{G}{P_1 \dots P_n}} is equivalent to some term of the form
\tm{\cpCut{x_1}{P_1}{\cpCut{x_2}{P_2}{\dots \cpCut{x_n}{P_{n-1}}{P_n} \dots}}} 
where $\tm{x_1} \dots \tm{x_{n-1}}$ are the channels bound in \tm{G}.
A similar equivalence was used by \citenat{lindley2015semantics} in their
semantics for \cp.
\begin{definition}[Maximum evaluation prefix]\label{def:cp-maximum-evaluation-prefix}
  We say that \tm{G} is the evaluation prefix of \tm{P} when there exist terms
  $\tm{P_1} \dots \tm{P_n}$ such that $\tm{P} = \tm{\cpPlug{G}{P_1 \dots P_n}}$.
  We say that \tm{G} is the maximum evaluation prefix if each \tm{P_i} is an
  action. 
  \qed
\end{definition}
\begin{lemma}\label{thm:cp-maximum-evaluation-prefix}
  Every term \tm{P} has a maximum evaluation prefix.
  \qed
\end{lemma}
\begin{proof}
  By induction on the structure of the term \tm{P}. 
  \qed
\end{proof}
We can now define what it means for a term to be in canonical form. Intuitively,
a process is in canonical form either when there is no top-level cut, or when it
is blocked on an external communication. We state this formally as follows:
\begin{definition}[Canonical forms]\label{def:cp-canonical-forms}
  A process \tm{P} is in canonical form if it is an action, or if it is of the
  form \tm{\cpPlug{G}{P_1 \dots P_n}}, where \tm{G} is the maximum evaluation
  prefix of \tm{P}, no \tm{P_i} is a link which acts on a bound channel, and no
  \tm{P_i} and \tm{P_j}, on different sides of at least one cut, act on the same
  channel.
  \qed
\end{definition}
In addition to evaluation prefixes, we also need a concept of evaluation
contexts to prove progress. Intuitively, evaluation contexts are one-holed term
contexts under which reduction can take place. For \cp, these consist solely of
cuts.
\begin{definition}[Evaluation contexts]\label{def:cp-evaluation-contexts}
  We define evaluation contexts as:
  \begin{align*}
    \tm{E} := \tm{\Box}
    \mid \tm{\cpCut{x}{E}{P}}
    \mid \tm{\cpCut{x}{P}{E}}
  \end{align*}
  \qed
\end{definition}
\begin{definition}[Plugging]\label{def:cp-evaluation-context-plugging}
  We define plugging for evaluation contexts as:
  \begin{gather*}
    \begin{array}{ll}
      \tm{\cpPlug{\Box}{R}}            
      & := \; \tm{R}
      \\
      \tm{\cpPlug{\cpCut{x}{E}{P}}{R}}
      & := \; \tm{\cpCut{x}{\cpPlug{E}{R}}{P}}
      \\
      \tm{\cpPlug{\cpCut{x}{P}{E}}{R}}
      & := \; \tm{\cpCut{x}{P}{\cpPlug{E}{R}}}
    \end{array}
  \end{gather*}
  \qed
\end{definition}
Next, we prove a lemma which states that we can push a \textnu-binder under any
evaluation context \tm{E}, as long as \tm{E} does not use the channel name
introduced by that \textnu-binder.
\begin{lemma}\label{thm:cp-display-cut}
  If $\seq[{ \tm{\cpCut{x}{\cpPlug{E}{P}}{Q}} }]{ \Gamma }$ and
  $\notFreeIn{x}{E}$, then $\tm{\cpCut{x}{\cpPlug{E}{P}}{Q}} \equiv
  \tm{\cpPlug{E}{\cpCut{x}{P}{Q}}}$. 
  \qed
\end{lemma}
\begin{proof}
  By induction on the structure of the evaluation context \tm{E}.
  \qed
\end{proof}
Last, we prove two lemmas relating evaluation contexts to evaluation prefixes,
showing that if we have a process acting on a channel \tm{x} under an evaluation
prefix, we can rewrite to reveal the binder introducing \tm{x}.
Each of these lemmas deals with one of the cases for reduction.
The first allows us to reveal the binder for links.
It is an equivalence instead of an equality because it uses commutativity of
links.
\begin{lemma}\label{thm:cp-progress-link}
  If $\seq[{ \cpPlug{G}{P_1 \dots P_n} }]{ \Gamma }$, and some \tm{P_i} is a
  link \tm{\cpLink{x}{y}}, then either \tm{x} and \tm{y} are not bound by
  \tm{G}, or there exist \tm{E}, \tm{E'} and \tm{Q} such that
  \(
  \tm{\cpPlug{G}{P_1 \dots P_n}} \equiv
  \tm{\cpPlug{E}{\cpCut{x}{\cpPlug{E'}{\cpLink{x}{y}}}{Q}}}
  \).
  \qed
\end{lemma}
\begin{proof}
  By induction on the structure of the evaluation prefix \tm{G}.
  \qed
\end{proof}
The second allows us to reveal the binder introducing the channel shared by two
actions. It is a plain equality.
\begin{lemma}\label{thm:cp-progress-beta}
  If $\seq[{ \cpPlug{G}{P_1 \dots P_n} }]{ \Gamma }$, and some \tm{P_i} and
  \tm{P_j}, on different sides of at least one cut, act on the same bound
  channel \tm{x}, then there exist \tm{E}, \tm{E_i} and \tm{E_j} such that 
  \(
  \tm{\cpPlug{G}{P_1 \dots P_n}} =
  \tm{\cpPlug{E}{\cpCut{x}{\cpPlug{E_i}{P_i}}{\cpPlug{E_j}{P_j}}}}
  \).
  \qed
\end{lemma}
\begin{proof}
  By induction on the structure of the evaluation prefix \tm{G}.
  \qed
\end{proof}

\begin{theorem}[Progress]\label{thm:cp-progress}
  If $\seq[{ P }]{ \Gamma }$, then either $\tm{P}$ is in canonical form, or
  there exists a $\tm{P'}$ such that $\reducesto{P}{P'}$. 
  \qed
\end{theorem}
\begin{proof}
  By induction on the structure of the derivation of \seq[P]{\Gamma}.
  The only interesting case is that in which \tm{P} was constructed using a cut,
  as all other rules introduce canonical forms.
  We consider the maximum evaluation prefix \tm{G} of \tm{P}. This consists of
  $n$ cuts, but composes $n+1$ actions. Therefore, one of the following must be
  true.
  \begin{itemize}
  \item
    One of the actions is a link on a bound channel.
    We rewrite by \cref{thm:cp-display-cut} and \cref{thm:cp-progress-link},
    and apply \cpRedAxCut1.
  \item
    Two of the processes act on the same bound channel.
    We rewrite by \cref{thm:cp-display-cut} and \cref{thm:cp-progress-beta},
    and apply one of the \textbeta-rules.
  \item
    At least one process acts on a free channel. No process is a link acting on
    a bound channel. No two process act on the same bound channel. Therefore,
    \tm{P} is canonical.
  \end{itemize}
  \qed
\end{proof}

\section{Shared Channels and Non-Determinism}\label{sec:cpnd}

\begin{definition}[Terms]\label{def:nc-terms}
  We extend \cref{def:cp-terms} with the following constructs:
  \[\!
    \begin{aligned}
      \tm{P}, \tm{Q}, \tm{R}
           :=& \; \dots
      \\ \mid& \; \tm{\ncCnt{x}{y}{P}} &&\text{client creation}
      \\ \mid& \; \tm{\ncSrv{x}{y}{P}} &&\text{server interaction}
      \\ \mid& \; \tm{\ncPool{P}{Q}}   &&\text{parallel composition of clients}
    \end{aligned}
  \]
  \qed
\end{definition}

\begin{definition}[Types]\label{def:nc-types}
  We extend \cref{def:cp-types} with the following types:
  \[\!
    \begin{aligned}
      \ty{A}, \ty{B}, \ty{C}
           :=& \; \dots
      \\ \mid& \; \ty{\take[n]{A}} &&\text{pool of} \; n \; \text{clients}
      \\ \mid& \; \ty{\give[n]{A}} &&n \; \text{server interactions}
    \end{aligned}
  \]  
  \qed
\end{definition}

\begin{lemma}[Involutive]\label{thm:nc-negation-involutive}
  We have $\ty{A^{\bot\bot}} = \ty{A}$.
  \qed
\end{lemma}

\begin{definition}[Structural congruence]\label{def:nc-equiv}
  We extend \cref{def:cp-equiv} with the following equivalences:
  \[
    \begin{array}{llll}
      \ncEquivPoolComm
      & \tm{\ncPool{P}{Q}}
      & \equiv \;
      & \tm{\ncPool{Q}{P}}
      \\
      \ncEquivPoolAss1
      & \tm{\ncPool{P}{\ncPool{Q}{R}}}
      & \equiv \;
      & \tm{\ncPool{\ncPool{P}{Q}}{R}}
      \\
      \ncRedKappaPool1
      & \tm{\cpCut{x}{\ncPool{P}{Q}}{R}}
      & \equiv \;
      & \tm{\ncPool{P}{\cpCut{x}{Q}{R}}} \quad \text{if} \; \notFreeIn{x}{P} 
      \\
      \ncRedKappaPool2
      & \tm{\ncPool{P}{\cpCut{x}{Q}{R}}}
      & \equiv \;
      & \tm{\cpCut{x}{\ncPool{P}{Q}}{R}}
    \end{array}
  \]
  \qed
\end{definition}

\begin{theorem}[Symmetry]\label{thm:nc-symmetry}
  If $\tm{P} \equiv \tm{Q}$, then $\tm{Q} \equiv \tm{P}$.
  \qed
\end{theorem}

\begin{definition}[Typing judgements]\label{def:nc-typing-judgement}
  We extend \cref{def:cp-typing-judgement} with the following rules:
  {\normalfont
    \begin{center} \ncInfTake1 \ncInfGive1 \end{center}
    \begin{center} \ncInfPool  \ncInfCont  \end{center}
  }
  \qed
\end{definition}

\begin{definition}[Term reduction]\label{def:nc-term-reduction-1}
  We extend \cref{def:cp-term-reduction-2} with the following reductions:
  \[
    \begin{array}{llll}
      \ncRedBetaStar{1}
      & \tm{\cpCut{x}{\ncCnt{x}{y}{P}}{\ncSrv{x}{z}{R}}}
      & \Longrightarrow \;
      & \tm{\cpCut{y}{P}{\cpSub{y}{z}{R}}}
      \\
      \ncRedBetaStar{n+1}
      & \tm{\cpCut{x}{\ncPool{\ncCnt{x}{y}{P}}{Q}}{\ncSrv{x}{z}{R}}}
      & \Longrightarrow \;
      & \tm{\cpCut{x}{Q}{\cpCut{y}{P}{\cpSub{y}{z}{R}}}}
    \end{array}
  \]
  \begin{prooftree}
    \AXC{\reducesto{P}{P'}}
    \SYM{\ncRedGammaPool}
    \UIC{\reducesto{\ncPool{P}{Q}}{\ncPool{P'}{Q}}}
  \end{prooftree}
  \qed
\end{definition}

\subsection{Preservation}

\begin{theorem}[Preservation for $\equiv$]\label{thm:nc-preservation-equiv}
  If $\seq[{ P }]{ \Gamma }$ and $\tm{P} \equiv \tm{Q}$,
  then $\seq[{ Q }]{\Gamma }$.
  \qed
\end{theorem}

\begin{theorem}[Preservation]\label{thm:nc-preservation}
  If $\seq[{ P }]{ \Gamma }$ and \reducesto{P}{Q},
  then $\seq[{ Q }]{ \Gamma }$.
  \qed
\end{theorem}

\subsection{Canonical Forms and Progress}

\begin{definition}[Action]\label{def:nc-action}
  We extend \cref{def:cp-action} with the following cases:
  \begin{itemize}[noitemsep,topsep=0pt,parsep=0pt,partopsep=0pt]
  \item \tm{\ncCnt{x}{y}{P'}}
  \item \tm{\ncSrv{x}{y}{P'}}
  \end{itemize}
  \qed
\end{definition}

\begin{definition}[Evaluation prefixes]\label{def:nc-evaluation-prefixes}
  We extend \cref{def:cp-evaluation-prefixes} with the following constructs:
  \begin{align*}
    \tm{G}, \tm{H} := \dots \mid \tm{\ncPool{G}{H}}
  \end{align*}
  We also define a special case of evaluation prefixes, which we will refer to
  as \emph{pooling prefixes}. These are evaluation prefixes which consist solely
  of pooling operators and holes. 
  \qed
\end{definition}

\begin{definition}[Plugging]\label{def:nc-evaluation-prefix-plugging}
  We extend \cref{def:cp-evaluation-prefix-plugging} with the following case:
  \[
    \begin{array}{ll}
      \tm{\cpPlug{\ncPool{G}{H}}{R_1 \dots R_m, R_{m+1} \dots R_{n}}}
      & := \; \tm{\ncPool{\cpPlug{G}{R_1 \dots R_m}}{\cpPlug{H}{R_{m+1} \dots R_n}}}
    \end{array}
  \]
  Note that in the this case, \tm{G} is an evaluation prefix with $m$ holes,
  and \tm{H} is an evaluation prefix with $(n-m)$ holes.
  \qed
\end{definition}

\begin{definition}[Evaluation contexts]\label{def:nc-evaluation-contexts}
  We extend \cref{def:cp-evaluation-contexts} with the following constructs:
  \begin{align*}
    \tm{E} := \dots \mid \tm{\ncPool{E}{P}} \mid \tm{\ncPool{P}{E}}
  \end{align*}
  We also define a special case of evaluation contexts, which we will refer to
  as \emph{pooling contexts}. These are evaluation contexts which consist solely
  of pooling operators and holes.
  \qed
\end{definition}

\begin{definition}[Plugging]\label{def:nc-evaluation-context-plugging}
  We extend \cref{def:cp-evaluation-context-plugging} with the following cases:
  \begin{gather*}
    \begin{array}{ll}
      \tm{\cpPlug{\ncPool{E}{P}}{R}}
      & := \; \tm{\ncPool{\cpPlug{E}{R}}{P}}
      \\
      \tm{\cpPlug{\ncPool{P}{E}}{R}}
      & := \; \tm{\ncPool{P}{\cpPlug{E}{R}}}
    \end{array}
  \end{gather*}
  \qed
\end{definition}

\begin{lemma}\label{thm:nc-progress-shared}
  If $\seq[{ \cpPlug{G}{P} }]{ \Gamma, \tmty{x}{\take[n]{A}} }$ and
  $\freeIn{x}{P}$, then there exists an $\tm{E}$ and $\tm{R_1}\dots\tm{R_{n-1}}$
  such that $\tm{\cpPlug{G}{P}} \equiv
  \tm{\cpPlug{E}{\ncPool{P}{\ncPool{R_1}{\ncPool{\dots}{R_{n-1}}\dots}}}}$,
  where $\notFreeIn{x}{E}$ and $\freeIn{x}{R_1},\dots,\freeIn{x}{R_{n-1}}$.
  \qed
\end{lemma}

\begin{theorem}[Progress]\label{thm:cp-progress}
  If $\seq[{ P }]{ \Gamma }$, then either $\tm{P}$ is in canonical form, or
  there exists a $\tm{P'}$ such that $\reducesto{P}{P'}$. 
  \qed
\end{theorem}

\section{Discussion and Future Work}\label{sec:discussion}

\printbibliography

\end{document}
%%% Local Variables:
%%% TeX-master: "main"
%%% End:
