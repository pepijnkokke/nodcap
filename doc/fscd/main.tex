\documentclass[a4paper,UKenglish]{lipics-v2016}
\usepackage{microtype}%
\bibliographystyle{plainurl}%
% Author macros::begin 
\title{Give or Take: Non-Determinism, Linearity, and Session Types}%
\titlerunning{Give or Take}%
\author[1]{Pepijn Kokke}%
\affil[1]{LFCS, University of Edinburgh, Edinburgh, United Kingdom\\
  \texttt{pepijn.kokke@ed.ac.uk}}%
\authorrunning{P. Kokke}%
\Copyright{Pepijn Kokke}%
\subjclass{
  F.4.1. [Mathematical Logic]: Lambda calculus and related systems;
  D.3.1. [Formal Definitions and Theory]: Syntax;
  D.3.2. [Language Classifications] Concurrent, distributed, and parallel languages
}%
\keywords{$\pi$-calculus, linear logic, session types, deadlock free, non-determinism}
% Author macros::end

% Editor-only macros::begin
\EventEditors{John Q. Open and Joan R. Acces}
\EventNoEds{2}
\EventLongTitle{42nd Conference on Very Important Topics (CVIT 2016)}
\EventShortTitle{CVIT 2016}
\EventAcronym{CVIT}
\EventYear{2016}
\EventDate{December 24--27, 2016}
\EventLocation{Little Whinging, United Kingdom}
\EventLogo{}
\SeriesVolume{42}
\ArticleNo{23}
% Editor-only macros::end

% Preamble macros::begin
\usepackage{cmap}
\usepackage{alphabeta}
\usepackage[greek,english]{babel}
\languageattribute{greek}{polutoniko}
\usepackage{hyperref}
\usepackage{comment}
\usepackage{lmodern}

%%% TikZ and QTree

\usepackage{tikz}
\usepackage{tikz-qtree}

%%% Term Language
\newcommand{\subst}[3]{\ensuremath{#1 \{ #2 / #3 \}}}
\newcommand{\case}[3]{%
\toks0={#2#3}%
\edef\param{\the\toks0}%
\ifx\param\empty
  \ensuremath{\text{case} \; #1 \; \{\}}
\else
  \ensuremath{\text{case} \; #1 \; \{ #2 ; #3 \}}
\fi}
\newcommand{\inl}[2]{%
  \ensuremath{#1[\text{inl}].#2}}
\newcommand{\inr}[2]{%
  \ensuremath{#1[\text{inr}].#2}}

%%% Local Variables:
%%% TeX-master: "main"
%%% End:
% Preamble macros::end

\begin{document}

\maketitle

\begin{abstract}
  The $\pi$-calculus is a great foundation for concurrent programming languages.
  With only a handful of constructions, it manages to model communicating
  processes faithfully---for instance, $\pi$-calculus programs can have races
  and deadlocks. 
  CP~\cite{wadler2012} is a session type system for the $\pi$-calculus, which
  has strong ties to classical linear logic~\cite{girard1987}. Programs typed by
  CP are guaranteed to be race and deadlock free. 
  Under the slogan that ``well-typed programs cannot `go wrong'\!\!
  ''~\cite{milner1978}, it makes sense to exclude deadlocks, as one rarely
  \emph{wants} a program to deadlock. 
  Races, on the other hand, may be the desired behaviour of a program---think
  of online auctions, vendors, or any other scenario where several processes
  compete for a limited resource. 

  In this paper, we introduce \nodcap\ (Nodcap), an extension of CP which is more
  permissive, and permits programs with races, without losing its strong ties
  to classical linear logic or the guarantee of deadlock freeness. 
\end{abstract}

\section{Overview}
\begin{itemize}
%\item Introduction
%  \begin{itemize} 
%  \item introduce example (racy vendor interaction)
%  \item mention that the $\pi$-calculus can model this example
%  \item mention the downsides of the $\pi$-calculus (i.e. deadlock)
%  \item mention that CP prevents deadlock
%  \item mention that CP \emph{cannot} model this example
%  \end{itemize}
\item Background
  \begin{itemize}
  \item introduce the $\pi$-calculus
  \item introduce CP
  \item mention local choice
  \end{itemize}
\item Main contribution
  \begin{itemize}
  \item introduce typing rules for \nodcap
  \item introduce evaluation for programs in \nodcap
  \item sketch a proof for server expansion
  \item sketch a proof for client interleaving
  \item sketch a proof for communication
  \end{itemize}
\item Extensions and future work
  \begin{itemize}
  \item mention relation to local choice (and proper cut elimination)
  \item mention possibility of recursion and creating servers from streams
  \item mention possibility of resource quantifiers
  \end{itemize} 
\end{itemize}

\section{Introduction}
Imagine the following scenario:
\begin{quote}
  John and Mary are working from home one morning when they each get a craving
  for a slice of cake. Being denizens of the web, they quickly find an nearby
  p\^atisserie that does home deliveries. Unfortunately for them, they both
  order their cake at the \emph{same} p\^atisserie, which only has one slice
  left. After that, all it can deliver is disappointment. 
\end{quote}
This scenario is a classic example of a race condition, and it is modeled in the
$\pi$-calculus by the program below, given that \john, \mary, and \ptis\ are
processes representing John, Mary and the p\^atisserie, respectively, and
\cake\ and \nocake\ are two channels over which \ptis\ will send a slice of
cake, and disappointment, respectively.
The arrows indicate two possible reductions of the program: one in which John
gets the cake and Mary is disappointed, and one in which Mary gets cake and John
is disappointed.  
\def\exampleprograma{\textcolor{red}{\ensuremath{%
      \cut{x}(\recv{x}{y}\john \mid \recv{x}{z}\mary \mid
      \send{x}{\cake}\send{x}{\nocake}\ptis)
    }}}
\def\exampleresultaa{\textcolor{red}{\ensuremath{%
      \cut{x}(\subst{\john}{\cake}{y} \mid \subst{\mary}{\nocake}{z} \mid \ptis)     
    }}}
\def\exampleresultab{\textcolor{red}{\ensuremath{%
      \cut{x}(\subst{\john}{\nocake}{y} \mid \subst{\mary}{\cake}{z} \mid \ptis)     
    }}}
\begin{center}
  \begin{tikzpicture}
    \node (program) at (0,0) {$\exampleprograma\quad$};
    \node[above=1cm, right=1cm, at=(program.east)] (resulta) {$\!\!^\star\quad\exampleresultaa$};
    \node[below=1cm, right=1cm, at=(program.east)] (resultb) {$\!\!^\star\quad\exampleresultab$};
    \draw[->] (program.east) to [out=045,in=180] (resulta.west);
    \draw[->] (program.east) to [out=315,in=180] (resultb.west);
  \end{tikzpicture}
\end{center}
Races show up in any client/server system where clients compete for limited
server resources. It is therefore important that any concurrent programming
language allows us to write them.
\\
While the $\pi$-calculus is up to the task of modeling such programs, it can
also model some undesirable programs, programs which `go wrong'. An example of
such a program is the following scenario:
\begin{quote}
  John managed to order the last slice of cake from the p\^atisserie. However,
  Mary \emph{really} wants it, and offers him some money for it. John agrees,
  but sadly there is no trust between the two. Mary demands to get the cake
  first, before she gives John the money. John demands to get the money first,
  before he gives Mary the cake.
\end{quote}
This scenario is a classic example of a deadlock, and it is modeled in the
$\pi$-calculus by the program below, given that \john\ and \mary\ are processes 
representing John and Mary, respectively, \cake\ is a channel over which
\john\ will send a slice of cake, and \money\ is a channel over which \mary\ %
will send money.
\def\exampleprogramb{\textcolor{red}{\ensuremath{%
      \cut{x}(\recv{x}{y}\send{x}{\cake}\john \mid \recv{x}{z}\send{x}{\money}\mary)
    }}}
\begin{center}
  \exampleprogramb
\end{center}
The program is stuck---John is refusing to send the cake until he receives money
from Mary, and Mary is refusing to send the money until John sends the cake, so
nothing will actually happen.
It `goes wrong', in the sense that a deadlock is unlikely to be the intended
behaviour of a program. 

Session types, a formalism proposed by~\cite{honda1993}, can provide static
guarantees that concurrent processes respect communication protocols, by
specifying the expected structure of communications among processes.
Recently, work by~\cite{caires2010} and~\cite{wadler2012} has yielded tight
correspondences between session typed $\pi$-calculi and intuitionistic and
classical linear logic~\cite{girard1987}, respectively.
These correspondences are tight in the following sense: not only do they
interpret session types as linear propositions, but they also show a
correspondence between reduction in the $\pi$-calculus and cut elimination in
linear logic.
Furthermore, by the very nature of their correspondence with linear logic,
programs in these calculi are guaranteed to be free from deadlocks.
Unfortunately, they are also free from \emph{races}.

In this paper, we introduce \nodcap\ (Nodcap), an extension of the type system
CP~\cite{wadler2012} which is more permissive, and permits programs with races,
without losing the tight correspondence with classical linear logic or the
guarantee of deadlock. 


\section{Background}

%\begin{figure*}[h]
  \centering

  % JGM: the syntax is already given in the text

  % \(
  %   \text{Type} \; \tm{A} , \tm{B} ::=
  %   \tm{\bot} \mid \tm{\one} \mid \tm{\top} \mid \tm{\nil} \mid
  %   \tm{A \parr B} \mid \tm{A \tens B} \mid
  %   \tm{A \with B} \mid \tm{A \plus B}
  % \)

  \begin{center}
    \begin{prooftree*}
      \AXC{$\vphantom{yA^\bot}$}
      \NOM{Ax}
      \UIC{$\seq[{ \link{x}{y} }]{ \tm[x]{A}, \tm[y]{A^\bot} }$}
    \end{prooftree*}
    \begin{prooftree*}
      \AXC{$\seq[{ P }]{ \Gamma, \tm[x]{A} }$}
      \AXC{$\seq[{ Q }]{ \Delta, \tm[y]{A^\bot} }$}
      \NOM{Cut}
      \BIC{$\seq[{ \cut{x}(P \mid Q) }]{ \Gamma, \Delta }$}
    \end{prooftree*}
  \end{center}

  \begin{center}
    \begin{prooftree*}
      \AXC{$\seq[ P ]{ \Gamma , \tm[y]{A} }$}
      \AXC{$\seq[ Q ]{ \Delta , \tm[x]{B} }$}
      \SYM{\tens}
      \BIC{$\seq[{ x[y].(P \mid Q) }]{ \Gamma , \Delta , \tm[x]{A \tens B} }$}
    \end{prooftree*}
    \begin{prooftree*}
      \AXC{$\seq[ P ]{ \Gamma , \tm[y]{A} , \tm[x]{B} }$}
      \SYM{\parr}
      \UIC{$\seq[ x(y).P ]{ \Gamma , \tm[x]{A \parr B} }$}
    \end{prooftree*}
  \end{center}

  \begin{center}
    \begin{prooftree*}
      \AXC{$\seq[ P ]{ \Gamma , \tm[x]{A} }$}
      \SYM{\plus_1}
      \UIC{$\seq[{ \inl{x}{P} }]{ \Gamma , \tm[x]{A \plus B} }$}
    \end{prooftree*}
    \begin{prooftree*}
      \AXC{$\seq[ P ]{ \Gamma , \tm[x]{B} }$}
      \SYM{\plus_2}
      \UIC{$\seq[{ \inr{x}{P} }]{ \Gamma , \tm[x]{A \plus B} }$}
    \end{prooftree*}
  \end{center}

  \begin{prooftree}
    \AXC{$\seq[ P ]{ \Gamma , \tm[x]{A} }$}
    \AXC{$\seq[ Q ]{ \Delta , \tm[x]{B} }$}
    \SYM{\with}
    \BIC{$\seq[ \case{x}{P}{Q} ]{ \Gamma , \Delta , \tm[x]{A \with B} }$}
  \end{prooftree}

  \begin{center}
    \begin{prooftree*}
      \AXC{}
      \SYM{\one}
      \UIC{$\seq[{ x[].0 }]{ \tm[x]{\one} }$}
    \end{prooftree*}
    \begin{prooftree*}
      \AXC{$\seq[P]{ \Gamma }$}
      \SYM{\bot}
      \UIC{$\seq[{x().P}]{ \Gamma , \tm[x]{\bot} }$}
    \end{prooftree*}
    (no rule for \nil)
    \begin{prooftree*}
      \AXC{}
      \SYM{\top}
      \UIC{$\seq[ \case{x}{}{} ]{ \tm[x]{\top} }$}
    \end{prooftree*}
  \end{center}

  \caption{RCP, the multiplicative additive subset of CP~\cite{wadler2012}}
  \label{fig:cp}
\end{figure*}
%%% Local Variables:
%%% TeX-master: "main"
%%% End:

%\begin{figure*}
  \centering
  \(
    \text{Type} \; A,B ::= \dots \mid \give[n]{A} \mid \take[n]{A}
  \)

  \begin{center}
    \begin{prooftree*}
      \AXC{$\seq[{P}]{ \Gamma , \tm[y]{A} }$}
      \SYM{\give[1]{}}
      \UIC{$\seq[{x[y].P}]{ \Gamma , \tm[x]{\give[1]{A}} }$}
    \end{prooftree*}
    \begin{prooftree*}
      \AXC{$\seq[{ P }]{ \Gamma , \tm[x]{\give[m]{A}} , \tm[y]{\give[n]{A}} }$}
      \NOM{Contract}
      \UIC{$\seq[{ \subst{P}{x}{y} }]{ \Gamma , \tm[x]{\give[m+n]{A}} }$}
    \end{prooftree*}
  \end{center}

  \begin{center}
    \begin{prooftree*}
      \AXC{$\seq[P]{ \Gamma , \tm[y]{A} }$}
      \SYM{\take[1]{}}
      \UIC{$\seq[{x(y).P}]{ \Gamma , \tm[x]{\take[1]{A}} }$}
    \end{prooftree*}
    \begin{prooftree*}
      \AXC{$\seq[{ P }]{ \Gamma , \tm[x]{\take[m]{A}} }$}
      \AXC{$\seq[{ Q }]{ \Delta , \tm[x]{\take[n]{A}} }$}
      \NOM{Pool}
      \BIC{$\seq[{ (P \mid Q) }]{ \Gamma , \Delta , \tm[x]{\take[m+n]{A}} }$}
    \end{prooftree*}
  \end{center}
  
  \caption{The type system \(\text{CP}_{\text{ND}}\)}
  \label{fig:nodcap}
\end{figure*}

%%% Local Variables:
%%% TeX-master: "main"
%%% End:
%\begin{figure*}
  \begin{center}
    \begin{proofbox}
      \AXC{}
      \SYM{\one}
      \UIC{$\seq[{ \halt{y} }]{ \tm[y]{\one} }$}
      \SYM{\bot}
      \UIC{$\seq[{ \wait{x}\halt{y} }]{ \tm[x]{\bot}, \tm[y]{\one} }$}
    \end{proofbox}
    \begin{proofbox}
      \AXC{}
      \SYM{\top}
      \UIC{$\seq[{ \case{x}{}{} }]{ \tm[x]{\top}, \tm[y]{\nil} }$}
    \end{proofbox}
  \end{center}
  
  \begin{proofblock}
    \AXC{}
    \NOM{Ax}
    \UIC{$
      \seq[{ \link{x}{y} }]
      { \tm[x]{A^\bot}, \tm[y]{A} }$}
    \AXC{}
    \NOM{Ax}
    \UIC{$
      \seq[{ \link{z}{w} }]
      { \tm[z]{B^\bot}, \tm[w]{B} }$}
    \SYM{\tens}
    \BIC{$
      \seq[{ \send{w}{y}(\link{x}{y}\mid\link{z}{w}) }]
      { \tm[x]{A^\bot}, \tm[z]{B^\bot}, \tm[w]{A \tens B} }$}
    \SYM{\parr}
    \UIC{$
      \seq[{ \recv{z}{x}\send{w}{y}(\link{x}{y}\mid\link{z}{w}) }]
      { \tm[z]{A^\bot \parr B^\bot}, \tm[w]{A \tens B}}$}
  \end{proofblock}
  
  \begin{proofblock}
    \AXC{}
    \NOM{Ax}
    \UIC{$
      \seq[{ \link{x}{y} }]
      { \tm[x]{A^\bot}, \tm[y]{A} }$}
    \SYM{\plus}
    \UIC{$
      \seq[{ \inl{y}\link{x}{y} }]
      { \tm[x]{A^\bot}, \tm[y]{A \plus B} }$}
    \AXC{}
    \NOM{Ax}
    \UIC{$
      \seq[{ \link{x}{y} }]
      { \tm[x]{B^\bot}, \tm[y]{B} }$}
    \SYM{\plus}
    \UIC{$
      \seq[{ \inr{y}\link{x}{y} }]
      { \tm[x]{B^\bot}, \tm[y]{A \plus B} }$}
    \SYM{\with}
    \BIC{$
      \seq[{ \case{x}{\inl{y}\link{x}{y}}{\inr{y}\link{x}{y}} }]
      { \tm[x]{A^\bot \with B^\bot}, \tm[y]{A \plus B} }$}
  \end{proofblock}

  \begin{center}
    \begin{proofbox}
      \AXC{}
      \NOM{Ax}
      \UIC{$
        \seq[{ \link{z}{w} }]
        { \tm[z]{A^\bot}, \tm[w]{B} }$}
      \UIC{$
        \seq[{ \take{\recv{y}{w}}\link{z}{w} }]
        { \tm[z]{A^\bot}, \tm[y]{\take[1]{A}} }$}
      \UIC{$
        \seq[{ \give{\send{x}{z}}\take{\recv{y}{w}}\link{z}{w} }]
        { \tm[x]{\give[1]{A^\bot}}, \tm[y]{\take[1]{A}} }$}
    \end{proofbox}  
    \begin{proofbox}
      \AXC{}
      \NOM{Ax}
      \UIC{$
        \seq[{ \link{x}{y} }]
        { \tm[x]{\give[1]{A^\bot}}, \tm[y]{\take[1]{A}} }$}
      \AXC{}
      \NOM{Ax}
      \UIC{$
        \seq[{ \link{z}{w} }]
        { \tm[z]{\give[n]{A^\bot}}, \tm[w]{\take[n]{A}} }$}
      \NOM{Pool}
      \BIC{$
        \seq[{ (\link{x}{y}\mid\link{z}{y}) }]
        { \tm[x]{\give[1]{A^\bot}}, \tm[z]{\give[n]{A^\bot}}, \tm[y]{\take[n+1]{A}} }$}
      \NOM{Contract}
      \UIC{$
        \seq[{ (\link{x}{y}\mid\link{x}{y}) }]
        { \tm[x]{\give[n+1]{A^\bot}}, \tm[y]{\take[n+1]{A}} }$}
    \end{proofbox}
  \end{center}

  \caption{Proof of identity expansion for \nodcap}
  \label{fig:nodcap-ax}
\end{figure*}
%%% Local Variables:
%%% TeX-master: "main"
%%% End:

%\begin{figure*}
  \centering
  \begin{tabular}{ccc}
    \begin{proofbox}[1]
      \AXC{$
        \seq[{ P }]
        { \Gamma, \tm[y]{A} }$}
      \SYM{\give[1]{}}
      \UIC{$
        \seq[{ \give{\send{x}{y}}P }]
        { \Gamma, \tm[x]{\give[1]{A}} }$}
      \NOM{Exp}
      \UIC{$
        \seq[{ \expn{x}{x_1}\give{\send{x}{y}}P }]
        { \Gamma, \tm[x_1]{A} }$}
    \end{proofbox}
    &$\Longrightarrow$&
    \begin{proofbox}[1]
      \AXC{$
        \seq[{ \subst{P}{x_1}{y} }]
        { \Gamma, \tm[x_1]{A} }$}
    \end{proofbox}
    \\\\
    \begin{proofbox}[1]
      \AXC{$
        \seq[{ P }]
        { \Gamma, \tm[x]{\give[m]{A}}, \tm[y]{\give[n]{A}} }$}
      \NOM{Cont}
      \UIC{$
        \seq[{ \subst{P}{x}{y} }]
        { \Gamma, \tm[x]{\give[m+n]{A}} }$}
      \NOM{Exp}
      \UIC{$
        \seq[{ \expn{x}{x_1 \cdots x_{m+n}}\subst{P}{x}{y} }]
        { \Gamma, \tm[x_1]{A}\cdots\tm[x_{m+n}]{A} }$}
    \end{proofbox}
    &$\Longrightarrow$
    \\\\
    \multicolumn{3}{r}{%
    \begin{proofbox}[1]
      \AXC{$
        \seq[{ P }]
        { \Gamma, \tm[x]{\give[m]{A}}, \tm[y]{\give[n]{A}} }$}
      \NOM{Exp}
      \UIC{$
        \seq[{ \expn{x}{x_1 \cdots x_m}P }]
        { \Gamma, \tm[x_1]{A}\cdots\tm[x_m], \tm[y]{\give[n]{A}} }$}
      \NOM{Exp}
      \UIC{$
        \seq[{ \expn{y}{x_{m+1} \cdots x_{m+n}}\expn{x}{x_1 \cdots x_m}P }]
        { \Gamma, \tm[x_1]{A}\cdots\tm[x_{m+n}]{A} }$}
    \end{proofbox}
    }
  \end{tabular}
  
  \caption{Principal expansions for \nodcap}
  \label{fig:nodcap-expand-prin}
\end{figure*}

%%% Local Variables:
%%% TeX-master: "main"
%%% End:

%\begin{figure*}
  \centering
  \bgroup
  \def\arraystretch{4}
  \begin{tabular}{lr}
    \begin{prooftree*}
      \AXC{$\seq[{ P }]{ \Gamma, \tm[x]{\give[n]{A^\bot}}}$}
      \AXC{$\seq[{ Q }]{ \Delta, \tm[x]{\take[n]{A}} }$}
      \NOM{Cut}
      \BIC{$\seq[{ \cut{x}(P \mid Q) }]{ \Gamma, \Delta }$}
    \end{prooftree*}
    &
      $\Longrightarrow$
    \\
    \multicolumn{2}{r}{%
    \begin{prooftree*}
      \AXC{$\seq[{ Q }]{ \Delta, \tm[x]{\take[n]{A}} }$} 
      \AXC{$\seq[{ P }]{ \Gamma, \tm[x]{\give[n]{A^\bot}}}$}
      \NOM{Exp}
      \UIC{$\seq
        [{ \expn{x}{x_1 \cdots x_n}P }]
        { \Gamma, \tm[x_1]{A^\bot}\cdots\tm[x_n]{A^\bot} }$}
      \NOM{Int}
      \BIC{$\seq
        [{ \intl{x}{y_1 \cdots y_n}(Q \mid (\expn{x}{x_1 \cdots x_n}P) }]
        { \Gamma, \Delta }$}
    \end{prooftree*}
    }
  \end{tabular}
  \egroup
  
  \caption{Principal cuts for \nodcap}
  \label{fig:nodcap-cut-prin}
\end{figure*}

%%% Local Variables:
%%% TeX-master: "main"
%%% End:

%\begin{figure*}
  \centering
  \bgroup
  \def\arraystretch{4}%
  \begin{tabular}{lr}
    \begin{prooftree*}
      \AXC{$
        \seq[{ P }]
        { \Gamma, \tm[x]{A^\bot}, \Delta, \tm[w]{B} }$}
      \SYM{\give[1]{}}
      \UIC{$
        \seq[{ \send{y}{w}P }]
        { \Gamma, \tm[x]{A^\bot}, \Delta, \tm[y]{\give[1]{B}} }$}
      \AXC{$
        \seq[{ R }]
        { \Theta, \tm[x]{A} }$}
      \NOM{Cut}
      \BIC{$
        \seq[{ \cut{x}(\send{y}{w}P \mid R) }]
        { \Gamma, \Delta, \Theta, \tm[z]{\give[1]{B}} }$}
    \end{prooftree*}
    &
      $\Longrightarrow$
    \\
    \multicolumn{2}{r}{%
    \begin{prooftree*}
      \AXC{$
        \seq[{ P }]
        { \Gamma, \tm[x]{A^\bot}, \Delta, \tm[w]{B} }$}
      \AXC{$
        \seq[{ R }]
        { \Theta, \tm[x]{A} }$}
      \NOM{Cut}
      \BIC{$
        \seq[{ \cut{x}(P \mid R) }]
        { \Gamma, \Delta, \Theta, \tm[w]{B} }$}
      \SYM{\give[1]{}}
      \UIC{$
        \seq[{ \send{y}{w}\cut{x}(P \mid R) }]
        { \Gamma, \Delta, \Theta, \tm[y]{\give[1]{B}} }$}
    \end{prooftree*}
    }
    \\
    \multicolumn{2}{c}{(case for $\take[1]{}$ as above)}
    \\
    \begin{prooftree*}
      \AXC{$
        \seq[{ P }]
        { \Gamma, \tm[x]{A^\bot}, \Delta, \tm[y]{\give[p]{B}}, \tm[w]{\give[q]{B}} }$}
      \NOM{Cont}
      \UIC{$
        \seq[{ \subst{P}{y}{w} }]
        { \Gamma, \tm[x]{A^\bot}, \Delta, \tm[y]{\give[p+q]{B}} }$}
      \AXC{$
        \seq[{ R }]
        { \Theta, \tm[x]{A} }$}
      \NOM{Cut}
      \BIC{$
        \seq[{ \cut{x}(\subst{P}{y}{w} \mid R) }]
        { \Gamma, \Delta, \Theta, \tm[y]{\give[p+q]{B}} }$}
    \end{prooftree*}
    &
      $\Longrightarrow$
    \\
    \multicolumn{2}{r}{%
    \begin{prooftree*}
      \AXC{$
        \seq[{ P }]
        { \Gamma, \tm[x]{A^\bot}, \Delta, \tm[y]{\give[p]{B}}, \tm[w]{\give[q]{B}} }$}
      \AXC{$
        \seq[{ R }]
        { \Theta, \tm[x]{A} }$}
      \NOM{Cut}
      \BIC{$
        \seq[{ \cut{x}(P \mid R) }]
        { \Gamma, \Delta, \Theta, \tm[y]{\give[p]{B}}, \tm[w]{\give[q]{B}} }$}
      \NOM{Cont}
      \UIC{$
        \seq[{ \cut{x}(\subst{P}{y}{z} \mid R) }]
        { \Gamma, \Delta, \Theta, \tm[y]{\give[p+q]{B}} }$}
    \end{prooftree*}
    }
    \\
    \begin{prooftree*}
      \AXC{$
        \seq[{ P }]
        { \Gamma, \tm[x]{A^\bot}, \Delta, \tm[y]{\take[p]{B}} }$}
      \AXC{$
        \seq[{ Q }]
        { \Theta, \tm[y]{\take[q]{B}} }$}
      \NOM{Pool}
      \BIC{$
        \seq[{ (P \mid Q) }]
        { \Gamma, \tm[x]{A^\bot}, \Delta, \Theta, \tm[y]{\take[p+q]{B}} }$}
      \AXC{$
        \seq[{ R }]
        { \Xi, \tm[x]{A} }$}
      \NOM{Cut}
      \BIC{$
        \seq[{ \cut{x}((P \mid Q) \mid R) }]
        { \Gamma, \Delta, \Theta, \Xi, \tm[y]{\take[p+q]{B}} }$}
    \end{prooftree*}
    &
      $\Longrightarrow$
    \\
    \multicolumn{2}{r}{%
    \begin{prooftree*}
      \AXC{$
        \seq[{ P }]
        { \Gamma, \tm[x]{A^\bot}, \Delta, \tm[y]{\take[p]{B}} }$}
      \AXC{$
        \seq[{ R }]
        { \Xi, \tm[x]{A} }$}
      \NOM{Cut}
      \BIC{$
        \seq[{ \cut{x}(P \mid R) }]
        { \Gamma, \Delta, \Xi, \tm[y]{\take[p]{B}} }$}
      \AXC{$
        \seq[{ Q }]
        { \Theta, \tm[y]{\take[q]{B}} }$}
      \NOM{Pool}
      \BIC{$
        \seq[{ (\cut{x}(P \mid R) \mid Q) }]
        { \Gamma, \Delta, \Theta, \Xi, \tm[y]{\take[p+q]{B}} }$}
    \end{prooftree*}
    }
  \end{tabular}
  \egroup

  \caption{Permutation cuts for \nodcap}
  \label{fig:nodcap-cut-perm}
\end{figure*}

%%% Local Variables:
%%% TeX-master: "main"
%%% End:


\bibliography{main}

\end{document}

%%% Local Variables:
%%% TeX-master: "main"
%%% End:
