\documentclass[twocolumn]{article}
\usepackage{cmap}
\usepackage{alphabeta}
\usepackage[greek,english]{babel}
\languageattribute{greek}{polutoniko}
\usepackage{hyperref}
\usepackage{comment}
\usepackage{lmodern}

%%% TikZ and QTree

\usepackage{tikz}
\usepackage{tikz-qtree}

%%% Term Language
\newcommand{\subst}[3]{\ensuremath{#1 \{ #2 / #3 \}}}
\newcommand{\case}[3]{%
\toks0={#2#3}%
\edef\param{\the\toks0}%
\ifx\param\empty
  \ensuremath{\text{case} \; #1 \; \{\}}
\else
  \ensuremath{\text{case} \; #1 \; \{ #2 ; #3 \}}
\fi}
\newcommand{\inl}[2]{%
  \ensuremath{#1[\text{inl}].#2}}
\newcommand{\inr}[2]{%
  \ensuremath{#1[\text{inr}].#2}}

%%% Local Variables:
%%% TeX-master: "main"
%%% End:
\usepackage{lipsum}

\title{Give or take \\
  {\large Non-Determinism, Linear Logic and Session Types}}
\author{Pepijn Kokke \\
  CDT Pervasive Parallelism \\
  \url{pepijn.kokke@ed.ac.uk}}

\begin{document}
\maketitle

\begin{abstract}
  The aim of this project is to extend the session-typed \textpi-calculus CP
  \citep{wadler2012} with \emph{global} non-determinism---i.e.\ without using an
  explicit program construct for non-determinism---while preserving the strong
  guarantees of termination and deadlock freedom that are offered by the
  correspondence with classical linear logic. 
  
  Our extended version of CP will allow us to model more complex concurrent
  processes, such as an online ticket vendor which can sell some finite number
  of tickets, with customers racing to obtain a ticket.
  %
  We will be able to model further complications, such as a ticket vendor for
  which all tickets are distinct---e.g.\ by being labeled with distinct serial
  numbers---and vendors who maintain a list detailing which customer bought
  which ticket.
  %
  All of these are open problems for logic-inspired session-typed systems which
  guarantee termination.
\end{abstract}

\section{Introduction}
\remark{Feedback on this section will not be crucial, so I will briefly outline
  what I am \emph{planning} to write.}\\
\outline{I will briefly introduce the problem---logic-inspired session-typed
  systems do not model non-deterministic processes. We would like to have
  non-determinism, but we don't want non-termination or deadlocks.}

\section{Background}
\remark{Feedback on this section will not be crucial, so I will briefly outline
  what I am \emph{planning} to write. Some of the content of this section is
  currently in the methodology section, in order to make the current document an
  easy read---e.g.\ the introduction of CP. This will be migrated upwards in the
  final version.}

\subsection{\textPi-calculus}
\outline{I will briefly introduce the \textpi-calculus---it's syntax formally,
  and its semantics informally, the latter due to the fact that the semantics of
  the \textpi-calculus and CP differ slightly, and I don't want to introduce two
  different semantics in the span of six or seven pages.} 

\subsection{Session Types and Linear Logic}
\outline{I will briefly outline the work by \citet{girard1987, caires2010,
    wadler2012}, and introduce the proof rules for the
  simply-typed subset of CP.}
\rfc{Should I mention previous work by e.g.\ \citet{abramsky1993, abramsky1994,
    bellin1994} without going into too much---or any---detail?} 

\subsection{Session Types and Non-Determinism}
\outline{I will briefly outline the work on integrating local choice into linear
logic based session types~\citep{caires2014,atkey2016,caires2017}. 
I will explain why local non-determinism---i.e.\ having an explicit programming
construct which says ``choose either, I don't care''---is an unpleasant way of
writing non-deterministic programs. I will also informally explain the
difference between global and local non-determinism.}

\subsection{Bounded Linear Logic}
\outline{I will briefly outline bounded linear logic~\citep{girard1992}, and its
intended use. I will not go into any detail on the proof rules of BLL, as we are
using a simplification of the rules of BLL, and even then our rules are very
similar to what we end up using for non-determinism.}

\section{Methodology}
Classical Processes~\citep[CP;][]{wadler2012} is a process calculus with a
Curry-Howard correspondence to classical linear logic, and for which computation
is identified with cut elimination. One consequence of this is that computation
is entirely deterministic.

In this section, we will discuss which changes we will make to CP to be able to
model non-deterministic computation. In addition, we will sketch process
semantics for these constructs.

\subsection{Give and take}
Our aim is to add some construct for global non-determinism to CP.
In the \textpi-calculus, the non-determinism comes from two things: channel
sharing and the structural congruence relation.
This allows you to write a process such as e.g.\ %
\[
  (x[u].x[v].P \mid x(y).Q \mid x(z).R)
\]
This process is non-deterministic, because it is not determined a priori which
value sent along $x$ will be received by which process. 

In CP, however, channels are strictly binary---all communication is introduced
by the \textsc{Cut}-rule, which creates a channel with two endpoints, and hands
each endpoint to \emph{one} process.
\begin{proofblock}
  \AXC{$\seq[P]{ \Gamma , \tm[x]{A} }$}
  \AXC{$\seq[Q]{ \Delta , \tm[x]{A^\bot} }$}
  \NOM{Cut}
  \BIC{$\seq[{\nu x.(P \mid Q)}]{ \Gamma , \Delta }$}
\end{proofblock}
Furthermore, CP defines its reduction semantics on \emph{typed} processes, and
does away with structural congruence.
As a consequence, computation in CP is completely deterministic.

Our aim, therefore, is to capture safe name sharing.
We do this by extending CP with a pair of exponentials---$\give[n]{A}$ (give)
and $\take[n]{A}$ (take), where $n \in \mathbb{N}$.
We refer to the resulting system as \gtcp.
The type $\give[n]{A}$ represents a process which communicates as $A$, $n$
times, and these communications can depend on one another.
For instance, a p\^{a}tisserie could use the income from its sales to buy the
ingredients for new cakes.
The type $\take[n]{A}$ represents a \emph{pool} of $n$ processes, each of which
communicates as $A$, and each of these communications should be independent.
It is this \emph{pooling} which allows for name sharing, and therefore
non-determinism.
\begin{center}
  \begin{proofbox}
    \AXC{$\seq[{P}]{ \Gamma , \tm[y]{A} }$}
    \SYM{\give{}}
    \UIC{$\seq[{x[y].P}]{ \Gamma , \tm[x]{\give[1]{A}} }$}
  \end{proofbox}
  \begin{proofbox}
    \AXC{$\seq[P]{ \Gamma , \tm[y]{A} }$}
    \SYM{\take{}}
    \UIC{$\seq[{x(y).P}]{ \Gamma , \tm[x]{\take[1]{A}} }$}
  \end{proofbox}
  \\[1\baselineskip]
  \begin{proofbox}
    \AXC{$\seq[P]{ \Gamma , \tm[x]{\give[m]{A}} , \tm[x']{\give[n]{A}} }$}
    \NOM{Cont}
    \UIC{$\seq[\subst{P}{x}{x'}]{ \Gamma , \tm[x]{\give[m+n]{A}} }$}
  \end{proofbox}
  \\[1\baselineskip]
  \begin{proofbox}
    \AXC{$\seq[P]{ \Gamma , \tm[x]{\take[m]{A}} }$}
    \AXC{$\seq[Q]{ \Delta , \tm[x]{\take[n]{A}} }$}
    \NOM{Pool}
    \BIC{$\seq[P \mid Q]{ \Gamma , \Delta , \tm[x]{\take[m+n]{A}} }$}
  \end{proofbox}
\end{center}
We eliminate the cut on a channel of type $\give[n]{A^\bot} / \take[n]{A}$
by linking each communication of the process $\give[n]{A^\bot}$ to an arbitrary
process from the pool of processes $\take[n]{A}$. 
Thus, we can rewrite the above cut into $n$ smaller cuts.
In addition, if the endsequent does not contain any channels of type
$\give[n]{A}$ or $\take[n]{A}$, then we should be able to eliminate \emph{all}
usage of the rules for non-determinism. This isn't surprising---after running a
non-deterministic program, we should end up in \emph{some} deterministic end
state.

\rfc{Should I briefly outline the reduction semantics for give and take, or can
  I assume that they'll be clear from the \textpi-calculus terms I've given?}

\subsection{Relation with bounded linear logic}
The connectives $\give[n]{A}$ and $\take[n]{A}$ were first used in the context
of bounded linear logic~\citep[BLL;][]{girard1992}, a type system based on
intuitionistic linear logic which guarantees polynomial time complexity.

Here we use the connectives in a classical setting.
However, this is not the only difference.
The connectives in BLL do not have a pooling rule---instead, they have a
\emph{storage} rule, which allows for arbitrary---but bounded---duplication.
\begin{center}
  \begin{proofbox}
    \AXC{$\seq{ \give[m]{\Gamma} , \take[m]{A} }$}
    \NOM{Store}
    \UIC{$\seq{ \give[mn]{\Gamma} , \take[mn]{A} }$}
  \end{proofbox}
\end{center}
We can show that, for $n \geq 1$, pooling is a generalization of
storage.

However, this does require making $n$ copies of the proof.
By induction.
\begin{proofblock}
  \AXC{IH}
  \noLine
  \UIC{$\vphantom{!}\smash[t]{\vdots}$}
  \noLine
  \UIC{$\seq{ \give[mn]{\Gamma} , \take[mn]{A} }$}
  \AXC{ARG}
  \noLine
  \UIC{$\vphantom{!}\smash[t]{\vdots}$}
  \noLine
  \UIC{$\seq{ \give[m]{\Gamma} , \take[m]{A} }$}
  \NOM{Pool}
  \BIC{$\seq{ \give[mn]{\Gamma} , \give[m]{\Gamma} , \take[m(n+1)]{A} }$}
  \NOM{Cont}
  \UIC{$\seq{ \give[m(n+1)]{\Gamma} , \take[m(n+1)]{A} }$}
\end{proofblock}
In our system, the processes pooled under $\take[n]{A}$ can be distinct.
Therefore, we most likely lose the time complexity guarantees offered by BLL.

BLL also has a weakening rule, which introduces a channel of type $\give[0]{A}$.
We have, so far, not found a use for weakening or the $0$-case. 

\subsection{A p\^{a}tisserie and its customer}
We can use the new connectives to implement non-deterministic programs.
For instance, in \autoref{fig:cake-store}, we model a p\^{a}tisserie which sells
exactly one cake, yet can have two customers race for the cake.

The first process represents a customer. Customers want a channel over which
they either can buy a cake---trading a coin for a cake---or which tells them
when cake is unavailable. This is represented using the type
$(\coin\mult\cake^\bot)\with\bot$.
The second process is the p\^{a}tisserie itself. It offers up a channel of type
$\give[2]{(\coin^\bot\parr\cake)\plus{1}}$---a channel over which two customers
can try to purchase a cake simultaneously.

In order to run the program, we pool two customers together, and have them
communicate with the store.
\begin{figure*}
  \begin{mdframed}
    \vspace*{1em}
    \begin{proofblock}  
      \AXC{$\vphantom{!}\smash[t]{\vdots}$}
      \noLine
      \UIC{$\seq[P]{ \Gamma , \tm[y]{\coin} }$}
      \AXC{$\vphantom{!}\smash[t]{\vdots}$}
      \noLine
      \UIC{$\seq[Q]{ \Delta , \tm[z]{\cake^\bot} }$}
      \BIC{$\seq[{y[z].(P \mid Q)}]{
          \Gamma , \Delta , \tm[y]{\coin\mult\cake^\bot} }$}
      \AXC{$\vphantom{!}\smash[t]{\vdots}$}
      \noLine
      \UIC{$\seq[R]{ \Gamma , \Delta }$}
      \SYM{\bot}
      \UIC{$\seq[{y().R}]{ \Gamma , \Delta , \tm[y]{\bot} }$}
      \SYM{\with}
      \BIC{$\seq[{\case{y}{y[z].(P \mid Q)}{y().R}}]{
          \Gamma , \Delta , \tm[y]{(\coin\mult\cake^\bot) \with \bot} }$}
      \SYM{\take{}}
      \UIC{$\seq[{!x(y).\case{y}{y[z].(P \mid Q)}{y().R}}]{
          \Gamma , \Delta ,
          \tm[x]{\take[1]{((\coin\mult\cake^\bot) \with \bot)}} }$}
    \end{proofblock}
    \vspace*{0pt}
    \begin{proofblock} 
      \AXC{}
      \SYM{1}
      \UIC{$\seq[{y[].0}]{ \tm[y]{1} }$}
      \SYM{\plus_2}
      \UIC{$\seq[{\inr{y}{y[].0}}]{
          \tm[y]{(\coin^\bot\parr\cake)\plus{1}} }$}
      \SYM{\give{}}
      \UIC{$\seq[{?x'[y].\inr{y}{y[].0}}]{
          \tm[x']{\give[1]{((\coin^\bot\parr\cake)\plus{1})}} }$}
      \noLine
      \UIC{$\vphantom{!}\smash[t]{\vdots}$}
      \noLine
      \UIC{$\seq[{\mathcal{E}[{\inr{x'}{x'[].0}}]}]{
          \Theta ,
          \tm[z]{\coin^\bot\parr\cake} ,
          \tm[x']{\give[1]{((\coin^\bot\parr\cake)\plus{1})}} }$}
      \SYM{\plus_1}
      \UIC{$\seq[\inl{z}{\mathcal{E}[{?x'(y).\inr{y}{y[].0}}]}]{
          \Theta ,
          \tm[z]{(\coin^\bot\parr\cake)\plus{1}} ,
          \tm[x']{\give[1]{((\coin^\bot\parr\cake)\plus{1})}} }$}
      \SYM{\give{}}
      \UIC{$\seq[{?x[z].\inl{z}{\mathcal{E}[{?x'[y].\inr{y}{y[].0}}]}}]{
          \Theta ,
          \tm[x]{\give[1]{((\coin^\bot\parr\cake)\plus{1})}} ,
          \tm[x']{\give[1]{((\coin^\bot\parr\cake)\plus{1})}} }$}
      \NOM{Cont}
      \UIC{$\seq[{?x[z].\inl{z}{\mathcal{E}[{?x[y].\inr{y}{y[].0}}]}}]{
          \Theta ,
          \tm[x]{\give[2]{((\coin^\bot\parr\cake)\plus{1})}} }$}
    \end{proofblock}
    \vspace*{0pt}
  \end{mdframed}
  \caption{A p\^{a}tisserie and its customer.}\label{fig:cake-store}
\end{figure*}


\subsection{Take tensors, give pars}
The bounded modalities give ($\give[n]{A}$) and take ($\take[n]{A}$) have a
strong connection to par ($\parr$) and tensor ($\mult$)---they are vectors of
pars and tensors.
We can demonstrate this by showing that we can unfold a give into a series
of pars.
\begin{proofblock}
  \AXC{}
  \NOM{Ax}
  \UIC{$\seq{ A^\bot , A }$}
  \SYM{\take{}}
  \UIC{$\seq{ \take[1]{A^\bot} , A }$}
  \AXC{}
  \NOM{Ax}
  \UIC{$\seq{ \take[n]{A^\bot} , \give[n]{A} }$}
  \NOM{Pool}
  \BIC{$\seq{ \take[n+1]{A^\bot} , A , \give[n]{A} }$}
  \SYM{\parr}
  \UIC{$\seq{ \take[n+1]{A^\bot} , A \parr \give[n]{A} }$}
  \SYM{\parr}
  \UIC{$\seq{ \give[n+1]{A} \limp A \parr \give[n]{A} }$}
\end{proofblock}
And vice versa---we can roll a series of pars into a give.
\begin{proofblock}
  \AXC{}
  \NOM{Ax}
  \UIC{$\seq{ A^\bot , A }$}
  \SYM{\give{}}
  \UIC{$\seq{ A^\bot , \give[1]{A} }$}
  \AXC{}
  \NOM{Ax}
  \UIC{$\seq{ \take[n]{A^\bot} , \give[n]{A} }$}
  \SYM{\mult}
  \BIC{$\seq{ A^\bot \mult \take[n]{A^\bot} , \give[1]{A} , \give[n]{A} }$}
  \NOM{Cont}
  \UIC{$\seq{ A^\bot \mult \take[n]{A^\bot} , \give[n+1]{A} }$}
  \SYM{\parr}
  \UIC{$\seq{ A \parr \give[n]{A} \limp \give[n+1]{A} }$}
\end{proofblock}
We can give similar proofs for take and tensor.
Thus, we have both $\give[n+1]{A} \lequiv A \parr \give[n]{A}$ and
$\take[n+1]{A} \lequiv A \mult \take[n]{A}$.

Given our description of $\give[n]{A}$ and $\take[n]{A}$ as vectors, it should
come as no surprise that we have $\give[1]{A} \lequiv \take[1]{A}$.
For unit vectors, it should not matter whether we build them using tensors or
pars.
\begin{proofblock}
  \AXC{}
  \RightLabel{Ax}
  \UIC{$\seq{A^\bot, A}$}
  \RightLabel{\take{}(2)}
  \UIC{$\seq{\take[1]{A^\bot}, \take[1]{A}}$}
  \RightLabel{\parr}
  \UIC{$\seq{\give[1]{A} \limp \take[1]{A}}$}
  \AXC{}
  \RightLabel{Ax}
  \UIC{$\seq{A^\bot, A}$}
  \RightLabel{\give{}(2)}
  \UIC{$\seq{\give[1]{A^\bot}, \give[1]{A}}$}
  \RightLabel{\parr}
  \UIC{$\seq{\take[1]{A} \limp \give[1]{A}}$}
  \RightLabel{\mult}
  \BIC{$\seq{\give[1]{A} \lequiv \take[1]{A}}$}
\end{proofblock}
As mentioned before, BLL has a weakening rule. In our analogy as vectors, this
rule corresponds to $\bot$, and the $0$-case for \textsc{Store} corresponds $1$.
\begin{center}
  \begin{proofbox}
    \AXC{$\seq{ \Gamma }$}
    \NOM{Weaken}
    \UIC{$\seq{ \Gamma , \give[0]{A} }$}
  \end{proofbox}
  \begin{proofbox}
    \AXC{$\vphantom{\Gamma}$}
    \NOM{Store}
    \UIC{$\seq{ \take[0]{A} }$}
  \end{proofbox}
\end{center}
\rfc{Should I briefly describe what these foldings and unfoldings of give and
  take do \emph{operationally}.}

\subsection{Conflating $\give[n]{A}$ and $\take[n]{A}$}
Give and take impose a strict structure on the kinds of non-deterministic which
we can model. Specifically, we always need one server process which offers to
communicate $n$ times, and $n$ independent client processes
There is a good reason for this. After cut elimination, we are left with a
single, central process---the server process.
However, this may not always be what we desire.
Imagine a matchmaking service.
This could be a dating site, an online market place---any process whose purpose
it is to link \emph{two} pools of independent processes.
In such a scenario, it is entirely reasonable to write programs which end up as
several independent processes. For instance,
\[
  (x[u].P \mid x[v].Q \mid x(y).R \mid x(z).S)
\]
In the above program, the names in $P$, $Q$, $R$ and $S$ are disjoint, aside
from $x$. This means that whichever way we choose to communicate along $x$, we
will end up with two completely independent processes.
Logically, allowing such processes corresponds to adding the \textsc{Mix}-rule.
\begin{proofblock}
  \AXC{$\seq[P]{ \Gamma }$}
  \AXC{$\seq[Q]{ \Delta }$}
  \NOM{Mix}
  \BIC{$\seq[P \mid Q]{ \Gamma , \Delta }$}
\end{proofblock}
If we add \textsc{Mix} to our type system, we get proofs of $\bot \limp 1$ and
$A \mult B \limp A \parr B$. Vice versa, if we have a proof of $\bot \limp 1$,
we can derive \textsc{Mix}.
This phenomenon was recently investigated in the context of CP
by~\citet{atkey2016}. 
Unsurprisingly, due to the relation between $\give[n]{}/\take[n]{}$ and
$\parr/\mult$, \textsc{Mix} also gives us $\take[n]{A}\limp\give[n]{A}$. 
\begin{proofblock}
  \AXC{}
  \NOM{Ax}
  \UIC{$\seq{ A^\bot , A }$}
  \SYM{\give{}(2)}
  \UIC{$\seq{ \give[1]{A^\bot} , \give[1]{A} }$}
  \AXC{IH}
  \noLine
  \UIC{$\vphantom{()}{\smash[t]{\vdots}}$}
  \noLine
  \UIC{$\seq{ \give[n]{A^\bot} , \give[n]{A} }$}
  \NOM{Mix}
  \BIC{$\seq{
      \give[1]{A^\bot} , \give[1]{A} , \give[n]{A^\bot} , \give[n]{A} }$}
  \NOM{Cont(2)}
  \UIC{$\seq{ \give[n+1]{A^\bot} , \give[n+1]{A} }$}
\end{proofblock}
This relation holds the other way around as well---if we have a proof of
$\give[2]{1}$, we can derive \textsc{Mix}.

From \citet{atkey2016}, we also know that if we add \textsc{MultiCut}, we get
$1 \limp \bot$ and $A \parr B \limp A \mult B$ (and vice versa).
\begin{proofblock}
  \AXC{$\seq{ \Gamma , A_1^\bot , \ldots , A_n^\bot }$}
  \AXC{$\seq{ \Delta , A_1     , \ldots , A_n     }$}
  \NOM{MultiCut}
  \BIC{$\seq{ \Gamma , \Delta }$}
\end{proofblock}
Adding \textsc{MultiCut} also gives us $\give[n]{A}\limp\take[n]{A}$, see
\autoref{fig:conflate-give-and-take}, resulting in a proof of $\take[n]{A}
\lequiv \give[n]{A}$.
\begin{figure*}
  \begin{mdframed}
    \begin{proofblock}
      \AXC{}
      \NOM{Ax}
      \UIC{$\seq{ A^\bot , A }$}
      \SYM{\take{}}
      \UIC{$\seq{ \take[1]{A^\bot} , A }$}
      \AXC{$\ldots$}
      \AXC{}
      \NOM{Ax}
      \UIC{$\seq{ A^\bot , A }$}
      \SYM{\take{}}
      \UIC{$\seq{ \take[1]{A^\bot} , A }$}
      \NOM{Pool}
      \TIC{$\seq{ \take[n]{A^\bot} , A , \ldots , A }$}
      \AXC{}
      \NOM{Ax}
      \UIC{$\seq{ A^\bot , A }$}
      \SYM{\take{}}
      \UIC{$\seq{ A^\bot , \take[1]{A} }$}
      \AXC{$\ldots$}
      \AXC{}
      \NOM{Ax}
      \UIC{$\seq{ A^\bot , A }$}
      \SYM{\take{}}
      \UIC{$\seq{ A^\bot , \take[1]{A} }$}
      \NOM{Pool}
      \TIC{$\seq{ A^\bot , \ldots , A^\bot , \take[n]{A} }$}
      \NOM{MultiCut}
      \BIC{$\seq{ \take[n]{A^\bot} , \take[n]{A} }$}
      \SYM{\parr}
      \UIC{$\seq{ \give[n]{A} \limp \take[n]{A} }$}
    \end{proofblock}
  \end{mdframed}
  \caption{Conflation of $\give[n]{A}$ and $\take[n]{A}$ in the presence of
    \textsc{MultiCut}.}
  \label{fig:conflate-give-and-take}
\end{figure*}
The other way around, however, this relation does not hold.

We formulate a system in which we conflate $\give[n]{A}$ and $\take[n]{A}$,
following \citet{atkey2016} in calling this connective $\nod[n]{A}$---though,
since they provide no pronunciation other than ``an access point of type $A$'',
I will pronounce it as ``nod''.
We add a rule for $\nod[1]{}$ elimination, a contraction rule and the
\textsc{Mix} rule.
\begin{center}
  \begin{proofbox}
    \AXC{$\seq[P]{ \Gamma , \tm[y]{A} }$}
    \SYM{\nod{}}
    \UIC{$\seq[x\nod{y}.P]{ \Gamma , \tm[x]{\nod[1]{A}} }$}
  \end{proofbox}
  \begin{proofbox}
    \AXC{$\seq[P]{ \Gamma , \tm[y]{\nod[m]{A}} , \tm[z]{\nod[n]{A}} }$}
    \NOM{Cont}
    \UIC{$\seq[\subst{P}{x}{y,z}]{ \Gamma , \tm[x]{\nod[m+n]{A}} }$}
  \end{proofbox}
\end{center}
There is no reason to add \textsc{Pool}, as it is derivable from \textsc{Cont}
and \textsc{Mix}.
\begin{proofblock}
  \AXC{$\seq{ \Gamma , \nod[m]{A} }$}
  \AXC{$\seq{ \Delta , \nod[n]{A} }$}
  \NOM{Mix}
  \BIC{$\seq{ \Gamma , \Delta , \nod[m]{A} , \nod[n]{A} }$}
  \NOM{Cont}
  \UIC{$\seq{ \Gamma , \Delta , \nod[m+n]{A} }$}
\end{proofblock}
We will refer to this system as \ndcp.

In \ndcp, we can derive a restricted version of \textsc{MultiCut}.
By applying \textsc{Cut} to a ``vector'' of $A$s, we can simulate communication
on multiple channels, as long as all of these channels have the same session
type.
\begin{proofblock}
  \AXC{$\seq{ \Gamma , A^\bot , \ldots , A^\bot  }$}
  \SYM{\nod{}}
  \UIC{$\seq{ \Gamma , \nod[1]{A^\bot} , \ldots , \nod[1]{A^\bot} }$}
  \NOM{Cont}
  \UIC{$\seq{ \Gamma , \nod[n]{A^\bot} }$}
  \AXC{$\seq{ \Delta , A , \ldots , A }$}
  \SYM{\nod{}}
  \UIC{$\seq{ \Delta , \nod[1]{A} , \ldots , \nod[1]{A}     }$}
  \NOM{Cont}
  \UIC{$\seq{ \Delta , \nod[n]{A} }$}
  \NOM{Cut}
  \BIC{$\seq{ \Gamma , \Delta }$}
\end{proofblock}
It is possible that cut elimination for \ndcp\ terminates.
The uses of \textsc{MultiCut} which lead to deadlocks invariably communicate on
two channels of dual types, as in
\begin{proofblock}
  \AXC{}
  \NOM{Ax}
  \UIC{$\seq[\link{x}{y}]{ \tm[x]{ A^\bot } , \tm[y]{ A } }$}
  \AXC{}
  \NOM{Ax}
  \UIC{$\seq[\link{y}{x}]{ \tm[y]{ A^\bot } , \tm[x]{ A } }$}
  \NOM{MultiCut}
  \BIC{$\seq[(\link{x}{y}\mid\link{y}{x})]{ }$}
\end{proofblock}
Why is \ndcp\ interesting?
It models an even more permissive form of non-determinism than
\gtcp+\textsc{Mix}.
In the latter, we can model a program in which two pools of processes are
non-deterministically linked together.
However, we cannot model a program in which two pools of sequentially
communicating processes are linked together. For instance,
\[
  (x[u_1].x[u_2].P \mid x[v].Q \mid x(w).R \mid x(z_1).x(z_2).S)
\]
We \emph{can} model such a process in \ndcp.
Why would we want to?
It may not unreasonable---in the case of a p\^{a}tisserie---to want to model a
customer buying \emph{two or more} pastries.
This would require us to have, on the one side, a process for the store which
communicates with $n$ customers sequentially.
On the other side, we would have a pool of processes which may communicate,
several times, with the store.
We could model this by having the store offer a transaction which can be used
some bounded number of times, e.g.\ %
\[
  \give[n]{( \coin\mult\cake^\bot )}
  \parr
  ( \forall{m \leq n}.{\give[m]{( \coin\mult\cake^\bot )}} ),
\]
but a simpler solution is to just require the customer to return to the pool
after the first transaction, and wait for another turn.

\rfc{Should my examples be cake-related throughout? For instance, many
  p\^{a}tisseries selling cakes to many customers, with each customer buying one
  or more cakes, would be an example that we can model using $\nod[n]{}$, but
  not using $\give[n]{}$ and $\take[n]{}$.}


\subsection{Relation to recursion}
There is still an obvious open question.
In many of the proofs throughout this proposal, we've left open branches,
labeled IH for induction hypothesis.
However, as it stands we are unable to express these induction proofs
\emph{within} the languages \gtcp\ or \ndcp.
To do so, we would have to add some sort of induction principle.
For instance,
\begin{proofblock}
  \AXC{$\seq{
      \give[1]{\Gamma} , \take[1]{\Delta} }$}
  \AXC{$\seq{
      \forall{n} .
      \take[n]{\Gamma^\bot} , \give[n]{\Delta^\bot} ,
      \give[n+1]{\Gamma} , \take[n+1]{\Delta} }$}
  \NOM{Ind}
  \BIC{$\seq{
      \forall{n} .
      \give[n]{\Gamma^\bot} , \take[n]{\Delta} }$}
\end{proofblock}
\begin{proofblock}
  \AXC{$\seq{\forall{n} . \give[n]{\Gamma} , \take[n]{\Delta}}$}
  \SYM{\forall}
  \UIC{$\seq{\give[m]{\Gamma} , \take[m]{\Delta}}$}
\end{proofblock}
This would add a restricted form of induction and polymorphism over resource
variables to our language.

\citet{lindley2016} describe an extension of CP, called \mucp, which adds
connectives for structural recursion.
It would be interesting to see what the interactions are between these recursion
principles and induction on resource variables.

Using the above induction principle, we can unfold a $\give[n]{A}/\take[n]{A}$
into a stream or pars or tensors, i.e.\ %
\begin{gather*}
  \give[n]{A} \mult (\mu X.A \parr X) \limp (\mu X.A \parr X),\\
  \take[n]{A} \mult (\mu X.A \mult X) \limp (\mu X.A \mult X),
\end{gather*}
or fold the first $n$ values from a stream into a $\give[n]{A}/\take[n]{A}$,
\begin{gather*}
  (\nu X.A \parr X) \limp (\give[n]{A} \parr \nu X.A \parr X),\\
  (\nu X.A \mult X) \limp (\take[n]{A} \parr \nu X.A \mult X).
\end{gather*}
Equivalently, we can unfold a $\give[n]{A}/\take[n]{A}$ into a list,
\begin{gather*}
  \give[n]{A} \limp (\mu X. (A \parr X) \plus 1),\\
  \take[n]{A} \limp (\mu X. (A \mult X) \plus 1),
\end{gather*}
though, in order to unfold a list into a $\take[n]{A}$, we would need to add
existential quantification over resource variables.

We could these constructions in order to, for example, implement a ticket server
which sells $n$ tickets, labeled $1$ to $n$, by recursion.

\subsection{Formalization}
We plan to formalize the systems \gtcp\ and \ndcp\ using Agda, and implement
their cut elimination procedures.
This will give us a strong guarantee that the described cut elimination
procedures are correct and terminating.
For this, we will use the techniques for separating types and usages used in
\citet{mcbride2016}.

\rfc{While recent conversations with Guillaume Allais have convinced me that
  formalizing these systems is practically feasible, I'm not sure about
  including such a formalization as a goal and research output in my proposal.
  Also, \citet{mcbride2016} is not the perfect body of work to cite, but I'm not 
  entirely sure if Conor has written any of this down. I've enquired with
  Guillaume.}


\section{Evaluation}
We can evaluate the success of this project by whether or not a
non-deterministic but terminating cut elimination procedure can be formulated
for \gtcp\ and \ndcp, and by the new kinds of non-deterministic programs which
these systems will be able to model.
We can reliably evaluate these conditions based on the formalizations of the
systems \gtcp\ and \ndcp.

\rfc{What should be the conditions for success w.r.t.\ induction and recursion?
  We cannot expression $\mu/\nu$-recursion using resource variable induction, as
  the latter is always bounded, but I'm not sure whether or not we can express
  resource variable induction using $\mu/\nu$-recursion, since the former is, in
  some sense, an induction principle, whereas the latter is a recursor.}

\section{Research outputs}
By the end of this project, I hope to have:
\begin{itemize}
\item%
  formulated the system \gtcp---which extends CP with a pair of dual
  exponentials, $\give[n]{A}$ and $\take[n]{A}$ for non-deterministic
  computation---and given a cut elimination procedure;
\item%
  formalized the system \gtcp;
\item%
  characterized the non-determinism in \gtcp;
\item%
  formulated the system \ndcp---which collapses the dual exponentials
  from \gtcp\ into a single, self-dual exponential $\nod[n]{A}$---and given a
  cut elimination procedure;
\item%
  formalized the system \ndcp;
\item%
  characterized the non-determinism in \ndcp;
\item%
  characterized the interactions between the structural recursion in \mucp\ and
  resource variable induction in \gtcp\ and \ndcp.
\end{itemize}
\rfc{Should I make (1) formulating the system, (2) giving a cut elimination
  procedure, and (3) proving termination and deadlock freedom for the cut
  elimination procedure separate research outputs?}

\section{Workplan}
\outline{What I'll say here is that I will work on \gtcp\ and \ndcp\ in
  parallel. Both of these projects have the structure (1) formulate system, (2)
  show cut elimination, (3) prove termination, with a formalization effort
  running in parallel. Finally, I'll write throughout, and change my focus to
  writing exclusively for the last month or two.}

\bibliographystyle{apalike}
\bibliography{main}
\nocite{*}

\end{document}